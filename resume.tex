Dans cette th\`ese, nous proposons des algorithmes proximaux, avec réduction de dimension automatique, pour des probl\`emes d’optimisation avec solutions parcimonieuses. Dans un premier temps, nous proposons une méthode générale de r\'eduction de dimension, exploitant la propri\'et\'e d’identification proximale, par des projections adaptées à la structure de l’it\'er\'e courant. Dans le cas parcimonieux, cet algorithme permet de travailler dans des sous-espaces al\'eatoires de petites dimensions plut\^ot que dans l’espace entier, possiblement de tr\`es grande dimension. Dans un deuxi\`eme temps, nous nous pla\c cons dans un cadre d’optimisation distribu\'ee asynchrone et utilisons la méthode pr\'ec\'edente pour réduire la taille des communications entre machines. Nous montrons tout d’abord, que l’application directe de notre m\'ethode de r\'eduction dimension dans ce cadre fonctionne si le probl\`eme est bien conditionn\'e. Pour attaquer les probl\`emes g\'en\'eraux, nous proposons ensuite un reconditionnement proximal donnant ainsi un algorithme avec garanties th\'eor\'etiques de convergence et de r\'eduction de communications. Des experiences num\'eriques montrent un gain important pour des probl\`emes classiques fortement parcimonieux.