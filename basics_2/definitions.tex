\section{Convexity and smoothness}\label{sec:basics_conv_and_smoothness}
In this section, we recall the basic definitions and properties that are used to analyse optimization methods for smooth and convex objective \cite{hiriart2012fundamentals}. 
\begin{definition}[Convex function]
A function $f:\RR^n\rightarrow \RR\cup{+\infty}$ is called convex if $\dom f$ is a convex set and for any $x, y\in \dom f$ and $\alpha \in[0,1]$
\begin{equation}\label{eq:conv_def}
f(\alpha x + (1-\alpha)y)\leq \alpha f(x) + (1-\alpha)f(y),
\end{equation}
where the domain of function $f$ is defined as
$$
\dom f = \{x:|f(x)|<\infty\}.
$$
Moreover, function $f$ is called $\mu$-strongly convex if function $g = f - \frac{\mu}{2}\|x\|_2^2$ is convex.
\end{definition}

From this definition we could immediately get the following interpretation for smooth functions $f$ that is more practical for further analysis. 

\begin{lemma}\label{lm:convex}[Theorem $2.1.2$ \cite{nesterov-book}]
    A continuously differentiable function $f:\RR^n\rightarrow \RR\cup{+\infty}$ is convex iff for any $x,y\in\mathbb{R}^n$, we have
    \begin{equation}\label{eq:convex_1}
        f(y) \geq f(x) + \langle \nabla f(x), y-x\rangle.
    \end{equation}
    Moreover, function $f:\RR^n\rightarrow \RR\cup{+\infty}$ is $\mu$-strongly convex iff for any $x,y$ in $\mathbb{R}^n$, we have
    \begin{equation}\label{eq:convex_2}
        f(y) \geq f(x) + \langle \nabla f(x), y-x\rangle + \frac{\mu}{2}\|y-x\|_2^2.
    \end{equation}
\end{lemma}
\begin{proof}[Proof of Lemma \ref{lm:convex}]
If $f$ is a convex function then for any $\alpha\in[0,1]$ we have 
$$
f(y)\geq f(x) +\frac{1}{1-\alpha}\left(f(\alpha x + (1-\alpha)y) - f(x)\right),
$$
where the computation of limit for $\alpha \rightarrow 1$ will give \eqref{eq:convex_1}.

Let us now prove the convexity of $f$ form \eqref{eq:convex_1}.
Summing the following inequalities with multipliers $\alpha$ and $1-\alpha$ correspondingly
$$
f(\alpha x + (1-\alpha)y) \leq f(x) - \langle \nabla f(\alpha x + (1-\alpha)y), (1-\alpha)(x-y)\rangle = f(x) + (1-\alpha) \langle \nabla f(\alpha x + (1-\alpha)y), y-x\rangle
$$
$$
f(\alpha x + (1-\alpha)y) \leq f(y) - \langle \nabla f(\alpha x + (1-\alpha)y), \alpha(y-x)\rangle = f(x) - \alpha \langle \nabla f(\alpha x + (1-\alpha)y), y-x\rangle
$$
we immediately get \eqref{eq:conv_def}. 

The proof of \eqref{eq:convex_2} immediately follows from the definition of the strongly convexity.
\end{proof}

Convexity implies that any stationary point is a global minima of $f$. In addition, the strong convexity implies the existence and uniqueness of $x^\star = \argmin_{x\in\mathbb{R}^n} f(x)$.

Now let us recall the definition of the $L$-smoothness. 

\begin{definition}[$L$-smoothness]
Differentiable function $f$ is called $L$-smooth if its gradient is $L$-Lipschitz
\begin{equation}
    \|\nabla f(x) - \nabla f(y)\|_2\leq L\|x-y\|_2
\end{equation}
for any $x,y\in\mathbb{R}^n$.
\end{definition}

If function $f$ is $L$-smooth then the following upper bound takes place.

\begin{lemma}\label{lm:descent}[Theorem $2.1.5$ \cite{nesterov-book}]
    Let us assume that $f:\RR^n\rightarrow \RR\cup{+\infty}$ is convex and $L$-smooth, then for any $x,y\in\mathbb{R}^n$
    \begin{equation}\label{eq:descent_lemma_1}
        f(y) - f(x) - \langle \nabla f(x), y-x\rangle\leq \frac{L}{2}\|x-y\|_2^2
    \end{equation}
    and
    \begin{equation}\label{eq:descent_lemma_2}
        \frac{1}{L}\|\nabla f(x) - \nabla f(y)\|_2^2 \leq\langle \nabla f(x) - \nabla f (y), x - y\rangle \leq L\|x-y\|_2^2.
    \end{equation}
\end{lemma}
\begin{proof}[Proof of Lemma \ref{lm:descent}]
   For any $x,y\in\mathbb{R}^n$ we have
   \begin{align}
        f(y) - f(x)  &= \int_0^1\langle\nabla f(x + \alpha(y-x)), y-x\rangle d\alpha\nonumber\\
         &= \langle \nabla f(x), y-x\rangle + \int_0^1\langle\nabla f(x + \alpha(y-x)) - \nabla f(x), y-x\rangle d\alpha
   \end{align}
Now, using  convexity of $f$ and Cauchy–Schwarz inequality we have
\begin{align}\label{eq:lipsh}
    f(y)-f(x) - \langle \nabla f(x), y-x\rangle &= |f(y)-f(x) - \langle \nabla f(x), y-x\rangle|\nonumber\\ &= \left|\int_0^1\langle\nabla f(x + \alpha(y-x)) - \nabla f(x), y-x\rangle d\alpha\right|\nonumber\\
 &\leq \int_0^1|\langle\nabla f(x + \alpha(y-x)) - \nabla f(x), y-x\rangle| d\alpha\nonumber\\
&\leq\int_0^1\|\nabla f(x + \alpha(y-x)) - \nabla f(x)\|_2\|y-x\|_2 d\alpha\nonumber\\
&\leq \int_0^1\tau L\|x-y\|_2^2 d\tau = \frac{L}{2}\|x-y\|_2^2.
\end{align}

To prove the second part let us sum \eqref{eq:lipsh} for the pairs of points $(x,y)$ and $(y,x)$ and immediately get the upper bound.
For the lower bound, let us define function $\varphi(x)  = f(x) - \langle \nabla f(x^\prime), x\rangle$ for some fixed $x^\prime\in\mathbb{R}^n$.
It is easy to see, that $\varphi(x)$ is $L$-smooth
$$
\|\nabla\varphi(x) - \nabla\varphi(y)\|_2 = \|\nabla f(x) - \nabla f(x^\prime) - \nabla f(y) + \nabla f(x^\prime)\|_2\leq L\|x-y\|_2.
$$
Since $x^\prime$ is a minimizer of $\varphi(x)$ we have
\begin{equation}
\varphi(x^\prime) \leq \varphi(x - \frac{1}{L}\nabla \varphi(x))\leq \varphi(x) - \frac{1}{2L}\|\nabla \varphi(x)\|_2^2. 
\end{equation}
Now setting $x^\prime = y$ we have the corresponding result.
\end{proof}

\begin{figure}[H]
\centering
\begin{tikzpicture}[
    thick,
    >=stealth',
    dot/.style = {
      draw,
      fill = white,
      circle,
      inner sep = 0pt,
      minimum size = 4pt
    }
  ]
  \draw [blue, dashed, xshift=4cm] plot [smooth] coordinates {( 0.0 , 0.532 )
( 0.25 , 0.38950000000000007 )
( 0.5 , 0.2660000000000001 )
( 0.75 , 0.1615000000000001 )
( 1.0 , 0.07600000000000015 )
( 1.25 , 0.009500000000000064 )
( 1.5 , -0.03800000000000002 )
( 1.75 , -0.06649999999999998 )
( 2.0 , -0.07599999999999998 )
( 2.25 , -0.0665 )
( 2.5 , -0.03799999999999999 )
( 2.75 , 0.009500000000000005 )
( 3.0 , 0.076 )
( 3.25 , 0.1615 )
( 3.5 , 0.26599999999999996 )
( 3.75 , 0.38949999999999996 )
( 4.0 , 0.5319999999999999 )
( 4.25 , 0.6935 )
( 4.5 , 0.874 )
( 4.75 , 1.0735 )
( 5.0 , 1.2920000000000003 )
( 5.25 , 1.5295 )
( 5.5 , 1.786 )
( 5.75 , 2.0615 )
( 6.0 , 2.3560000000000003 )
( 6.25 , 2.6694999999999998 )
( 6.5 , 3.0020000000000002 )
( 6.75 , 3.3534999999999995 )
( 7.0 , 3.7239999999999998 )
( 7.25 , 4.1135 )
};
\node[] at (14, 3.6) {\color{blue} $f(y) + \langle \nabla f(y), x - y\rangle +\frac{\mu}{2}\|x-y\|_2^2$};
\draw [black, xshift=4cm] plot [smooth] coordinates {( 0.0 , 1.8999999999999997 )
( 0.25 , 1.539 )
( 0.5 , 1.216 )
( 0.75 , 0.9309999999999998 )
( 1.0 , 0.6839999999999999 )
( 1.25 , 0.4749999999999999 )
( 1.5 , 0.304 )
( 1.75 , 0.17099999999999999 )
( 2.0 , 0.076 )
( 2.25 , 0.019 )
( 2.5 , 0.0 )
( 2.75 , 0.019 )
( 3.0 , 0.076 )
( 3.25 , 0.17099999999999999 )
( 3.5 , 0.304 )
( 3.75 , 0.4749999999999999 )
( 4.0 , 0.6839999999999999 )
( 4.25 , 0.9309999999999998 )
( 4.5 , 1.216 )
( 4.75 , 1.539 )
( 5.0 , 1.8999999999999997 )
( 5.25 , 2.299 )
( 5.5 , 2.7359999999999998 )
( 5.75 , 3.211 )
( 6.0 , 3.7239999999999993 )
( 6.25 , 4.275 )
( 6.5 , 4.864 )
( 6.75 , 5.491 )
( 7.0 , 6.156 )
};
\node[] at (11.3, 5.7) {$f(x)$};
\draw [yellow!50!black, dashed, xshift=4cm] plot [smooth] coordinates {( 0.0 , 3.964 )
( 0.25 , 3.2733333333333334 )
( 0.5 , 2.6493333333333333 )
( 0.75 , 2.092 )
( 1.0 , 1.6013333333333335 )
( 1.25 , 1.1773333333333333 )
( 1.5 , 0.8200000000000001 )
( 1.75 , 0.5293333333333334 )
( 2.0 , 0.3053333333333334 )
( 2.25 , 0.148 )
( 2.5 , 0.057333333333333326 )
( 2.75 , 0.03333333333333333 )
( 3.0 , 0.076 )
( 3.25 , 0.18533333333333332 )
( 3.5 , 0.36133333333333334 )
( 3.75 , 0.6039999999999999 )
( 4.0 , 0.9133333333333333 )
( 4.25 , 1.2893333333333332 )
( 4.5 , 1.732 )
( 4.75 , 2.241333333333333 )
( 5.0 , 2.817333333333333 )
( 5.25 , 3.46 )
( 5.5 , 4.169333333333333 )
( 5.75 , 4.945333333333334 )
( 6.0 , 5.787999999999999 )
( 6.25 , 6.697333333333333 )
( 6.5 , 7.673333333333333 )
};
\node[] at (13.5, 7.3) {\color{yellow!50!black} $f(y) + \langle \nabla f(y), x - y\rangle +\frac{L}{2}\|x-y\|_2^2$};
\draw [red, xshift=4cm] plot [smooth] coordinates {( 0.0 , -0.836 )
( 0.25 , -0.7599999999999999 )
( 0.5 , -0.6839999999999998 )
( 0.75 , -0.608 )
( 1.0 , -0.5319999999999999 )
( 1.25 , -0.456 )
( 1.5 , -0.38 )
( 1.75 , -0.304 )
( 2.0 , -0.228 )
( 2.25 , -0.15200000000000002 )
( 2.5 , -0.076 )
( 2.75 , 0.0 )
( 3.0 , 0.076 )
( 3.25 , 0.152 )
( 3.5 , 0.228 )
( 3.75 , 0.304 )
( 4.0 , 0.37999999999999995 )
( 4.25 , 0.4559999999999999 )
( 4.5 , 0.5319999999999999 )
( 4.75 , 0.608 )
( 5.0 , 0.6839999999999999 )
( 5.25 , 0.76 )
( 5.5 , 0.836 )
( 5.75 , 0.912 )
( 6.0 , 0.988 )
( 6.25 , 1.0639999999999998 )
( 6.5 , 1.14 )
( 6.75 , 1.216 )
( 7.0 , 1.292 )
( 7.25 , 1.3679999999999999 )};
\node[] at (13, 1) {\color{red}$f(y) + \langle \nabla f(y), x - y\rangle $};
\node[] at (7.8 , -0.2 ) {$\left(y, f(y)\right)$};
\node[] at (7.03, 0.07) {$\bullet$};
\end{tikzpicture}
\caption{Graphical illustration of lower and upper bound for $L$-smooth and $\mu$-strongly convex function $f:\RR\rightarrow \RR$}
\label{fig:functional_approximations}
\end{figure}

In Figure \ref{fig:functional_approximations} we present the graphical illustration of lower \eqref{eq:convex_2} and upper \eqref{eq:descent_lemma_1} bounds for the $L$-smooth and $\mu$-strongly convex objective function $f$. {\color{red}As we could see from this figure, the quadratic lower bound provided by \eqref{eq:convex_2} approximates the functional value much better than first-order approximation, that leads to the intuition that $\mu$-strongly convex functions are easier to analyse.} The quality of these approximations could be characterized by {\textbf{condition number}} of the problem
$$
\kappa =  \frac{L}{\mu}.
$$
When this number is close to $1$ ({\textbf{well-conditioned}}) the problem is extremely well approximated and when it is big ({\textbf{ill-conditioned}}) problems have a weak approximation. It impacts on the speed of optimization algorithms: better-conditioned problems are easier to solve.

{\color{blue}
Finally, let us present the following auxiliary lemma (Lemma $3.11$ \cite{bubeck2015convex}) that is widely used in the convergence analysis of first-order methods for $L$-smooth and $\mu$-strongly convex objectives.

\begin{lemma}\label{lm:bubeck}
Let us assume that $f$ is $L$-smooth and $\mu$-strongly convex, then for any $x,y\in\mathbb{R}^n$ holds
\begin{equation}\label{eq:bubeck}
\langle \nabla f(x)- \nabla f(y), x - y\rangle\geq \frac{\mu L}{\mu + L}\|x-y\|_2^2 + \frac{1}{\mu + L}\|\nabla f(x) - \nabla f(y)\|_2^2.
\end{equation}
\end{lemma}
\begin{proof}[Proof of Lemma \ref{lm:bubeck}]
Consider the case $\mu = L$ then from the $\mu$-strong convexity of $f$ we have
$$
f(y) + f(x) \geq f(x) + \langle \nabla f(x), y-x\rangle + \frac{\mu}{2}\|y-x\|_2^2 + f(y) - \langle \nabla f(y), y-x\rangle + \frac{\mu}{2}\|y-x\|_2^2
$$
that implies 
\begin{equation}
    \langle \nabla f(y) - \nabla f(x), y-x\rangle\geq \mu\|y-x\|_2^2.
\end{equation}
Now, using $L$-smoothness we have
\begin{equation}
    \langle \nabla f(y) - \nabla f(x), y-x\rangle\geq \frac{\mu}{2}\|y-x\|_2^2 + \frac{\mu}{2L^2}\|\nabla f(x) - \nabla f(y)\|_2^2,
\end{equation}
that proves the statement of the lemma.

Consider the case $L>\mu$. Denote by $\varphi(x)=f(x) -\frac{\mu}{2}\|x\|_2^2$, then $\nabla \varphi(x) = \nabla f(x) - \mu x$. This function is $L-\mu$-smooth, using Cauchy–Schwarz inequality and \eqref{eq:descent_lemma_2} we have
\begin{align}
\langle \nabla\varphi(x) -\nabla\varphi(y), x-y\rangle\geq \frac{1}{L-\mu}\|\nabla \varphi(x) - \nabla \varphi(y)\|_2^2.
\end{align}
Now, substituting the expression for $\nabla \varphi$ 
\begin{align}
\langle \nabla f(x)- \nabla f(y), x - y\rangle - \mu\|x-y\|_2^2 &\geq \frac{1}{L-\mu}\left(\|\nabla f(x) - \nabla f(y)\|_2^2 + \mu^2\|x-y\|_2^2\right)\nonumber\\
&- \frac{2\mu}{L-\mu} \langle \nabla f(x)- \nabla f(y), x - y\rangle,
\end{align}
that proves the result.
\end{proof}
}