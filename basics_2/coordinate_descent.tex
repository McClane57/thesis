Coordinate descent methods \dg{is a class of iterative methods in which only one coordinate (block) is updated on every iteration.} Let us see the interest of such methods for the problem with a least-squares objective function
$$
f(x) = \|Ax-b\|_2^2,
$$
where $A$ is a matrix of examples, and $b$ is a vector \dg{of observations}. For such objective function, the computational cost of one coordinate of the gradient is $n$ times smaller than the full gradient computation thanks to the function's separable structure.

The idea of the underlying coordinate descent methods is the following: on every iteration, we compute one (or some) coordinate of gradient and make a move in that direction with some stepsize. There are different ways to select stepsize and the coordinate-based, for example, on the smoothness constants of $i$-th coordinates of gradient \cite{richtarik2012efficient} or on the maximal absolute value of the gradient's coordinates correspondingly \cite{nesterov2012efficiency}. 

Let us consider smooth optimization problem \eqref{eq:pb_simple}, where function $f$ has coordinate-wise Lipschitz continuous gradient
\begin{equation}\label{eq:lipsch_coord}
|\nabla_{[i]} f(x+\dg{h}e_i) - \nabla_{[i]} f(x)|\leq M|h|,
\end{equation}
where subscript $[i]$ means the $i$-th coordinate, $e_i$ is a $i$-th standar\dg{d} basis vector, and $h\in\RR$.
\begin{algorithm}
    \caption{Coordinate Descent (\texttt{CD}) for \eqref{eq:pb_simple}}
    \label{algo:cd}
    \begin{algorithmic}
        \STATE Initialize $x^0\in\mathbb{R}^n$
        \FOR{$k\geq 0$}{
            \STATE Select coordinate $i^k=\argmax_{1\leq i^k\leq n}|\nabla_{i^k}f(x^k)|$
            \STATE $x^{k+1}\leftarrow x^k - \frac{1}{M} \nabla_{[i^k]} f(x^k)$
        }
        \ENDFOR
    \end{algorithmic}
\end{algorithm}
For such function, it is easy to see that
$$
f(x^{k}) - f(x^{k+1})\geq \frac{1}{2M}|\nabla_{i^k} f(x^k)|^2\geq \frac{1}{2nM}\|\nabla f(x^k)\|_2^2,
$$
\dg{where the first inequality follows from \eqref{eq:lipsch_coord} and the second follows from the coordinate selection.}
It immediately implies (see the proof of Theorem \ref{th:gd}) the following rate for the non-strongly convex objective
$$
f(x^k) - f^\star\leq \frac{2nM}{k+4}\|x^0-x^\star\|_2^2,
$$
where we use the same notations as before. 

As we could see from this result, the rate is $n$ times worse than the rate of gradient descent, and every iteration requires \dg{a} full gradient computation for coordinate selection. Moreover, it could happen that $M\gg L$ where $L$ is \dg{the} smoothness constant of $f$ that makes algorithm worse \dg{in terms of the amount of gradient evaluations to reach the same accuracy} even if only $1$ coordinate of the gradient is required.

Together with the greedy strategy, randomized and cyclic variations of this method also appear widely in the literature. \dg{Besides, some modifications propose selecting the block of coordinates on every iteration}. Thus in practice, it is more popular to use a cyclic or randomized selection of coordinate to be updated \dg{(see e.g. \cite{wright2015coordinate})}.

Now, let us consider composite optimization problem \eqref{eq:pb_composite}, where $f$ has coordinate-wise Lipschitz continuous gradient and moreover regularizer $r$ is coordinate-wise separable
$$
r(x) = \sum_{i=1}^nr_i(x_{[i]}).
$$
The best know example of coordinate-separable regularizers is $\ell_1$ regularizer with $r_i(x_{[i]}) = \lambda |x_{[i]}|$. This type of regularizers together with block-separable (for example $\ell_{1,2}$ regularization \cite{bach2012optimization}) is usual assumption for (block) coordinate descent methods, because it implies (block) separability of proximal operator
$$
\rprox(x)_{[i]} = \prox_{\gamma r_i}(x_{[i]}).
$$

The key idea of coordinate descent method for \eqref{eq:pb_composite} is to make a majorization-minimization algorithm where in \eqref{eq:step_pgd} \dg{the upper bound is based on the coordinate smoothness}% instead of full upper bound $\hat{f}(z, x^k)$ the coordinate one used
$$
\hat{f}_i(z, x^k) = f(x^k) + \nabla_{[i]} f(x^k) (z-x^k)_{[i]}+\frac{1}{2\gamma}(x^k-z)_{[i]}^2.
$$
The simplest version of the coordinate descent for \eqref{eq:pb_composite} is presented in Algorithm \ref{algo:cd_composite} (see Theorem $1$ of \cite{richtarik2012efficient} for the proof).

\begin{algorithm}
    \caption{Coordinate Descent (\texttt{CD}) for \eqref{eq:pb_composite}}
    \label{algo:cd_composite}
    \begin{algorithmic}
        \STATE Initialize $x^0\in\mathbb{R}^n$
        \FOR{$k\geq 0$}{
            \STATE Select coordinate $i^k\in\{1,\ldots,n\}$
            \STATE $x^{k+1}_{[i^k]}\leftarrow \prox_{\gamma r_{i^k}}\left(x^k_{[i^k]} - \gamma \nabla_{[i^k]} f(x^k)\right)$\hfill with $\gamma = \frac{1}{M}$
            \STATE $x^{k+1}_{[j]}\leftarrow x^k_{[j]}$\hfill for all $j\neq i^k$
        }
        \ENDFOR
    \end{algorithmic}
\end{algorithm}

Since a lot of common regularizers are not separable (for example $1$-d Total Variation \cite{bach2012optimization}) coordinate methods Algorithm \ref{algo:cd_composite} could not be widely used to solve arbitrary regularized ERM problem \eqref{eq:erm_reg}. This calls for some modern modifications that allow to use non-separable proximal term \cite{hanzely2018sega}. 

{{Part of our research considers the extension of Algorithm \ref{algo:cd_composite} with a surprising application to the non-separable regularizers.}}