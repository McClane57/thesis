\subsection{Moreau-Yosida regularization}\label{sec:basics_moreau-yosida}
Let us define a Moreau envelope, that is also called Moreau-Yosida regularization \cite{moreau1962fonctions, yosida2012functional}.

\begin{definition}
Given $\lambda>0$, the Moreau envelope $M_{\lambda r}$ of the function $r$ with parameter $\lambda$ is defined as 
\begin{equation}\label{eq:M-Y}
M_{\lambda r}(y) = \inf_x\left(r(x) + \frac{1}{2\lambda}\|x-y\|_2^2\right).
\end{equation}
\end{definition}

Moreau-Yosida regularization of function $r$ is continuously differentiable, even if $r$ is not \cite[Fact $17.17$]{yamada2011minimizing} and its gradient is given by
\begin{equation}\label{eq:M-Y_grad}
\nabla M_{\lambda r} = \frac{1}{\lambda}(x-\prox_{\lambda r}(x)).
\end{equation}
Moreover, the sets of minimums of $r$ and $M_f$ are the same. Let us rewrite proximal operator as
$$
\prox_{\lambda r}(x) = x - \lambda\nabla M_{\lambda r}(x),
$$
which shows that proximal operator could be viewed as a gradient step and Algorithm \ref{algo:pm} is a gradient descent algorithm with stepsize $\lambda$ for minimizing $M_{\lambda r}$ \cite{rockafellar1976monotone}. Taking into account that Moreau envelope has the same minimizers as $r$ we have the convergence of proximal point method. In Figure \ref{fig:M-Y} we present a geometrical interpretation of Moreau envelope for the non-smooth objective function $r$.

\begin{figure}[H]
    \centering
    \begin{tikzpicture}[
    thick,
    >=stealth',
    dot/.style = {
      draw,
      fill = white,
      circle,
      inner sep = 0pt,
      minimum size = 4pt
    }
  ]

\draw [black, xshift=4cm] plot [smooth] coordinates {(0,5) (5,0)};
\draw [black, xshift=4cm] plot [smooth] coordinates {(5,0) (11, 3)};
\node[ xshift=4cm] at (2, 3) {\color{black}$\bullet$};
\node[ xshift=4cm] at (1, 3.5) {\color{blue}$\bullet$};
\draw [yellow!50!black, dashed, xshift = 4cm] plot [smooth] coordinates{
( -1.0 , 1.5 )
( -0.8 , 1.88 )
( -0.6000000000000001 , 2.2199999999999998 )
( -0.3999999999999999 , 2.52 )
( -0.19999999999999996 , 2.7800000000000002 )
( 0.0 , 3.0 )
( 0.19999999999999996 , 3.18 )
( 0.4 , 3.32 )
( 0.6 , 3.42 )
( 0.8 , 3.48 )
( 1.0 , 3.5 )
( 1.2 , 3.48 )
( 1.4 , 3.42 )
( 1.6 , 3.32 )
( 1.8 , 3.18 )
( 2.0 , 3.0 )
( 2.2 , 2.7800000000000002 )
( 2.4 , 2.52 )
( 2.6 , 2.2199999999999998 )
( 2.8 , 1.88 )
};

\node[ xshift=4cm] at (5, 0) {\color{black}$\bullet$};
\draw [yellow!50!black, dashed, xshift = 4cm] plot [smooth] coordinates{
( 3.0 , -2.0 )
( 3.2 , -1.62 )
( 3.4 , -1.28 )
( 3.6 , -0.98 )
( 3.8 , -0.72 )
( 4.0 , -0.5 )
( 4.2 , -0.32 )
( 4.4 , -0.18 )
( 4.6 , -0.08 )
( 4.8 , -0.02 )
( 5.0 , 0.0 )
( 5.2 , -0.02 )
( 5.4 , -0.08 )
( 5.6 , -0.18 )
( 5.8 , -0.32 )
( 6.0 , -0.5 )
( 6.2 , -0.72 )
( 6.4 , -0.98 )
( 6.6 , -1.28 )
( 6.8 , -1.62 )
};

\node[ xshift=4cm] at (9, 2) {\color{black}$\bullet$};
\node[ xshift=4cm] at (9.5, 2.125) {\color{blue}$\bullet$};
\draw [yellow!50!black, dashed, xshift = 4cm] plot [smooth] coordinates{
( 7.5 , 0.125 )
( 7.7 , 0.5049999999999999 )
( 7.9 , 0.845 )
( 8.1 , 1.145 )
( 8.3 , 1.405 )
( 8.5 , 1.625 )
( 8.7 , 1.805 )
( 8.9 , 1.945 )
( 9.1 , 2.045 )
( 9.3 , 2.105 )
( 9.5 , 2.125 )
( 9.7 , 2.105 )
( 9.9 , 2.045 )
( 10.1 , 1.945 )
( 10.3 , 1.805 )
( 10.5 , 1.625 )
( 10.7 , 1.405 )
( 10.9 , 1.145 )
( 11.1 , 0.845 )
( 11.3 , 0.5049999999999999 )
};


\draw [red, xshift = 4cm] plot [smooth] coordinates{
(-1, 5.5) ( 4.0 , 0.5 )};

\draw [red, xshift = 4cm] plot [smooth] coordinates{
( 4.0 , 0.5 )
( 4.1 , 0.405 )
( 4.2 , 0.32 )
( 4.3 , 0.245 )
( 4.4 , 0.18 )
( 4.5 , 0.125 )
( 4.6 , 0.08 )
( 4.7 , 0.045 )
( 4.8 , 0.02 )
( 4.9 , 0.005 )
( 5.0 , 0.0 )
( 5.1 , 0.005 )
( 5.2 , 0.02 )
( 5.3 , 0.045 )
( 5.4 , 0.08 )
( 5.5 , 0.125)};

\draw [red, xshift = 4cm] plot [smooth] coordinates{
( 5.5 , 0.125)
( 6.5 , 0.625)
( 11.5 , 3.125)
};

\node[xshift = 4cm] at (0.7, 5) {$r(x)$};
\node[xshift = 4cm] at (4.3, 3.2) {$\left(\prox_{r}(y), r\left(\prox_{r}(y)\right)\right)$};

\node[xshift = 4cm] at (-0.4, 3.6) {\color{blue}$\left(y, M_{\lambda r}\left(y\right)\right))$};
\node[xshift = 4cm] at (-0.3, 5.6) {\color{red}$M_{\lambda r}(x)$};

\node[xshift = 4cm] at (-1, 1) {\color{yellow!50!black}$-\frac{\lambda}{2}\|x-y\|_2^2 + a_{\max}$};
\end{tikzpicture}
    \caption{Geometrical interpretation of Moreau envelope for $r = \max\left\{-x, 0.5x\right\}$ from $\RR$ to $\RR$.}
    \label{fig:M-Y}
\end{figure}

Let us talk a little bit about the properties of Moreau-Yosida regularization. Since we mentioned that the proximal point method is a gradient descent method on Moreau envelope of the function, let us present smoothness and strong convexity properties of this envelope. From \eqref{eq:M-Y_grad} we have that 
\begin{equation}\label{eq:M-Y_smoothness}
L_{M_{\lambda r}} = \lambda^{-1}
\end{equation}
independent on the smoothness parameter of $r$. However, Moreau-Yosida regularization is strongly convex if the objective function $r$ is strongly convex, moreover, their strongly convexity constants relates as
\begin{equation}\label{eq:M-Y_strongly}
\mu_{M_{\lambda r}} = \frac{\mu\lambda^{-1}}{\mu + \lambda^{-1}},
\end{equation}
where $\mu$ is the strongly convexity constant of $r$ (see Theorem 2.2 of \cite{lemarechal1997practical} for the proof). Thus, the condition number of this Moreau-Yosida regularization of $\mu$-strongly convex function $r$ is 
\begin{equation}\label{eq:M-Y_condition}
\kappa_{M_{\lambda r}} = \frac{L_{M_{\lambda r}}}{\mu_{M_{\lambda r}}} = \frac{\mu + \lambda^{-1}}{\mu}.
\end{equation}

As a result, this problem becomes exceptionally well-conditioned when $\lambda$ is selected big.






