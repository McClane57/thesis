\section{Useful tools for stochastic processes analysis.}\label{sec:basics-stoch-basics}
{\color{blue}
In this section, we recall some definitions and results from the probability theory that we will use in our further proofs.

Let us start from the law of total expectation \cite[Chapter V, Section $4$]{kolmogorov1933grundbegriffe}.

\begin{lemma}[Tower property]\label{lm:tower_exp}
For any random variables $X$ and $Y$, we have $\EE[X] = \EE\left[\EE[X|Y]\right]$.
\end{lemma}
\begin{proof}[Proof of Lemma \ref{lm:tower_exp}]
    Let us provide the proof for discrete random variables only.
\begin{align}
\EE[X] &= \sum_x x P(X=x) = \sum_x x\sum_yP(X=x)\&P(Y=y) = \sum_x\sum_y x P(X=x)\&P(Y=y)\nonumber\\
&= \sum_x\sum_y P(Y=y)x\frac{P(X=x\&Y=y)}{P(Y = y)}  = \sum_y P(Y=y) \sum_x xP(X=x\&Y=y)\nonumber\\
& = \sum_y P(Y=y) \EE[X|Y=y] = \EE\left[\EE[X|Y]\right],
\end{align}
which finishes the proof.
\end{proof}

Now, let us present the notion of almost surely convergence \cite{stout1974almost}.
\begin{definition}[Almost sure convergence]
    We say that the sequence $\left\{X_t\right\}$ of random variables $X_t:\Omega\rightarrow \mathcal{S}$ converges almost surely (a.s.) to $X\in\mathcal{S}$ if 
    $$
    \mathbf{P}\left[\omega\in\Omega:\lim_{n\rightarrow\infty}X_n(s) = X\right] = 1.
    $$
\end{definition}

The following theorem allows proving almost sure convergence for almost supermartingales \cite[Theorem $1$]{robbins1971convergence}.
\begin{theorem}\label{th:r-s_theorem}
Consider probability space $(\Omega, \mathcal{F}, \PP)$ with the sequence of $\sigma$-algebras $\FF_1\subset\FF_2\subset\ldots$. For each $n = 1,2,\ldots$ let $z_n, \beta_n, \xi_n,$ and $\zeta_n$ be non-negative $\FF_n$-measurable random variables such that
\begin{equation}\label{eq:a-super-martingale}
\EE[z_{n+1}|\FF_n]\leq z_n(1+\beta_n) + \xi_n - \zeta_n.
\end{equation}
Assume that $\sum_{i=1}^\infty \beta_i<\infty$ and $\sum_{i=1}^\infty \xi_i<\infty$ then $\lim_{n\rightarrow\infty}z_n$ exist and is finite and $\sum_{i=1}^\infty \zeta_i<\infty$ a.s.
\end{theorem}
}
