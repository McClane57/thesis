\section{Numerical illustrations}\label{sec:mor-numerical}
We report preliminary numerical experiments illustrating the behavior of our randomized proximal algorithms on standard problems involving $\ell_1$/TV regularizations. We provide an empirical comparison of our algorithms
%\algo, \adaalgo, 
with the standard proximal (full and coordinate) gradient algorithms %\cite{teboulle2018simplified,wright2015coordinate} 
%(\pgd) 
and a recent proximal sketching algorithm.
%\sega.
%\cite{hanzely2018sega}.


% ==========================================================================
\subsection{Experimental setup}
% ==========================================================================


% ==========================================================================
%\paragraph{Problems.}
%\textbf{Problems.}
We consider the standard regularized logistic regression with three different regularization terms,
%"epsilon" from LibSVM data set with 100,000 observations and 2,000 features. 
% estimated over a training set $\mathcal S=\{(z_j,y_j), j\in\{1,\ldots,n\}\}$
%\begin{align*}%\label{eq:reg-loss}
which can be written for given $(a_i,b_i)\in\RR^{n+1}$ ($i=1,\ldots,m$) and parameters $\lambda_1,\lambda_2>0$
\begin{subequations}
\begin{align}
 &+ ~\lambda_1\!\left\|x\right\|_1\label{eq:logl1}\\
 \min_{x\in \RR^n}~~~\frac{1}{m}\sum\limits_{i=1}^m \log\left(1 + \exp\left(-b_ia_i^\top x\right)\right) + \frac{\lambda_2}{2}\|x\|_2^2 &+ ~  \lambda_1\!\left\|x\right\|_{1,2}\label{eq:logl12}\\
 &+~\lambda_1\!\mathbf{TV}(x)\label{eq:logtv}
\end{align}
\end{subequations}
%\end{align*}
We use two standard data-sets from the LibSVM repository: the 
\emph{a1a} data-set ($m=1,605$ $n=123$) for the $\mathbf{TV}$ regularizer
%with hyperparameters $\lambda_2$ and $\lambda_1$, chosen to reach 80\% sparsity;
the \emph{rcv1\_train} data-set ($m=20,242$ $n=47,236$) for the $\ell_1$ and $\ell_{1,2}$ regularizers. We fix the parameters $\lambda_2=1/m$ and $\lambda_1$ to reach a final sparsity of roughly 90\%. 
%$\lambda_2$ and $\lambda_1$ are chosen to reach a sparsity  80\% (for \eqref{eq:logl1}), 90\% (for \eqref{eq:logl12}) and 96\% (for \eqref{eq:logtv}) respectively.

The subspace collections are taken naturally adapted to the regularizers: by coordinate for \eqref{eq:logl1} and \eqref{eq:logl12}, and by variation for \eqref{eq:logtv}. {The adaptation strategies are the ones described in Section~\ref{sec:ex_ada}.}

% For the coordinate-wise problems \eqref{eq:logl1} and \eqref{eq:logl12}, adaptation is performed at each iteration following Example~\ref{ex:adapt_coord}; for the variation~\eqref{eq:logtv} where the adaptation is both more costly to compute and more challenging theoretically, we adopt the strategy of Example~\ref{ex:adapt_fixed}, leading to seldom but efficient adaptations (see forthcoming Figure\;\ref{fig:a11}).

% We also examine a synthetic least squares problems
%  \begin{align*}
%  &~~\lambda_1\!\left\|x\right\|_1\\
%  \min_{x\in \RR^n}~~\| Ax - b\|^2 &+ ~  \lambda_1\!\left\|x\right\|_{1,2}\\
%  &~~\lambda_1\!\mathbf{TV}(x)
% \end{align*}
% with $m= 10,000$ examples and $n = 100$ features.  $A$ is $m\times n$ matrix generated from the standard normal distribution, $b = Ax_0 +e$ where $x_0$ is a 90\% sparse vector and $e\in\mathbb{R}^m$ is taken from the normal distribution with standard deviation $0.01$.  
% %\end{align*}
% We take $\lambda_1$ to reach the sparsity of 90\%.


% It measures the amount of coordinates that are updated during all the iterations. Especially this measure is interesting for the problems when the computational time of iteration $k$ depends on $\rank(P^k)$ (for example in LASSO problems with coordinate-wise projections).

% % ==========================================================================
% \paragraph{Projections.} 
% More precisely, for $\ell_{1,2}$ and $\ell_1$ in nonadaptive case the set of projections is associated with all elements of $\C_\ell$ with dimension $d$, where $d$ is a parameter of the algorithm. In adaptive case we select the set of projections $\PP(x)$ associated with all linear subspaces $\{C_i\}_i$ of $\C_\ell$ such that $T(x)\subset C_i$ and $\dim(C_i) = \dim(T(x)) + d$, where $d$ is a parameter of the algorithm.

% For $\bP$ associated with some subcollection of $\C^\mathbf{TV}$ there is no closed formula and the only way to compute it just to calculate the weighted sum of all projections matrices. According to this exponentially big number of projections in $\PP$ as in $\ell_1$ case above is permitted. Because of this, we select in nonadaptive case the set of projections associated with all elements of $C_l\in\C^\mathbf{TV}$ with jumps set $\jumps(C_l) = \{l, l+1,\dots, l+d-1\}$, where $d$ is a parameter of the algorithm. 
% To describe the selection on adaptive case, let us define 
% \begin{equation}
%     \overline{\jumps}(x) = \{i\in[2,n]\colon i\notin\jumps(x)\}.
% \end{equation}
% In adaptive case we select the set of projections $\PP(x)$ associated with all linear subspaces $C_l\in\C^\mathbf{TV}$ such that 
% \begin{equation}
%     \jumps(C_l) = \jumps(x) + \{\overline{\jumps}(x)_l, \overline{\jumps}(x)_{l+1},\dots,\overline{\jumps}(x)_{l+d-1}\},
% \end{equation}
% where $d$ is a parameter of the algorithm. Remark \ref{rm:el} guarantees the eligibility of both of these sets.


% \subsubsection{$\mathbf{TV}$ case}
% This case is different from all the others as far as there is no closed formula for expected projection $\bP$ and for its square root of the inverse $\bQ$. It means, that for computation of $\bQ$ we need to take into account all projections from the set $\PP$. To make it feasible we assume that against of selecting $s$ jump positions in random we select only the first one and all the other are consecutive free places just after the first one. According to the Lemma \ref{lm:eligible} this set is eligible so it is fair to use it.


% ==========================================================================
%\paragraph{Algorithms.}
%textbf{Algorithms.} 
We consider five algorithms:\\[0.2cm]
\resizebox{\linewidth}{!}{
\begin{tabular}{|c|c|c|c|}
\hline
    Name & Reference & Description & Randomness  \\
    \hline
   \pgd  &  &  vanilla proximal gradient descent  & None \\
   x\footnotemark \, \rpcd & \cite{nesterov2012efficiency} & standard proximal coordinate descent &  x coordinates selected for each update \\
   x  \sega & \cite{hanzely2018sega} & Algorithm \sega~with coordinate sketches &  $\rank(S^k) = \text{x}$ \\
   x  \algo &  Algorithm \ref{alg:strata_nondis} & (non-adaptive) random subspace descent & Option 2 of Table~\ref{tab:comp} with $s =  \text{x}$ \\
   x  \adaalgo & Algorithm \ref{alg:ada_strata_nondis} & adaptive random subspace descent &  Option 2 of  Table~\ref{tab:comp} with $s =  \text{x}$\\
 \hline
\end{tabular}}
\footnotetext{In the following, x is often given in percentage of the possible subspaces, i.e. x\% of $|\mathcal{C}|$, that is x\% of $n$ for coordinate projections and x\% of $n-1$ for variation projections.}

\vspace*{0.5cm}


For the produced iterates, we measure the sparsity of a point $x$ by $\|\S(x_k)\|_1$, which correspond to the size of the supports for the $\ell_1$ case and the number of jumps for the TV case. We also consider the quantity: 
\[
\text{Number of subspaces explored at time $k$} \displaystyle ~=~ \sum^k_{t=1} \|\S(x^t)\|_1.
\]
We then compare the performance of the algorithms on three criteria:
\begin{itemize}
    \item functional suboptimality vs iterations (standard comparison);
    \item size of the sparsity pattern vs iterations (showing the identification properties);
    \item functional suboptimality vs number of subspaces explored (showing the gain of adaptivity).
\end{itemize}

% ==========================================================================
\subsection{Illustrations for coordinate-structured problems}
% ==========================================================================


% ==========================================================================
\subsubsection{Comparison with standard methods}
% ==========================================================================

{We consider first $\ell_1$-regularized logistic regression \eqref{eq:logl1}; in this setup, the non-adaptive \algo~boils down to the usual randomized proximal gradient descent (see Section\;\ref{sec:coordproj}). We compare the proximal gradient to its adaptive and non-adaptive randomized counterparts.

First, we observe that the iterates of \pgd~and \adaalgo~coincide. This is due to the fact that the sparsity of iterates only decreases ($\S(x_k)\leq \S(x_{k+1})$) along the convergence, and according to Option 2 all the non-zero coordinates are selected at each iteration and thus set to the same value as with \pgd. However, a single iteration of $10\%$-\adaalgo costs less in terms of number of subspaces explored, leading the speed-up of the right-most plot.  Contrary to the adaptive \adaalgo, the structure-blind \algo~identifies much later then \pgd~and shows poor convergence.}


\begin{figure}[H]
\begin{center}
\scalebox{.85}{% This file was created by matplotlib2tikz v0.6.18.
\begin{tikzpicture}

\begin{axis}[
legend cell align={left},
legend columns=1,
legend entries={{\pgd},{10\% \algo},{10\% \adaalgo}},
legend style={at={(0.8,0.99)}, anchor=north},
tick align=outside,
tick pos=left,
xlabel={Iteration},
xmajorgrids,
xmin=0, xmax=30000,
ylabel={Iterate sparsity},
ymajorgrids,
ymin=0, ymax=100
]

\addlegendimage{ black,thick,mark=square*}
\addlegendimage{ blue,mark=*}
\addlegendimage{ red,mark=diamond*,mark size = 3pt}

\addplot [black,thick,mark=square*, mark repeat=100]
table [row sep=\\]{%
0	99.4919129477517 \\
100	98.9012617495131 \\
200	98.2830891692777 \\
300	97.6035227368956 \\
400	96.9133711575917 \\
500	96.2105174019815 \\
600	95.4356846473029 \\
700	94.6756710983148 \\
800	93.8606147853332 \\
900	92.9608772969769 \\
1000	92.1564061309171 \\
1100	91.2164450842578 \\
1200	90.2426115674486 \\
1300	89.1608095520366 \\
1400	88.0514861546278 \\
1500	86.9802692861377 \\
1600	85.8010839190448 \\
1700	84.6028452874926 \\
1800	83.3707341857905 \\
1900	82.1746125836227 \\
2000	80.9043949530019 \\
2100	79.5389110000847 \\
2200	78.0950969599458 \\
2300	76.634346684732 \\
2400	75.120670674909 \\
2500	73.5371326954018 \\
2600	71.8816157168261 \\
2700	70.3044288254721 \\
2800	68.5959861122872 \\
2900	66.8875433991024 \\
3000	65.1727495977644 \\
3100	63.4473706495046 \\
3200	61.5081717334237 \\
3300	59.6070793462613 \\
3400	57.6869336946397 \\
3500	55.6778728088746 \\
3600	53.6984503344906 \\
3700	51.8121771530189 \\
3800	49.8242865610975 \\
3900	47.8427470573292 \\
4000	45.8188669658735 \\
4100	43.8415615208739 \\
4200	41.6716064018969 \\
4300	39.6731306630536 \\
4400	37.7339317469727 \\
4500	35.7629773901262 \\
4600	34.0503006181726 \\
4700	32.1682614954696 \\
4800	30.2798712846134 \\
4900	28.5058006605132 \\
5000	26.7338470657973 \\
5100	25.0656279109154 \\
5200	23.5921754593954 \\
5300	22.1462443898721 \\
5400	20.7384198492675 \\
5500	19.4131594546532 \\
5600	18.2297400287916 \\
5700	17.1161825726141 \\
5800	16.0322635278178 \\
5900	15.0965365399272 \\
6000	14.2772461681768 \\
6100	13.5341688542637 \\
6200	12.9075281564908 \\
6300	12.3168769582522 \\
6400	11.7812685240071 \\
6500	11.3007028537556 \\
6600	10.9619781522568 \\
6700	10.6126683038361 \\
6800	10.364975865865 \\
6900	10.1236345160471 \\
7000	9.9712084003726 \\
7100	9.79126090270133 \\
7200	9.65577102210179 \\
7300	9.54568549411466 \\
7400	9.43348293674316 \\
7500	9.31492929121856 \\
7600	9.22601405707511 \\
7700	9.1349817935473 \\
7800	9.05876873571005 \\
7900	9.00372597171649 \\
8000	8.93598103141672 \\
8100	8.86188500296384 \\
8200	8.80895926835464 \\
8300	8.77720382758913 \\
8400	8.74544838682361 \\
8500	8.70522482852062 \\
8600	8.66923532898637 \\
8700	8.63959691760522 \\
8800	8.60995850622407 \\
8900	8.58032009484292 \\
9000	8.5485646540774 \\
9100	8.53586247777119 \\
9200	8.51469218392751 \\
9300	8.4892878313151 \\
9400	8.46176644931832 \\
9500	8.44483021424337 \\
9600	8.43424506732153 \\
9700	8.41307477347786 \\
9800	8.40248962655602 \\
9900	8.38555339148107 \\
10000	8.38131933271234 \\
10100	8.37920230332797 \\
10200	8.3707341857905 \\
10300	8.36226606825303 \\
10400	8.34532983317808 \\
10500	8.3241595393344 \\
10600	8.30934033364383 \\
10700	8.30722330425946 \\
10800	8.29875518672199 \\
10900	8.29452112795325 \\
11000	8.28181895164705 \\
11100	8.27546786349394 \\
11200	8.27335083410958 \\
11300	8.26911677534084 \\
11400	8.258531628419 \\
11500	8.25429756965027 \\
11600	8.24794648149716 \\
11700	8.23736133457532 \\
11800	8.23101024642222 \\
11900	8.22465915826912 \\
12000	8.22465915826912 \\
12100	8.21407401134728 \\
12200	8.20983995257854 \\
12300	8.20772292319417 \\
12400	8.20560589380981 \\
12500	8.20560589380981 \\
12600	8.20560589380981 \\
12700	8.20348886442544 \\
12800	8.1992548056567 \\
12900	8.19078668811923 \\
13000	8.18866965873486 \\
13100	8.18231857058176 \\
13200	8.17808451181302 \\
13300	8.17596748242866 \\
13400	8.17596748242866 \\
13500	8.17173342365992 \\
13600	8.16961639427555 \\
13700	8.16961639427555 \\
13800	8.16749936489119 \\
13900	8.15903124735371 \\
14000	8.15903124735371 \\
14100	8.15691421796934 \\
14200	8.15479718858498 \\
14300	8.15268015920061 \\
14400	8.14844610043187 \\
14500	8.14844610043187 \\
14600	8.14844610043187 \\
14700	8.14632907104751 \\
14800	8.14632907104751 \\
14900	8.14632907104751 \\
15000	8.14632907104751 \\
15100	8.14632907104751 \\
15200	8.14209501227877 \\
15300	8.1399779828944 \\
15400	8.1399779828944 \\
15500	8.13786095351004 \\
15600	8.13574392412567 \\
15700	8.13574392412567 \\
15800	8.13150986535693 \\
15900	8.13150986535693 \\
16000	8.13150986535693 \\
16100	8.13150986535693 \\
16200	8.12939283597256 \\
16300	8.12939283597256 \\
16400	8.12939283597256 \\
16500	8.12939283597256 \\
16600	8.12727580658819 \\
16700	8.12727580658819 \\
16800	8.12727580658819 \\
16900	8.12727580658819 \\
17000	8.12515877720383 \\
17100	8.12515877720383 \\
17200	8.12515877720383 \\
17300	8.12515877720383 \\
17400	8.12304174781946 \\
17500	8.12304174781946 \\
17600	8.12304174781946 \\
17700	8.12304174781946 \\
17800	8.12304174781946 \\
17900	8.12092471843509 \\
18000	8.12092471843509 \\
18100	8.12092471843509 \\
18200	8.12092471843509 \\
18300	8.12092471843509 \\
18400	8.12092471843509 \\
18500	8.11880768905072 \\
18600	8.11880768905072 \\
18700	8.11880768905072 \\
18800	8.11880768905072 \\
18900	8.11880768905072 \\
19000	8.11880768905072 \\
19100	8.11880768905072 \\
19200	8.11669065966636 \\
19300	8.11669065966636 \\
19400	8.11669065966636 \\
19500	8.11669065966636 \\
19600	8.11669065966636 \\
19700	8.11669065966636 \\
19800	8.11669065966636 \\
19900	8.11669065966636 \\
20000	8.11669065966636 \\
20100	8.11669065966636 \\
20200	8.11669065966636 \\
20300	8.11669065966636 \\
20400	8.11669065966636 \\
20500	8.11669065966636 \\
20600	8.11669065966636 \\
20700	8.11669065966636 \\
20800	8.11457363028199 \\
20900	8.11457363028199 \\
21000	8.11457363028199 \\
21100	8.11457363028199 \\
21200	8.11457363028199 \\
21300	8.11457363028199 \\
21400	8.11457363028199 \\
21500	8.11457363028199 \\
21600	8.11457363028199 \\
21700	8.11457363028199 \\
21800	8.11457363028199 \\
21900	8.11457363028199 \\
22000	8.11457363028199 \\
22100	8.11457363028199 \\
22200	8.11457363028199 \\
22300	8.11457363028199 \\
22400	8.11457363028199 \\
22500	8.11457363028199 \\
22600	8.11457363028199 \\
22700	8.11457363028199 \\
22800	8.11457363028199 \\
22900	8.11457363028199 \\
23000	8.11457363028199 \\
23100	8.11457363028199 \\
23200	8.11457363028199 \\
23300	8.11457363028199 \\
23400	8.11245660089762 \\
23500	8.11245660089762 \\
23600	8.11245660089762 \\
23700	8.11245660089762 \\
23800	8.11245660089762 \\
23900	8.11245660089762 \\
24000	8.11245660089762 \\
24100	8.11245660089762 \\
24200	8.11245660089762 \\
24300	8.11245660089762 \\
24400	8.11245660089762 \\
24500	8.11245660089762 \\
24600	8.11245660089762 \\
24700	8.11245660089762 \\
24800	8.11245660089762 \\
24900	8.11245660089762 \\
25000	8.11245660089762 \\
25100	8.11245660089762 \\
25200	8.11245660089762 \\
25300	8.11245660089762 \\
25400	8.11245660089762 \\
25500	8.11245660089762 \\
25600	8.11245660089762 \\
25700	8.11245660089762 \\
25800	8.11245660089762 \\
25900	8.11245660089762 \\
26000	8.11245660089762 \\
26100	8.11245660089762 \\
26200	8.11245660089762 \\
26300	8.11245660089762 \\
26400	8.11245660089762 \\
26500	8.11245660089762 \\
26600	8.11245660089762 \\
26700	8.11245660089762 \\
26800	8.11245660089762 \\
26900	8.11245660089762 \\
27000	8.11245660089762 \\
27100	8.11245660089762 \\
27200	8.11245660089762 \\
27300	8.11245660089762 \\
27400	8.11245660089762 \\
27500	8.11245660089762 \\
27600	8.11245660089762 \\
27700	8.11245660089762 \\
27800	8.11245660089762 \\
27900	8.11245660089762 \\
28000	8.11245660089762 \\
28100	8.11245660089762 \\
28200	8.11245660089762 \\
28300	8.11245660089762 \\
28400	8.11245660089762 \\
28500	8.11245660089762 \\
28600	8.11245660089762 \\
28700	8.11245660089762 \\
28800	8.11245660089762 \\
28900	8.11245660089762 \\
29000	8.11245660089762 \\
29100	8.11245660089762 \\
29200	8.11245660089762 \\
29300	8.11245660089762 \\
29400	8.11245660089762 \\
29500	8.11245660089762 \\
29600	8.11245660089762 \\
29700	8.11245660089762 \\
29800	8.11245660089762 \\
29900	8.11245660089762 \\
};
\addplot [ blue,mark=*, mark repeat=100]
table [row sep=\\]{%
0	99.8729782369379 \\
100	99.7692437971039 \\
200	99.6655093572699 \\
300	99.5723600643577 \\
400	99.4538064188331 \\
500	99.3394868320772 \\
600	99.2315183334745 \\
700	99.119315776103 \\
800	98.9817088661191 \\
900	98.8652722499788 \\
1000	98.7869421627572 \\
1100	98.653569311542 \\
1200	98.5392497247862 \\
1300	98.3910576678804 \\
1400	98.2767380811246 \\
1500	98.1708866119062 \\
1600	98.0671521720721 \\
1700	97.9718858497756 \\
1800	97.8448640867135 \\
1900	97.6966720298078 \\
2000	97.5717672961301 \\
2100	97.4489795918367 \\
2200	97.2859683292404 \\
2300	97.1610635955627 \\
2400	97.036158861885 \\
2500	96.9112541282073 \\
2600	96.731306630536 \\
2700	96.6064018968583 \\
2800	96.4772631044119 \\
2900	96.3438902531967 \\
3000	96.2041663138284 \\
3100	96.09831484461 \\
3200	95.9522398170887 \\
3300	95.8167499364891 \\
3400	95.6833770852739 \\
3500	95.4991955288339 \\
3600	95.378524853925 \\
3700	95.2112795325599 \\
3800	95.0630874756542 \\
3900	94.929714624439 \\
4000	94.7561182149208 \\
4100	94.584638834787 \\
4200	94.3962232195783 \\
4300	94.2268608688289 \\
4400	94.0892539588449 \\
4500	93.9347108137861 \\
4600	93.7780506393429 \\
4700	93.6340926412059 \\
4800	93.4922516724532 \\
4900	93.3165382335507 \\
5000	93.1619950884918 \\
5100	92.9799305614362 \\
5200	92.7788127699212 \\
5300	92.5776949784063 \\
5400	92.4400880684224 \\
5500	92.2495554238293 \\
5600	92.0420865441612 \\
5700	91.86213904649 \\
5800	91.6525531374375 \\
5900	91.4472012871539 \\
6000	91.2757219070201 \\
6100	91.0767211448895 \\
6200	90.8925395884495 \\
6300	90.7422305021594 \\
6400	90.5580489457194 \\
6500	90.3929206537387 \\
6600	90.1727495977644 \\
6700	89.9991531882463 \\
6800	89.8149716318062 \\
6900	89.6329071047506 \\
7000	89.4444914895419 \\
7100	89.2581929037175 \\
7200	89.0972986705055 \\
7300	88.9152341434499 \\
7400	88.69718011686 \\
7500	88.489711237192 \\
7600	88.2780082987552 \\
7700	88.0747734778559 \\
7800	87.8694216275722 \\
7900	87.655601659751 \\
8000	87.4015581336269 \\
8100	87.2046744008807 \\
8200	86.9506308747565 \\
8300	86.7219917012448 \\
8400	86.5039376746549 \\
8500	86.3176390888306 \\
8600	86.0762977390126 \\
8700	85.8815310356508 \\
8800	85.669828097214 \\
8900	85.3988483360149 \\
9000	85.1934964857312 \\
9100	84.9817935472944 \\
9200	84.7764416970108 \\
9300	84.5520365822678 \\
9400	84.3466847319841 \\
9500	84.1138114997036 \\
9600	83.8682360911169 \\
9700	83.626894741299 \\
9800	83.3707341857905 \\
9900	83.1569142179693 \\
10000	82.9176898975358 \\
10100	82.7081039884834 \\
10200	82.4561774917436 \\
10300	82.2275383182319 \\
10400	81.9607926158015 \\
10500	81.6813447370649 \\
10600	81.4018968583284 \\
10700	81.1774917435854 \\
10800	80.8789906003895 \\
10900	80.5910746041155 \\
11000	80.3476162249132 \\
11100	80.0851045812516 \\
11200	79.8479972902024 \\
11300	79.5833686171564 \\
11400	79.3547294436447 \\
11500	79.1049199762893 \\
11600	78.844525362012 \\
11700	78.596832924041 \\
11800	78.3364383097637 \\
11900	78.0591074604116 \\
12000	77.8347023456686 \\
12100	77.5573714963164 \\
12200	77.3033279701922 \\
12300	76.9921246506902 \\
12400	76.6830383605724 \\
12500	76.3951223642984 \\
12600	76.1495469557117 \\
12700	75.897620458972 \\
12800	75.6181725802354 \\
12900	75.3387247014989 \\
13000	75.0550427639936 \\
13100	74.7692437971039 \\
13200	74.4453383012956 \\
13300	74.1574223050216 \\
13400	73.8631552205945 \\
13500	73.5540689304768 \\
13600	73.2809721398933 \\
13700	72.9761199085443 \\
13800	72.6310441188924 \\
13900	72.3748835633839 \\
14000	72.0446269794225 \\
14100	71.748242865611 \\
14200	71.4751460750275 \\
14300	71.1173681090694 \\
14400	70.8336861715641 \\
14500	70.5500042340588 \\
14600	70.2557371496316 \\
14700	69.9275975950546 \\
14800	69.6312134812431 \\
14900	69.3242442205098 \\
15000	69.0405622830045 \\
15100	68.6997205521213 \\
15200	68.3525277330849 \\
15300	68.0243881785079 \\
15400	67.679312388856 \\
15500	67.3596409518164 \\
15600	67.0272673384707 \\
15700	66.6800745194343 \\
15800	66.3265306122449 \\
15900	65.9772207638242 \\
16000	65.6448471504784 \\
16100	65.2595478025235 \\
16200	64.9483444830214 \\
16300	64.6223219578288 \\
16400	64.2941824032518 \\
16500	63.9533406723685 \\
16600	63.5680413244136 \\
16700	63.2271995935304 \\
16800	62.8821238038784 \\
16900	62.5666864256076 \\
17000	62.2152595478025 \\
17100	61.8850029638411 \\
17200	61.5632144974172 \\
17300	61.2075535608434 \\
17400	60.8751799474977 \\
17500	60.5428063341519 \\
17600	60.1723261918875 \\
17700	59.7849098145482 \\
17800	59.4483021424337 \\
17900	59.0566517063257 \\
18000	58.6925226522144 \\
18100	58.3453298331781 \\
18200	57.9833178084512 \\
18300	57.6445931069523 \\
18400	57.3185705817597 \\
18500	56.9607926158015 \\
18600	56.6199508849183 \\
18700	56.2685240071132 \\
18800	55.8938098060801 \\
18900	55.519095605047 \\
19000	55.1486154627826 \\
19100	54.7590820560589 \\
19200	54.3377932085697 \\
19300	53.9694300956897 \\
19400	53.5693115420442 \\
19500	53.2072995173173 \\
19600	52.8283512575155 \\
19700	52.4726903209417 \\
19800	52.1022101786773 \\
19900	51.7973579473283 \\
20000	51.3612498941485 \\
20100	50.9823016343467 \\
20200	50.6605131679228 \\
20300	50.2900330256584 \\
20400	49.9026166483191 \\
20500	49.5130832415954 \\
20600	49.1574223050216 \\
20700	48.791176221526 \\
20800	48.4079939029554 \\
20900	48.0459818782285 \\
21000	47.6818528241172 \\
21100	47.324074858159 \\
21200	46.9387755102041 \\
21300	46.5047844864087 \\
21400	46.0940807858413 \\
21500	45.7193665848082 \\
21600	45.3488864425438 \\
21700	45.0038106528919 \\
21800	44.6185113049369 \\
21900	44.2501481920569 \\
22000	43.8563807265645 \\
22100	43.4181556440003 \\
22200	42.9629943263612 \\
22300	42.5565246845626 \\
22400	42.2495554238293 \\
22500	41.9044796341773 \\
22600	41.5636379032941 \\
22700	41.2016258785672 \\
22800	40.8290287069185 \\
22900	40.4373782708104 \\
23000	40.0499618934711 \\
23100	39.6265560165975 \\
23200	39.2031501397239 \\
23300	38.8305529680752 \\
23400	38.4579557964265 \\
23500	38.0154966550936 \\
23600	37.6513676009823 \\
23700	37.2576001354899 \\
23800	36.8892370226099 \\
23900	36.5293420272673 \\
24000	36.1736810906935 \\
24100	35.8116690659666 \\
24200	35.3882631890931 \\
24300	34.9500381065289 \\
24400	34.588026081802 \\
24500	34.2387162333813 \\
24600	33.8872893555763 \\
24700	33.5506816834618 \\
24800	33.1717334236599 \\
24900	32.8203065458549 \\
25000	32.4392412566686 \\
25100	32.0645270556355 \\
25200	31.7490896773647 \\
25300	31.3849606232534 \\
25400	31.0229485985265 \\
25500	30.663053603184 \\
25600	30.2841053433822 \\
25700	29.9178592598865 \\
25800	29.5791345583877 \\
25900	29.2806334151918 \\
26000	28.9292065373867 \\
26100	28.6222372766534 \\
26200	28.2623422813109 \\
26300	27.9066813447371 \\
26400	27.555254466932 \\
26500	27.2080616478957 \\
26600	26.9349648573122 \\
26700	26.6195274790414 \\
26800	26.2638665424676 \\
26900	25.9547802523499 \\
27000	25.632991785926 \\
27100	25.3090862901177 \\
27200	25.0127021763062 \\
27300	24.7057329155729 \\
27400	24.4241680074519 \\
27500	24.1214328054873 \\
27600	23.8165805741384 \\
27700	23.5731221949361 \\
27800	23.2788551105089 \\
27900	23.0226945550004 \\
28000	22.749597764417 \\
28100	22.4976712676772 \\
28200	22.2139893301719 \\
28300	21.9620628334321 \\
28400	21.6720298077737 \\
28500	21.4328054873402 \\
28600	21.195698196291 \\
28700	20.8929629943264 \\
28800	20.6537386738928 \\
28900	20.3996951477687 \\
29000	20.1202472690321 \\
29100	19.8725548310611 \\
29200	19.6163942755525 \\
29300	19.377169955119 \\
29400	19.1548818697604 \\
29500	18.9452959607079 \\
29600	18.7208908459649 \\
29700	18.479549496147 \\
29800	18.2593784401727 \\
29900	18.0349733254298 \\
};


\addplot [ red, mark=diamond*,mark repeat=100,mark size=3pt]
table [row sep=\\]{%
0	99.4770937420611 \\
100	98.9097298670506 \\
200	98.2767380811246 \\
300	97.6585655008892 \\
400	96.9091370988229 \\
500	96.2105174019815 \\
600	95.4462697942247 \\
700	94.6481497163181 \\
800	93.8140401388771 \\
900	92.9397070031332 \\
1000	92.0442035735456 \\
1100	91.0873062918113 \\
1200	90.0647810991617 \\
1300	89.0041493775934 \\
1400	87.9519857735625 \\
1500	86.9040562283004 \\
1600	85.7650944195105 \\
1700	84.6451858751799 \\
1800	83.4617664493183 \\
1900	82.2296553476162 \\
2000	80.9488525700737 \\
2100	79.6108899991532 \\
2200	78.0950969599458 \\
2300	76.5644847150478 \\
2400	75.0825641459903 \\
2500	73.5815903124735 \\
2600	71.9917012448133 \\
2700	70.3594715894657 \\
2800	68.6531459056652 \\
2900	66.9552883394022 \\
3000	65.0711321873148 \\
3100	63.2335506816835 \\
3200	61.3176390888306 \\
3300	59.444068083665 \\
3400	57.504869167584 \\
3500	55.6461173681091 \\
3600	53.7492590397155 \\
3700	51.7020916250318 \\
3800	49.7480735032602 \\
3900	47.7707680582606 \\
4000	45.6918452028114 \\
4100	43.663731052587 \\
4200	41.5191802862224 \\
4300	39.554577017529 \\
4400	37.4756541620798 \\
4500	35.5301041578457 \\
4600	33.6459480057583 \\
4700	31.7427385892116 \\
4800	29.9686679651114 \\
4900	28.3406723685325 \\
5000	26.7190278601067 \\
5100	25.0910322635278 \\
5200	23.5138453721738 \\
5300	22.0552121263443 \\
5400	20.7108984672707 \\
5500	19.4110424252689 \\
5600	18.2932509103226 \\
5700	17.1648742484546 \\
5800	16.1402320264205 \\
5900	15.2743670082141 \\
6000	14.4381404013888 \\
6100	13.7014141756288 \\
6200	12.9858582437124 \\
6300	12.4544838682361 \\
6400	11.9442797866034 \\
6500	11.5187568803455 \\
6600	11.1144042679312 \\
6700	10.8159031247354 \\
6800	10.5279871284613 \\
6900	10.2972309255652 \\
7000	10.0939961046659 \\
7100	9.9246337539165 \\
7200	9.77644169701076 \\
7300	9.65788805148616 \\
7400	9.53721737657719 \\
7500	9.45253620120247 \\
7600	9.37208908459649 \\
7700	9.28740790922178 \\
7800	9.21966296892201 \\
7900	9.1413328817004 \\
8000	9.05876873571005 \\
8100	9.01007705986959 \\
8200	8.95080023710729 \\
8300	8.88517232619189 \\
8400	8.83648065035143 \\
8500	8.79837412143281 \\
8600	8.75815056312982 \\
8700	8.72427809297993 \\
8800	8.68405453467694 \\
8900	8.65441612329579 \\
9000	8.61842662376154 \\
9100	8.59513930053349 \\
9200	8.57608603607418 \\
9300	8.5549157422305 \\
9400	8.53162841900246 \\
9500	8.50834109577441 \\
9600	8.4892878313151 \\
9700	8.46811753747142 \\
9800	8.45329833178084 \\
9900	8.4363620967059 \\
10000	8.42577694978406 \\
10100	8.41519180286222 \\
10200	8.40884071470912 \\
10300	8.38555339148107 \\
10400	8.36861715640613 \\
10500	8.36014903886866 \\
10600	8.34956389194682 \\
10700	8.33051062748751 \\
10800	8.32204250995004 \\
10900	8.31569142179693 \\
11000	8.3051062748751 \\
11100	8.29452112795325 \\
11200	8.28393598103142 \\
11300	8.27758489287831 \\
11400	8.27546786349394 \\
11500	8.27123380472521 \\
11600	8.26911677534084 \\
11700	8.258531628419 \\
11800	8.25429756965027 \\
11900	8.24794648149716 \\
12000	8.24159539334406 \\
12100	8.23312727580659 \\
12200	8.22465915826912 \\
12300	8.22042509950038 \\
12400	8.20983995257854 \\
12500	8.20772292319417 \\
12600	8.20772292319417 \\
12700	8.20772292319417 \\
12800	8.20772292319417 \\
12900	8.20560589380981 \\
13000	8.20348886442544 \\
13100	8.1929037175036 \\
13200	8.18866965873486 \\
13300	8.18655262935049 \\
13400	8.18443559996613 \\
13500	8.18020154119739 \\
13600	8.17596748242866 \\
13700	8.17596748242866 \\
13800	8.17385045304429 \\
13900	8.16961639427555 \\
14000	8.16749936489119 \\
14100	8.16326530612245 \\
14200	8.15479718858498 \\
14300	8.15479718858498 \\
14400	8.15479718858498 \\
14500	8.15056312981624 \\
14600	8.14844610043187 \\
14700	8.14844610043187 \\
14800	8.14632907104751 \\
14900	8.14632907104751 \\
15000	8.14421204166314 \\
15100	8.14209501227877 \\
15200	8.1399779828944 \\
15300	8.13574392412567 \\
15400	8.1336268947413 \\
15500	8.1336268947413 \\
15600	8.13150986535693 \\
15700	8.12939283597256 \\
15800	8.12939283597256 \\
15900	8.12515877720383 \\
16000	8.12515877720383 \\
16100	8.12304174781946 \\
16200	8.12092471843509 \\
16300	8.12092471843509 \\
16400	8.12092471843509 \\
16500	8.11880768905072 \\
16600	8.11880768905072 \\
16700	8.11880768905072 \\
16800	8.11880768905072 \\
16900	8.11669065966636 \\
17000	8.11457363028199 \\
17100	8.11245660089762 \\
17200	8.11245660089762 \\
17300	8.11245660089762 \\
17400	8.11245660089762 \\
17500	8.11245660089762 \\
17600	8.11245660089762 \\
17700	8.11245660089762 \\
17800	8.11245660089762 \\
17900	8.11245660089762 \\
18000	8.11245660089762 \\
18100	8.11245660089762 \\
18200	8.11245660089762 \\
18300	8.11245660089762 \\
18400	8.11245660089762 \\
18500	8.11245660089762 \\
18600	8.11245660089762 \\
18700	8.11245660089762 \\
18800	8.11245660089762 \\
18900	8.11245660089762 \\
19000	8.11245660089762 \\
19100	8.11245660089762 \\
19200	8.11245660089762 \\
19300	8.11245660089762 \\
19400	8.11245660089762 \\
19500	8.11245660089762 \\
19600	8.11245660089762 \\
19700	8.11245660089762 \\
19800	8.11245660089762 \\
19900	8.11245660089762 \\
20000	8.11245660089762 \\
20100	8.11245660089762 \\
20200	8.11245660089762 \\
20300	8.11245660089762 \\
20400	8.11245660089762 \\
20500	8.11245660089762 \\
20600	8.11245660089762 \\
20700	8.11245660089762 \\
20800	8.11245660089762 \\
20900	8.11245660089762 \\
21000	8.11245660089762 \\
21100	8.11245660089762 \\
21200	8.11245660089762 \\
21300	8.11245660089762 \\
21400	8.11245660089762 \\
21500	8.11245660089762 \\
21600	8.11245660089762 \\
21700	8.11245660089762 \\
21800	8.11245660089762 \\
21900	8.11245660089762 \\
22000	8.11245660089762 \\
22100	8.11245660089762 \\
22200	8.11245660089762 \\
22300	8.11245660089762 \\
22400	8.11245660089762 \\
22500	8.11245660089762 \\
22600	8.11245660089762 \\
22700	8.11245660089762 \\
22800	8.11245660089762 \\
22900	8.11245660089762 \\
23000	8.11245660089762 \\
23100	8.11245660089762 \\
23200	8.11245660089762 \\
23300	8.11245660089762 \\
23400	8.11245660089762 \\
23500	8.11245660089762 \\
23600	8.11245660089762 \\
23700	8.11245660089762 \\
23800	8.11245660089762 \\
23900	8.11245660089762 \\
24000	8.11245660089762 \\
24100	8.11245660089762 \\
24200	8.11245660089762 \\
24300	8.11033957151325 \\
24400	8.11033957151325 \\
24500	8.11033957151325 \\
24600	8.11033957151325 \\
24700	8.10822254212889 \\
24800	8.10822254212889 \\
24900	8.10822254212889 \\
25000	8.10822254212889 \\
25100	8.10822254212889 \\
25200	8.10822254212889 \\
25300	8.10822254212889 \\
25400	8.10822254212889 \\
25500	8.10822254212889 \\
25600	8.10822254212889 \\
25700	8.10822254212889 \\
25800	8.10822254212889 \\
25900	8.10822254212889 \\
26000	8.10822254212889 \\
26100	8.10822254212889 \\
26200	8.10822254212889 \\
26300	8.10822254212889 \\
26400	8.10822254212889 \\
26500	8.10822254212889 \\
26600	8.10822254212889 \\
26700	8.10822254212889 \\
26800	8.10822254212889 \\
26900	8.10822254212889 \\
27000	8.10822254212889 \\
27100	8.10822254212889 \\
27200	8.10822254212889 \\
27300	8.10822254212889 \\
27400	8.10822254212889 \\
27500	8.10822254212889 \\
27600	8.10822254212889 \\
27700	8.10822254212889 \\
27800	8.10822254212889 \\
27900	8.10822254212889 \\
28000	8.10822254212889 \\
28100	8.10822254212889 \\
28200	8.10822254212889 \\
28300	8.10822254212889 \\
28400	8.10822254212889 \\
28500	8.10822254212889 \\
28600	8.10822254212889 \\
28700	8.10822254212889 \\
28800	8.10822254212889 \\
28900	8.10822254212889 \\
29000	8.10822254212889 \\
29100	8.10822254212889 \\
29200	8.10822254212889 \\
29300	8.10822254212889 \\
29400	8.10822254212889 \\
29500	8.10822254212889 \\
29600	8.10822254212889 \\
29700	8.10822254212889 \\
29800	8.10822254212889 \\
29900	8.10822254212889 \\
};

\end{axis}

\end{tikzpicture}
}
\scalebox{.85}{% This file was created by matplotlib2tikz v0.6.18.
\begin{tikzpicture}

\begin{axis}[
legend cell align={left},
legend columns=1,
legend entries={{\pgd},{10\% \algo},{10\% \adaalgo}},
legend style={at={(0.8,0.99)}, anchor=north},
tick align=outside,
tick pos=left,
xlabel={Iteration},
xmajorgrids,
xmin= 0, xmax=30000,
ylabel={Suboptimality},
ymajorgrids,
ymin=1e-9, ymax=16.9012056357722,
ymode=log
]


\addlegendimage{ black,thick,mark=square*}
\addlegendimage{ blue,mark=*}
\addlegendimage{ red,mark=diamond*,mark size = 3pt}

\addplot [ black,thick,mark=square*, mark repeat=100]
table [row sep=\\]{%
0	3.11873290739497 \\
100	2.77535925327753 \\
200	2.50369224345817 \\
300	2.27308798061162 \\
400	2.07075078636474 \\
500	1.89007396092497 \\
600	1.72710069283051 \\
700	1.57925071778381 \\
800	1.44462446322043 \\
900	1.32164655423222 \\
1000	1.2091312139345 \\
1100	1.10604503598994 \\
1200	1.01152652486429 \\
1300	0.924792640324409 \\
1400	0.845191494099079 \\
1500	0.772105699083621 \\
1600	0.705001571513682 \\
1700	0.643374246868564 \\
1800	0.586773953015203 \\
1900	0.534790155024508 \\
2000	0.487040252143533 \\
2100	0.44318773927003 \\
2200	0.402930967731763 \\
2300	0.365992571118125 \\
2400	0.33210802611626 \\
2500	0.301045231971698 \\
2600	0.272579645461304 \\
2700	0.246508660988616 \\
2800	0.222644913899345 \\
2900	0.200817509483236 \\
3000	0.180867602814448 \\
3100	0.162646221544455 \\
3200	0.146016697892941 \\
3300	0.13086008966058 \\
3400	0.117051447369624 \\
3500	0.104494008581944 \\
3600	0.0931005624708253 \\
3700	0.0827567172703523 \\
3800	0.0733865207542882 \\
3900	0.064917771240966 \\
4000	0.0572748943775027 \\
4100	0.0503864545083988 \\
4200	0.0441982819481087 \\
4300	0.0386588305436489 \\
4400	0.0337135309718574 \\
4500	0.0293077448968004 \\
4600	0.025384129222869 \\
4700	0.0219019266446579 \\
4800	0.0188296504524348 \\
4900	0.0161309605958988 \\
5000	0.0137711030533562 \\
5100	0.0117206418236253 \\
5200	0.00993596414789544 \\
5300	0.00839123085221932 \\
5400	0.00705981647626686 \\
5500	0.00592306724551972 \\
5600	0.00495377236691852 \\
5700	0.00413288828235037 \\
5800	0.00344278138528376 \\
5900	0.00286770264705927 \\
6000	0.00238699454507962 \\
6100	0.0019879885421813 \\
6200	0.00165998459442318 \\
6300	0.00138789106214765 \\
6400	0.00116616435194389 \\
6500	0.000987589317950788 \\
6600	0.00084117040830467 \\
6700	0.000720990671235322 \\
6800	0.000622160137551453 \\
6900	0.000539946089365972 \\
7000	0.000470375220936348 \\
7100	0.000411722200673093 \\
7200	0.000361898971029173 \\
7300	0.000318944104594709 \\
7400	0.000281831280821376 \\
7500	0.000250020415022423 \\
7600	0.00022248404899422 \\
7700	0.000198675669167536 \\
7800	0.00017798616787118 \\
7900	0.00015962173593248 \\
8000	0.00014350394983742 \\
8100	0.000129529752432411 \\
8200	0.000117221173490833 \\
8300	0.000106287082733691 \\
8400	9.64412387001734e-05 \\
8500	8.75460266331896e-05 \\
8600	7.95379641899685e-05 \\
8700	7.22952985284797e-05 \\
8800	6.57502969421686e-05 \\
8900	5.98372469407527e-05 \\
9000	5.44980727199551e-05 \\
9100	4.96745624037342e-05 \\
9200	4.52971344500064e-05 \\
9300	4.13194703208775e-05 \\
9400	3.77009753881885e-05 \\
9500	3.44154877140457e-05 \\
9600	3.14231500777584e-05 \\
9700	2.87089473014479e-05 \\
9800	2.62309387236703e-05 \\
9900	2.39676545443768e-05 \\
10000	2.19083193538361e-05 \\
10100	2.00229553654063e-05 \\
10200	1.82922871623248e-05 \\
10300	1.67078663880216e-05 \\
10400	1.52668286419466e-05 \\
10500	1.39599014984393e-05 \\
10600	1.27772308953222e-05 \\
10700	1.16996191808494e-05 \\
10800	1.07122389941816e-05 \\
10900	9.80753638235576e-06 \\
11000	8.98149593459019e-06 \\
11100	8.22519658799648e-06 \\
11200	7.53371385692203e-06 \\
11300	6.90069086728418e-06 \\
11400	6.32161885927296e-06 \\
11500	5.79275745216101e-06 \\
11600	5.30964677025736e-06 \\
11700	4.86763496249099e-06 \\
11800	4.46462252112756e-06 \\
11900	4.09518824523447e-06 \\
12000	3.75621651760305e-06 \\
12100	3.44541611796423e-06 \\
12200	3.16086346197064e-06 \\
12300	2.89990122237516e-06 \\
12400	2.6604341071268e-06 \\
12500	2.4413718927585e-06 \\
12600	2.24010717092549e-06 \\
12700	2.05541381831376e-06 \\
12800	1.88594172068068e-06 \\
12900	1.73080063003983e-06 \\
13000	1.58838616443058e-06 \\
13100	1.457801096727e-06 \\
13200	1.33786950484716e-06 \\
13300	1.22771024829715e-06 \\
13400	1.12652808181846e-06 \\
13500	1.03364107251203e-06 \\
13600	9.4853946364104e-07 \\
13700	8.70405270858576e-07 \\
13800	7.98666252299629e-07 \\
13900	7.32941065229031e-07 \\
14000	6.73027460318387e-07 \\
14100	6.18062610702186e-07 \\
14200	5.67830185038698e-07 \\
14300	5.21755191940088e-07 \\
14400	4.79453572399535e-07 \\
14500	4.4063592014032e-07 \\
14600	4.04955367294679e-07 \\
14700	3.72172756168965e-07 \\
14800	3.42039570777608e-07 \\
14900	3.14339887119708e-07 \\
15000	2.88876649379155e-07 \\
15100	2.65468807825187e-07 \\
15200	2.4397135783838e-07 \\
15300	2.24236513579612e-07 \\
15400	2.06096238020148e-07 \\
15500	1.89421238028231e-07 \\
15600	1.74101937755466e-07 \\
15700	1.60030969476033e-07 \\
15800	1.47110345960488e-07 \\
15900	1.35248632071594e-07 \\
16000	1.24342690255297e-07 \\
16100	1.14315321264602e-07 \\
16200	1.05101985725664e-07 \\
16300	9.66319777528923e-08 \\
16400	8.88438688728677e-08 \\
16500	8.16826342675547e-08 \\
16600	7.51041135393926e-08 \\
16700	6.90598931929465e-08 \\
16800	6.35019077321886e-08 \\
16900	5.83909551821193e-08 \\
17000	5.3698409607783e-08 \\
17100	4.93851148308799e-08 \\
17200	4.54185787868155e-08 \\
17300	4.17708652666882e-08 \\
17400	3.8416315295553e-08 \\
17500	3.53313091006058e-08 \\
17600	3.24941348828212e-08 \\
17700	2.98848429336473e-08 \\
17800	2.74850929793402e-08 \\
17900	2.52795809840833e-08 \\
18000	2.32511830899895e-08 \\
18100	2.13855956832099e-08 \\
18200	1.96697252929923e-08 \\
18300	1.80915338821031e-08 \\
18400	1.6639953359654e-08 \\
18500	1.53054053875223e-08 \\
18600	1.40781508117804e-08 \\
18700	1.29493096245881e-08 \\
18800	1.191097559472e-08 \\
18900	1.09558790928332e-08 \\
19000	1.00773354105854e-08 \\
19100	9.26919774268953e-09 \\
19200	8.52636639070425e-09 \\
19300	7.84406906095825e-09 \\
19400	7.2164286213372e-09 \\
19500	6.63905919129348e-09 \\
19600	6.10792733057863e-09 \\
19700	5.61932417264543e-09 \\
19800	5.16983844622843e-09 \\
19900	4.75633338270498e-09 \\
20000	4.37592395652331e-09 \\
20100	4.02595706772146e-09 \\
20200	3.70399272364708e-09 \\
20300	3.40778644192241e-09 \\
20400	3.13527448447815e-09 \\
20500	2.88455831443102e-09 \\
20600	2.65389210607481e-09 \\
20700	2.44166992180439e-09 \\
20800	2.24658741609929e-09 \\
20900	2.0671566680619e-09 \\
21000	1.90206839167928e-09 \\
21100	1.75017428327706e-09 \\
21200	1.61041840973652e-09 \\
21300	1.48182910386652e-09 \\
21400	1.36351308022142e-09 \\
21500	1.25464821865151e-09 \\
21600	1.15447851278816e-09 \\
21700	1.06230846341759e-09 \\
21800	9.77498249010722e-10 \\
21900	8.99459395853341e-10 \\
22000	8.27650614709796e-10 \\
22100	7.61573637486634e-10 \\
22200	7.00770441675047e-10 \\
22300	6.44819531103735e-10 \\
22400	5.933332158925e-10 \\
22500	5.45954614850075e-10 \\
22600	5.02355657072684e-10 \\
22700	4.6223441740878e-10 \\
22800	4.25313062546451e-10 \\
22900	3.91336185678881e-10 \\
23000	3.60068530547153e-10 \\
23100	3.31293825706069e-10 \\
23200	3.04813119189618e-10 \\
23300	2.80443224198734e-10 \\
23400	2.58021215504556e-10 \\
23500	2.37388553259166e-10 \\
23600	2.18400020290943e-10 \\
23700	2.00924388238377e-10 \\
23800	1.84841086880994e-10 \\
23900	1.70038927382876e-10 \\
24000	1.56415824736911e-10 \\
24100	1.4387768754176e-10 \\
24200	1.32338029423806e-10 \\
24300	1.21717302903335e-10 \\
24400	1.11942177749569e-10 \\
24500	1.02945207913763e-10 \\
24600	9.46644984622935e-11 \\
24700	8.70428729093931e-11 \\
24800	8.00278177059965e-11 \\
24900	7.3571093661684e-11 \\
25000	6.76280698108656e-11 \\
25100	6.21579454573862e-11 \\
25200	5.71229730184086e-11 \\
25300	5.24885690467158e-11 \\
25400	4.82227036080474e-11 \\
25500	4.4296288859158e-11 \\
25600	4.06820133136421e-11 \\
25700	3.73551745092016e-11 \\
25800	3.4292846340378e-11 \\
25900	3.14739900808547e-11 \\
26000	2.88792878500033e-11 \\
26100	2.64906985236735e-11 \\
26200	2.42921238680083e-11 \\
26300	2.22682428052678e-11 \\
26400	2.04051775476444e-11 \\
26500	1.86901050192034e-11 \\
26600	1.71114233893377e-11 \\
26700	1.56580859389521e-11 \\
26800	1.43202116831276e-11 \\
26900	1.30886967930621e-11 \\
27000	1.19548260180125e-11 \\
27100	1.09111053525623e-11 \\
27200	9.95031834705173e-12 \\
27300	9.06574815218164e-12 \\
27400	8.25145507477032e-12 \\
27500	7.50183248854341e-12 \\
27600	6.81171785643642e-12 \\
27700	6.17639273059467e-12 \\
27800	5.59141621891968e-12 \\
27900	5.05290254082524e-12 \\
28000	4.55718796033011e-12 \\
28100	4.10077527490671e-12 \\
28200	3.68055586008609e-12 \\
28300	3.29369864715545e-12 \\
28400	2.9375391008557e-12 \\
28500	2.60963473053266e-12 \\
28600	2.30776509013708e-12 \\
28700	2.02982075592217e-12 \\
28800	1.77391434874608e-12 \\
28900	1.53826951176939e-12 \\
29000	1.32138744390886e-12 \\
29100	1.1216583217788e-12 \\
29200	9.3780538890087e-13 \\
29300	7.68496377645533e-13 \\
29400	6.12621064988161e-13 \\
29500	4.69069227904129e-13 \\
29600	3.36841665671272e-13 \\
29700	2.15272244474818e-13 \\
29800	1.0325074129014e-13 \\
29900	0 \\
};
\addplot [ blue,mark=*, mark repeat=100]
table [row sep=\\]{%
0	3.4972822901323 \\
100	3.39510545487811 \\
200	3.30275597011237 \\
300	3.21810305246387 \\
400	3.13923637750734 \\
500	3.06585180132961 \\
600	2.99721861283229 \\
700	2.93264365616751 \\
800	2.87070178916345 \\
900	2.81189750803698 \\
1000	2.75516030918367 \\
1100	2.7008488465797 \\
1200	2.6483828333318 \\
1300	2.59773522497975 \\
1400	2.5487177178056 \\
1500	2.50096015713199 \\
1600	2.4547685733796 \\
1700	2.40997406975267 \\
1800	2.36628086519521 \\
1900	2.32380717178399 \\
2000	2.2823368273581 \\
2100	2.24180116410383 \\
2200	2.20226211790837 \\
2300	2.16360540684118 \\
2400	2.12577552303587 \\
2500	2.08878147931863 \\
2600	2.0525767336408 \\
2700	2.01710307472315 \\
2800	1.9822822268085 \\
2900	1.94821291850321 \\
3000	1.9147821199043 \\
3100	1.88207229937136 \\
3200	1.8499686063572 \\
3300	1.81840077052322 \\
3400	1.78750197338921 \\
3500	1.75712504957064 \\
3600	1.72736660094836 \\
3700	1.69810164153657 \\
3800	1.6693373940913 \\
3900	1.64116018231041 \\
4000	1.61344252030888 \\
4100	1.5862267701218 \\
4200	1.55946746209051 \\
4300	1.53313806237496 \\
4400	1.50735695503711 \\
4500	1.48199730767284 \\
4600	1.4570791033744 \\
4700	1.43260511405886 \\
4800	1.4085391840132 \\
4900	1.38489295431133 \\
5000	1.36162120659215 \\
5100	1.3387038101714 \\
5200	1.31617393278362 \\
5300	1.29407899383225 \\
5400	1.27229438753743 \\
5500	1.25090010394861 \\
5600	1.22990925105329 \\
5700	1.20919860294222 \\
5800	1.18882524836858 \\
5900	1.16882171528337 \\
6000	1.14911391348541 \\
6100	1.12974885710723 \\
6200	1.11072199059652 \\
6300	1.0919821245389 \\
6400	1.07356909855578 \\
6500	1.05545873934999 \\
6600	1.03759604685021 \\
6700	1.02009918732549 \\
6800	1.00283677495893 \\
6900	0.985883308147115 \\
7000	0.969198283836797 \\
7100	0.952802833508156 \\
7200	0.936691751863281 \\
7300	0.920809210695115 \\
7400	0.90518109207932 \\
7500	0.889790730865599 \\
7600	0.874670858066954 \\
7700	0.859776491481002 \\
7800	0.845120020615617 \\
7900	0.830724927833163 \\
8000	0.81655983253222 \\
8100	0.802606211012568 \\
8200	0.788862790396885 \\
8300	0.775350494821452 \\
8400	0.762071134249867 \\
8500	0.748984604025046 \\
8600	0.736131543929189 \\
8700	0.723464358566196 \\
8800	0.71099642623536 \\
8900	0.698736640421892 \\
9000	0.686663423374142 \\
9100	0.674775272492144 \\
9200	0.663117991953427 \\
9300	0.651624672036636 \\
9400	0.640337594404101 \\
9500	0.629212902632271 \\
9600	0.618250054743158 \\
9700	0.607495814643505 \\
9800	0.596900779755652 \\
9900	0.586464863889119 \\
10000	0.576193616384681 \\
10100	0.566118173723163 \\
10200	0.556190731090684 \\
10300	0.546413994880667 \\
10400	0.536801289310856 \\
10500	0.527361788544908 \\
10600	0.518062527709103 \\
10700	0.508903396627252 \\
10800	0.499899479429161 \\
10900	0.491036331879212 \\
11000	0.482331813052693 \\
11100	0.473757130341422 \\
11200	0.465328080250561 \\
11300	0.457021513934213 \\
11400	0.448860891341354 \\
11500	0.440829673680575 \\
11600	0.432922873811665 \\
11700	0.425140224533427 \\
11800	0.417500012126798 \\
11900	0.409990132632135 \\
12000	0.402594884415163 \\
12100	0.395312177798678 \\
12200	0.388154281692778 \\
12300	0.38109740123952 \\
12400	0.374154790136995 \\
12500	0.367342296343573 \\
12600	0.360623000001894 \\
12700	0.354015501152999 \\
12800	0.347529195508312 \\
12900	0.341142507668412 \\
13000	0.334881376479442 \\
13100	0.328717896508915 \\
13200	0.322654591680694 \\
13300	0.316679111883088 \\
13400	0.310800595445393 \\
13500	0.305023680480247 \\
13600	0.299339422103704 \\
13700	0.293736783528956 \\
13800	0.288238113412875 \\
13900	0.282834877852907 \\
14000	0.277532519311496 \\
14100	0.27231151921171 \\
14200	0.267177879591692 \\
14300	0.262127347051416 \\
14400	0.257154733654282 \\
14500	0.252261457676039 \\
14600	0.24744649704671 \\
14700	0.242714009308533 \\
14800	0.238067700943617 \\
14900	0.233496534941536 \\
15000	0.228994342774265 \\
15100	0.224587262031189 \\
15200	0.220245973554343 \\
15300	0.21597899489172 \\
15400	0.211771922577363 \\
15500	0.207653366589174 \\
15600	0.203591800277056 \\
15700	0.199606382347605 \\
15800	0.195686116648392 \\
15900	0.191824989177648 \\
16000	0.188036810054605 \\
16100	0.184313337673383 \\
16200	0.180653872682677 \\
16300	0.177052227031997 \\
16400	0.173518911423533 \\
16500	0.170040383064209 \\
16600	0.16663051999768 \\
16700	0.16327671501398 \\
16800	0.159982159037717 \\
16900	0.156739102669397 \\
17000	0.153555528014838 \\
17100	0.150425332454145 \\
17200	0.147351822055532 \\
17300	0.144329291624162 \\
17400	0.141362504536084 \\
17500	0.138438125570222 \\
17600	0.135572047156652 \\
17700	0.132763074868144 \\
17800	0.129998956313561 \\
17900	0.127281976597798 \\
18000	0.12462248953996 \\
18100	0.122009514186896 \\
18200	0.119436382112319 \\
18300	0.116907429164548 \\
18400	0.114420686291659 \\
18500	0.1119827383498 \\
18600	0.109587712017956 \\
18700	0.107235405406261 \\
18800	0.1049217474808 \\
18900	0.102658810367907 \\
19000	0.100434794134166 \\
19100	0.0982521519678025 \\
19200	0.0961045246816444 \\
19300	0.0939993538900678 \\
19400	0.09193756788046 \\
19500	0.0899148533039029 \\
19600	0.0879247131409889 \\
19700	0.0859727417040819 \\
19800	0.0840551264792341 \\
19900	0.0821746171239623 \\
20000	0.0803265571596781 \\
20100	0.0785177014595917 \\
20200	0.0767368303836675 \\
20300	0.0749926811985097 \\
20400	0.0732798445486831 \\
20500	0.0716011708735884 \\
20600	0.0699529457352243 \\
20700	0.0683360302034564 \\
20800	0.06675009246955 \\
20900	0.0651926629234637 \\
21000	0.0636697766826078 \\
21100	0.0621732412293891 \\
21200	0.0607061797171847 \\
21300	0.0592701980173865 \\
21400	0.0578620493572647 \\
21500	0.05647634065754 \\
21600	0.055124613294272 \\
21700	0.0538008171105127 \\
21800	0.052501021399452 \\
21900	0.0512266083826023 \\
22000	0.0499792847564014 \\
22100	0.0487571103732222 \\
22200	0.0475592128216455 \\
22300	0.0463888179357008 \\
22400	0.0452389475142634 \\
22500	0.0441161805816802 \\
22600	0.0430152003442734 \\
22700	0.0419360846390645 \\
22800	0.0408812276482906 \\
22900	0.0398450723385866 \\
23000	0.0388328014057585 \\
23100	0.0378438507260243 \\
23200	0.036874697678807 \\
23300	0.035924384368542 \\
23400	0.0349966789049647 \\
23500	0.0340902307251995 \\
23600	0.0331990907870891 \\
23700	0.0323285503006707 \\
23800	0.0314798841104632 \\
23900	0.0306469447819048 \\
24000	0.0298350331661943 \\
24100	0.0290394600847722 \\
24200	0.0282621277312829 \\
24300	0.0274990071187898 \\
24400	0.0267581981182599 \\
24500	0.0260333585967785 \\
24600	0.0253217970354946 \\
24700	0.0246291198650062 \\
24800	0.0239524667955522 \\
24900	0.0232908373643099 \\
25000	0.0226449357884584 \\
25100	0.0220132384240403 \\
25200	0.0213976814577822 \\
25300	0.0207975221488923 \\
25400	0.0202109695406452 \\
25500	0.0196375725053343 \\
25600	0.0190779603693775 \\
25700	0.0185338926035765 \\
25800	0.0180024662374108 \\
25900	0.0174848307273814 \\
26000	0.0169783737136292 \\
26100	0.0164855741828572 \\
26200	0.0160035203416988 \\
26300	0.0155371468614129 \\
26400	0.0150806156924632 \\
26500	0.0146350688499903 \\
26600	0.0142032350306457 \\
26700	0.0137806356098456 \\
26800	0.0133694537788405 \\
26900	0.01296832607183 \\
27000	0.0125770713735133 \\
27100	0.0121950971711584 \\
27200	0.0118263263121198 \\
27300	0.0114667869982677 \\
27400	0.0111166676064888 \\
27500	0.0107759237814063 \\
27600	0.0104450775738532 \\
27700	0.0101216544649749 \\
27800	0.00980728798282876 \\
27900	0.00950070456672131 \\
28000	0.00920324773360165 \\
28100	0.00891295947275234 \\
28200	0.0086315459021723 \\
28300	0.00835809648663238 \\
28400	0.00809184896824849 \\
28500	0.00783439853464185 \\
28600	0.00758281092754398 \\
28700	0.00733972445331826 \\
28800	0.00710299896966105 \\
28900	0.0068728995331212 \\
29000	0.00664971822601551 \\
29100	0.00643337626054841 \\
29200	0.00622330891544509 \\
29300	0.00602101689745954 \\
29400	0.00582347624661816 \\
29500	0.00563152683137846 \\
29600	0.00544532846348722 \\
29700	0.00526454333145154 \\
29800	0.0050898045775199 \\
29900	0.00492096512750201 \\
};



\addplot [red,mark=diamond*,mark size = 3pt, mark repeat=100]
table [row sep=\\]{%
0	3.0857899701446 \\
100	2.75182738187063 \\
200	2.48515358019898 \\
300	2.25766076430116 \\
400	2.05750400028381 \\
500	1.87851104001453 \\
600	1.71691138282605 \\
700	1.57020806605537 \\
800	1.4364984263023 \\
900	1.31431026836838 \\
1000	1.20247888255991 \\
1100	1.10000116124795 \\
1200	1.00602100923469 \\
1300	0.919777744863729 \\
1400	0.840589133109599 \\
1500	0.767856303970862 \\
1600	0.701049958409764 \\
1700	0.639696415173068 \\
1800	0.583325631831081 \\
1900	0.531541132291824 \\
2000	0.483994582827203 \\
2100	0.440347105958501 \\
2200	0.400281847389078 \\
2300	0.363526839596628 \\
2400	0.329822630342297 \\
2500	0.298929767307152 \\
2600	0.270621910098718 \\
2700	0.24469509064499 \\
2800	0.220962806245919 \\
2900	0.199260582000378 \\
3000	0.179432655956278 \\
3100	0.161338975514801 \\
3200	0.144828343166643 \\
3300	0.129784241797847 \\
3400	0.116083855101735 \\
3500	0.103615851220774 \\
3600	0.0922924241537478 \\
3700	0.0820283964720403 \\
3800	0.07273875576908 \\
3900	0.0643394693932832 \\
4000	0.0567637457971095 \\
4100	0.0499378468000212 \\
4200	0.0438107532035085 \\
4300	0.0383223298423189 \\
4400	0.0334209360110357 \\
4500	0.0290602985792681 \\
4600	0.0251825792950961 \\
4700	0.0217528961888003 \\
4800	0.0187238317379649 \\
4900	0.0160554642174976 \\
5000	0.0137097447548956 \\
5100	0.0116652747690236 \\
5200	0.00989056469978372 \\
5300	0.00835687589247591 \\
5400	0.0070367189227758 \\
5500	0.00590841419091453 \\
5600	0.00494630630985637 \\
5700	0.00413081933524995 \\
5800	0.00344525735471207 \\
5900	0.00286824093133481 \\
6000	0.00238533370287908 \\
6100	0.00198707260600633 \\
6200	0.00165855135094378 \\
6300	0.00138925868825562 \\
6400	0.00116829376702376 \\
6500	0.000987499874919806 \\
6600	0.000839069364573786 \\
6700	0.000716769622773339 \\
6800	0.000616094786904553 \\
6900	0.000533323563152499 \\
7000	0.00046504822902671 \\
7100	0.000406785856248593 \\
7200	0.000358010372176754 \\
7300	0.000316779454863358 \\
7400	0.000281341099442722 \\
7500	0.000250373942971671 \\
7600	0.000223337930834511 \\
7700	0.000199701985422462 \\
7800	0.000178973479890709 \\
7900	0.000160689444895112 \\
8000	0.000144687700523727 \\
8100	0.000130516283458337 \\
8200	0.000117880008541849 \\
8300	0.000106558217798647 \\
8400	9.64295334152654e-05 \\
8500	8.73744707606261e-05 \\
8600	7.91968253810449e-05 \\
8700	7.18370356934117e-05 \\
8800	6.52731194321077e-05 \\
8900	5.93962535194725e-05 \\
9000	5.40817397133853e-05 \\
9100	4.92955680780893e-05 \\
9200	4.49611354697566e-05 \\
9300	4.10072659488003e-05 \\
9400	3.74068816460515e-05 \\
9500	3.41373603580108e-05 \\
9600	3.11699377684715e-05 \\
9700	2.84640499313982e-05 \\
9800	2.59986153393066e-05 \\
9900	2.37608390418198e-05 \\
10000	2.17149659883509e-05 \\
10100	1.98529079248888e-05 \\
10200	1.81486513662343e-05 \\
10300	1.66023774511026e-05 \\
10400	1.51973655217463e-05 \\
10500	1.39115223488084e-05 \\
10600	1.27398209850416e-05 \\
10700	1.16687672707672e-05 \\
10800	1.06911797850828e-05 \\
10900	9.79442938220387e-06 \\
11000	8.97443373354401e-06 \\
11100	8.22449466603148e-06 \\
11200	7.53864367497581e-06 \\
11300	6.91025440346937e-06 \\
11400	6.33400321237154e-06 \\
11500	5.8053829887772e-06 \\
11600	5.32042084933071e-06 \\
11700	4.87619484157431e-06 \\
11800	4.47128516450901e-06 \\
11900	4.10002939893461e-06 \\
12000	3.75965417764856e-06 \\
12100	3.44750961855933e-06 \\
12200	3.16150255419423e-06 \\
12300	2.89896512201127e-06 \\
12400	2.65870525506529e-06 \\
12500	2.43868918048395e-06 \\
12600	2.23655672093592e-06 \\
12700	2.05080486004316e-06 \\
12800	1.88010155338647e-06 \\
12900	1.72330178932478e-06 \\
13000	1.57937200773173e-06 \\
13100	1.44794872308784e-06 \\
13200	1.3278234629599e-06 \\
13300	1.21769551730821e-06 \\
13400	1.11650343115333e-06 \\
13500	1.02407215085698e-06 \\
13600	9.39255871190703e-07 \\
13700	8.61565852661172e-07 \\
13800	7.90211709678079e-07 \\
13900	7.24643119121016e-07 \\
14000	6.64777473624589e-07 \\
14100	6.09807695783893e-07 \\
14200	5.59679305012217e-07 \\
14300	5.13886361241944e-07 \\
14400	4.7179294376809e-07 \\
14500	4.33213250994324e-07 \\
14600	3.97867592261125e-07 \\
14700	3.65389825507378e-07 \\
14800	3.35550131902451e-07 \\
14900	3.08119922798333e-07 \\
15000	2.82904468296152e-07 \\
15100	2.59730204121844e-07 \\
15200	2.38566742583046e-07 \\
15300	2.19166354342715e-07 \\
15400	2.01364136587223e-07 \\
15500	1.85004301100911e-07 \\
15600	1.6998955476577e-07 \\
15700	1.56201220569141e-07 \\
15800	1.43530227925126e-07 \\
15900	1.31884193677045e-07 \\
16000	1.21178212464379e-07 \\
16100	1.11338682939888e-07 \\
16200	1.02304212601378e-07 \\
16300	9.3997072792007e-08 \\
16400	8.63585447041793e-08 \\
16500	7.93681862454321e-08 \\
16600	7.2954425056615e-08 \\
16700	6.70567888994e-08 \\
16800	6.16336505165904e-08 \\
16900	5.66484843478321e-08 \\
17000	5.20866698106026e-08 \\
17100	4.79032054023776e-08 \\
17200	4.40591325645556e-08 \\
17300	4.05241211698382e-08 \\
17400	3.72732692488498e-08 \\
17500	3.42836862787799e-08 \\
17600	3.15343301471316e-08 \\
17700	2.90058573826357e-08 \\
17800	2.66804857651515e-08 \\
17900	2.4541867926775e-08 \\
18000	2.25749752225113e-08 \\
18100	2.07659912043745e-08 \\
18200	1.9102213477673e-08 \\
18300	1.75719636064109e-08 \\
18400	1.61645046237169e-08 \\
18500	1.48699645374784e-08 \\
18600	1.36792669969132e-08 \\
18700	1.25840665665677e-08 \\
18800	1.15766899400072e-08 \\
18900	1.06500814833765e-08 \\
19000	9.79775388598725e-09 \\
19100	9.01374153094991e-09 \\
19200	8.29255925038908e-09 \\
19300	7.6291629125258e-09 \\
19400	7.01891422760781e-09 \\
19500	6.45754788530795e-09 \\
19600	5.94114152319136e-09 \\
19700	5.4660881931845e-09 \\
19800	5.0290707709344e-09 \\
19900	4.62703875214743e-09 \\
20000	4.25718682528498e-09 \\
20100	3.91693472101551e-09 \\
20200	3.6039098927354e-09 \\
20300	3.3159298085117e-09 \\
20400	3.05098762920508e-09 \\
20500	2.80723744250366e-09 \\
20600	2.58298127331358e-09 \\
20700	2.37665787050645e-09 \\
20800	2.18683088304417e-09 \\
20900	2.01217931206088e-09 \\
21000	1.85148824050074e-09 \\
21100	1.70363984031141e-09 \\
21200	1.56760637803899e-09 \\
21300	1.4424422767334e-09 \\
21400	1.32727762114371e-09 \\
21500	1.22131249558066e-09 \\
21600	1.12381093320124e-09 \\
21700	1.03409586449388e-09 \\
21800	9.51544676386362e-10 \\
21900	8.75584438286836e-10 \\
22000	8.05688293858964e-10 \\
22100	7.41371242174438e-10 \\
22200	6.82187584200022e-10 \\
22300	6.27726870483514e-10 \\
22400	5.77611736218842e-10 \\
22500	5.31494959155054e-10 \\
22600	4.89057128127968e-10 \\
22700	4.50004311591812e-10 \\
22800	4.14066225751242e-10 \\
22900	3.80994014115288e-10 \\
23000	3.50558970740877e-10 \\
23100	3.22550430809088e-10 \\
23200	2.96774715913273e-10 \\
23300	2.73053579746829e-10 \\
23400	2.5122304236902e-10 \\
23500	2.31132279981949e-10 \\
23600	2.12642514707539e-10 \\
23700	1.95625737831051e-10 \\
23800	1.79964820823386e-10 \\
23900	1.6555135040619e-10 \\
24000	1.52285961618759e-10 \\
24100	1.40077061061561e-10 \\
24200	1.28840327295876e-10 \\
24300	1.18498433288039e-10 \\
24400	1.08980047208718e-10 \\
24500	1.00219388343703e-10 \\
24600	9.21560605604554e-11 \\
24700	8.47346637300461e-11 \\
24800	7.79038500375862e-11 \\
24900	7.16165460268314e-11 \\
25000	6.5829564022124e-11 \\
25100	6.05029915057287e-11 \\
25200	5.56001356066815e-11 \\
25300	5.10873010561852e-11 \\
25400	4.69334016095502e-11 \\
25500	4.31098490238924e-11 \\
25600	3.95904420358306e-11 \\
25700	3.63507557388232e-11 \\
25800	3.33687522058312e-11 \\
25900	3.06237257774455e-11 \\
26000	2.80969136845499e-11 \\
26100	2.57710519591114e-11 \\
26200	2.36299868561218e-11 \\
26300	2.1659063431656e-11 \\
26400	1.98447369648136e-11 \\
26500	1.817462846887e-11 \\
26600	1.66371916243691e-11 \\
26700	1.52218793125769e-11 \\
26800	1.39189215708768e-11 \\
26900	1.27196586596767e-11 \\
27000	1.16154308393845e-11 \\
27100	1.0598966149189e-11 \\
27200	9.66321467288367e-12 \\
27300	8.80179262807701e-12 \\
27400	8.00876032158726e-12 \\
27500	7.27867766059376e-12 \\
27600	6.60654864148569e-12 \\
27700	5.98776583871086e-12 \\
27800	5.41811040477569e-12 \\
27900	4.89364104794276e-12 \\
28000	4.41080505453328e-12 \\
28100	3.96621624432214e-12 \\
28200	3.55698803744531e-12 \\
28300	3.18023385403876e-12 \\
28400	2.83334466999463e-12 \\
28500	2.51393350580997e-12 \\
28600	2.219890937738e-12 \\
28700	1.94916305318316e-12 \\
28800	1.69991798415481e-12 \\
28900	1.47037937381356e-12 \\
29000	1.25910393222739e-12 \\
29100	1.06459285831306e-12 \\
29200	8.85402862138562e-13 \\
29300	7.20479231830495e-13 \\
29400	5.68656233213005e-13 \\
29500	4.28879154412698e-13 \\
29600	3.00148794707411e-13 \\
29700	1.81576975677444e-13 \\
29800	7.24420523567915e-14 \\
29900	-2.80331313717852e-14 \\
};
\end{axis}

\end{tikzpicture}
}
\scalebox{.85}{% This file was created by matplotlib2tikz v0.6.18.
\begin{tikzpicture}

\begin{axis}[
legend cell align={left},
legend columns=1,
legend entries={{\pgd},{10\% \algo},{10\% \adaalgo}},
legend style={at={(0.8,0.99)}, anchor=north},
tick align=outside,
tick pos=left,
xlabel={Number of Subspaces explored},
xmajorgrids,
xmin=0, xmax=1e9,
ylabel={Suboptimality},
ymajorgrids,
ymin=1e-8, ymax=16.9012056357722,
ymode=log
]

\addlegendimage{ black,thick,mark=square*}
\addlegendimage{ blue,mark=*}
\addlegendimage{ red,mark=diamond*,mark size = 3pt}

\addplot [black,thick,mark=square*, mark repeat=100]
table [row sep=\\]{%
4723600	3.11873290739497 \\
9447200	2.77535925327753 \\
14170800	2.50369224345817 \\
18894400	2.27308798061162 \\
23618000	2.07075078636474 \\
28341600	1.89007396092497 \\
33065200	1.72710069283051 \\
37788800	1.57925071778381 \\
42512400	1.44462446322043 \\
47236000	1.32164655423222 \\
51959600	1.2091312139345 \\
56683200	1.10604503598994 \\
61406800	1.01152652486429 \\
66130400	0.924792640324409 \\
70854000	0.845191494099079 \\
75577600	0.772105699083621 \\
80301200	0.705001571513682 \\
85024800	0.643374246868564 \\
89748400	0.586773953015203 \\
94472000	0.534790155024508 \\
99195600	0.487040252143533 \\
103919200	0.44318773927003 \\
108642800	0.402930967731763 \\
113366400	0.365992571118125 \\
118090000	0.33210802611626 \\
122813600	0.301045231971698 \\
127537200	0.272579645461304 \\
132260800	0.246508660988616 \\
136984400	0.222644913899345 \\
141708000	0.200817509483236 \\
146431600	0.180867602814448 \\
151155200	0.162646221544455 \\
155878800	0.146016697892941 \\
160602400	0.13086008966058 \\
165326000	0.117051447369624 \\
170049600	0.104494008581944 \\
174773200	0.0931005624708253 \\
179496800	0.0827567172703523 \\
184220400	0.0733865207542882 \\
188944000	0.064917771240966 \\
193667600	0.0572748943775027 \\
198391200	0.0503864545083988 \\
203114800	0.0441982819481087 \\
207838400	0.0386588305436489 \\
212562000	0.0337135309718574 \\
217285600	0.0293077448968004 \\
222009200	0.025384129222869 \\
226732800	0.0219019266446579 \\
231456400	0.0188296504524348 \\
236180000	0.0161309605958988 \\
240903600	0.0137711030533562 \\
245627200	0.0117206418236253 \\
250350800	0.00993596414789544 \\
255074400	0.00839123085221932 \\
259798000	0.00705981647626686 \\
264521600	0.00592306724551972 \\
269245200	0.00495377236691852 \\
273968800	0.00413288828235037 \\
278692400	0.00344278138528376 \\
283416000	0.00286770264705927 \\
288139600	0.00238699454507962 \\
292863200	0.0019879885421813 \\
297586800	0.00165998459442318 \\
302310400	0.00138789106214765 \\
307034000	0.00116616435194389 \\
311757600	0.000987589317950788 \\
316481200	0.00084117040830467 \\
321204800	0.000720990671235322 \\
325928400	0.000622160137551453 \\
330652000	0.000539946089365972 \\
335375600	0.000470375220936348 \\
340099200	0.000411722200673093 \\
344822800	0.000361898971029173 \\
349546400	0.000318944104594709 \\
354270000	0.000281831280821376 \\
358993600	0.000250020415022423 \\
363717200	0.00022248404899422 \\
368440800	0.000198675669167536 \\
373164400	0.00017798616787118 \\
377888000	0.00015962173593248 \\
382611600	0.00014350394983742 \\
387335200	0.000129529752432411 \\
392058800	0.000117221173490833 \\
396782400	0.000106287082733691 \\
401506000	9.64412387001734e-05 \\
406229600	8.75460266331896e-05 \\
410953200	7.95379641899685e-05 \\
415676800	7.22952985284797e-05 \\
420400400	6.57502969421686e-05 \\
425124000	5.98372469407527e-05 \\
429847600	5.44980727199551e-05 \\
434571200	4.96745624037342e-05 \\
439294800	4.52971344500064e-05 \\
444018400	4.13194703208775e-05 \\
448742000	3.77009753881885e-05 \\
453465600	3.44154877140457e-05 \\
458189200	3.14231500777584e-05 \\
462912800	2.87089473014479e-05 \\
467636400	2.62309387236703e-05 \\
472360000	2.39676545443768e-05 \\
477083600	2.19083193538361e-05 \\
481807200	2.00229553654063e-05 \\
486530800	1.82922871623248e-05 \\
491254400	1.67078663880216e-05 \\
495978000	1.52668286419466e-05 \\
500701600	1.39599014984393e-05 \\
505425200	1.27772308953222e-05 \\
510148800	1.16996191808494e-05 \\
514872400	1.07122389941816e-05 \\
519596000	9.80753638235576e-06 \\
524319600	8.98149593459019e-06 \\
529043200	8.22519658799648e-06 \\
533766800	7.53371385692203e-06 \\
538490400	6.90069086728418e-06 \\
543214000	6.32161885927296e-06 \\
547937600	5.79275745216101e-06 \\
552661200	5.30964677025736e-06 \\
557384800	4.86763496249099e-06 \\
562108400	4.46462252112756e-06 \\
566832000	4.09518824523447e-06 \\
571555600	3.75621651760305e-06 \\
576279200	3.44541611796423e-06 \\
581002800	3.16086346197064e-06 \\
585726400	2.89990122237516e-06 \\
590450000	2.6604341071268e-06 \\
595173600	2.4413718927585e-06 \\
599897200	2.24010717092549e-06 \\
604620800	2.05541381831376e-06 \\
609344400	1.88594172068068e-06 \\
614068000	1.73080063003983e-06 \\
618791600	1.58838616443058e-06 \\
623515200	1.457801096727e-06 \\
628238800	1.33786950484716e-06 \\
632962400	1.22771024829715e-06 \\
637686000	1.12652808181846e-06 \\
642409600	1.03364107251203e-06 \\
647133200	9.4853946364104e-07 \\
651856800	8.70405270858576e-07 \\
656580400	7.98666252299629e-07 \\
661304000	7.32941065229031e-07 \\
666027600	6.73027460318387e-07 \\
670751200	6.18062610702186e-07 \\
675474800	5.67830185038698e-07 \\
680198400	5.21755191940088e-07 \\
684922000	4.79453572399535e-07 \\
689645600	4.4063592014032e-07 \\
694369200	4.04955367294679e-07 \\
699092800	3.72172756168965e-07 \\
703816400	3.42039570777608e-07 \\
708540000	3.14339887119708e-07 \\
713263600	2.88876649379155e-07 \\
717987200	2.65468807825187e-07 \\
722710800	2.4397135783838e-07 \\
727434400	2.24236513579612e-07 \\
732158000	2.06096238020148e-07 \\
736881600	1.89421238028231e-07 \\
741605200	1.74101937755466e-07 \\
746328800	1.60030969476033e-07 \\
751052400	1.47110345960488e-07 \\
755776000	1.35248632071594e-07 \\
760499600	1.24342690255297e-07 \\
765223200	1.14315321264602e-07 \\
769946800	1.05101985725664e-07 \\
774670400	9.66319777528923e-08 \\
779394000	8.88438688728677e-08 \\
784117600	8.16826342675547e-08 \\
788841200	7.51041135393926e-08 \\
793564800	6.90598931929465e-08 \\
798288400	6.35019077321886e-08 \\
803012000	5.83909551821193e-08 \\
807735600	5.3698409607783e-08 \\
812459200	4.93851148308799e-08 \\
817182800	4.54185787868155e-08 \\
821906400	4.17708652666882e-08 \\
826630000	3.8416315295553e-08 \\
831353600	3.53313091006058e-08 \\
836077200	3.24941348828212e-08 \\
840800800	2.98848429336473e-08 \\
845524400	2.74850929793402e-08 \\
850248000	2.52795809840833e-08 \\
854971600	2.32511830899895e-08 \\
859695200	2.13855956832099e-08 \\
864418800	1.96697252929923e-08 \\
869142400	1.80915338821031e-08 \\
873866000	1.6639953359654e-08 \\
878589600	1.53054053875223e-08 \\
883313200	1.40781508117804e-08 \\
888036800	1.29493096245881e-08 \\
892760400	1.191097559472e-08 \\
897484000	1.09558790928332e-08 \\
902207600	1.00773354105854e-08 \\
906931200	9.26919774268953e-09 \\
911654800	8.52636639070425e-09 \\
916378400	7.84406906095825e-09 \\
921102000	7.2164286213372e-09 \\
925825600	6.63905919129348e-09 \\
930549200	6.10792733057863e-09 \\
935272800	5.61932417264543e-09 \\
939996400	5.16983844622843e-09 \\
944720000	4.75633338270498e-09 \\
949443600	4.37592395652331e-09 \\
954167200	4.02595706772146e-09 \\
958890800	3.70399272364708e-09 \\
963614400	3.40778644192241e-09 \\
968338000	3.13527448447815e-09 \\
973061600	2.88455831443102e-09 \\
977785200	2.65389210607481e-09 \\
982508800	2.44166992180439e-09 \\
987232400	2.24658741609929e-09 \\
991956000	2.0671566680619e-09 \\
996679600	1.90206839167928e-09 \\
1001403200	1.75017428327706e-09 \\
1006126800	1.61041840973652e-09 \\
1010850400	1.48182910386652e-09 \\
1015574000	1.36351308022142e-09 \\
1020297600	1.25464821865151e-09 \\
1025021200	1.15447851278816e-09 \\
1029744800	1.06230846341759e-09 \\
1034468400	9.77498249010722e-10 \\
1039192000	8.99459395853341e-10 \\
1043915600	8.27650614709796e-10 \\
1048639200	7.61573637486634e-10 \\
1053362800	7.00770441675047e-10 \\
1058086400	6.44819531103735e-10 \\
1062810000	5.933332158925e-10 \\
1067533600	5.45954614850075e-10 \\
1072257200	5.02355657072684e-10 \\
1076980800	4.6223441740878e-10 \\
1081704400	4.25313062546451e-10 \\
1086428000	3.91336185678881e-10 \\
1091151600	3.60068530547153e-10 \\
1095875200	3.31293825706069e-10 \\
1100598800	3.04813119189618e-10 \\
1105322400	2.80443224198734e-10 \\
1110046000	2.58021215504556e-10 \\
1114769600	2.37388553259166e-10 \\
1119493200	2.18400020290943e-10 \\
1124216800	2.00924388238377e-10 \\
1128940400	1.84841086880994e-10 \\
1133664000	1.70038927382876e-10 \\
1138387600	1.56415824736911e-10 \\
1143111200	1.4387768754176e-10 \\
1147834800	1.32338029423806e-10 \\
1152558400	1.21717302903335e-10 \\
1157282000	1.11942177749569e-10 \\
1162005600	1.02945207913763e-10 \\
1166729200	9.46644984622935e-11 \\
1171452800	8.70428729093931e-11 \\
1176176400	8.00278177059965e-11 \\
1180900000	7.3571093661684e-11 \\
1185623600	6.76280698108656e-11 \\
1190347200	6.21579454573862e-11 \\
1195070800	5.71229730184086e-11 \\
1199794400	5.24885690467158e-11 \\
1204518000	4.82227036080474e-11 \\
1209241600	4.4296288859158e-11 \\
1213965200	4.06820133136421e-11 \\
1218688800	3.73551745092016e-11 \\
1223412400	3.4292846340378e-11 \\
1228136000	3.14739900808547e-11 \\
1232859600	2.88792878500033e-11 \\
1237583200	2.64906985236735e-11 \\
1242306800	2.42921238680083e-11 \\
1247030400	2.22682428052678e-11 \\
1251754000	2.04051775476444e-11 \\
1256477600	1.86901050192034e-11 \\
1261201200	1.71114233893377e-11 \\
1265924800	1.56580859389521e-11 \\
1270648400	1.43202116831276e-11 \\
1275372000	1.30886967930621e-11 \\
1280095600	1.19548260180125e-11 \\
1284819200	1.09111053525623e-11 \\
1289542800	9.95031834705173e-12 \\
1294266400	9.06574815218164e-12 \\
1298990000	8.25145507477032e-12 \\
1303713600	7.50183248854341e-12 \\
1308437200	6.81171785643642e-12 \\
1313160800	6.17639273059467e-12 \\
1317884400	5.59141621891968e-12 \\
1322608000	5.05290254082524e-12 \\
1327331600	4.55718796033011e-12 \\
1332055200	4.10077527490671e-12 \\
1336778800	3.68055586008609e-12 \\
1341502400	3.29369864715545e-12 \\
1346226000	2.9375391008557e-12 \\
1350949600	2.60963473053266e-12 \\
1355673200	2.30776509013708e-12 \\
1360396800	2.02982075592217e-12 \\
1365120400	1.77391434874608e-12 \\
1369844000	1.53826951176939e-12 \\
1374567600	1.32138744390886e-12 \\
1379291200	1.1216583217788e-12 \\
1384014800	9.3780538890087e-13 \\
1388738400	7.68496377645533e-13 \\
1393462000	6.12621064988161e-13 \\
1398185600	4.69069227904129e-13 \\
1402909200	3.36841665671272e-13 \\
1407632800	2.15272244474818e-13 \\
1412356400	1.0325074129014e-13 \\
1417080000	0 \\
};
\addplot [blue,mark=*, mark repeat= 100]
table [row sep=\\]{%
900000	3.4972822901323 \\
1800000	3.39510545487811 \\
2700000	3.30275597011237 \\
3600000	3.21810305246387 \\
4500000	3.13923637750734 \\
5400000	3.06585180132961 \\
6300000	2.99721861283229 \\
7200000	2.93264365616751 \\
8100000	2.87070178916345 \\
9000000	2.81189750803698 \\
9900000	2.75516030918367 \\
10800000	2.7008488465797 \\
11700000	2.6483828333318 \\
12600000	2.59773522497975 \\
13500000	2.5487177178056 \\
14400000	2.50096015713199 \\
15300000	2.4547685733796 \\
16200000	2.40997406975267 \\
17100000	2.36628086519521 \\
18000000	2.32380717178399 \\
18900000	2.2823368273581 \\
19800000	2.24180116410383 \\
20700000	2.20226211790837 \\
21600000	2.16360540684118 \\
22500000	2.12577552303587 \\
23400000	2.08878147931863 \\
24300000	2.0525767336408 \\
25200000	2.01710307472315 \\
26100000	1.9822822268085 \\
27000000	1.94821291850321 \\
27900000	1.9147821199043 \\
28800000	1.88207229937136 \\
29700000	1.8499686063572 \\
30600000	1.81840077052322 \\
31500000	1.78750197338921 \\
32400000	1.75712504957064 \\
33300000	1.72736660094836 \\
34200000	1.69810164153657 \\
35100000	1.6693373940913 \\
36000000	1.64116018231041 \\
36900000	1.61344252030888 \\
37800000	1.5862267701218 \\
38700000	1.55946746209051 \\
39600000	1.53313806237496 \\
40500000	1.50735695503711 \\
41400000	1.48199730767284 \\
42300000	1.4570791033744 \\
43200000	1.43260511405886 \\
44100000	1.4085391840132 \\
45000000	1.38489295431133 \\
45900000	1.36162120659215 \\
46800000	1.3387038101714 \\
47700000	1.31617393278362 \\
48600000	1.29407899383225 \\
49500000	1.27229438753743 \\
50400000	1.25090010394861 \\
51300000	1.22990925105329 \\
52200000	1.20919860294222 \\
53100000	1.18882524836858 \\
54000000	1.16882171528337 \\
54900000	1.14911391348541 \\
55800000	1.12974885710723 \\
56700000	1.11072199059652 \\
57600000	1.0919821245389 \\
58500000	1.07356909855578 \\
59400000	1.05545873934999 \\
60300000	1.03759604685021 \\
61200000	1.02009918732549 \\
62100000	1.00283677495893 \\
63000000	0.985883308147115 \\
63900000	0.969198283836797 \\
64800000	0.952802833508156 \\
65700000	0.936691751863281 \\
66600000	0.920809210695115 \\
67500000	0.90518109207932 \\
68400000	0.889790730865599 \\
69300000	0.874670858066954 \\
70200000	0.859776491481002 \\
71100000	0.845120020615617 \\
72000000	0.830724927833163 \\
72900000	0.81655983253222 \\
73800000	0.802606211012568 \\
74700000	0.788862790396885 \\
75600000	0.775350494821452 \\
76500000	0.762071134249867 \\
77400000	0.748984604025046 \\
78300000	0.736131543929189 \\
79200000	0.723464358566196 \\
80100000	0.71099642623536 \\
81000000	0.698736640421892 \\
81900000	0.686663423374142 \\
82800000	0.674775272492144 \\
83700000	0.663117991953427 \\
84600000	0.651624672036636 \\
85500000	0.640337594404101 \\
86400000	0.629212902632271 \\
87300000	0.618250054743158 \\
88200000	0.607495814643505 \\
89100000	0.596900779755652 \\
90000000	0.586464863889119 \\
90900000	0.576193616384681 \\
91800000	0.566118173723163 \\
92700000	0.556190731090684 \\
93600000	0.546413994880667 \\
94500000	0.536801289310856 \\
95400000	0.527361788544908 \\
96300000	0.518062527709103 \\
97200000	0.508903396627252 \\
98100000	0.499899479429161 \\
99000000	0.491036331879212 \\
99900000	0.482331813052693 \\
100800000	0.473757130341422 \\
101700000	0.465328080250561 \\
102600000	0.457021513934213 \\
103500000	0.448860891341354 \\
104400000	0.440829673680575 \\
105300000	0.432922873811665 \\
106200000	0.425140224533427 \\
107100000	0.417500012126798 \\
108000000	0.409990132632135 \\
108900000	0.402594884415163 \\
109800000	0.395312177798678 \\
110700000	0.388154281692778 \\
111600000	0.38109740123952 \\
112500000	0.374154790136995 \\
113400000	0.367342296343573 \\
114300000	0.360623000001894 \\
115200000	0.354015501152999 \\
116100000	0.347529195508312 \\
117000000	0.341142507668412 \\
117900000	0.334881376479442 \\
118800000	0.328717896508915 \\
119700000	0.322654591680694 \\
120600000	0.316679111883088 \\
121500000	0.310800595445393 \\
122400000	0.305023680480247 \\
123300000	0.299339422103704 \\
124200000	0.293736783528956 \\
125100000	0.288238113412875 \\
126000000	0.282834877852907 \\
126900000	0.277532519311496 \\
127800000	0.27231151921171 \\
128700000	0.267177879591692 \\
129600000	0.262127347051416 \\
130500000	0.257154733654282 \\
131400000	0.252261457676039 \\
132300000	0.24744649704671 \\
133200000	0.242714009308533 \\
134100000	0.238067700943617 \\
135000000	0.233496534941536 \\
135900000	0.228994342774265 \\
136800000	0.224587262031189 \\
137700000	0.220245973554343 \\
138600000	0.21597899489172 \\
139500000	0.211771922577363 \\
140400000	0.207653366589174 \\
141300000	0.203591800277056 \\
142200000	0.199606382347605 \\
143100000	0.195686116648392 \\
144000000	0.191824989177648 \\
144900000	0.188036810054605 \\
145800000	0.184313337673383 \\
146700000	0.180653872682677 \\
147600000	0.177052227031997 \\
148500000	0.173518911423533 \\
149400000	0.170040383064209 \\
150300000	0.16663051999768 \\
151200000	0.16327671501398 \\
152100000	0.159982159037717 \\
153000000	0.156739102669397 \\
153900000	0.153555528014838 \\
154800000	0.150425332454145 \\
155700000	0.147351822055532 \\
156600000	0.144329291624162 \\
157500000	0.141362504536084 \\
158400000	0.138438125570222 \\
159300000	0.135572047156652 \\
160200000	0.132763074868144 \\
161100000	0.129998956313561 \\
162000000	0.127281976597798 \\
162900000	0.12462248953996 \\
163800000	0.122009514186896 \\
164700000	0.119436382112319 \\
165600000	0.116907429164548 \\
166500000	0.114420686291659 \\
167400000	0.1119827383498 \\
168300000	0.109587712017956 \\
169200000	0.107235405406261 \\
170100000	0.1049217474808 \\
171000000	0.102658810367907 \\
171900000	0.100434794134166 \\
172800000	0.0982521519678025 \\
173700000	0.0961045246816444 \\
174600000	0.0939993538900678 \\
175500000	0.09193756788046 \\
176400000	0.0899148533039029 \\
177300000	0.0879247131409889 \\
178200000	0.0859727417040819 \\
179100000	0.0840551264792341 \\
180000000	0.0821746171239623 \\
180900000	0.0803265571596781 \\
181800000	0.0785177014595917 \\
182700000	0.0767368303836675 \\
183600000	0.0749926811985097 \\
184500000	0.0732798445486831 \\
185400000	0.0716011708735884 \\
186300000	0.0699529457352243 \\
187200000	0.0683360302034564 \\
188100000	0.06675009246955 \\
189000000	0.0651926629234637 \\
189900000	0.0636697766826078 \\
190800000	0.0621732412293891 \\
191700000	0.0607061797171847 \\
192600000	0.0592701980173865 \\
193500000	0.0578620493572647 \\
194400000	0.05647634065754 \\
195300000	0.055124613294272 \\
196200000	0.0538008171105127 \\
197100000	0.052501021399452 \\
198000000	0.0512266083826023 \\
198900000	0.0499792847564014 \\
199800000	0.0487571103732222 \\
200700000	0.0475592128216455 \\
201600000	0.0463888179357008 \\
202500000	0.0452389475142634 \\
203400000	0.0441161805816802 \\
204300000	0.0430152003442734 \\
205200000	0.0419360846390645 \\
206100000	0.0408812276482906 \\
207000000	0.0398450723385866 \\
207900000	0.0388328014057585 \\
208800000	0.0378438507260243 \\
209700000	0.036874697678807 \\
210600000	0.035924384368542 \\
211500000	0.0349966789049647 \\
212400000	0.0340902307251995 \\
213300000	0.0331990907870891 \\
214200000	0.0323285503006707 \\
215100000	0.0314798841104632 \\
216000000	0.0306469447819048 \\
216900000	0.0298350331661943 \\
217800000	0.0290394600847722 \\
218700000	0.0282621277312829 \\
219600000	0.0274990071187898 \\
220500000	0.0267581981182599 \\
221400000	0.0260333585967785 \\
222300000	0.0253217970354946 \\
223200000	0.0246291198650062 \\
224100000	0.0239524667955522 \\
225000000	0.0232908373643099 \\
225900000	0.0226449357884584 \\
226800000	0.0220132384240403 \\
227700000	0.0213976814577822 \\
228600000	0.0207975221488923 \\
229500000	0.0202109695406452 \\
230400000	0.0196375725053343 \\
231300000	0.0190779603693775 \\
232200000	0.0185338926035765 \\
233100000	0.0180024662374108 \\
234000000	0.0174848307273814 \\
234900000	0.0169783737136292 \\
235800000	0.0164855741828572 \\
236700000	0.0160035203416988 \\
237600000	0.0155371468614129 \\
238500000	0.0150806156924632 \\
239400000	0.0146350688499903 \\
240300000	0.0142032350306457 \\
241200000	0.0137806356098456 \\
242100000	0.0133694537788405 \\
243000000	0.01296832607183 \\
243900000	0.0125770713735133 \\
244800000	0.0121950971711584 \\
245700000	0.0118263263121198 \\
246600000	0.0114667869982677 \\
247500000	0.0111166676064888 \\
248400000	0.0107759237814063 \\
249300000	0.0104450775738532 \\
250200000	0.0101216544649749 \\
251100000	0.00980728798282876 \\
252000000	0.00950070456672131 \\
252900000	0.00920324773360165 \\
253800000	0.00891295947275234 \\
254700000	0.0086315459021723 \\
255600000	0.00835809648663238 \\
256500000	0.00809184896824849 \\
257400000	0.00783439853464185 \\
258300000	0.00758281092754398 \\
259200000	0.00733972445331826 \\
260100000	0.00710299896966105 \\
261000000	0.0068728995331212 \\
261900000	0.00664971822601551 \\
262800000	0.00643337626054841 \\
263700000	0.00622330891544509 \\
264600000	0.00602101689745954 \\
265500000	0.00582347624661816 \\
266400000	0.00563152683137846 \\
267300000	0.00544532846348722 \\
268200000	0.00526454333145154 \\
269100000	0.0050898045775199 \\
270000000	0.00492096512750201 \\
};

\addplot [red,mark=diamond*,mark size = 3pt,mark repeat = 100]
table [row sep=\\]{%
4723600	3.0857899701446 \\
9447200	2.75182738187063 \\
14170800	2.48515358019898 \\
18894400	2.25766076430116 \\
23618000	2.05750400028381 \\
28341600	1.87851104001453 \\
33065200	1.71691138282605 \\
37788800	1.57020806605537 \\
42512400	1.4364984263023 \\
47236000	1.31431026836838 \\
51959600	1.20247888255991 \\
56683200	1.10000116124795 \\
61406800	1.00602100923469 \\
66109673	0.919777744863729 \\
70762731	0.840589133109599 \\
75366342	0.767856303970862 \\
79916453	0.701049958409764 \\
84414041	0.639696415173068 \\
88857144	0.583325631831081 \\
93242928	0.531541132291824 \\
97570228	0.483994582827203 \\
101834933	0.440347105958501 \\
106033319	0.400281847389078 \\
110158467	0.363526839596628 \\
114213180	0.329822630342297 \\
118197405	0.298929767307152 \\
122107898	0.270621910098718 \\
125942272	0.24469509064499 \\
129698309	0.220962806245919 \\
133372535	0.199260582000378 \\
136963549	0.179432655956278 \\
140466451	0.161338975514801 \\
143881724	0.144828343166643 \\
147205878	0.129784241797847 \\
150439153	0.116083855101735 \\
153585366	0.103615851220774 \\
156642135	0.0922924241537478 \\
159605784	0.0820283964720403 \\
162474549	0.07273875576908 \\
165250372	0.0643394693932832 \\
167928761	0.0567637457971095 \\
170511894	0.0499378468000212 \\
172994776	0.0438107532035085 \\
175382318	0.0383223298423189 \\
177674688	0.0334209360110357 \\
179870620	0.0290602985792681 \\
181978085	0.0251825792950961 \\
183994326	0.0217528961888003 \\
185924840	0.0187238317379649 \\
187774411	0.0160554642174976 \\
189547829	0.0137097447548956 \\
191242760	0.0116652747690236 \\
192862968	0.00989056469978372 \\
194412359	0.00835687589247591 \\
195895221	0.0070367189227758 \\
197314549	0.00590841419091453 \\
198677735	0.00494630630985637 \\
199987261	0.00413081933524995 \\
201244752	0.00344525735471207 \\
202459420	0.00286824093133481 \\
203634512	0.00238533370287908 \\
204771330	0.00198707260600633 \\
205873898	0.00165855135094378 \\
206946603	0.00138925868825562 \\
207994707	0.00116829376702376 \\
209020696	0.000987499874919806 \\
210026756	0.000839069364573786 \\
211016625	0.000716769622773339 \\
211992498	0.000616094786904553 \\
212956493	0.000533323563152499 \\
213909947	0.00046504822902671 \\
214855484	0.000406785856248593 \\
215793435	0.000358010372176754 \\
216725053	0.000316779454863358 \\
217650697	0.000281341099442722 \\
218571715	0.000250373942971671 \\
219488824	0.000223337930834511 \\
220401909	0.000199701985422462 \\
221311294	0.000178973479890709 \\
222217251	0.000160689444895112 \\
223119323	0.000144687700523727 \\
224018584	0.000130516283458337 \\
224915079	0.000117880008541849 \\
225808617	0.000106558217798647 \\
226699726	9.64295334152654e-05 \\
227588784	8.73744707606261e-05 \\
228475903	7.91968253810449e-05 \\
229361325	7.18370356934117e-05 \\
230244788	6.52731194321077e-05 \\
231126570	5.93962535194725e-05 \\
232006856	5.40817397133853e-05 \\
232886001	4.92955680780893e-05 \\
233763998	4.49611354697566e-05 \\
234640965	4.10072659488003e-05 \\
235517039	3.74068816460515e-05 \\
236391980	3.41373603580108e-05 \\
237265776	3.11699377684715e-05 \\
238138881	2.84640499313982e-05 \\
239010933	2.59986153393066e-05 \\
239882211	2.37608390418198e-05 \\
240752916	2.17149659883509e-05 \\
241623080	1.98529079248888e-05 \\
242492828	1.81486513662343e-05 \\
243361881	1.66023774511026e-05 \\
244230016	1.51973655217463e-05 \\
245097590	1.39115223488084e-05 \\
245964650	1.27398209850416e-05 \\
246831127	1.16687672707672e-05 \\
247696942	1.06911797850828e-05 \\
248562366	9.79442938220387e-06 \\
249427268	8.97443373354401e-06 \\
250291710	8.22449466603148e-06 \\
251155579	7.53864367497581e-06 \\
252019124	6.91025440346937e-06 \\
252882487	6.33400321237154e-06 \\
253745753	5.8053829887772e-06 \\
254608784	5.32042084933071e-06 \\
255471500	4.87619484157431e-06 \\
256333945	4.47128516450901e-06 \\
257196113	4.10002939893461e-06 \\
258057972	3.75965417764856e-06 \\
258919488	3.44750961855933e-06 \\
259780575	3.16150255419423e-06 \\
260641467	2.89896512201127e-06 \\
261501859	2.65870525506529e-06 \\
262362023	2.43868918048395e-06 \\
263222123	2.23655672093592e-06 \\
264082223	2.05080486004316e-06 \\
264942323	1.88010155338647e-06 \\
265802387	1.72330178932478e-06 \\
266662349	1.57937200773173e-06 \\
267521957	1.44794872308784e-06 \\
268381246	1.3278234629599e-06 \\
269240360	1.21769551730821e-06 \\
270099454	1.11650343115333e-06 \\
270958287	1.02407215085698e-06 \\
271816991	9.39255871190703e-07 \\
272675591	8.61565852661172e-07 \\
273534103	7.90211709678079e-07 \\
274392569	7.24643119121016e-07 \\
275250728	6.64777473624589e-07 \\
276108808	6.09807695783893e-07 \\
276966572	5.59679305012217e-07 \\
277824172	5.13886361241944e-07 \\
278681772	4.7179294376809e-07 \\
279539263	4.33213250994324e-07 \\
280396635	3.97867592261125e-07 \\
281253935	3.65389825507378e-07 \\
282111142	3.35550131902451e-07 \\
282968342	3.08119922798333e-07 \\
283825452	2.82904468296152e-07 \\
284682534	2.59730204121844e-07 \\
285539469	2.38566742583046e-07 \\
286396366	2.19166354342715e-07 \\
287252993	2.01364136587223e-07 \\
288109593	1.85004301100911e-07 \\
288966130	1.6998955476577e-07 \\
289822607	1.56201220569141e-07 \\
290679007	1.43530227925126e-07 \\
291535312	1.31884193677045e-07 \\
292391512	1.21178212464379e-07 \\
293247687	1.11338682939888e-07 \\
294103692	1.02304212601378e-07 \\
294959692	9.3997072792007e-08 \\
295815692	8.63585447041793e-08 \\
296671623	7.93681862454321e-08 \\
297527523	7.2954425056615e-08 \\
298383423	6.70567888994e-08 \\
299239323	6.16336505165904e-08 \\
300095192	5.66484843478321e-08 \\
300950905	5.20866698106026e-08 \\
301806532	4.79032054023776e-08 \\
302662132	4.40591325645556e-08 \\
303517732	4.05241211698382e-08 \\
304373332	3.72732692488498e-08 \\
305228932	3.42836862787799e-08 \\
306084532	3.15343301471316e-08 \\
306940132	2.90058573826357e-08 \\
307795732	2.66804857651515e-08 \\
308651332	2.4541867926775e-08 \\
309506932	2.25749752225113e-08 \\
310362532	2.07659912043745e-08 \\
311218132	1.9102213477673e-08 \\
312073732	1.75719636064109e-08 \\
312929332	1.61645046237169e-08 \\
313784932	1.48699645374784e-08 \\
314640532	1.36792669969132e-08 \\
315496132	1.25840665665677e-08 \\
316351732	1.15766899400072e-08 \\
317207332	1.06500814833765e-08 \\
318062932	9.79775388598725e-09 \\
318918532	9.01374153094991e-09 \\
319774132	8.29255925038908e-09 \\
320629732	7.6291629125258e-09 \\
321485332	7.01891422760781e-09 \\
322340932	6.45754788530795e-09 \\
323196532	5.94114152319136e-09 \\
324052132	5.4660881931845e-09 \\
324907732	5.0290707709344e-09 \\
325763332	4.62703875214743e-09 \\
326618932	4.25718682528498e-09 \\
327474532	3.91693472101551e-09 \\
328330132	3.6039098927354e-09 \\
329185732	3.3159298085117e-09 \\
330041332	3.05098762920508e-09 \\
330896932	2.80723744250366e-09 \\
331752532	2.58298127331358e-09 \\
332608132	2.37665787050645e-09 \\
333463732	2.18683088304417e-09 \\
334319332	2.01217931206088e-09 \\
335174932	1.85148824050074e-09 \\
336030532	1.70363984031141e-09 \\
336886132	1.56760637803899e-09 \\
337741732	1.4424422767334e-09 \\
338597332	1.32727762114371e-09 \\
339452932	1.22131249558066e-09 \\
340308532	1.12381093320124e-09 \\
341164132	1.03409586449388e-09 \\
342019732	9.51544676386362e-10 \\
342875332	8.75584438286836e-10 \\
343730932	8.05688293858964e-10 \\
344586532	7.41371242174438e-10 \\
345442132	6.82187584200022e-10 \\
346297732	6.27726870483514e-10 \\
347153332	5.77611736218842e-10 \\
348008932	5.31494959155054e-10 \\
348864532	4.89057128127968e-10 \\
349720132	4.50004311591812e-10 \\
350575732	4.14066225751242e-10 \\
351431332	3.80994014115288e-10 \\
352286932	3.50558970740877e-10 \\
353142532	3.22550430809088e-10 \\
353998132	2.96774715913273e-10 \\
354853732	2.73053579746829e-10 \\
355709332	2.5122304236902e-10 \\
356564932	2.31132279981949e-10 \\
357420532	2.12642514707539e-10 \\
358276132	1.95625737831051e-10 \\
359131732	1.79964820823386e-10 \\
359987332	1.6555135040619e-10 \\
360842932	1.52285961618759e-10 \\
361698532	1.40077061061561e-10 \\
362554132	1.28840327295876e-10 \\
363409706	1.18498433288039e-10 \\
364265206	1.08980047208718e-10 \\
365120706	1.00219388343703e-10 \\
365976206	9.21560605604554e-11 \\
366831638	8.47346637300461e-11 \\
367687038	7.79038500375862e-11 \\
368542438	7.16165460268314e-11 \\
369397838	6.5829564022124e-11 \\
370253238	6.05029915057287e-11 \\
371108638	5.56001356066815e-11 \\
371964038	5.10873010561852e-11 \\
372819438	4.69334016095502e-11 \\
373674838	4.31098490238924e-11 \\
374530238	3.95904420358306e-11 \\
375385638	3.63507557388232e-11 \\
376241038	3.33687522058312e-11 \\
377096438	3.06237257774455e-11 \\
377951838	2.80969136845499e-11 \\
378807238	2.57710519591114e-11 \\
379662638	2.36299868561218e-11 \\
380518038	2.1659063431656e-11 \\
381373438	1.98447369648136e-11 \\
382228838	1.817462846887e-11 \\
383084238	1.66371916243691e-11 \\
383939638	1.52218793125769e-11 \\
384795038	1.39189215708768e-11 \\
385650438	1.27196586596767e-11 \\
386505838	1.16154308393845e-11 \\
387361238	1.0598966149189e-11 \\
388216638	9.66321467288367e-12 \\
389072038	8.80179262807701e-12 \\
389927438	8.00876032158726e-12 \\
390782838	7.27867766059376e-12 \\
391638238	6.60654864148569e-12 \\
392493638	5.98776583871086e-12 \\
393349038	5.41811040477569e-12 \\
394204438	4.89364104794276e-12 \\
395059838	4.41080505453328e-12 \\
395915238	3.96621624432214e-12 \\
396770638	3.55698803744531e-12 \\
397626038	3.18023385403876e-12 \\
398481438	2.83334466999463e-12 \\
399336838	2.51393350580997e-12 \\
400192238	2.219890937738e-12 \\
401047638	1.94916305318316e-12 \\
401903038	1.69991798415481e-12 \\
402758438	1.47037937381356e-12 \\
403613838	1.25910393222739e-12 \\
404469238	1.06459285831306e-12 \\
405324638	8.85402862138562e-13 \\
406180038	7.20479231830495e-13 \\
407035438	5.68656233213005e-13 \\
407890838	4.28879154412698e-13 \\
408746238	3.00148794707411e-13 \\
409601638	1.81576975677444e-13 \\
410457038	7.24420523567915e-14 \\
411312438	-2.80331313717852e-14 \\
};
\end{axis}

\end{tikzpicture}
}
\end{center}
\caption{$\ell_1$-regularized logistic regression \eqref{eq:logl1}
}
\label{fig:rcv1}
\end{figure}

% ==========================================================================
\subsubsection{Comparison with \sega}\label{sec:num:sega}


In Figure~\ref{fig:rcv1_l12}, we compare \adaalgo~algorithm with \sega~algorithm featuring coordinate sketches\;\cite{hanzely2018sega}. While the focus of \sega~is not to produce an efficient coordinate descent method but rather to use sketched gradients, \sega~and \algo~are similar algorithmically and reach similar rates (see Section \ref{sec:comparison}). As mentioned in \cite[Apx.\;G2]{hanzely2018sega}, \sega~is slightly slower than plain randomized proximal coordinate descent (10\% \algo) but still competitive, which corresponds to our experiments. Thanks to the use of identification, \adaalgo~shows a clear improvement over other methods in terms of efficiency with respect to the number of subspaces explored.

\begin{figure}[H]
\begin{center}
\scalebox{.85}{% This file was created by matplotlib2tikz v0.6.18.
\begin{tikzpicture}


\begin{axis}[
legend cell align={left},
legend columns=1,
legend entries={{\pgd},{10\% \sega},{10\% \algo},{10\% \adaalgo},},
legend style={at={(0.8,0.99)}, anchor=north,},
tick align=outside,
tick pos=left,
xlabel={Iteration},
xmajorgrids,
xmin=0, xmax=5000,
ylabel={Iterate sparsity},
ymajorgrids,
ymin=0, ymax=100
]


\addlegendimage{ black,thick,mark=square*,mark repeat = 100}
\addlegendimage{ green!50!black,mark=text, text mark={s},mark repeat = 100}
\addlegendimage{ blue,mark=*,mark repeat = 100}
\addlegendimage{ red,mark=diamond*,mark repeat = 100}



\addplot [black,thick,mark=square*,mark repeat = 100]
table [row sep=\\]{%
0	100 \\
1	100 \\
2	100 \\
3	100 \\
4	100 \\
5	100 \\
6	100 \\
7	100 \\
8	100 \\
9	100 \\
10	100 \\
11	100 \\
12	100 \\
13	100 \\
14	100 \\
15	100 \\
16	100 \\
17	100 \\
18	100 \\
19	100 \\
20	100 \\
21	100 \\
22	100 \\
23	100 \\
24	100 \\
25	100 \\
26	100 \\
27	100 \\
28	100 \\
29	100 \\
30	100 \\
31	100 \\
32	100 \\
33	100 \\
34	100 \\
35	100 \\
36	100 \\
37	100 \\
38	100 \\
39	100 \\
40	100 \\
41	100 \\
42	100 \\
43	100 \\
44	100 \\
45	100 \\
46	100 \\
47	100 \\
48	100 \\
49	100 \\
50	100 \\
51	100 \\
52	100 \\
53	100 \\
54	100 \\
55	100 \\
56	100 \\
57	100 \\
58	100 \\
59	100 \\
60	100 \\
61	100 \\
62	100 \\
63	100 \\
64	100 \\
65	100 \\
66	100 \\
67	100 \\
68	100 \\
69	100 \\
70	100 \\
71	100 \\
72	100 \\
73	100 \\
74	100 \\
75	100 \\
76	100 \\
77	100 \\
78	100 \\
79	100 \\
80	100 \\
81	100 \\
82	100 \\
83	100 \\
84	100 \\
85	100 \\
86	100 \\
87	100 \\
88	100 \\
89	100 \\
90	100 \\
91	100 \\
92	100 \\
93	100 \\
94	100 \\
95	99.0007621305784 \\
96	98.0015242611567 \\
97	95.0038106528919 \\
98	85.93445677026 \\
99	78.9397916843086 \\
100	67.9481751206707 \\
101	55.9573206876111 \\
102	49.9618934710814 \\
103	45.9649419933949 \\
104	42.96722838513 \\
105	38.9702769074435 \\
106	35.9725632991786 \\
107	31.9756118214921 \\
108	29.9771360826488 \\
109	26.9794224743839 \\
110	25.9801846049623 \\
111	23.9817088661191 \\
112	21.9832331272758 \\
113	21.9832331272758 \\
114	20.9839952578542 \\
115	19.9847573884325 \\
116	17.9862816495893 \\
117	16.9870437801677 \\
118	16.9870437801677 \\
119	16.9870437801677 \\
120	16.9870437801677 \\
121	15.987805910746 \\
122	15.987805910746 \\
123	14.9885680413244 \\
124	11.9908544330595 \\
125	11.9908544330595 \\
126	11.9908544330595 \\
127	10.9916165636379 \\
128	10.9916165636379 \\
129	10.9916165636379 \\
130	10.9916165636379 \\
131	8.99314082479465 \\
132	6.99466508595139 \\
133	6.99466508595139 \\
134	6.99466508595139 \\
135	5.99542721652977 \\
136	4.99618934710814 \\
137	4.99618934710814 \\
138	4.99618934710814 \\
139	4.99618934710814 \\
140	4.99618934710814 \\
141	4.99618934710814 \\
142	3.99695147768651 \\
143	3.99695147768651 \\
144	3.99695147768651 \\
145	3.99695147768651 \\
146	3.99695147768651 \\
147	3.99695147768651 \\
148	3.99695147768651 \\
149	3.99695147768651 \\
150	3.99695147768651 \\
151	3.99695147768651 \\
152	3.99695147768651 \\
153	3.99695147768651 \\
154	3.99695147768651 \\
155	3.99695147768651 \\
156	3.99695147768651 \\
157	3.99695147768651 \\
158	3.99695147768651 \\
159	3.99695147768651 \\
160	3.99695147768651 \\
161	3.99695147768651 \\
162	3.99695147768651 \\
163	3.99695147768651 \\
164	3.99695147768651 \\
165	3.99695147768651 \\
166	3.99695147768651 \\
167	3.99695147768651 \\
168	3.99695147768651 \\
169	3.99695147768651 \\
170	3.99695147768651 \\
171	3.99695147768651 \\
172	3.99695147768651 \\
173	3.99695147768651 \\
174	3.99695147768651 \\
175	3.99695147768651 \\
176	3.99695147768651 \\
177	3.99695147768651 \\
178	3.99695147768651 \\
179	3.99695147768651 \\
180	3.99695147768651 \\
181	3.99695147768651 \\
182	3.99695147768651 \\
183	3.99695147768651 \\
184	3.99695147768651 \\
185	3.99695147768651 \\
186	3.99695147768651 \\
187	3.99695147768651 \\
188	3.99695147768651 \\
189	3.99695147768651 \\
190	3.99695147768651 \\
191	3.99695147768651 \\
192	3.99695147768651 \\
193	3.99695147768651 \\
194	3.99695147768651 \\
195	3.99695147768651 \\
196	3.99695147768651 \\
197	3.99695147768651 \\
198	3.99695147768651 \\
199	3.99695147768651 \\
200	3.99695147768651 \\
201	3.99695147768651 \\
202	3.99695147768651 \\
203	3.99695147768651 \\
204	3.99695147768651 \\
205	3.99695147768651 \\
206	3.99695147768651 \\
207	3.99695147768651 \\
208	3.99695147768651 \\
209	3.99695147768651 \\
210	3.99695147768651 \\
211	3.99695147768651 \\
212	3.99695147768651 \\
213	3.99695147768651 \\
214	3.99695147768651 \\
215	3.99695147768651 \\
216	3.99695147768651 \\
217	3.99695147768651 \\
218	3.99695147768651 \\
219	3.99695147768651 \\
220	3.99695147768651 \\
221	3.99695147768651 \\
222	3.99695147768651 \\
223	3.99695147768651 \\
224	3.99695147768651 \\
225	3.99695147768651 \\
226	3.99695147768651 \\
227	3.99695147768651 \\
228	3.99695147768651 \\
229	3.99695147768651 \\
230	3.99695147768651 \\
231	3.99695147768651 \\
232	3.99695147768651 \\
233	3.99695147768651 \\
234	3.99695147768651 \\
235	3.99695147768651 \\
236	3.99695147768651 \\
237	3.99695147768651 \\
238	3.99695147768651 \\
239	3.99695147768651 \\
240	3.99695147768651 \\
241	3.99695147768651 \\
242	3.99695147768651 \\
243	3.99695147768651 \\
244	3.99695147768651 \\
245	3.99695147768651 \\
246	3.99695147768651 \\
247	3.99695147768651 \\
248	3.99695147768651 \\
249	3.99695147768651 \\
250	3.99695147768651 \\
251	3.99695147768651 \\
252	3.99695147768651 \\
253	3.99695147768651 \\
254	3.99695147768651 \\
255	3.99695147768651 \\
256	3.99695147768651 \\
257	3.99695147768651 \\
258	3.99695147768651 \\
259	3.99695147768651 \\
260	3.99695147768651 \\
261	3.99695147768651 \\
262	3.99695147768651 \\
263	3.99695147768651 \\
264	3.99695147768651 \\
265	3.99695147768651 \\
266	3.99695147768651 \\
267	3.99695147768651 \\
268	3.99695147768651 \\
269	3.99695147768651 \\
270	3.99695147768651 \\
271	3.99695147768651 \\
272	3.99695147768651 \\
273	3.99695147768651 \\
274	3.99695147768651 \\
275	3.99695147768651 \\
276	3.99695147768651 \\
277	3.99695147768651 \\
278	3.99695147768651 \\
279	3.99695147768651 \\
280	3.99695147768651 \\
281	3.99695147768651 \\
282	3.99695147768651 \\
283	3.99695147768651 \\
284	3.99695147768651 \\
285	3.99695147768651 \\
286	3.99695147768651 \\
287	3.99695147768651 \\
288	3.99695147768651 \\
289	3.99695147768651 \\
290	3.99695147768651 \\
291	3.99695147768651 \\
292	3.99695147768651 \\
293	3.99695147768651 \\
294	3.99695147768651 \\
295	3.99695147768651 \\
296	3.99695147768651 \\
297	3.99695147768651 \\
298	3.99695147768651 \\
299	3.99695147768651 \\
};





\addplot [green!50!black,mark=text, text mark={s},mark repeat = 100]
table [row sep=\\]{%
0	100 \\
20	100 \\
40	100 \\
60	100 \\
80	100 \\
100	100 \\
120	100 \\
140	100 \\
160	100 \\
180	100 \\
200	100 \\
220	100 \\
240	100 \\
260	100 \\
280	100 \\
300	100 \\
320	100 \\
340	100 \\
360	100 \\
380	100 \\
400	100 \\
420	100 \\
440	100 \\
460	100 \\
480	100 \\
500	100 \\
520	100 \\
540	100 \\
560	100 \\
580	100 \\
600	100 \\
620	100 \\
640	100 \\
660	100 \\
680	100 \\
700	100 \\
720	100 \\
740	100 \\
760	100 \\
780	100 \\
800	100 \\
820	100 \\
840	100 \\
860	100 \\
880	100 \\
900	100 \\
920	100 \\
940	100 \\
960	100 \\
980	100 \\
1000	100 \\
1020	100 \\
1040	100 \\
1060	100 \\
1080	100 \\
1100	100 \\
1120	100 \\
1140	100 \\
1160	100 \\
1180	100 \\
1200	100 \\
1220	100 \\
1240	100 \\
1260	100 \\
1280	100 \\
1300	100 \\
1320	100 \\
1340	100 \\
1360	100 \\
1380	100 \\
1400	100 \\
1420	100 \\
1440	100 \\
1460	100 \\
1480	100 \\
1500	100 \\
1520	100 \\
1540	100 \\
1560	100 \\
1580	100 \\
1600	100 \\
1620	100 \\
1640	100 \\
1660	100 \\
1680	100 \\
1700	100 \\
1720	100 \\
1740	100 \\
1760	100 \\
1780	100 \\
1800	100 \\
1820	100 \\
1840	100 \\
1860	100 \\
1880	100 \\
1900	100 \\
1920	100 \\
1940	100 \\
1960	100 \\
1980	100 \\
2000	100 \\
2020	100 \\
2040	100 \\
2060	100 \\
2080	100 \\
2100	100 \\
2120	100 \\
2140	100 \\
2160	100 \\
2180	100 \\
2200	100 \\
2220	100 \\
2240	100 \\
2260	100 \\
2280	100 \\
2300	100 \\
2320	97.0022863917351 \\
2340	96.0030485223135 \\
2360	91.0068591752054 \\
2380	83.0129562198323 \\
2400	78.0167668727242 \\
2420	70.0228639173512 \\
2440	64.0274367008214 \\
2460	61.0297230925565 \\
2480	56.0335337454484 \\
2500	47.9634177322381 \\
2520	41.9679905157084 \\
2540	36.9718011686002 \\
2560	33.9740875603353 \\
2580	29.9771360826488 \\
2600	27.9786603438056 \\
2620	26.9794224743839 \\
2640	26.9794224743839 \\
2660	26.9794224743839 \\
2680	24.9809467355407 \\
2700	22.9824709966974 \\
2720	20.9839952578542 \\
2740	20.9839952578542 \\
2760	19.9847573884325 \\
2780	18.9855195190109 \\
2800	17.9862816495893 \\
2820	16.9870437801677 \\
2840	14.9885680413244 \\
2860	13.9893301719028 \\
2880	11.9908544330595 \\
2900	11.9908544330595 \\
2920	11.9908544330595 \\
2940	11.9908544330595 \\
2960	10.9916165636379 \\
2980	10.9916165636379 \\
3000	9.99237869421627 \\
3020	8.99314082479465 \\
3040	7.99390295537302 \\
3060	7.99390295537302 \\
3080	7.99390295537302 \\
3100	5.99542721652977 \\
3120	5.99542721652977 \\
3140	5.99542721652977 \\
3160	5.99542721652977 \\
3180	5.99542721652977 \\
3200	5.99542721652977 \\
3220	5.99542721652977 \\
3240	5.99542721652977 \\
3260	5.99542721652977 \\
3280	5.99542721652977 \\
3300	4.99618934710814 \\
3320	4.99618934710814 \\
3340	4.99618934710814 \\
3360	3.99695147768651 \\
3380	3.99695147768651 \\
3400	3.99695147768651 \\
3420	3.99695147768651 \\
3440	3.99695147768651 \\
3460	3.99695147768651 \\
3480	3.99695147768651 \\
3500	3.99695147768651 \\
3520	3.99695147768651 \\
3540	3.99695147768651 \\
3560	3.99695147768651 \\
3580	3.99695147768651 \\
3600	3.99695147768651 \\
3620	3.99695147768651 \\
3640	3.99695147768651 \\
3660	3.99695147768651 \\
3680	3.99695147768651 \\
3700	3.99695147768651 \\
3720	3.99695147768651 \\
3740	3.99695147768651 \\
3760	3.99695147768651 \\
3780	3.99695147768651 \\
3800	3.99695147768651 \\
3820	3.99695147768651 \\
3840	3.99695147768651 \\
3860	3.99695147768651 \\
3880	3.99695147768651 \\
3900	3.99695147768651 \\
3920	3.99695147768651 \\
3940	3.99695147768651 \\
3960	3.99695147768651 \\
3980	3.99695147768651 \\
4000	3.99695147768651 \\
4020	3.99695147768651 \\
4040	3.99695147768651 \\
4060	3.99695147768651 \\
4080	3.99695147768651 \\
4100	3.99695147768651 \\
4120	3.99695147768651 \\
4140	3.99695147768651 \\
4160	3.99695147768651 \\
4180	3.99695147768651 \\
4200	3.99695147768651 \\
4220	3.99695147768651 \\
4240	3.99695147768651 \\
4260	3.99695147768651 \\
4280	3.99695147768651 \\
4300	3.99695147768651 \\
4320	3.99695147768651 \\
4340	3.99695147768651 \\
4360	3.99695147768651 \\
4380	3.99695147768651 \\
4400	3.99695147768651 \\
4420	3.99695147768651 \\
4440	3.99695147768651 \\
4460	3.99695147768651 \\
4480	3.99695147768651 \\
4500	3.99695147768651 \\
4520	3.99695147768651 \\
4540	3.99695147768651 \\
4560	3.99695147768651 \\
4580	3.99695147768651 \\
4600	3.99695147768651 \\
4620	3.99695147768651 \\
4640	3.99695147768651 \\
4660	3.99695147768651 \\
4680	3.99695147768651 \\
4700	3.99695147768651 \\
4720	3.99695147768651 \\
4740	3.99695147768651 \\
4760	3.99695147768651 \\
4780	3.99695147768651 \\
4800	3.99695147768651 \\
4820	3.99695147768651 \\
4840	3.99695147768651 \\
4860	3.99695147768651 \\
4880	3.99695147768651 \\
4900	3.99695147768651 \\
4920	3.99695147768651 \\
4940	3.99695147768651 \\
4960	3.99695147768651 \\
4980	3.99695147768651 \\
5000	3.99695147768651 \\
5020	3.99695147768651 \\
5040	3.99695147768651 \\
5060	3.99695147768651 \\
5080	3.99695147768651 \\
5100	3.99695147768651 \\
5120	3.99695147768651 \\
5140	3.99695147768651 \\
5160	3.99695147768651 \\
5180	3.99695147768651 \\
5200	3.99695147768651 \\
5220	3.99695147768651 \\
5240	3.99695147768651 \\
5260	3.99695147768651 \\
5280	3.99695147768651 \\
5300	3.99695147768651 \\
5320	3.99695147768651 \\
5340	3.99695147768651 \\
5360	3.99695147768651 \\
5380	3.99695147768651 \\
5400	3.99695147768651 \\
5420	3.99695147768651 \\
5440	3.99695147768651 \\
5460	3.99695147768651 \\
5480	3.99695147768651 \\
5500	3.99695147768651 \\
5520	3.99695147768651 \\
5540	3.99695147768651 \\
5560	3.99695147768651 \\
5580	3.99695147768651 \\
5600	3.99695147768651 \\
5620	3.99695147768651 \\
5640	3.99695147768651 \\
5660	3.99695147768651 \\
5680	3.99695147768651 \\
5700	3.99695147768651 \\
5720	3.99695147768651 \\
5740	3.99695147768651 \\
5760	3.99695147768651 \\
5780	3.99695147768651 \\
5800	3.99695147768651 \\
5820	3.99695147768651 \\
5840	3.99695147768651 \\
5860	3.99695147768651 \\
5880	3.99695147768651 \\
5900	3.99695147768651 \\
5920	3.99695147768651 \\
5940	3.99695147768651 \\
5960	3.99695147768651 \\
5980	3.99695147768651 \\
};



\addplot [blue,mark=*,mark repeat = 100]
table [row sep=\\]{%
0	100 \\
20	100 \\
40	100 \\
60	100 \\
80	100 \\
100	100 \\
120	100 \\
140	100 \\
160	100 \\
180	100 \\
200	100 \\
220	100 \\
240	100 \\
260	100 \\
280	100 \\
300	100 \\
320	100 \\
340	100 \\
360	100 \\
380	100 \\
400	100 \\
420	100 \\
440	100 \\
460	100 \\
480	100 \\
500	100 \\
520	100 \\
540	100 \\
560	100 \\
580	100 \\
600	100 \\
620	100 \\
640	100 \\
660	100 \\
680	100 \\
700	100 \\
720	100 \\
740	100 \\
760	100 \\
780	100 \\
800	100 \\
820	100 \\
840	100 \\
860	100 \\
880	100 \\
900	100 \\
920	100 \\
940	100 \\
960	100 \\
980	100 \\
1000	94.9275975950546 \\
1020	87.9329325091032 \\
1040	72.9443644677788 \\
1060	49.9618934710814 \\
1080	43.9664662545516 \\
1100	38.9702769074435 \\
1120	29.9771360826488 \\
1140	24.9809467355407 \\
1160	22.9824709966974 \\
1180	17.9862816495893 \\
1200	17.9862816495893 \\
1220	16.9870437801677 \\
1240	14.9885680413244 \\
1260	14.9885680413244 \\
1280	13.9893301719028 \\
1300	11.9908544330595 \\
1320	11.9908544330595 \\
1340	10.9916165636379 \\
1360	9.99237869421627 \\
1380	8.99314082479465 \\
1400	6.99466508595139 \\
1420	6.99466508595139 \\
1440	6.99466508595139 \\
1460	4.99618934710814 \\
1480	3.99695147768651 \\
1500	3.99695147768651 \\
1520	3.99695147768651 \\
1540	3.99695147768651 \\
1560	3.99695147768651 \\
1580	3.99695147768651 \\
1600	3.99695147768651 \\
1620	3.99695147768651 \\
1640	3.99695147768651 \\
1660	3.99695147768651 \\
1680	3.99695147768651 \\
1700	3.99695147768651 \\
1720	3.99695147768651 \\
1740	3.99695147768651 \\
1760	3.99695147768651 \\
1780	3.99695147768651 \\
1800	3.99695147768651 \\
1820	3.99695147768651 \\
1840	3.99695147768651 \\
1860	3.99695147768651 \\
1880	3.99695147768651 \\
1900	3.99695147768651 \\
1920	3.99695147768651 \\
1940	3.99695147768651 \\
1960	3.99695147768651 \\
1980	3.99695147768651 \\
2000	3.99695147768651 \\
2020	3.99695147768651 \\
2040	3.99695147768651 \\
2060	3.99695147768651 \\
2080	3.99695147768651 \\
2100	3.99695147768651 \\
2120	3.99695147768651 \\
2140	3.99695147768651 \\
2160	3.99695147768651 \\
2180	3.99695147768651 \\
2200	3.99695147768651 \\
2220	3.99695147768651 \\
2240	3.99695147768651 \\
2260	3.99695147768651 \\
2280	3.99695147768651 \\
2300	3.99695147768651 \\
2320	3.99695147768651 \\
2340	3.99695147768651 \\
2360	3.99695147768651 \\
2380	3.99695147768651 \\
2400	3.99695147768651 \\
2420	3.99695147768651 \\
2440	3.99695147768651 \\
2460	3.99695147768651 \\
2480	3.99695147768651 \\
2500	3.99695147768651 \\
2520	3.99695147768651 \\
2540	3.99695147768651 \\
2560	3.99695147768651 \\
2580	3.99695147768651 \\
2600	3.99695147768651 \\
2620	3.99695147768651 \\
2640	3.99695147768651 \\
2660	3.99695147768651 \\
2680	3.99695147768651 \\
2700	3.99695147768651 \\
2720	3.99695147768651 \\
2740	3.99695147768651 \\
2760	3.99695147768651 \\
2780	3.99695147768651 \\
2800	3.99695147768651 \\
2820	3.99695147768651 \\
2840	3.99695147768651 \\
2860	3.99695147768651 \\
2880	3.99695147768651 \\
2900	3.99695147768651 \\
2920	3.99695147768651 \\
2940	3.99695147768651 \\
2960	3.99695147768651 \\
2980	3.99695147768651 \\
3000	3.99695147768651 \\
3020	3.99695147768651 \\
3040	3.99695147768651 \\
3060	3.99695147768651 \\
3080	3.99695147768651 \\
3100	3.99695147768651 \\
3120	3.99695147768651 \\
3140	3.99695147768651 \\
3160	3.99695147768651 \\
3180	3.99695147768651 \\
3200	3.99695147768651 \\
3220	3.99695147768651 \\
3240	3.99695147768651 \\
3260	3.99695147768651 \\
3280	3.99695147768651 \\
3300	3.99695147768651 \\
3320	3.99695147768651 \\
3340	3.99695147768651 \\
3360	3.99695147768651 \\
3380	3.99695147768651 \\
3400	3.99695147768651 \\
3420	3.99695147768651 \\
3440	3.99695147768651 \\
3460	3.99695147768651 \\
3480	3.99695147768651 \\
3500	3.99695147768651 \\
3520	3.99695147768651 \\
3540	3.99695147768651 \\
3560	3.99695147768651 \\
3580	3.99695147768651 \\
3600	3.99695147768651 \\
3620	3.99695147768651 \\
3640	3.99695147768651 \\
3660	3.99695147768651 \\
3680	3.99695147768651 \\
3700	3.99695147768651 \\
3720	3.99695147768651 \\
3740	3.99695147768651 \\
3760	3.99695147768651 \\
3780	3.99695147768651 \\
3800	3.99695147768651 \\
3820	3.99695147768651 \\
3840	3.99695147768651 \\
3860	3.99695147768651 \\
3880	3.99695147768651 \\
3900	3.99695147768651 \\
3920	3.99695147768651 \\
3940	3.99695147768651 \\
3960	3.99695147768651 \\
3980	3.99695147768651 \\
4000	3.99695147768651 \\
4020	3.99695147768651 \\
4040	3.99695147768651 \\
4060	3.99695147768651 \\
4080	3.99695147768651 \\
4100	3.99695147768651 \\
4120	3.99695147768651 \\
4140	3.99695147768651 \\
4160	3.99695147768651 \\
4180	3.99695147768651 \\
4200	3.99695147768651 \\
4220	3.99695147768651 \\
4240	3.99695147768651 \\
4260	3.99695147768651 \\
4280	3.99695147768651 \\
4300	3.99695147768651 \\
4320	3.99695147768651 \\
4340	3.99695147768651 \\
4360	3.99695147768651 \\
4380	3.99695147768651 \\
4400	3.99695147768651 \\
4420	3.99695147768651 \\
4440	3.99695147768651 \\
4460	3.99695147768651 \\
4480	3.99695147768651 \\
4500	3.99695147768651 \\
4520	3.99695147768651 \\
4540	3.99695147768651 \\
4560	3.99695147768651 \\
4580	3.99695147768651 \\
4600	3.99695147768651 \\
4620	3.99695147768651 \\
4640	3.99695147768651 \\
4660	3.99695147768651 \\
4680	3.99695147768651 \\
4700	3.99695147768651 \\
4720	3.99695147768651 \\
4740	3.99695147768651 \\
4760	3.99695147768651 \\
4780	3.99695147768651 \\
4800	3.99695147768651 \\
4820	3.99695147768651 \\
4840	3.99695147768651 \\
4860	3.99695147768651 \\
4880	3.99695147768651 \\
4900	3.99695147768651 \\
4920	3.99695147768651 \\
4940	3.99695147768651 \\
4960	3.99695147768651 \\
4980	3.99695147768651 \\
5000	3.99695147768651 \\
5020	3.99695147768651 \\
5040	3.99695147768651 \\
5060	3.99695147768651 \\
5080	3.99695147768651 \\
5100	3.99695147768651 \\
5120	3.99695147768651 \\
5140	3.99695147768651 \\
5160	3.99695147768651 \\
5180	3.99695147768651 \\
5200	3.99695147768651 \\
5220	3.99695147768651 \\
5240	3.99695147768651 \\
5260	3.99695147768651 \\
5280	3.99695147768651 \\
5300	3.99695147768651 \\
5320	3.99695147768651 \\
5340	3.99695147768651 \\
5360	3.99695147768651 \\
5380	3.99695147768651 \\
5400	3.99695147768651 \\
5420	3.99695147768651 \\
5440	3.99695147768651 \\
5460	3.99695147768651 \\
5480	3.99695147768651 \\
5500	3.99695147768651 \\
5520	3.99695147768651 \\
5540	3.99695147768651 \\
5560	3.99695147768651 \\
5580	3.99695147768651 \\
5600	3.99695147768651 \\
5620	3.99695147768651 \\
5640	3.99695147768651 \\
5660	3.99695147768651 \\
5680	3.99695147768651 \\
5700	3.99695147768651 \\
5720	3.99695147768651 \\
5740	3.99695147768651 \\
5760	3.99695147768651 \\
5780	3.99695147768651 \\
5800	3.99695147768651 \\
5820	3.99695147768651 \\
5840	3.99695147768651 \\
5860	3.99695147768651 \\
5880	3.99695147768651 \\
5900	3.99695147768651 \\
5920	3.99695147768651 \\
5940	3.99695147768651 \\
5960	3.99695147768651 \\
5980	3.99695147768651 \\
};


\addplot [red,mark=diamond*,mark repeat = 100]
table [row sep=\\]{%
0	100 \\
1	100 \\
2	100 \\
3	100 \\
4	100 \\
5	100 \\
6	100 \\
7	100 \\
8	100 \\
9	100 \\
10	100 \\
11	100 \\
12	100 \\
13	100 \\
14	100 \\
15	100 \\
16	100 \\
17	100 \\
18	100 \\
19	100 \\
20	100 \\
21	100 \\
22	100 \\
23	100 \\
24	100 \\
25	100 \\
26	100 \\
27	100 \\
28	100 \\
29	100 \\
30	100 \\
31	100 \\
32	100 \\
33	100 \\
34	100 \\
35	100 \\
36	100 \\
37	100 \\
38	100 \\
39	100 \\
40	100 \\
41	100 \\
42	100 \\
43	100 \\
44	100 \\
45	100 \\
46	100 \\
47	100 \\
48	100 \\
49	100 \\
50	100 \\
51	100 \\
52	100 \\
53	100 \\
54	100 \\
55	100 \\
56	100 \\
57	100 \\
58	100 \\
59	100 \\
60	100 \\
61	100 \\
62	100 \\
63	100 \\
64	100 \\
65	100 \\
66	100 \\
67	100 \\
68	100 \\
69	100 \\
70	100 \\
71	100 \\
72	100 \\
73	100 \\
74	100 \\
75	100 \\
76	100 \\
77	100 \\
78	100 \\
79	100 \\
80	100 \\
81	100 \\
82	100 \\
83	100 \\
84	100 \\
85	100 \\
86	100 \\
87	100 \\
88	100 \\
89	100 \\
90	100 \\
91	100 \\
92	100 \\
93	100 \\
94	100 \\
95	100 \\
96	98.0015242611567 \\
97	95.0038106528919 \\
98	89.0083834363621 \\
99	80.0152426115674 \\
100	68.0243881785079 \\
101	57.9557964264544 \\
102	48.9626556016597 \\
103	44.9657041239732 \\
104	41.9679905157084 \\
105	39.9695147768651 \\
106	34.973325429757 \\
107	30.9763739520705 \\
108	27.9786603438056 \\
109	26.9794224743839 \\
110	25.9801846049623 \\
111	23.9817088661191 \\
112	22.9824709966974 \\
113	20.9839952578542 \\
114	18.9855195190109 \\
115	17.9862816495893 \\
116	17.9862816495893 \\
117	16.9870437801677 \\
118	16.9870437801677 \\
119	16.9870437801677 \\
120	16.9870437801677 \\
121	14.9885680413244 \\
122	12.9900923024812 \\
123	12.9900923024812 \\
124	12.9900923024812 \\
125	11.9908544330595 \\
126	10.9916165636379 \\
127	10.9916165636379 \\
128	10.9916165636379 \\
129	9.99237869421627 \\
130	7.99390295537302 \\
131	6.99466508595139 \\
132	5.99542721652977 \\
133	5.99542721652977 \\
134	4.99618934710814 \\
135	4.99618934710814 \\
136	4.99618934710814 \\
137	4.99618934710814 \\
138	4.99618934710814 \\
139	4.99618934710814 \\
140	4.99618934710814 \\
141	4.99618934710814 \\
142	4.99618934710814 \\
143	3.99695147768651 \\
144	3.99695147768651 \\
145	3.99695147768651 \\
146	3.99695147768651 \\
147	3.99695147768651 \\
148	3.99695147768651 \\
149	3.99695147768651 \\
150	3.99695147768651 \\
151	3.99695147768651 \\
152	3.99695147768651 \\
153	3.99695147768651 \\
154	3.99695147768651 \\
155	3.99695147768651 \\
156	3.99695147768651 \\
157	3.99695147768651 \\
158	3.99695147768651 \\
159	3.99695147768651 \\
160	3.99695147768651 \\
161	3.99695147768651 \\
162	3.99695147768651 \\
163	3.99695147768651 \\
164	3.99695147768651 \\
165	3.99695147768651 \\
166	3.99695147768651 \\
167	3.99695147768651 \\
168	3.99695147768651 \\
169	3.99695147768651 \\
170	3.99695147768651 \\
171	3.99695147768651 \\
172	3.99695147768651 \\
173	3.99695147768651 \\
174	3.99695147768651 \\
175	3.99695147768651 \\
176	3.99695147768651 \\
177	3.99695147768651 \\
178	3.99695147768651 \\
179	3.99695147768651 \\
180	3.99695147768651 \\
181	3.99695147768651 \\
182	3.99695147768651 \\
183	3.99695147768651 \\
184	3.99695147768651 \\
185	3.99695147768651 \\
186	3.99695147768651 \\
187	3.99695147768651 \\
188	3.99695147768651 \\
189	3.99695147768651 \\
190	3.99695147768651 \\
191	3.99695147768651 \\
192	3.99695147768651 \\
193	3.99695147768651 \\
194	3.99695147768651 \\
195	3.99695147768651 \\
196	3.99695147768651 \\
197	3.99695147768651 \\
198	3.99695147768651 \\
199	3.99695147768651 \\
200	3.99695147768651 \\
201	3.99695147768651 \\
202	3.99695147768651 \\
203	3.99695147768651 \\
204	3.99695147768651 \\
205	3.99695147768651 \\
206	3.99695147768651 \\
207	3.99695147768651 \\
208	3.99695147768651 \\
209	3.99695147768651 \\
210	3.99695147768651 \\
211	3.99695147768651 \\
212	3.99695147768651 \\
213	3.99695147768651 \\
214	3.99695147768651 \\
215	3.99695147768651 \\
216	3.99695147768651 \\
217	3.99695147768651 \\
218	3.99695147768651 \\
219	3.99695147768651 \\
220	3.99695147768651 \\
221	3.99695147768651 \\
222	3.99695147768651 \\
223	3.99695147768651 \\
224	3.99695147768651 \\
225	3.99695147768651 \\
226	3.99695147768651 \\
227	3.99695147768651 \\
228	3.99695147768651 \\
229	3.99695147768651 \\
230	3.99695147768651 \\
231	3.99695147768651 \\
232	3.99695147768651 \\
233	3.99695147768651 \\
234	3.99695147768651 \\
235	3.99695147768651 \\
236	3.99695147768651 \\
237	3.99695147768651 \\
238	3.99695147768651 \\
239	3.99695147768651 \\
240	3.99695147768651 \\
241	3.99695147768651 \\
242	3.99695147768651 \\
243	3.99695147768651 \\
244	3.99695147768651 \\
245	3.99695147768651 \\
246	3.99695147768651 \\
247	3.99695147768651 \\
248	3.99695147768651 \\
249	3.99695147768651 \\
250	3.99695147768651 \\
251	3.99695147768651 \\
252	3.99695147768651 \\
253	3.99695147768651 \\
254	3.99695147768651 \\
255	3.99695147768651 \\
256	3.99695147768651 \\
257	3.99695147768651 \\
258	3.99695147768651 \\
259	3.99695147768651 \\
260	3.99695147768651 \\
261	3.99695147768651 \\
262	3.99695147768651 \\
263	3.99695147768651 \\
264	3.99695147768651 \\
265	3.99695147768651 \\
266	3.99695147768651 \\
267	3.99695147768651 \\
268	3.99695147768651 \\
269	3.99695147768651 \\
270	3.99695147768651 \\
271	3.99695147768651 \\
272	3.99695147768651 \\
273	3.99695147768651 \\
274	3.99695147768651 \\
275	3.99695147768651 \\
276	3.99695147768651 \\
277	3.99695147768651 \\
278	3.99695147768651 \\
279	3.99695147768651 \\
280	3.99695147768651 \\
281	3.99695147768651 \\
282	3.99695147768651 \\
283	3.99695147768651 \\
284	3.99695147768651 \\
285	3.99695147768651 \\
286	3.99695147768651 \\
287	3.99695147768651 \\
288	3.99695147768651 \\
289	3.99695147768651 \\
290	3.99695147768651 \\
291	3.99695147768651 \\
292	3.99695147768651 \\
293	3.99695147768651 \\
294	3.99695147768651 \\
295	3.99695147768651 \\
296	3.99695147768651 \\
297	3.99695147768651 \\
298	3.99695147768651 \\
299	3.99695147768651 \\
};
\end{axis}

\end{tikzpicture}
}
\scalebox{.85}{% This file was created by matplotlib2tikz v0.6.18.
\begin{tikzpicture}

\definecolor{color0}{rgb}{0.75,0.75,0}

\begin{axis}[
legend cell align={left},
legend columns=1,
legend entries={{\pgd},{10\% \sega}, {10\% \algo},{10\% \adaalgo}},
legend style={at={(0.8,0.99)}, anchor=north},
tick align=outside,
tick pos=left,
xlabel={Iteration},
xmajorgrids,
xmin=0, xmax=5000,
ylabel={Suboptimality},
ymajorgrids,
ymin=1e-4, ymax=18087.4434074586,
ymode=log
]


\addlegendimage{ black,thick,mark=square*,mark repeat = 100}
\addlegendimage{ green!50!black,mark=text, text mark={s},mark repeat = 100}
\addlegendimage{ blue,mark=*,mark repeat = 100}
\addlegendimage{ red,mark=diamond*,mark repeat = 100}



\addplot [black,thick,mark=square*,mark repeat = 100]
table [row sep=\\]{%
0	2310.8803612917 \\
1	2211.53318678313 \\
2	2116.65039725421 \\
3	2026.01187598724 \\
4	1939.40962430536 \\
5	1856.64702015789 \\
6	1777.53812600643 \\
7	1701.90704254618 \\
8	1629.58730504891 \\
9	1560.42131934677 \\
10	1494.25983469264 \\
11	1430.9614509322 \\
12	1370.39215760844 \\
13	1312.42490279097 \\
14	1256.93918958126 \\
15	1203.82069839242 \\
16	1152.96093323829 \\
17	1104.25689039335 \\
18	1057.61074790175 \\
19	1012.92957452251 \\
20	970.125056798296 \\
21	929.113243028751 \\
22	889.814303015186 \\
23	852.15230252408 \\
24	816.054991490722 \\
25	781.453605053275 \\
26	748.282676571378 \\
27	716.479861842569 \\
28	685.985773784739 \\
29	656.743826903731 \\
30	628.700090912491 \\
31	601.803152911997 \\
32	576.003987584914 \\
33	551.25583489069 \\
34	527.514084785866 \\
35	504.736168525981 \\
36	482.881456135643 \\
37	461.911159661492 \\
38	441.788241848833 \\
39	422.477329907002 \\
40	403.944634051064 \\
41	386.157870528399 \\
42	369.086188858226 \\
43	352.700103030225 \\
44	336.971426425298 \\
45	321.873210237167 \\
46	307.379685188138 \\
47	293.466206345921 \\
48	280.109200861057 \\
49	267.286118456288 \\
50	254.975384510163 \\
51	243.15635558739 \\
52	231.809277277979 \\
53	220.91524421606 \\
54	210.456162157514 \\
55	200.414712003258 \\
56	190.774315662158 \\
57	181.519103654173 \\
58	172.633884360556 \\
59	164.104114833606 \\
60	155.915873083806 \\
61	148.055831767071 \\
62	140.51123319932 \\
63	133.26986562974 \\
64	126.320040707852 \\
65	119.650572082935 \\
66	113.250755077383 \\
67	107.110347378315 \\
68	101.219550694119 \\
69	95.5689933246376 \\
70	90.1497135953951 \\
71	84.9531441076054 \\
72	79.9710967567007 \\
73	75.1957484727789 \\
74	70.6196276366722 \\
75	66.2356011253113 \\
76	62.036861939687 \\
77	58.0169173680178 \\
78	54.1695776357466 \\
79	50.4889449927625 \\
80	46.969403186885 \\
81	43.6056072713442 \\
82	40.39247369306 \\
83	37.325170608543 \\
84	34.3991083761672 \\
85	31.6099301790976 \\
86	28.9535027453273 \\
87	26.4259071557736 \\
88	24.0234297794359 \\
89	21.7425534699767 \\
90	19.5799493548577 \\
91	17.532469981661 \\
92	15.5971456393623 \\
93	13.7711886405867 \\
94	12.0520210772522 \\
95	10.4400761524391 \\
96	8.96017474839175 \\
97	7.61085893994912 \\
98	6.43224660463946 \\
99	5.44383218682478 \\
100	4.63875735309595 \\
101	4.01337350124374 \\
102	3.52913804368414 \\
103	3.11657343608689 \\
104	2.75007312349651 \\
105	2.4331107269804 \\
106	2.16461237309359 \\
107	1.93165711540296 \\
108	1.73346084554685 \\
109	1.55791526019514 \\
110	1.40772694409368 \\
111	1.27332723221149 \\
112	1.15758605841933 \\
113	1.05255162119959 \\
114	0.954973050174627 \\
115	0.864500127317723 \\
116	0.789428900928137 \\
117	0.719718983909603 \\
118	0.657534272772214 \\
119	0.597997129818025 \\
120	0.540989152071345 \\
121	0.489500255804279 \\
122	0.440695487395648 \\
123	0.39495442290653 \\
124	0.35718693921726 \\
125	0.326277251715657 \\
126	0.296655047648732 \\
127	0.269953497704607 \\
128	0.245376784979378 \\
129	0.221826509510787 \\
130	0.199259458288364 \\
131	0.182021312264225 \\
132	0.166707873646684 \\
133	0.15502627018349 \\
134	0.143826184150629 \\
135	0.134513849664251 \\
136	0.126413575199693 \\
137	0.11997371016397 \\
138	0.113798830034974 \\
139	0.107878003964861 \\
140	0.102200757813249 \\
141	0.0967570536822766 \\
142	0.0918692427830381 \\
143	0.0881038354629582 \\
144	0.0844926121091831 \\
145	0.0810292536842177 \\
146	0.0777077010948872 \\
147	0.0745221442604294 \\
148	0.0714670117027119 \\
149	0.0685369606176665 \\
150	0.0657268673937743 \\
151	0.0630318185486523 \\
152	0.0604471020587547 \\
153	0.0579681990603356 \\
154	0.0555907759022056 \\
155	0.0533106765327651 \\
156	0.0511239152053307 \\
157	0.0490266694870312 \\
158	0.0470152735576028 \\
159	0.0450862117852999 \\
160	0.0432361125678948 \\
161	0.0414617424274248 \\
162	0.0397600003479139 \\
163	0.0381279123458638 \\
164	0.0365626262637682 \\
165	0.0350614067773749 \\
166	0.0336216306078341 \\
167	0.0322407819302515 \\
168	0.0309164479705488 \\
169	0.0296463147828745 \\
170	0.0284281632001289 \\
171	0.0272598649504956 \\
172	0.0261393789331579 \\
173	0.025064747646663 \\
174	0.0240340937636754 \\
175	0.0230456168461113 \\
176	0.0220975901948979 \\
177	0.0211883578288372 \\
178	0.0203163315872841 \\
179	0.0194799883515575 \\
180	0.018677867380227 \\
181	0.0179085677535959 \\
182	0.0171707459229133 \\
183	0.0164631133600235 \\
184	0.0157844343033271 \\
185	0.0151335235961172 \\
186	0.0145092446134965 \\
187	0.0139105072742525 \\
188	0.013336266134198 \\
189	0.012785518557648 \\
190	0.0122573029638267 \\
191	0.0117506971451309 \\
192	0.0112648166543092 \\
193	0.0107988132577281 \\
194	0.0103518734520205 \\
195	0.00992321704151222 \\
196	0.00951209577394607 \\
197	0.00911779203209995 \\
198	0.00873961757902331 \\
199	0.00837691235467863 \\
200	0.0080290433218948 \\
201	0.00769540335960328 \\
202	0.00737541020142118 \\
203	0.0070685054177213 \\
204	0.00677415343941057 \\
205	0.00649184062170338 \\
206	0.00622107434625074 \\
207	0.00596138216005859 \\
208	0.00571231094968128 \\
209	0.00547342614924262 \\
210	0.00524431098091083 \\
211	0.00502456572648313 \\
212	0.00481380702880974 \\
213	0.00461166722183948 \\
214	0.00441779368810691 \\
215	0.00423184824254053 \\
216	0.00405350654151104 \\
217	0.00388245751608918 \\
218	0.00371840282851466 \\
219	0.00356105635093273 \\
220	0.00341014366548009 \\
221	0.00326540158484478 \\
222	0.00312657769247027 \\
223	0.00299342990158458 \\
224	0.00286572603229596 \\
225	0.00274324340600662 \\
226	0.00262576845643792 \\
227	0.00251309635659025 \\
228	0.00240503066097397 \\
229	0.0023013829624976 \\
230	0.00220197256340793 \\
231	0.00210662615970136 \\
232	0.00201517753846447 \\
233	0.00192746728760707 \\
234	0.0018433425174802 \\
235	0.00176265659389596 \\
236	0.00168526888207987 \\
237	0.00161104450110661 \\
238	0.00153985408839308 \\
239	0.00147157357383354 \\
240	0.00140608396318331 \\
241	0.00134327113031141 \\
242	0.00128302561796145 \\
243	0.00122524244666855 \\
244	0.00116982093149876 \\
245	0.00111666450629633 \\
246	0.0010656805551208 \\
247	0.00101678025058938 \\
248	0.00096987839883611 \\
249	0.000924893290819906 \\
250	0.000881746559717067 \\
251	0.000840363044155579 \\
252	0.00080067065704581 \\
253	0.000762600259781143 \\
254	0.00072608554158915 \\
255	0.000691062903823259 \\
256	0.000657471348985528 \\
257	0.000625252374300445 \\
258	0.000594349869635025 \\
259	0.000564710019608006 \\
260	0.000536281209698064 \\
261	0.000509013936198066 \\
262	0.000482860719855926 \\
263	0.000457776023047973 \\
264	0.000433716170344489 \\
265	0.000410639272324431 \\
266	0.000388505152510321 \\
267	0.000367275277290524 \\
268	0.000346912688713896 \\
269	0.000327381940031124 \\
270	0.000308649033875064 \\
271	0.000290681362972389 \\
272	0.000273447653280412 \\
273	0.000256917909448706 \\
274	0.000241063362519611 \\
275	0.000225856419758586 \\
276	0.000211270616547576 \\
277	0.000197280570240155 \\
278	0.000183861935908469 \\
279	0.000170991363898532 \\
280	0.000158646459124778 \\
281	0.000146805742027301 \\
282	0.000135448611130018 \\
283	0.000124555307127627 \\
284	0.000114106878444042 \\
285	0.000104085148198152 \\
286	9.44726825233833e-05 \\
287	8.52527601815645e-05 \\
288	7.64093434211244e-05 \\
289	6.79270500265616e-05 \\
290	5.97911265116657e-05 \\
291	5.19874224107486e-05 \\
292	4.45023656188148e-05 \\
293	3.73229387440333e-05 \\
294	3.04366564229941e-05 \\
295	2.38315435703296e-05 \\
296	1.74961145114061e-05 \\
297	1.14193529725526e-05 \\
298	5.59069288974534e-06 \\
299	0 \\
};





\addplot [ green!50!black,mark=text, text mark={s},mark repeat = 100]
table [row sep=\\]{%
0	5118.19810677661 \\
20	5156.79094809061 \\
40	4988.23788149984 \\
60	4786.98719582755 \\
80	4579.00032913794 \\
100	4375.90803128651 \\
120	4182.82943050911 \\
140	3999.04823488744 \\
160	3824.05550823249 \\
180	3657.25948489222 \\
200	3498.30395249591 \\
220	3346.75400815268 \\
240	3202.18029256636 \\
260	3064.28907979919 \\
280	2932.71561566201 \\
300	2807.15142380639 \\
320	2687.29388505692 \\
340	2572.86588199057 \\
360	2463.52267383409 \\
380	2359.08428842239 \\
400	2259.23652810634 \\
420	2163.92887717618 \\
440	2072.71150374388 \\
460	1985.54365156167 \\
480	1902.17976308624 \\
500	1822.45363009979 \\
520	1746.1829139727 \\
540	1673.21105612787 \\
560	1603.37762876826 \\
580	1536.55692605971 \\
600	1472.57952743264 \\
620	1411.31164729256 \\
640	1352.63753538713 \\
660	1296.41492464636 \\
680	1242.61293528437 \\
700	1191.02119534394 \\
720	1141.58445536406 \\
740	1094.19805184971 \\
760	1048.74240087473 \\
780	1005.24462941131 \\
800	963.534319354642 \\
820	923.531032149195 \\
840	885.158482372068 \\
860	848.380650761963 \\
880	813.07471042326 \\
900	779.213168550373 \\
920	746.738780641375 \\
940	715.571389762593 \\
960	685.659371755516 \\
980	656.959127616953 \\
1000	629.417925860482 \\
1020	602.982650743085 \\
1040	577.622057311306 \\
1060	553.250646700286 \\
1080	529.853366619237 \\
1100	507.403329528555 \\
1120	485.819053019103 \\
1140	465.122511404107 \\
1160	445.235878734245 \\
1180	426.113314819017 \\
1200	407.769938411664 \\
1220	390.161964815785 \\
1240	373.255077575855 \\
1260	357.010721188903 \\
1280	341.398431874376 \\
1300	326.396364706156 \\
1320	311.992395323858 \\
1340	298.152689547975 \\
1360	284.848854963539 \\
1380	272.071594692461 \\
1400	259.805579975242 \\
1420	248.012599983664 \\
1440	236.68051802111 \\
1460	225.796974591652 \\
1480	215.340084517457 \\
1500	205.28916657388 \\
1520	195.636341155293 \\
1540	186.361485249588 \\
1560	177.459531783695 \\
1580	168.902456367874 \\
1600	160.68609826467 \\
1620	152.789526921417 \\
1640	145.20840216103 \\
1660	137.920566376422 \\
1680	130.922434409231 \\
1700	124.200538626428 \\
1720	117.742959778987 \\
1740	111.547082829434 \\
1760	105.594661910847 \\
1780	99.8814607096787 \\
1800	94.4021128259466 \\
1820	89.1415178218626 \\
1840	84.0936357337545 \\
1860	79.2508069559346 \\
1880	74.6047891852429 \\
1900	70.1526901138013 \\
1920	65.8834554627503 \\
1940	61.7901400193989 \\
1960	57.8684420702027 \\
1980	54.114102177453 \\
2000	50.5221852980118 \\
2020	47.0848232037644 \\
2040	43.7987971199032 \\
2060	40.655053078342 \\
2080	37.6512730110352 \\
2100	34.7852851873841 \\
2120	32.0513881213278 \\
2140	29.4471103123797 \\
2160	26.9653954942391 \\
2180	24.6060341734179 \\
2200	22.364670652853 \\
2220	20.2390763387535 \\
2240	18.2227739897265 \\
2260	16.3154549606001 \\
2280	14.5123116907002 \\
2300	12.8134437983709 \\
2320	11.2350016318027 \\
2340	9.7776311508194 \\
2360	8.46199827169052 \\
2380	7.3207179386756 \\
2400	6.32632500690619 \\
2420	5.45431763618444 \\
2440	4.71867886411573 \\
2460	4.07680756870057 \\
2480	3.51450141583708 \\
2500	3.07177673464935 \\
2520	2.7004550884522 \\
2540	2.39208786366211 \\
2560	2.13063262867491 \\
2580	1.90838344358468 \\
2600	1.71922933228157 \\
2620	1.55255664427784 \\
2640	1.39739475068902 \\
2660	1.24970982163907 \\
2680	1.11672946278222 \\
2700	1.00284737708698 \\
2720	0.907336437169514 \\
2740	0.818395207742907 \\
2760	0.737082148039421 \\
2780	0.662586190817373 \\
2800	0.595957965875252 \\
2820	0.53664919997138 \\
2840	0.484821327888823 \\
2860	0.440944986716957 \\
2880	0.405090430250839 \\
2900	0.372763443758392 \\
2920	0.341601168917772 \\
2940	0.311746125706362 \\
2960	0.2849371853371 \\
2980	0.259874031879505 \\
3000	0.236818470108087 \\
3020	0.217756662614517 \\
3040	0.201757479208176 \\
3060	0.186904854084361 \\
3080	0.172600698944909 \\
3100	0.160147761556666 \\
3120	0.150634973579162 \\
3140	0.141474200815684 \\
3160	0.132695608749722 \\
3180	0.12429991584608 \\
3200	0.116211113120471 \\
3220	0.108489662008521 \\
3240	0.10110013599439 \\
3260	0.0939538391942314 \\
3280	0.0871353248941693 \\
3300	0.0805923972751155 \\
3320	0.0755588363875679 \\
3340	0.0707076835570497 \\
3360	0.0674145444884093 \\
3380	0.0647059100194678 \\
3400	0.0620954658108008 \\
3420	0.0595926076942417 \\
3440	0.057196602547809 \\
3460	0.0548966807350333 \\
3480	0.0526976650228161 \\
3500	0.0505704736504 \\
3520	0.0485424057911354 \\
3540	0.0465833892352194 \\
3560	0.0447075948120834 \\
3580	0.0429220321786721 \\
3600	0.0411953433609458 \\
3620	0.0395437712866613 \\
3640	0.0379577676926885 \\
3660	0.0364177311983838 \\
3680	0.0349489360780864 \\
3700	0.0335241500701446 \\
3720	0.0321695377380911 \\
3740	0.0308671811975914 \\
3760	0.0296064644838849 \\
3780	0.0284125804220408 \\
3800	0.0272632407703477 \\
3820	0.0261515163074892 \\
3840	0.0251024222411202 \\
3860	0.0240835881735644 \\
3880	0.0231094359554018 \\
3900	0.0221798927846222 \\
3920	0.0212865654246199 \\
3940	0.0204278600017438 \\
3960	0.0196059140537748 \\
3980	0.0188087089856461 \\
4000	0.0180546645492852 \\
4020	0.0173224790169617 \\
4040	0.0166115263671669 \\
4060	0.0159264679208859 \\
4080	0.0152851666590823 \\
4100	0.0146583087568934 \\
4120	0.0140643917707004 \\
4140	0.0134929303058295 \\
4160	0.0129467541947341 \\
4180	0.0124251765177725 \\
4200	0.011920445136679 \\
4220	0.0114345203452002 \\
4240	0.0109696243839335 \\
4260	0.0105250950541489 \\
4280	0.0100972728939217 \\
4300	0.00968374499380564 \\
4320	0.00928789346096082 \\
4340	0.0089100058343432 \\
4360	0.00854672354243924 \\
4380	0.00819866663376523 \\
4400	0.00786183347244473 \\
4420	0.00754059073256541 \\
4440	0.0072342830215717 \\
4460	0.00693979856879534 \\
4480	0.00665608527417039 \\
4500	0.00638464808292438 \\
4520	0.00612369763452891 \\
4540	0.00587242423557344 \\
4560	0.00562813638429005 \\
4580	0.00539488153203083 \\
4600	0.00517417287869582 \\
4620	0.00496168824862897 \\
4640	0.00475715504217722 \\
4660	0.00456066956647527 \\
4680	0.00437320511396599 \\
4700	0.00419203453511585 \\
4720	0.00401931303156267 \\
4740	0.00385276008358515 \\
4760	0.00369354536336797 \\
4780	0.00354094512814385 \\
4800	0.00339339934251237 \\
4820	0.00325071987671821 \\
4840	0.00311548607001999 \\
4860	0.00298553004172941 \\
4880	0.00285997180962738 \\
4900	0.0027391757048898 \\
4920	0.00262381212843188 \\
4940	0.00251357338848157 \\
4960	0.00240648552478984 \\
4980	0.00230478630642228 \\
5000	0.00220627365118697 \\
5020	0.0021115676570187 \\
5040	0.00202139016458336 \\
5060	0.00193523724449141 \\
5080	0.00185256864048622 \\
5100	0.00177278657322666 \\
5120	0.0016961412812817 \\
5140	0.00162274519097605 \\
5160	0.00155161867797804 \\
5180	0.00148398242802839 \\
5200	0.00141912631192387 \\
5220	0.00135587450323094 \\
5240	0.00129628839762552 \\
5260	0.00123915331607227 \\
5280	0.00118419825388028 \\
5300	0.00113119789871718 \\
5320	0.00108049245340891 \\
5340	0.00103190361943861 \\
5360	0.000985195055735577 \\
5380	0.000940283477031567 \\
5400	0.000897223793660862 \\
5420	0.000856221675901381 \\
5440	0.000816697965103863 \\
5460	0.000778665954801028 \\
5480	0.000742286451011154 \\
5500	0.000707324283729216 \\
5520	0.000673773857202908 \\
5540	0.00064137139949505 \\
5560	0.000610403051375608 \\
5580	0.00058049972476959 \\
5600	0.000551816445459874 \\
5620	0.000524216466932792 \\
5640	0.0004980202743603 \\
5660	0.000472677531387244 \\
5680	0.000448467282707199 \\
5700	0.000425161760051473 \\
5720	0.000402773885388896 \\
5740	0.000381117534484776 \\
5760	0.000360293562763569 \\
5780	0.000340746730422836 \\
5800	0.000321765641868588 \\
5820	0.000303539445225542 \\
5840	0.000286038588972026 \\
5860	0.000269367777260276 \\
5880	0.000253320926741463 \\
5900	0.000237818848667359 \\
5920	0.000223104856724055 \\
5940	0.000208903001279293 \\
5960	0.000195115944018109 \\
5980	0.000182007849647503 \\
};




\addplot [blue,mark=*,mark repeat = 100]
table [row sep=\\]{%
0	2215.46769112275 \\
20	2037.32560309681 \\
40	1871.53172955521 \\
60	1719.86780163509 \\
80	1580.97313434325 \\
100	1454.02619401535 \\
120	1336.99261082007 \\
140	1230.0816734375 \\
160	1132.00382770868 \\
180	1041.58816862701 \\
200	958.080965617693 \\
220	880.890694921656 \\
240	810.180934882506 \\
260	745.544286193405 \\
280	685.629080841174 \\
300	629.812094785211 \\
320	579.190768174642 \\
340	532.130896138681 \\
360	489.172782838581 \\
380	448.970823143715 \\
400	411.994605737835 \\
420	377.693202602656 \\
440	345.763442788726 \\
460	316.643011844905 \\
480	289.844089082958 \\
500	265.024422361409 \\
520	242.081677210233 \\
540	220.714376589643 \\
560	200.935199737049 \\
580	182.728935360078 \\
600	165.862181521112 \\
620	150.471039392253 \\
640	136.120063528095 \\
660	122.828796576557 \\
680	110.528900101399 \\
700	99.0263817536842 \\
720	88.6853140177956 \\
740	79.0551655778828 \\
760	70.2097547980831 \\
780	62.0004196830372 \\
800	54.5104341964193 \\
820	47.5936765603387 \\
840	41.2011267944248 \\
860	35.4688343431486 \\
880	30.2767660866124 \\
900	25.5511202754026 \\
920	21.2319613734782 \\
940	17.3278512600208 \\
960	13.9199451050115 \\
980	10.859732434644 \\
1000	8.17313254415244 \\
1020	5.94325956108226 \\
1040	4.35963591572909 \\
1060	3.3421330045379 \\
1080	2.57613510754713 \\
1100	2.03820735660192 \\
1120	1.64367561622815 \\
1140	1.36156093612851 \\
1160	1.11936054194711 \\
1180	0.920783269891287 \\
1200	0.777564894147422 \\
1220	0.674774956391493 \\
1240	0.564576060032792 \\
1260	0.476257560869569 \\
1280	0.398959161599076 \\
1300	0.330402582279344 \\
1320	0.285865379536206 \\
1340	0.230223380572313 \\
1360	0.193699030181417 \\
1380	0.156101035183382 \\
1400	0.123808682362502 \\
1420	0.112532301128868 \\
1440	0.0919036840094851 \\
1460	0.0801469015283101 \\
1480	0.0744979564892159 \\
1500	0.0659229973911128 \\
1520	0.0593549896667656 \\
1540	0.0539383296603342 \\
1560	0.0505767264184604 \\
1580	0.0454427378330433 \\
1600	0.0411626747204714 \\
1620	0.0381080701952439 \\
1640	0.0341606443218394 \\
1660	0.0310249481552582 \\
1680	0.0292879934831141 \\
1700	0.0261010160525519 \\
1720	0.0240787807597438 \\
1740	0.0222265333849654 \\
1760	0.0201760694256783 \\
1780	0.018910463818731 \\
1800	0.0179198263212341 \\
1820	0.0170788544349327 \\
1840	0.0160438698251628 \\
1860	0.0151730231472071 \\
1880	0.0139130223494337 \\
1900	0.0127487341802963 \\
1920	0.0116338357363084 \\
1940	0.0110614089184784 \\
1960	0.0105866605553218 \\
1980	0.0101781086193464 \\
2000	0.00949870566529998 \\
2020	0.0086610859496361 \\
2040	0.00817496420412445 \\
2060	0.00755391092746205 \\
2080	0.0069215091406778 \\
2100	0.00604755883339236 \\
2120	0.00557701444636094 \\
2140	0.00500930751814233 \\
2160	0.00443851511086879 \\
2180	0.00410317158312656 \\
2200	0.00353827209036295 \\
2220	0.00333287020226747 \\
2240	0.00304215475018976 \\
2260	0.00279974788854487 \\
2280	0.00257876450404226 \\
2300	0.00245956091685384 \\
2320	0.0022133374003992 \\
2340	0.00203646817651526 \\
2360	0.00187730869269265 \\
2380	0.00175532041033755 \\
2400	0.00166920396008985 \\
2420	0.00150131302121692 \\
2440	0.00136255943398278 \\
2460	0.00123096287088176 \\
2480	0.001100204640887 \\
2500	0.00101623481152457 \\
2520	0.000949441637570603 \\
2540	0.000884432583142525 \\
2560	0.000812510916769149 \\
2580	0.000742472866251731 \\
2600	0.000677155607360591 \\
2620	0.000629517082185505 \\
2640	0.000591175604756522 \\
2660	0.000541115506787815 \\
2680	0.000485105868365254 \\
2700	0.00045045411086897 \\
2720	0.000416273480553508 \\
2740	0.000372095084320456 \\
2760	0.000334533445461638 \\
2780	0.0003030299428326 \\
2800	0.000260348643132069 \\
2820	0.000205180484260215 \\
2840	0.00017823551181495 \\
2860	0.000155740123426007 \\
2880	0.000134748697486531 \\
2900	0.000118303867667979 \\
2920	9.94370597839911e-05 \\
2940	9.29016066277821e-05 \\
2960	7.79697129249524e-05 \\
2980	5.74764181768828e-05 \\
3000	4.08328144851922e-05 \\
3020	2.64600165156281e-05 \\
3040	1.14609381742792e-05 \\
3060	5.08335999960252e-06 \\
3080	-7.46151959241104e-06 \\
3100	-1.62555974296819e-05 \\
3120	-2.428763716944e-05 \\
3140	-3.05905163024978e-05 \\
3160	-4.0448479383004e-05 \\
3180	-4.84575076118432e-05 \\
3200	-5.46238686962575e-05 \\
3220	-6.14516177717128e-05 \\
3240	-6.57563403123618e-05 \\
3260	-7.04176673247403e-05 \\
3280	-7.52610921714858e-05 \\
3300	-7.97950795738966e-05 \\
3320	-8.52381957967552e-05 \\
3340	-8.77465387929544e-05 \\
3360	-9.16434927653587e-05 \\
3380	-9.51337436920774e-05 \\
3400	-9.73324485245008e-05 \\
3420	-0.000100852120405559 \\
3440	-0.000104069143702423 \\
3460	-0.000105408343014979 \\
3480	-0.000107404486987761 \\
3500	-0.000108747075628246 \\
3520	-0.000109888051108786 \\
3540	-0.000111686225726659 \\
3560	-0.000112856931558714 \\
3580	-0.000114610466437792 \\
3600	-0.000115808470956935 \\
3620	-0.000117399967864751 \\
3640	-0.000118875312976385 \\
3660	-0.000119867969355925 \\
3680	-0.000120942099396393 \\
3700	-0.000121592089585842 \\
3720	-0.000122310606665899 \\
3740	-0.000123241807776786 \\
3760	-0.000123786380473945 \\
3780	-0.000124471386827807 \\
3800	-0.000125289914478577 \\
3820	-0.000125779770783652 \\
3840	-0.000126129219248927 \\
3860	-0.000126483222937512 \\
3880	-0.000126747901387114 \\
3900	-0.000127176899718862 \\
3920	-0.000127632966547608 \\
3940	-0.000127930903118134 \\
3960	-0.000128196758494292 \\
3980	-0.00012844135499912 \\
4000	-0.000128671393262225 \\
4020	-0.000128971600738659 \\
4040	-0.000129119051188287 \\
4060	-0.000129298619535145 \\
4080	-0.000129532742533023 \\
4100	-0.000129686324016332 \\
4120	-0.000129784222598239 \\
4140	-0.000129843054706757 \\
4160	-0.000129994736016492 \\
4180	-0.000130098253828592 \\
4200	-0.000130199090270677 \\
4220	-0.000130337754423726 \\
4240	-0.000130451485293115 \\
4260	-0.000130503190100661 \\
4280	-0.000130577868739934 \\
4300	-0.000130645328820567 \\
4320	-0.000130712106684694 \\
4340	-0.00013074938673352 \\
4360	-0.000130818125866572 \\
4380	-0.000130854139438252 \\
4400	-0.000130904364110673 \\
4420	-0.000130936181654162 \\
4440	-0.000130976199283817 \\
4460	-0.000131027241370596 \\
4480	-0.000131058692006469 \\
4500	-0.000131099130946666 \\
4520	-0.000131121177700422 \\
4540	-0.00013114345024956 \\
4560	-0.000131167040977154 \\
4580	-0.000131186694467766 \\
4600	-0.000131204589046918 \\
4620	-0.000131226199857393 \\
4640	-0.000131243259162916 \\
4660	-0.00013126479936254 \\
4680	-0.000131274840880824 \\
4700	-0.000131285462410258 \\
4720	-0.000131304193822457 \\
4740	-0.000131314130968008 \\
4760	-0.000131323639352754 \\
4780	-0.000131333251127241 \\
4800	-0.00013133869121118 \\
4820	-0.000131345493680213 \\
4840	-0.00013135004975906 \\
4860	-0.000131356808398042 \\
4880	-0.000131362678915004 \\
4900	-0.000131368025171552 \\
4920	-0.000131372311478639 \\
4940	-0.000131374087628089 \\
4960	-0.00013137644225214 \\
4980	-0.000131378952843209 \\
5000	-0.000131382926427559 \\
5020	-0.000131386146071888 \\
5040	-0.000131388671373411 \\
5060	-0.000131391475943765 \\
5080	-0.000131392768053296 \\
5100	-0.000131395193757156 \\
5120	-0.000131397333228644 \\
5140	-0.000131398643900216 \\
5160	-0.000131399784067954 \\
5180	-0.000131401461318958 \\
5200	-0.000131402476356124 \\
5220	-0.000131403616031811 \\
5240	-0.000131404192929452 \\
5260	-0.000131404544881919 \\
5280	-0.000131405423160924 \\
5300	-0.000131407006322082 \\
5320	-0.000131407680529216 \\
5340	-0.000131408015970225 \\
5360	-0.000131408518285525 \\
5380	-0.000131409081909117 \\
5400	-0.000131409254442882 \\
5420	-0.000131409713992836 \\
5440	-0.000131410108713759 \\
5460	-0.000131410469435211 \\
5480	-0.000131411069090426 \\
5500	-0.000131411429115769 \\
5520	-0.000131411683051308 \\
5540	-0.000131411997066344 \\
5560	-0.000131412152264865 \\
5580	-0.000131412414276832 \\
5600	-0.000131412618254778 \\
5620	-0.000131412801197328 \\
5640	-0.000131412862325098 \\
5660	-0.000131412965585387 \\
5680	-0.000131413058624297 \\
5700	-0.000131413116695622 \\
5720	-0.000131413233589228 \\
5740	-0.000131413402971736 \\
5760	-0.000131413461915475 \\
5780	-0.000131413581587747 \\
5800	-0.000131413732362695 \\
5820	-0.000131413779161704 \\
5840	-0.000131413899852273 \\
5860	-0.000131413944675973 \\
5880	-0.000131414075740466 \\
5900	-0.000131414141408159 \\
5920	-0.000131414170485566 \\
5940	-0.000131414199224356 \\
5960	-0.000131414249647577 \\
5980	-0.000131414324923584 \\
};





\addplot [red,mark=diamond*,mark repeat = 100]
table [row sep=\\]{%
0	2300.19291388783 \\
1	2201.32505971244 \\
2	2106.89795102146 \\
3	2016.69277113798 \\
4	1930.50274206657 \\
5	1848.13238835034 \\
6	1769.39684985688 \\
7	1694.12124005497 \\
8	1622.14004659352 \\
9	1553.29657122546 \\
10	1487.44240633372 \\
11	1424.43694551452 \\
12	1364.14692585735 \\
13	1306.44599973096 \\
14	1251.21433404237 \\
15	1198.33823508217 \\
16	1147.70979720457 \\
17	1099.22657371601 \\
18	1052.79126846244 \\
19	1008.311446713 \\
20	965.699264037489 \\
21	924.871211967653 \\
22	885.74787931785 \\
23	848.253728120298 \\
24	812.316883203574 \\
25	777.868934511467 \\
26	744.844751322552 \\
27	713.182307589627 \\
28	682.822517672612 \\
29	653.709081789065 \\
30	625.788340553373 \\
31	599.009138019179 \\
32	573.322692680007 \\
33	548.682475920535 \\
34	525.044097445784 \\
35	502.365197247801 \\
36	480.605343699445 \\
37	459.725937392764 \\
38	439.690120365379 \\
39	420.462690382331 \\
40	402.010019963277 \\
41	384.299979865678 \\
42	367.301866754001 \\
43	350.986334802934 \\
44	335.325330999366 \\
45	320.292033923432 \\
46	305.860795803446 \\
47	292.007087653034 \\
48	278.707447311312 \\
49	265.939430218688 \\
50	253.681562771743 \\
51	241.913298110785 \\
52	230.614974203138 \\
53	219.767774094008 \\
54	209.353688204972 \\
55	199.355478567754 \\
56	189.756644888053 \\
57	180.541392340743 \\
58	171.69460100393 \\
59	163.201796844939 \\
60	155.049124176593 \\
61	147.223319506902 \\
62	139.711686709708 \\
63	132.502073447806 \\
64	125.582848783701 \\
65	118.942881916318 \\
66	112.571521984843 \\
67	106.45857888323 \\
68	100.594305030907 \\
69	94.969378046764 \\
70	89.574884274564 \\
71	84.4023031085797 \\
72	79.4434920683459 \\
73	74.690672571072 \\
74	70.1364163493668 \\
75	65.7736324605588 \\
76	61.5955548320614 \\
77	57.5957302850215 \\
78	53.7680069760221 \\
79	50.1065231941022 \\
80	46.6056964481226 \\
81	43.2602127779974 \\
82	40.0650162231841 \\
83	37.0152983839567 \\
84	34.106488016596 \\
85	31.3342406144279 \\
86	28.6944279450966 \\
87	26.1831275445655 \\
88	23.7966122171377 \\
89	21.5313396722713 \\
90	19.3839425734867 \\
91	17.3512195538147 \\
92	15.430128350294 \\
93	13.6177836590535 \\
94	11.9114664357797 \\
95	10.3086664828268 \\
96	8.81706545792113 \\
97	7.45740842672546 \\
98	6.2788523707335 \\
99	5.2641539147017 \\
100	4.45784746309847 \\
101	3.83275209709886 \\
102	3.35308726217701 \\
103	2.94483842422316 \\
104	2.59075045514111 \\
105	2.28337744159368 \\
106	2.0168530343678 \\
107	1.80597429629088 \\
108	1.62209218127343 \\
109	1.46252456297161 \\
110	1.31776898357308 \\
111	1.19340780651337 \\
112	1.07857350791643 \\
113	0.974553393591858 \\
114	0.886951233613247 \\
115	0.812088236022116 \\
116	0.741387341231388 \\
117	0.674156699923656 \\
118	0.613841742867664 \\
119	0.556099377788208 \\
120	0.500814342237803 \\
121	0.450022711058062 \\
122	0.406587781660225 \\
123	0.37058040724784 \\
124	0.336086235777852 \\
125	0.303431865461464 \\
126	0.276775488889247 \\
127	0.251885118404891 \\
128	0.228036229158716 \\
129	0.20659244448976 \\
130	0.18980192129818 \\
131	0.175567697613862 \\
132	0.165188992685961 \\
133	0.155393564122782 \\
134	0.146561050260203 \\
135	0.139289346229385 \\
136	0.132317093093035 \\
137	0.125631903087486 \\
138	0.119221914032722 \\
139	0.113075763820598 \\
140	0.107182566886773 \\
141	0.101531892372706 \\
142	0.0961137437395931 \\
143	0.0919567807760056 \\
144	0.0881864020184207 \\
145	0.0845704647093735 \\
146	0.0811026374611046 \\
147	0.0777768493316384 \\
148	0.0745872788887074 \\
149	0.0715283437919481 \\
150	0.0685946908534825 \\
151	0.0657811865435234 \\
152	0.0630829079126565 \\
153	0.0604951339063109 \\
154	0.0580133370499114 \\
155	0.055633175485551 \\
156	0.0533504853428817 \\
157	0.0511612734284053 \\
158	0.0490617102185777 \\
159	0.0470481231431457 \\
160	0.0451169901459996 \\
161	0.0432649335115776 \\
162	0.0414887139454909 \\
163	0.0397852248986439 \\
164	0.0381514871246278 \\
165	0.0365846434606651 \\
166	0.0350819538228144 \\
167	0.0336407904065674 \\
168	0.0322586330843548 \\
169	0.0309330649918387 \\
170	0.029661768295226 \\
171	0.0284425201321521 \\
172	0.0272731887190081 \\
173	0.0261517296178715 \\
174	0.0250761821564907 \\
175	0.0240446659950493 \\
176	0.0230553778336757 \\
177	0.0221065882549396 \\
178	0.0211966386957889 \\
179	0.0203239385436262 \\
180	0.0194869623514315 \\
181	0.0186842471670503 \\
182	0.0179143899719669 \\
183	0.0171760452250753 \\
184	0.0164679225071414 \\
185	0.0157887842618256 \\
186	0.0151374436293157 \\
187	0.0145127623687609 \\
188	0.0139136488658762 \\
189	0.0133390562222218 \\
190	0.0127879804228059 \\
191	0.0122594585788045 \\
192	0.0117525672423122 \\
193	0.0112664207901823 \\
194	0.010800169874104 \\
195	0.0103529999342231 \\
196	0.00992412977368251 \\
197	0.00951281019159778 \\
198	0.00911832267206103 \\
199	0.00873997812688598 \\
200	0.00837711568988286 \\
201	0.00802910156055292 \\
202	0.00769532789518057 \\
203	0.00737521174337141 \\
204	0.00706819402818049 \\
205	0.00677373856803865 \\
206	0.00649133113876621 \\
207	0.00622047857402941 \\
208	0.00596070790266201 \\
209	0.00571156552134466 \\
210	0.00547261640118535 \\
211	0.00524344332682203 \\
212	0.00502364616670015 \\
213	0.00481284117326108 \\
214	0.00461066031180413 \\
215	0.0044167506168542 \\
216	0.0042307735749012 \\
217	0.00405240453243683 \\
218	0.00388133212824604 \\
219	0.0037172577489617 \\
220	0.00355989500692866 \\
221	0.00340896923946188 \\
222	0.00326421702862145 \\
223	0.00312538574066701 \\
224	0.00299223308437746 \\
225	0.00286452668747117 \\
226	0.00274204369038022 \\
227	0.00262457035666652 \\
228	0.0025119016993993 \\
229	0.0024038411228442 \\
230	0.00230020007882348 \\
231	0.00220079773716342 \\
232	0.00210546066963602 \\
233	0.00201402254685368 \\
234	0.00192632384757596 \\
235	0.0018422115799277 \\
236	0.00176153901403864 \\
237	0.00168416542563099 \\
238	0.00160995585011392 \\
239	0.00153878084675041 \\
240	0.0014705162724824 \\
241	0.00140504306502365 \\
242	0.00134224703483299 \\
243	0.0012820186656135 \\
244	0.00122425292297668 \\
245	0.00116884907095027 \\
246	0.00111571049599646 \\
247	0.00106474453823946 \\
248	0.00101586232960682 \\
249	0.000968978638597129 \\
250	0.000924011721406215 \\
251	0.000880883179152692 \\
252	0.000839517820946822 \\
253	0.000799843532569344 \\
254	0.000761791150528879 \\
255	0.000725294341274552 \\
256	0.00069028948535621 \\
257	0.000656715566329069 \\
258	0.000624514064206494 \\
259	0.000593628853277295 \\
260	0.000564006104103232 \\
261	0.000535594189535971 \\
262	0.000508343594574745 \\
263	0.000482206829923282 \\
264	0.000457138349077679 \\
265	0.000433094468816231 \\
266	0.00041003329293976 \\
267	0.000387914639137898 \\
268	0.000366699968846307 \\
269	0.000346352319978038 \\
270	0.000326836242408923 \\
271	0.000308117736101732 \\
272	0.000290164191768305 \\
273	0.000272944333953973 \\
274	0.000256428166459655 \\
275	0.000240586919992181 \\
276	0.000225393001960672 \\
277	0.000210819948328167 \\
278	0.00019684237743256 \\
279	0.000183435945699584 \\
280	0.00017057730516612 \\
281	0.000158244062746338 \\
282	0.000146414741162282 \\
283	0.0001350687414714 \\
284	0.00012418630713551 \\
285	0.000113748489550813 \\
286	0.000103737114997005 \\
287	9.41347529335168e-05 \\
288	8.49246855971764e-05 \\
289	7.6090878840418e-05 \\
290	6.76179541667654e-05 \\
291	5.94911619089533e-05 \\
292	5.16963555037275e-05 \\
293	4.42199668224674e-05 \\
294	3.70489825074483e-05 \\
295	3.01709212804369e-05 \\
296	2.35738121767692e-05 \\
297	1.72461736718255e-05 \\
298	1.11769936614881e-05 \\
299	5.35571026150095e-06 \\
};
\end{axis}

\end{tikzpicture}
}
\scalebox{.85}{% This file was created by matplotlib2tikz v0.6.18.
\begin{tikzpicture}

\definecolor{color0}{rgb}{0.75,0.75,0}

\begin{axis}[
legend cell align={left},
legend columns=1,
legend entries={{\pgd},{10\% \sega},{10\% \algo},{10\% \adaalgo}},
legend style={at={(0.8,0.99)}, anchor=north},
tick align=outside,
tick pos=left,
xlabel={Number of subspaces explored},
xmajorgrids,
xmin=0, xmax=25000000,
ylabel={Suboptimality},
ymajorgrids,
ymin=1e-04, ymax=20455.3594534592,
ymode=log
]



\addlegendimage{ black,thick,mark=square*,mark repeat = 100}
\addlegendimage{ green!50!black,mark=text, text mark={s},mark repeat = 100}
\addlegendimage{ blue,mark=*,mark repeat = 100}
\addlegendimage{ red,mark=diamond*,mark repeat = 100}


\addplot [black,thick,mark=square*,mark repeat = 100]
table [row sep=\\]{%
47236	2310.8803612917 \\
94472	2211.53318678313 \\
141708	2116.65039725421 \\
188944	2026.01187598724 \\
236180	1939.40962430536 \\
283416	1856.64702015789 \\
330652	1777.53812600643 \\
377888	1701.90704254618 \\
425124	1629.58730504891 \\
472360	1560.42131934677 \\
519596	1494.25983469264 \\
566832	1430.9614509322 \\
614068	1370.39215760844 \\
661304	1312.42490279097 \\
708540	1256.93918958126 \\
755776	1203.82069839242 \\
803012	1152.96093323829 \\
850248	1104.25689039335 \\
897484	1057.61074790175 \\
944720	1012.92957452251 \\
991956	970.125056798296 \\
1039192	929.113243028751 \\
1086428	889.814303015186 \\
1133664	852.15230252408 \\
1180900	816.054991490722 \\
1228136	781.453605053275 \\
1275372	748.282676571378 \\
1322608	716.479861842569 \\
1369844	685.985773784739 \\
1417080	656.743826903731 \\
1464316	628.700090912491 \\
1511552	601.803152911997 \\
1558788	576.003987584914 \\
1606024	551.25583489069 \\
1653260	527.514084785866 \\
1700496	504.736168525981 \\
1747732	482.881456135643 \\
1794968	461.911159661492 \\
1842204	441.788241848833 \\
1889440	422.477329907002 \\
1936676	403.944634051064 \\
1983912	386.157870528399 \\
2031148	369.086188858226 \\
2078384	352.700103030225 \\
2125620	336.971426425298 \\
2172856	321.873210237167 \\
2220092	307.379685188138 \\
2267328	293.466206345921 \\
2314564	280.109200861057 \\
2361800	267.286118456288 \\
2409036	254.975384510163 \\
2456272	243.15635558739 \\
2503508	231.809277277979 \\
2550744	220.91524421606 \\
2597980	210.456162157514 \\
2645216	200.414712003258 \\
2692452	190.774315662158 \\
2739688	181.519103654173 \\
2786924	172.633884360556 \\
2834160	164.104114833606 \\
2881396	155.915873083806 \\
2928632	148.055831767071 \\
2975868	140.51123319932 \\
3023104	133.26986562974 \\
3070340	126.320040707852 \\
3117576	119.650572082935 \\
3164812	113.250755077383 \\
3212048	107.110347378315 \\
3259284	101.219550694119 \\
3306520	95.5689933246376 \\
3353756	90.1497135953951 \\
3400992	84.9531441076054 \\
3448228	79.9710967567007 \\
3495464	75.1957484727789 \\
3542700	70.6196276366722 \\
3589936	66.2356011253113 \\
3637172	62.036861939687 \\
3684408	58.0169173680178 \\
3731644	54.1695776357466 \\
3778880	50.4889449927625 \\
3826116	46.969403186885 \\
3873352	43.6056072713442 \\
3920588	40.39247369306 \\
3967824	37.325170608543 \\
4015060	34.3991083761672 \\
4062296	31.6099301790976 \\
4109532	28.9535027453273 \\
4156768	26.4259071557736 \\
4204004	24.0234297794359 \\
4251240	21.7425534699767 \\
4298476	19.5799493548577 \\
4345712	17.532469981661 \\
4392948	15.5971456393623 \\
4440184	13.7711886405867 \\
4487420	12.0520210772522 \\
4534656	10.4400761524391 \\
4581892	8.96017474839175 \\
4629128	7.61085893994912 \\
4676364	6.43224660463946 \\
4723600	5.44383218682478 \\
4770836	4.63875735309595 \\
4818072	4.01337350124374 \\
4865308	3.52913804368414 \\
4912544	3.11657343608689 \\
4959780	2.75007312349651 \\
5007016	2.4331107269804 \\
5054252	2.16461237309359 \\
5101488	1.93165711540296 \\
5148724	1.73346084554685 \\
5195960	1.55791526019514 \\
5243196	1.40772694409368 \\
5290432	1.27332723221149 \\
5337668	1.15758605841933 \\
5384904	1.05255162119959 \\
5432140	0.954973050174627 \\
5479376	0.864500127317723 \\
5526612	0.789428900928137 \\
5573848	0.719718983909603 \\
5621084	0.657534272772214 \\
5668320	0.597997129818025 \\
5715556	0.540989152071345 \\
5762792	0.489500255804279 \\
5810028	0.440695487395648 \\
5857264	0.39495442290653 \\
5904500	0.35718693921726 \\
5951736	0.326277251715657 \\
5998972	0.296655047648732 \\
6046208	0.269953497704607 \\
6093444	0.245376784979378 \\
6140680	0.221826509510787 \\
6187916	0.199259458288364 \\
6235152	0.182021312264225 \\
6282388	0.166707873646684 \\
6329624	0.15502627018349 \\
6376860	0.143826184150629 \\
6424096	0.134513849664251 \\
6471332	0.126413575199693 \\
6518568	0.11997371016397 \\
6565804	0.113798830034974 \\
6613040	0.107878003964861 \\
6660276	0.102200757813249 \\
6707512	0.0967570536822766 \\
6754748	0.0918692427830381 \\
6801984	0.0881038354629582 \\
6849220	0.0844926121091831 \\
6896456	0.0810292536842177 \\
6943692	0.0777077010948872 \\
6990928	0.0745221442604294 \\
7038164	0.0714670117027119 \\
7085400	0.0685369606176665 \\
7132636	0.0657268673937743 \\
7179872	0.0630318185486523 \\
7227108	0.0604471020587547 \\
7274344	0.0579681990603356 \\
7321580	0.0555907759022056 \\
7368816	0.0533106765327651 \\
7416052	0.0511239152053307 \\
7463288	0.0490266694870312 \\
7510524	0.0470152735576028 \\
7557760	0.0450862117852999 \\
7604996	0.0432361125678948 \\
7652232	0.0414617424274248 \\
7699468	0.0397600003479139 \\
7746704	0.0381279123458638 \\
7793940	0.0365626262637682 \\
7841176	0.0350614067773749 \\
7888412	0.0336216306078341 \\
7935648	0.0322407819302515 \\
7982884	0.0309164479705488 \\
8030120	0.0296463147828745 \\
8077356	0.0284281632001289 \\
8124592	0.0272598649504956 \\
8171828	0.0261393789331579 \\
8219064	0.025064747646663 \\
8266300	0.0240340937636754 \\
8313536	0.0230456168461113 \\
8360772	0.0220975901948979 \\
8408008	0.0211883578288372 \\
8455244	0.0203163315872841 \\
8502480	0.0194799883515575 \\
8549716	0.018677867380227 \\
8596952	0.0179085677535959 \\
8644188	0.0171707459229133 \\
8691424	0.0164631133600235 \\
8738660	0.0157844343033271 \\
8785896	0.0151335235961172 \\
8833132	0.0145092446134965 \\
8880368	0.0139105072742525 \\
8927604	0.013336266134198 \\
8974840	0.012785518557648 \\
9022076	0.0122573029638267 \\
9069312	0.0117506971451309 \\
9116548	0.0112648166543092 \\
9163784	0.0107988132577281 \\
9211020	0.0103518734520205 \\
9258256	0.00992321704151222 \\
9305492	0.00951209577394607 \\
9352728	0.00911779203209995 \\
9399964	0.00873961757902331 \\
9447200	0.00837691235467863 \\
9494436	0.0080290433218948 \\
9541672	0.00769540335960328 \\
9588908	0.00737541020142118 \\
9636144	0.0070685054177213 \\
9683380	0.00677415343941057 \\
9730616	0.00649184062170338 \\
9777852	0.00622107434625074 \\
9825088	0.00596138216005859 \\
9872324	0.00571231094968128 \\
9919560	0.00547342614924262 \\
9966796	0.00524431098091083 \\
10014032	0.00502456572648313 \\
10061268	0.00481380702880974 \\
10108504	0.00461166722183948 \\
10155740	0.00441779368810691 \\
10202976	0.00423184824254053 \\
10250212	0.00405350654151104 \\
10297448	0.00388245751608918 \\
10344684	0.00371840282851466 \\
10391920	0.00356105635093273 \\
10439156	0.00341014366548009 \\
10486392	0.00326540158484478 \\
10533628	0.00312657769247027 \\
10580864	0.00299342990158458 \\
10628100	0.00286572603229596 \\
10675336	0.00274324340600662 \\
10722572	0.00262576845643792 \\
10769808	0.00251309635659025 \\
10817044	0.00240503066097397 \\
10864280	0.0023013829624976 \\
10911516	0.00220197256340793 \\
10958752	0.00210662615970136 \\
11005988	0.00201517753846447 \\
11053224	0.00192746728760707 \\
11100460	0.0018433425174802 \\
11147696	0.00176265659389596 \\
11194932	0.00168526888207987 \\
11242168	0.00161104450110661 \\
11289404	0.00153985408839308 \\
11336640	0.00147157357383354 \\
11383876	0.00140608396318331 \\
11431112	0.00134327113031141 \\
11478348	0.00128302561796145 \\
11525584	0.00122524244666855 \\
11572820	0.00116982093149876 \\
11620056	0.00111666450629633 \\
11667292	0.0010656805551208 \\
11714528	0.00101678025058938 \\
11761764	0.00096987839883611 \\
11809000	0.000924893290819906 \\
11856236	0.000881746559717067 \\
11903472	0.000840363044155579 \\
11950708	0.00080067065704581 \\
11997944	0.000762600259781143 \\
12045180	0.00072608554158915 \\
12092416	0.000691062903823259 \\
12139652	0.000657471348985528 \\
12186888	0.000625252374300445 \\
12234124	0.000594349869635025 \\
12281360	0.000564710019608006 \\
12328596	0.000536281209698064 \\
12375832	0.000509013936198066 \\
12423068	0.000482860719855926 \\
12470304	0.000457776023047973 \\
12517540	0.000433716170344489 \\
12564776	0.000410639272324431 \\
12612012	0.000388505152510321 \\
12659248	0.000367275277290524 \\
12706484	0.000346912688713896 \\
12753720	0.000327381940031124 \\
12800956	0.000308649033875064 \\
12848192	0.000290681362972389 \\
12895428	0.000273447653280412 \\
12942664	0.000256917909448706 \\
12989900	0.000241063362519611 \\
13037136	0.000225856419758586 \\
13084372	0.000211270616547576 \\
13131608	0.000197280570240155 \\
13178844	0.000183861935908469 \\
13226080	0.000170991363898532 \\
13273316	0.000158646459124778 \\
13320552	0.000146805742027301 \\
13367788	0.000135448611130018 \\
13415024	0.000124555307127627 \\
13462260	0.000114106878444042 \\
13509496	0.000104085148198152 \\
13556732	9.44726825233833e-05 \\
13603968	8.52527601815645e-05 \\
13651204	7.64093434211244e-05 \\
13698440	6.79270500265616e-05 \\
13745676	5.97911265116657e-05 \\
13792912	5.19874224107486e-05 \\
13840148	4.45023656188148e-05 \\
13887384	3.73229387440333e-05 \\
13934620	3.04366564229941e-05 \\
13981856	2.38315435703296e-05 \\
14029092	1.74961145114061e-05 \\
14076328	1.14193529725526e-05 \\
14123564	5.59069288974534e-06 \\
14170800	0 \\
};





\addplot [green!50!black,mark=text, text mark={s},mark repeat = 100]
table [row sep=\\]{%
94460	5118.19810677661 \\
188920	5156.79094809061 \\
283380	4988.23788149984 \\
377840	4786.98719582755 \\
472300	4579.00032913794 \\
566760	4375.90803128651 \\
661220	4182.82943050911 \\
755680	3999.04823488744 \\
850140	3824.05550823249 \\
944600	3657.25948489222 \\
1039060	3498.30395249591 \\
1133520	3346.75400815268 \\
1227980	3202.18029256636 \\
1322440	3064.28907979919 \\
1416900	2932.71561566201 \\
1511360	2807.15142380639 \\
1605820	2687.29388505692 \\
1700280	2572.86588199057 \\
1794740	2463.52267383409 \\
1889200	2359.08428842239 \\
1983660	2259.23652810634 \\
2078120	2163.92887717618 \\
2172580	2072.71150374388 \\
2267040	1985.54365156167 \\
2361500	1902.17976308624 \\
2455960	1822.45363009979 \\
2550420	1746.1829139727 \\
2644880	1673.21105612787 \\
2739340	1603.37762876826 \\
2833800	1536.55692605971 \\
2928260	1472.57952743264 \\
3022720	1411.31164729256 \\
3117180	1352.63753538713 \\
3211640	1296.41492464636 \\
3306100	1242.61293528437 \\
3400560	1191.02119534394 \\
3495020	1141.58445536406 \\
3589480	1094.19805184971 \\
3683940	1048.74240087473 \\
3778400	1005.24462941131 \\
3872860	963.534319354642 \\
3967320	923.531032149195 \\
4061780	885.158482372068 \\
4156240	848.380650761963 \\
4250700	813.07471042326 \\
4345160	779.213168550373 \\
4439620	746.738780641375 \\
4534080	715.571389762593 \\
4628540	685.659371755516 \\
4723000	656.959127616953 \\
4817460	629.417925860482 \\
4911920	602.982650743085 \\
5006380	577.622057311306 \\
5100840	553.250646700286 \\
5195300	529.853366619237 \\
5289760	507.403329528555 \\
5384220	485.819053019103 \\
5478680	465.122511404107 \\
5573140	445.235878734245 \\
5667600	426.113314819017 \\
5762060	407.769938411664 \\
5856520	390.161964815785 \\
5950980	373.255077575855 \\
6045440	357.010721188903 \\
6139900	341.398431874376 \\
6234360	326.396364706156 \\
6328820	311.992395323858 \\
6423280	298.152689547975 \\
6517740	284.848854963539 \\
6612200	272.071594692461 \\
6706660	259.805579975242 \\
6801120	248.012599983664 \\
6895580	236.68051802111 \\
6990040	225.796974591652 \\
7084500	215.340084517457 \\
7178960	205.28916657388 \\
7273420	195.636341155293 \\
7367880	186.361485249588 \\
7462340	177.459531783695 \\
7556800	168.902456367874 \\
7651260	160.68609826467 \\
7745720	152.789526921417 \\
7840180	145.20840216103 \\
7934640	137.920566376422 \\
8029100	130.922434409231 \\
8123560	124.200538626428 \\
8218020	117.742959778987 \\
8312480	111.547082829434 \\
8406940	105.594661910847 \\
8501400	99.8814607096787 \\
8595860	94.4021128259466 \\
8690320	89.1415178218626 \\
8784780	84.0936357337545 \\
8879240	79.2508069559346 \\
8973700	74.6047891852429 \\
9068160	70.1526901138013 \\
9162620	65.8834554627503 \\
9257080	61.7901400193989 \\
9351540	57.8684420702027 \\
9446000	54.114102177453 \\
9540460	50.5221852980118 \\
9634920	47.0848232037644 \\
9729380	43.7987971199032 \\
9823840	40.655053078342 \\
9918300	37.6512730110352 \\
10012760	34.7852851873841 \\
10107220	32.0513881213278 \\
10201680	29.4471103123797 \\
10296140	26.9653954942391 \\
10390600	24.6060341734179 \\
10485060	22.364670652853 \\
10579520	20.2390763387535 \\
10673980	18.2227739897265 \\
10768440	16.3154549606001 \\
10862900	14.5123116907002 \\
10957360	12.8134437983709 \\
11051820	11.2350016318027 \\
11146280	9.7776311508194 \\
11240740	8.46199827169052 \\
11335200	7.3207179386756 \\
11429660	6.32632500690619 \\
11524120	5.45431763618444 \\
11618580	4.71867886411573 \\
11713040	4.07680756870057 \\
11807500	3.51450141583708 \\
11901960	3.07177673464935 \\
11996420	2.7004550884522 \\
12090880	2.39208786366211 \\
12185340	2.13063262867491 \\
12279800	1.90838344358468 \\
12374260	1.71922933228157 \\
12468720	1.55255664427784 \\
12563180	1.39739475068902 \\
12657640	1.24970982163907 \\
12752100	1.11672946278222 \\
12846560	1.00284737708698 \\
12941020	0.907336437169514 \\
13035480	0.818395207742907 \\
13129940	0.737082148039421 \\
13224400	0.662586190817373 \\
13318860	0.595957965875252 \\
13413320	0.53664919997138 \\
13507780	0.484821327888823 \\
13602240	0.440944986716957 \\
13696700	0.405090430250839 \\
13791160	0.372763443758392 \\
13885620	0.341601168917772 \\
13980080	0.311746125706362 \\
14074540	0.2849371853371 \\
14169000	0.259874031879505 \\
14263460	0.236818470108087 \\
14357920	0.217756662614517 \\
14452380	0.201757479208176 \\
14546840	0.186904854084361 \\
14641300	0.172600698944909 \\
14735760	0.160147761556666 \\
14830220	0.150634973579162 \\
14924680	0.141474200815684 \\
15019140	0.132695608749722 \\
15113600	0.12429991584608 \\
15208060	0.116211113120471 \\
15302520	0.108489662008521 \\
15396980	0.10110013599439 \\
15491440	0.0939538391942314 \\
15585900	0.0871353248941693 \\
15680360	0.0805923972751155 \\
15774820	0.0755588363875679 \\
15869280	0.0707076835570497 \\
15963740	0.0674145444884093 \\
16058200	0.0647059100194678 \\
16152660	0.0620954658108008 \\
16247120	0.0595926076942417 \\
16341580	0.057196602547809 \\
16436040	0.0548966807350333 \\
16530500	0.0526976650228161 \\
16624960	0.0505704736504 \\
16719420	0.0485424057911354 \\
16813880	0.0465833892352194 \\
16908340	0.0447075948120834 \\
17002800	0.0429220321786721 \\
17097260	0.0411953433609458 \\
17191720	0.0395437712866613 \\
17286180	0.0379577676926885 \\
17380640	0.0364177311983838 \\
17475100	0.0349489360780864 \\
17569560	0.0335241500701446 \\
17664020	0.0321695377380911 \\
17758480	0.0308671811975914 \\
17852940	0.0296064644838849 \\
17947400	0.0284125804220408 \\
18041860	0.0272632407703477 \\
18136320	0.0261515163074892 \\
18230780	0.0251024222411202 \\
18325240	0.0240835881735644 \\
18419700	0.0231094359554018 \\
18514160	0.0221798927846222 \\
18608620	0.0212865654246199 \\
18703080	0.0204278600017438 \\
18797540	0.0196059140537748 \\
18892000	0.0188087089856461 \\
18986460	0.0180546645492852 \\
19080920	0.0173224790169617 \\
19175380	0.0166115263671669 \\
19269840	0.0159264679208859 \\
19364300	0.0152851666590823 \\
19458760	0.0146583087568934 \\
19553220	0.0140643917707004 \\
19647680	0.0134929303058295 \\
19742140	0.0129467541947341 \\
19836600	0.0124251765177725 \\
19931060	0.011920445136679 \\
20025520	0.0114345203452002 \\
20119980	0.0109696243839335 \\
20214440	0.0105250950541489 \\
20308900	0.0100972728939217 \\
20403360	0.00968374499380564 \\
20497820	0.00928789346096082 \\
20592280	0.0089100058343432 \\
20686740	0.00854672354243924 \\
20781200	0.00819866663376523 \\
20875660	0.00786183347244473 \\
20970120	0.00754059073256541 \\
21064580	0.0072342830215717 \\
21159040	0.00693979856879534 \\
21253500	0.00665608527417039 \\
21347960	0.00638464808292438 \\
21442420	0.00612369763452891 \\
21536880	0.00587242423557344 \\
21631340	0.00562813638429005 \\
21725800	0.00539488153203083 \\
21820260	0.00517417287869582 \\
21914720	0.00496168824862897 \\
22009180	0.00475715504217722 \\
22103640	0.00456066956647527 \\
22198100	0.00437320511396599 \\
22292560	0.00419203453511585 \\
22387020	0.00401931303156267 \\
22481480	0.00385276008358515 \\
22575940	0.00369354536336797 \\
22670400	0.00354094512814385 \\
22764860	0.00339339934251237 \\
22859320	0.00325071987671821 \\
22953780	0.00311548607001999 \\
23048240	0.00298553004172941 \\
23142700	0.00285997180962738 \\
23237160	0.0027391757048898 \\
23331620	0.00262381212843188 \\
23426080	0.00251357338848157 \\
23520540	0.00240648552478984 \\
23615000	0.00230478630642228 \\
23709460	0.00220627365118697 \\
23803920	0.0021115676570187 \\
23898380	0.00202139016458336 \\
23992840	0.00193523724449141 \\
24087300	0.00185256864048622 \\
24181760	0.00177278657322666 \\
24276220	0.0016961412812817 \\
24370680	0.00162274519097605 \\
24465140	0.00155161867797804 \\
24559600	0.00148398242802839 \\
24654060	0.00141912631192387 \\
24748520	0.00135587450323094 \\
24842980	0.00129628839762552 \\
24937440	0.00123915331607227 \\
25031900	0.00118419825388028 \\
25126360	0.00113119789871718 \\
25220820	0.00108049245340891 \\
25315280	0.00103190361943861 \\
25409740	0.000985195055735577 \\
25504200	0.000940283477031567 \\
25598660	0.000897223793660862 \\
25693120	0.000856221675901381 \\
25787580	0.000816697965103863 \\
25882040	0.000778665954801028 \\
25976500	0.000742286451011154 \\
26070960	0.000707324283729216 \\
26165420	0.000673773857202908 \\
26259880	0.00064137139949505 \\
26354340	0.000610403051375608 \\
26448800	0.00058049972476959 \\
26543260	0.000551816445459874 \\
26637720	0.000524216466932792 \\
26732180	0.0004980202743603 \\
26826640	0.000472677531387244 \\
26921100	0.000448467282707199 \\
27015560	0.000425161760051473 \\
27110020	0.000402773885388896 \\
27204480	0.000381117534484776 \\
27298940	0.000360293562763569 \\
27393400	0.000340746730422836 \\
27487860	0.000321765641868588 \\
27582320	0.000303539445225542 \\
27676780	0.000286038588972026 \\
27771240	0.000269367777260276 \\
27865700	0.000253320926741463 \\
27960160	0.000237818848667359 \\
28054620	0.000223104856724055 \\
28149080	0.000208903001279293 \\
28243540	0.000195115944018109 \\
28338000	0.000182007849647503 \\
};




\addplot [blue,mark=*,mark repeat = 100]
table [row sep=\\]{%
94460	2215.46769112275 \\
188920	2037.32560309681 \\
283380	1871.53172955521 \\
377840	1719.86780163509 \\
472300	1580.97313434325 \\
566760	1454.02619401535 \\
661220	1336.99261082007 \\
755680	1230.0816734375 \\
850140	1132.00382770868 \\
944600	1041.58816862701 \\
1039060	958.080965617693 \\
1133520	880.890694921656 \\
1227980	810.180934882506 \\
1322440	745.544286193405 \\
1416900	685.629080841174 \\
1511360	629.812094785211 \\
1605820	579.190768174642 \\
1700280	532.130896138681 \\
1794740	489.172782838581 \\
1889200	448.970823143715 \\
1983660	411.994605737835 \\
2078120	377.693202602656 \\
2172580	345.763442788726 \\
2267040	316.643011844905 \\
2361500	289.844089082958 \\
2455960	265.024422361409 \\
2550420	242.081677210233 \\
2644880	220.714376589643 \\
2739340	200.935199737049 \\
2833800	182.728935360078 \\
2928260	165.862181521112 \\
3022720	150.471039392253 \\
3117180	136.120063528095 \\
3211640	122.828796576557 \\
3306100	110.528900101399 \\
3400560	99.0263817536842 \\
3495020	88.6853140177956 \\
3589480	79.0551655778828 \\
3683940	70.2097547980831 \\
3778400	62.0004196830372 \\
3872860	54.5104341964193 \\
3967320	47.5936765603387 \\
4061780	41.2011267944248 \\
4156240	35.4688343431486 \\
4250700	30.2767660866124 \\
4345160	25.5511202754026 \\
4439620	21.2319613734782 \\
4534080	17.3278512600208 \\
4628540	13.9199451050115 \\
4723000	10.859732434644 \\
4817460	8.17313254415244 \\
4911920	5.94325956108226 \\
5006380	4.35963591572909 \\
5100840	3.3421330045379 \\
5195300	2.57613510754713 \\
5289760	2.03820735660192 \\
5384220	1.64367561622815 \\
5478680	1.36156093612851 \\
5573140	1.11936054194711 \\
5667600	0.920783269891287 \\
5762060	0.777564894147422 \\
5856520	0.674774956391493 \\
5950980	0.564576060032792 \\
6045440	0.476257560869569 \\
6139900	0.398959161599076 \\
6234360	0.330402582279344 \\
6328820	0.285865379536206 \\
6423280	0.230223380572313 \\
6517740	0.193699030181417 \\
6612200	0.156101035183382 \\
6706660	0.123808682362502 \\
6801120	0.112532301128868 \\
6895580	0.0919036840094851 \\
6990040	0.0801469015283101 \\
7084500	0.0744979564892159 \\
7178960	0.0659229973911128 \\
7273420	0.0593549896667656 \\
7367880	0.0539383296603342 \\
7462340	0.0505767264184604 \\
7556800	0.0454427378330433 \\
7651260	0.0411626747204714 \\
7745720	0.0381080701952439 \\
7840180	0.0341606443218394 \\
7934640	0.0310249481552582 \\
8029100	0.0292879934831141 \\
8123560	0.0261010160525519 \\
8218020	0.0240787807597438 \\
8312480	0.0222265333849654 \\
8406940	0.0201760694256783 \\
8501400	0.018910463818731 \\
8595860	0.0179198263212341 \\
8690320	0.0170788544349327 \\
8784780	0.0160438698251628 \\
8879240	0.0151730231472071 \\
8973700	0.0139130223494337 \\
9068160	0.0127487341802963 \\
9162620	0.0116338357363084 \\
9257080	0.0110614089184784 \\
9351540	0.0105866605553218 \\
9446000	0.0101781086193464 \\
9540460	0.00949870566529998 \\
9634920	0.0086610859496361 \\
9729380	0.00817496420412445 \\
9823840	0.00755391092746205 \\
9918300	0.0069215091406778 \\
10012760	0.00604755883339236 \\
10107220	0.00557701444636094 \\
10201680	0.00500930751814233 \\
10296140	0.00443851511086879 \\
10390600	0.00410317158312656 \\
10485060	0.00353827209036295 \\
10579520	0.00333287020226747 \\
10673980	0.00304215475018976 \\
10768440	0.00279974788854487 \\
10862900	0.00257876450404226 \\
10957360	0.00245956091685384 \\
11051820	0.0022133374003992 \\
11146280	0.00203646817651526 \\
11240740	0.00187730869269265 \\
11335200	0.00175532041033755 \\
11429660	0.00166920396008985 \\
11524120	0.00150131302121692 \\
11618580	0.00136255943398278 \\
11713040	0.00123096287088176 \\
11807500	0.001100204640887 \\
11901960	0.00101623481152457 \\
11996420	0.000949441637570603 \\
12090880	0.000884432583142525 \\
12185340	0.000812510916769149 \\
12279800	0.000742472866251731 \\
12374260	0.000677155607360591 \\
12468720	0.000629517082185505 \\
12563180	0.000591175604756522 \\
12657640	0.000541115506787815 \\
12752100	0.000485105868365254 \\
12846560	0.00045045411086897 \\
12941020	0.000416273480553508 \\
13035480	0.000372095084320456 \\
13129940	0.000334533445461638 \\
13224400	0.0003030299428326 \\
13318860	0.000260348643132069 \\
13413320	0.000205180484260215 \\
13507780	0.00017823551181495 \\
13602240	0.000155740123426007 \\
13696700	0.000134748697486531 \\
13791160	0.000118303867667979 \\
13885620	9.94370597839911e-05 \\
13980080	9.29016066277821e-05 \\
14074540	7.79697129249524e-05 \\
14169000	5.74764181768828e-05 \\
14263460	4.08328144851922e-05 \\
14357920	2.64600165156281e-05 \\
14452380	1.14609381742792e-05 \\
14546840	5.08335999960252e-06 \\
14641300	-7.46151959241104e-06 \\
14735760	-1.62555974296819e-05 \\
14830220	-2.428763716944e-05 \\
14924680	-3.05905163024978e-05 \\
15019140	-4.0448479383004e-05 \\
15113600	-4.84575076118432e-05 \\
15208060	-5.46238686962575e-05 \\
15302520	-6.14516177717128e-05 \\
15396980	-6.57563403123618e-05 \\
15491440	-7.04176673247403e-05 \\
15585900	-7.52610921714858e-05 \\
15680360	-7.97950795738966e-05 \\
15774820	-8.52381957967552e-05 \\
15869280	-8.77465387929544e-05 \\
15963740	-9.16434927653587e-05 \\
16058200	-9.51337436920774e-05 \\
16152660	-9.73324485245008e-05 \\
16247120	-0.000100852120405559 \\
16341580	-0.000104069143702423 \\
16436040	-0.000105408343014979 \\
16530500	-0.000107404486987761 \\
16624960	-0.000108747075628246 \\
16719420	-0.000109888051108786 \\
16813880	-0.000111686225726659 \\
16908340	-0.000112856931558714 \\
17002800	-0.000114610466437792 \\
17097260	-0.000115808470956935 \\
17191720	-0.000117399967864751 \\
17286180	-0.000118875312976385 \\
17380640	-0.000119867969355925 \\
17475100	-0.000120942099396393 \\
17569560	-0.000121592089585842 \\
17664020	-0.000122310606665899 \\
17758480	-0.000123241807776786 \\
17852940	-0.000123786380473945 \\
17947400	-0.000124471386827807 \\
18041860	-0.000125289914478577 \\
18136320	-0.000125779770783652 \\
18230780	-0.000126129219248927 \\
18325240	-0.000126483222937512 \\
18419700	-0.000126747901387114 \\
18514160	-0.000127176899718862 \\
18608620	-0.000127632966547608 \\
18703080	-0.000127930903118134 \\
18797540	-0.000128196758494292 \\
18892000	-0.00012844135499912 \\
18986460	-0.000128671393262225 \\
19080920	-0.000128971600738659 \\
19175380	-0.000129119051188287 \\
19269840	-0.000129298619535145 \\
19364300	-0.000129532742533023 \\
19458760	-0.000129686324016332 \\
19553220	-0.000129784222598239 \\
19647680	-0.000129843054706757 \\
19742140	-0.000129994736016492 \\
19836600	-0.000130098253828592 \\
19931060	-0.000130199090270677 \\
20025520	-0.000130337754423726 \\
20119980	-0.000130451485293115 \\
20214440	-0.000130503190100661 \\
20308900	-0.000130577868739934 \\
20403360	-0.000130645328820567 \\
20497820	-0.000130712106684694 \\
20592280	-0.00013074938673352 \\
20686740	-0.000130818125866572 \\
20781200	-0.000130854139438252 \\
20875660	-0.000130904364110673 \\
20970120	-0.000130936181654162 \\
21064580	-0.000130976199283817 \\
21159040	-0.000131027241370596 \\
21253500	-0.000131058692006469 \\
21347960	-0.000131099130946666 \\
21442420	-0.000131121177700422 \\
21536880	-0.00013114345024956 \\
21631340	-0.000131167040977154 \\
21725800	-0.000131186694467766 \\
21820260	-0.000131204589046918 \\
21914720	-0.000131226199857393 \\
22009180	-0.000131243259162916 \\
22103640	-0.00013126479936254 \\
22198100	-0.000131274840880824 \\
22292560	-0.000131285462410258 \\
22387020	-0.000131304193822457 \\
22481480	-0.000131314130968008 \\
22575940	-0.000131323639352754 \\
22670400	-0.000131333251127241 \\
22764860	-0.00013133869121118 \\
22859320	-0.000131345493680213 \\
22953780	-0.00013135004975906 \\
23048240	-0.000131356808398042 \\
23142700	-0.000131362678915004 \\
23237160	-0.000131368025171552 \\
23331620	-0.000131372311478639 \\
23426080	-0.000131374087628089 \\
23520540	-0.00013137644225214 \\
23615000	-0.000131378952843209 \\
23709460	-0.000131382926427559 \\
23803920	-0.000131386146071888 \\
23898380	-0.000131388671373411 \\
23992840	-0.000131391475943765 \\
24087300	-0.000131392768053296 \\
24181760	-0.000131395193757156 \\
24276220	-0.000131397333228644 \\
24370680	-0.000131398643900216 \\
24465140	-0.000131399784067954 \\
24559600	-0.000131401461318958 \\
24654060	-0.000131402476356124 \\
24748520	-0.000131403616031811 \\
24842980	-0.000131404192929452 \\
24937440	-0.000131404544881919 \\
25031900	-0.000131405423160924 \\
25126360	-0.000131407006322082 \\
25220820	-0.000131407680529216 \\
25315280	-0.000131408015970225 \\
25409740	-0.000131408518285525 \\
25504200	-0.000131409081909117 \\
25598660	-0.000131409254442882 \\
25693120	-0.000131409713992836 \\
25787580	-0.000131410108713759 \\
25882040	-0.000131410469435211 \\
25976500	-0.000131411069090426 \\
26070960	-0.000131411429115769 \\
26165420	-0.000131411683051308 \\
26259880	-0.000131411997066344 \\
26354340	-0.000131412152264865 \\
26448800	-0.000131412414276832 \\
26543260	-0.000131412618254778 \\
26637720	-0.000131412801197328 \\
26732180	-0.000131412862325098 \\
26826640	-0.000131412965585387 \\
26921100	-0.000131413058624297 \\
27015560	-0.000131413116695622 \\
27110020	-0.000131413233589228 \\
27204480	-0.000131413402971736 \\
27298940	-0.000131413461915475 \\
27393400	-0.000131413581587747 \\
27487860	-0.000131413732362695 \\
27582320	-0.000131413779161704 \\
27676780	-0.000131413899852273 \\
27771240	-0.000131413944675973 \\
27865700	-0.000131414075740466 \\
27960160	-0.000131414141408159 \\
28054620	-0.000131414170485566 \\
28149080	-0.000131414199224356 \\
28243540	-0.000131414249647577 \\
28338000	-0.000131414324923584 \\
};




\addplot [red,mark=diamond*,mark repeat = 100]
table [row sep=\\]{%
47236	2300.19291388783 \\
94472	2201.32505971244 \\
141708	2106.89795102146 \\
188944	2016.69277113798 \\
236180	1930.50274206657 \\
283416	1848.13238835034 \\
330652	1769.39684985688 \\
377888	1694.12124005497 \\
425124	1622.14004659352 \\
472360	1553.29657122546 \\
519596	1487.44240633372 \\
566832	1424.43694551452 \\
614068	1364.14692585735 \\
661304	1306.44599973096 \\
708540	1251.21433404237 \\
755776	1198.33823508217 \\
803012	1147.70979720457 \\
850248	1099.22657371601 \\
897484	1052.79126846244 \\
944720	1008.311446713 \\
991956	965.699264037489 \\
1039192	924.871211967653 \\
1086428	885.74787931785 \\
1133664	848.253728120298 \\
1180900	812.316883203574 \\
1228136	777.868934511467 \\
1275372	744.844751322552 \\
1322608	713.182307589627 \\
1369844	682.822517672612 \\
1417080	653.709081789065 \\
1464316	625.788340553373 \\
1511552	599.009138019179 \\
1558788	573.322692680007 \\
1606024	548.682475920535 \\
1653260	525.044097445784 \\
1700496	502.365197247801 \\
1747732	480.605343699445 \\
1794968	459.725937392764 \\
1842204	439.690120365379 \\
1889440	420.462690382331 \\
1936676	402.010019963277 \\
1983912	384.299979865678 \\
2031148	367.301866754001 \\
2078384	350.986334802934 \\
2125620	335.325330999366 \\
2172856	320.292033923432 \\
2220092	305.860795803446 \\
2267328	292.007087653034 \\
2314564	278.707447311312 \\
2361800	265.939430218688 \\
2409036	253.681562771743 \\
2456272	241.913298110785 \\
2503508	230.614974203138 \\
2550744	219.767774094008 \\
2597980	209.353688204972 \\
2645216	199.355478567754 \\
2692452	189.756644888053 \\
2739688	180.541392340743 \\
2786924	171.69460100393 \\
2834160	163.201796844939 \\
2881396	155.049124176593 \\
2928632	147.223319506902 \\
2975868	139.711686709708 \\
3023104	132.502073447806 \\
3070340	125.582848783701 \\
3117576	118.942881916318 \\
3164812	112.571521984843 \\
3212048	106.45857888323 \\
3259284	100.594305030907 \\
3306520	94.969378046764 \\
3353756	89.574884274564 \\
3400992	84.4023031085797 \\
3448228	79.4434920683459 \\
3495464	74.690672571072 \\
3542700	70.1364163493668 \\
3589936	65.7736324605588 \\
3637172	61.5955548320614 \\
3684408	57.5957302850215 \\
3731644	53.7680069760221 \\
3778880	50.1065231941022 \\
3826116	46.6056964481226 \\
3873352	43.2602127779974 \\
3920588	40.0650162231841 \\
3967824	37.0152983839567 \\
4015060	34.106488016596 \\
4062296	31.3342406144279 \\
4109532	28.6944279450966 \\
4156768	26.1831275445655 \\
4204004	23.7966122171377 \\
4251240	21.5313396722713 \\
4298476	19.3839425734867 \\
4345712	17.3512195538147 \\
4392948	15.430128350294 \\
4440184	13.6177836590535 \\
4487420	11.9114664357797 \\
4534656	10.3086664828268 \\
4581892	8.81706545792113 \\
4629128	7.45740842672546 \\
4676364	6.2788523707335 \\
4723132	5.2641539147017 \\
4765652	4.45784746309847 \\
4802508	3.83275209709886 \\
4834608	3.35308726217701 \\
4862460	2.94483842422316 \\
4888424	2.59075045514111 \\
4912972	2.28337744159368 \\
4936576	2.0168530343678 \\
4957820	1.80597429629088 \\
4977176	1.62209218127343 \\
4995116	1.46252456297161 \\
5012584	1.31776898357308 \\
5029580	1.19340780651337 \\
5045632	1.07857350791643 \\
5061212	0.974553393591858 \\
5075848	0.886951233613247 \\
5089540	0.812088236022116 \\
5102760	0.741387341231388 \\
5115980	0.674156699923656 \\
5128728	0.613841742867664 \\
5141476	0.556099377788208 \\
5154224	0.500814342237803 \\
5166972	0.450022711058062 \\
5178776	0.406587781660225 \\
5189636	0.37058040724784 \\
5200496	0.336086235777852 \\
5211356	0.303431865461464 \\
5221744	0.276775488889247 \\
5231660	0.251885118404891 \\
5241576	0.228036229158716 \\
5251492	0.20659244448976 \\
5260936	0.18980192129818 \\
5269436	0.175567697613862 \\
5277464	0.165188992685961 \\
5285020	0.155393564122782 \\
5292576	0.146561050260203 \\
5299660	0.139289346229385 \\
5306744	0.132317093093035 \\
5313828	0.125631903087486 \\
5320912	0.119221914032722 \\
5327996	0.113075763820598 \\
5335080	0.107182566886773 \\
5342164	0.101531892372706 \\
5349248	0.0961137437395931 \\
5356332	0.0919567807760056 \\
5362944	0.0881864020184207 \\
5369556	0.0845704647093735 \\
5376168	0.0811026374611046 \\
5382780	0.0777768493316384 \\
5389392	0.0745872788887074 \\
5396004	0.0715283437919481 \\
5402616	0.0685946908534825 \\
5409228	0.0657811865435234 \\
5415840	0.0630829079126565 \\
5422452	0.0604951339063109 \\
5429064	0.0580133370499114 \\
5435676	0.055633175485551 \\
5442288	0.0533504853428817 \\
5448900	0.0511612734284053 \\
5455512	0.0490617102185777 \\
5462124	0.0470481231431457 \\
5468736	0.0451169901459996 \\
5475348	0.0432649335115776 \\
5481960	0.0414887139454909 \\
5488572	0.0397852248986439 \\
5495184	0.0381514871246278 \\
5501796	0.0365846434606651 \\
5508408	0.0350819538228144 \\
5515020	0.0336407904065674 \\
5521632	0.0322586330843548 \\
5528244	0.0309330649918387 \\
5534856	0.029661768295226 \\
5541468	0.0284425201321521 \\
5548080	0.0272731887190081 \\
5554692	0.0261517296178715 \\
5561304	0.0250761821564907 \\
5567916	0.0240446659950493 \\
5574528	0.0230553778336757 \\
5581140	0.0221065882549396 \\
5587752	0.0211966386957889 \\
5594364	0.0203239385436262 \\
5600976	0.0194869623514315 \\
5607588	0.0186842471670503 \\
5614200	0.0179143899719669 \\
5620812	0.0171760452250753 \\
5627424	0.0164679225071414 \\
5634036	0.0157887842618256 \\
5640648	0.0151374436293157 \\
5647260	0.0145127623687609 \\
5653872	0.0139136488658762 \\
5660484	0.0133390562222218 \\
5667096	0.0127879804228059 \\
5673708	0.0122594585788045 \\
5680320	0.0117525672423122 \\
5686932	0.0112664207901823 \\
5693544	0.010800169874104 \\
5700156	0.0103529999342231 \\
5706768	0.00992412977368251 \\
5713380	0.00951281019159778 \\
5719992	0.00911832267206103 \\
5726604	0.00873997812688598 \\
5733216	0.00837711568988286 \\
5739828	0.00802910156055292 \\
5746440	0.00769532789518057 \\
5753052	0.00737521174337141 \\
5759664	0.00706819402818049 \\
5766276	0.00677373856803865 \\
5772888	0.00649133113876621 \\
5779500	0.00622047857402941 \\
5786112	0.00596070790266201 \\
5792724	0.00571156552134466 \\
5799336	0.00547261640118535 \\
5805948	0.00524344332682203 \\
5812560	0.00502364616670015 \\
5819172	0.00481284117326108 \\
5825784	0.00461066031180413 \\
5832396	0.0044167506168542 \\
5839008	0.0042307735749012 \\
5845620	0.00405240453243683 \\
5852232	0.00388133212824604 \\
5858844	0.0037172577489617 \\
5865456	0.00355989500692866 \\
5872068	0.00340896923946188 \\
5878680	0.00326421702862145 \\
5885292	0.00312538574066701 \\
5891904	0.00299223308437746 \\
5898516	0.00286452668747117 \\
5905128	0.00274204369038022 \\
5911740	0.00262457035666652 \\
5918352	0.0025119016993993 \\
5924964	0.0024038411228442 \\
5931576	0.00230020007882348 \\
5938188	0.00220079773716342 \\
5944800	0.00210546066963602 \\
5951412	0.00201402254685368 \\
5958024	0.00192632384757596 \\
5964636	0.0018422115799277 \\
5971248	0.00176153901403864 \\
5977860	0.00168416542563099 \\
5984472	0.00160995585011392 \\
5991084	0.00153878084675041 \\
5997696	0.0014705162724824 \\
6004308	0.00140504306502365 \\
6010920	0.00134224703483299 \\
6017532	0.0012820186656135 \\
6024144	0.00122425292297668 \\
6030756	0.00116884907095027 \\
6037368	0.00111571049599646 \\
6043980	0.00106474453823946 \\
6050592	0.00101586232960682 \\
6057204	0.000968978638597129 \\
6063816	0.000924011721406215 \\
6070428	0.000880883179152692 \\
6077040	0.000839517820946822 \\
6083652	0.000799843532569344 \\
6090264	0.000761791150528879 \\
6096876	0.000725294341274552 \\
6103488	0.00069028948535621 \\
6110100	0.000656715566329069 \\
6116712	0.000624514064206494 \\
6123324	0.000593628853277295 \\
6129936	0.000564006104103232 \\
6136548	0.000535594189535971 \\
6143160	0.000508343594574745 \\
6149772	0.000482206829923282 \\
6156384	0.000457138349077679 \\
6162996	0.000433094468816231 \\
6169608	0.00041003329293976 \\
6176220	0.000387914639137898 \\
6182832	0.000366699968846307 \\
6189444	0.000346352319978038 \\
6196056	0.000326836242408923 \\
6202668	0.000308117736101732 \\
6209280	0.000290164191768305 \\
6215892	0.000272944333953973 \\
6222504	0.000256428166459655 \\
6229116	0.000240586919992181 \\
6235728	0.000225393001960672 \\
6242340	0.000210819948328167 \\
6248952	0.00019684237743256 \\
6255564	0.000183435945699584 \\
6262176	0.00017057730516612 \\
6268788	0.000158244062746338 \\
6275400	0.000146414741162282 \\
6282012	0.0001350687414714 \\
6288624	0.00012418630713551 \\
6295236	0.000113748489550813 \\
6301848	0.000103737114997005 \\
6308460	9.41347529335168e-05 \\
6315072	8.49246855971764e-05 \\
6321684	7.6090878840418e-05 \\
6328296	6.76179541667654e-05 \\
6334908	5.94911619089533e-05 \\
6341520	5.16963555037275e-05 \\
6348132	4.42199668224674e-05 \\
6354744	3.70489825074483e-05 \\
6361356	3.01709212804369e-05 \\
6367968	2.35738121767692e-05 \\
6374580	1.72461736718255e-05 \\
6381192	1.11769936614881e-05 \\
6387804	5.35571026150095e-06 \\
};
\end{axis}

\end{tikzpicture}
}
\end{center}
\caption{$\ell_{1,2}$ regularized logistic regression \eqref{eq:logl12}
}
\label{fig:rcv1_l12}
\end{figure}

% ==========================================================================
\subsection{Illustrations for total variation regularization}

We focus here on the case of total variation \eqref{eq:logtv} which is a typical usecase for our adaptive algorithm and subspace descent in general. Figure~\ref{fig:a11} displays a comparison between the vanilla proximal gradient and various versions of our subspace descent methods.

We observe first that  \algo, not exploiting the problem structure, fails to reach satisfying performances as it identifies lately and converges slowly. In contrast, the adaptive versions \adaalgo~perform similarly to the vanilla proximal gradient in terms of sparsification and suboptimality with respect to iterations. As a consequence, in terms of number of subspaces explored, \adaalgo~becomes much faster once a near-optimal structure is identified. More precisely, all adaptive algorithms (except 1 \adaalgo, see the next paragraph) identify a subspace of size $\approx 8\%$ (10 jumps in the entries of the iterates) after having explored around $10^5$ subspaces. Subsequently, each iteration involves a subspace of size 22,32,62 (out of a total dimension of 123) for 10\%,20\%,50\% \adaalgo~respectively, resulting in the different slopes in the red plots on the rightmost figure.

\begin{figure}[H]
\begin{center}
\scalebox{.85}{% This file was created by matplotlib2tikz v0.6.18.
\begin{tikzpicture}



\begin{axis}[
legend cell align={left},
legend columns=1,
legend entries={{\pgd},{20\% \algo},{1 \adaalgo},{10\% \adaalgo},{20\% \adaalgo},{50\% \adaalgo},},
legend style={at={(0.8,0.99)}, anchor=north},
tick align=outside,
tick pos=left,
xlabel={Iteration},
xmajorgrids,
xmin=0, xmax=4000,
ylabel={Iterate sparsity},
ymajorgrids,
ymin=3.78048780487805, ymax=99.4715447154472
]

\addlegendimage{ black,thick,mark=square*,mark repeat = 100}
\addlegendimage{ blue,mark=*,mark repeat = 100}
\addlegendimage{ red }
\addlegendimage{ red,mark=text, text mark={\tiny 10\%}}
\addlegendimage{ dashed, red,mark=text, text mark={\tiny 20\%}}
\addlegendimage{ dotted, red,mark=text, text mark={\tiny 50\%}}

\addplot [black,thick,mark=square*,mark repeat = 100]
table [row sep=\\]{%
0	90.2439024390244 \\
10	85.3658536585366 \\
20	78.8617886178862 \\
30	77.2357723577236 \\
40	74.7967479674797 \\
50	68.2926829268293 \\
60	65.0406504065041 \\
70	62.6016260162602 \\
80	59.349593495935 \\
90	56.9105691056911 \\
100	53.6585365853659 \\
110	56.0975609756098 \\
120	47.9674796747967 \\
130	44.7154471544715 \\
140	41.4634146341463 \\
150	41.4634146341463 \\
160	39.8373983739837 \\
170	36.5853658536585 \\
180	33.3333333333333 \\
190	30.8943089430894 \\
200	30.8943089430894 \\
210	31.7073170731707 \\
220	29.2682926829268 \\
230	28.4552845528455 \\
240	26.0162601626016 \\
250	24.390243902439 \\
260	23.5772357723577 \\
270	21.9512195121951 \\
280	21.9512195121951 \\
290	21.9512195121951 \\
300	21.1382113821138 \\
310	20.3252032520325 \\
320	19.5121951219512 \\
330	19.5121951219512 \\
340	18.6991869918699 \\
350	18.6991869918699 \\
360	18.6991869918699 \\
370	18.6991869918699 \\
380	17.8861788617886 \\
390	17.0731707317073 \\
400	16.260162601626 \\
410	16.260162601626 \\
420	16.260162601626 \\
430	16.260162601626 \\
440	15.4471544715447 \\
450	15.4471544715447 \\
460	15.4471544715447 \\
470	14.6341463414634 \\
480	14.6341463414634 \\
490	14.6341463414634 \\
500	14.6341463414634 \\
510	13.8211382113821 \\
520	13.8211382113821 \\
530	13.8211382113821 \\
540	13.8211382113821 \\
550	13.8211382113821 \\
560	13.8211382113821 \\
570	13.8211382113821 \\
580	13.8211382113821 \\
590	13.8211382113821 \\
600	13.8211382113821 \\
610	13.8211382113821 \\
620	13.0081300813008 \\
630	13.0081300813008 \\
640	12.1951219512195 \\
650	11.3821138211382 \\
660	11.3821138211382 \\
670	11.3821138211382 \\
680	10.5691056910569 \\
690	10.5691056910569 \\
700	10.5691056910569 \\
710	10.5691056910569 \\
720	10.5691056910569 \\
730	10.5691056910569 \\
740	10.5691056910569 \\
750	10.5691056910569 \\
760	10.5691056910569 \\
770	10.5691056910569 \\
780	10.5691056910569 \\
790	10.5691056910569 \\
800	10.5691056910569 \\
810	9.75609756097561 \\
820	9.75609756097561 \\
830	9.75609756097561 \\
840	9.75609756097561 \\
850	9.75609756097561 \\
860	9.75609756097561 \\
870	9.75609756097561 \\
880	9.75609756097561 \\
890	9.75609756097561 \\
900	9.75609756097561 \\
910	9.75609756097561 \\
920	9.75609756097561 \\
930	9.75609756097561 \\
940	9.75609756097561 \\
950	9.75609756097561 \\
960	9.75609756097561 \\
970	9.75609756097561 \\
980	9.75609756097561 \\
990	9.75609756097561 \\
1000	9.75609756097561 \\
1010	9.75609756097561 \\
1020	9.75609756097561 \\
1030	9.75609756097561 \\
1040	9.75609756097561 \\
1050	9.75609756097561 \\
1060	9.75609756097561 \\
1070	9.75609756097561 \\
1080	9.75609756097561 \\
1090	9.75609756097561 \\
1100	9.75609756097561 \\
1110	9.75609756097561 \\
1120	9.75609756097561 \\
1130	9.75609756097561 \\
1140	9.75609756097561 \\
1150	9.75609756097561 \\
1160	9.75609756097561 \\
1170	9.75609756097561 \\
1180	9.75609756097561 \\
1190	9.75609756097561 \\
1200	9.75609756097561 \\
1210	8.94308943089431 \\
1220	8.94308943089431 \\
1230	8.13008130081301 \\
1240	8.13008130081301 \\
1250	8.13008130081301 \\
1260	8.13008130081301 \\
1270	8.13008130081301 \\
1280	8.13008130081301 \\
1290	8.13008130081301 \\
1300	8.13008130081301 \\
1310	8.13008130081301 \\
1320	8.13008130081301 \\
1330	8.13008130081301 \\
1340	8.13008130081301 \\
1350	8.13008130081301 \\
1360	8.13008130081301 \\
1370	8.13008130081301 \\
1380	8.13008130081301 \\
1390	8.13008130081301 \\
1400	8.13008130081301 \\
1410	8.13008130081301 \\
1420	8.13008130081301 \\
1430	8.13008130081301 \\
1440	8.13008130081301 \\
1450	8.13008130081301 \\
1460	8.13008130081301 \\
1470	8.13008130081301 \\
1480	8.13008130081301 \\
1490	8.13008130081301 \\
1500	8.13008130081301 \\
1510	8.13008130081301 \\
1520	8.13008130081301 \\
1530	8.13008130081301 \\
1540	8.13008130081301 \\
1550	8.13008130081301 \\
1560	8.13008130081301 \\
1570	8.13008130081301 \\
1580	8.13008130081301 \\
1590	8.13008130081301 \\
1600	8.13008130081301 \\
1610	8.13008130081301 \\
1620	8.13008130081301 \\
1630	8.13008130081301 \\
1640	8.13008130081301 \\
1650	8.13008130081301 \\
1660	8.13008130081301 \\
1670	8.13008130081301 \\
1680	8.13008130081301 \\
1690	8.13008130081301 \\
1700	8.13008130081301 \\
1710	8.13008130081301 \\
1720	8.13008130081301 \\
1730	8.13008130081301 \\
1740	8.13008130081301 \\
1750	8.13008130081301 \\
1760	8.13008130081301 \\
1770	8.13008130081301 \\
1780	8.13008130081301 \\
1790	8.13008130081301 \\
1800	8.13008130081301 \\
1810	8.13008130081301 \\
1820	8.13008130081301 \\
1830	8.13008130081301 \\
1840	8.13008130081301 \\
1850	8.13008130081301 \\
1860	8.13008130081301 \\
1870	8.13008130081301 \\
1880	8.13008130081301 \\
1890	8.13008130081301 \\
1900	8.13008130081301 \\
1910	8.13008130081301 \\
1920	8.13008130081301 \\
1930	8.13008130081301 \\
1940	8.13008130081301 \\
1950	8.13008130081301 \\
1960	8.13008130081301 \\
1970	8.13008130081301 \\
1980	8.13008130081301 \\
1990	8.13008130081301 \\
2000	8.13008130081301 \\
2010	8.13008130081301 \\
2020	8.13008130081301 \\
2030	8.13008130081301 \\
2040	8.13008130081301 \\
2050	8.13008130081301 \\
2060	8.13008130081301 \\
2070	8.13008130081301 \\
2080	8.13008130081301 \\
2090	8.13008130081301 \\
2100	8.13008130081301 \\
2110	8.13008130081301 \\
2120	8.13008130081301 \\
2130	8.13008130081301 \\
2140	8.13008130081301 \\
2150	8.13008130081301 \\
2160	8.13008130081301 \\
2170	8.13008130081301 \\
2180	8.13008130081301 \\
2190	8.13008130081301 \\
2200	8.13008130081301 \\
2210	8.13008130081301 \\
2220	8.13008130081301 \\
2230	8.13008130081301 \\
2240	8.13008130081301 \\
2250	8.13008130081301 \\
2260	8.13008130081301 \\
2270	8.13008130081301 \\
2280	8.13008130081301 \\
2290	8.13008130081301 \\
2300	8.13008130081301 \\
2310	8.13008130081301 \\
2320	8.13008130081301 \\
2330	8.13008130081301 \\
2340	8.13008130081301 \\
2350	8.13008130081301 \\
2360	8.13008130081301 \\
2370	8.13008130081301 \\
2380	8.13008130081301 \\
2390	8.13008130081301 \\
2400	8.13008130081301 \\
2410	8.13008130081301 \\
2420	8.13008130081301 \\
2430	8.13008130081301 \\
2440	8.13008130081301 \\
2450	8.13008130081301 \\
2460	8.13008130081301 \\
2470	8.13008130081301 \\
2480	8.13008130081301 \\
2490	8.13008130081301 \\
2500	8.13008130081301 \\
2510	8.13008130081301 \\
2520	8.13008130081301 \\
2530	8.13008130081301 \\
2540	8.13008130081301 \\
2550	8.13008130081301 \\
2560	8.13008130081301 \\
2570	8.13008130081301 \\
2580	8.13008130081301 \\
2590	8.13008130081301 \\
2600	8.13008130081301 \\
2610	8.13008130081301 \\
2620	8.13008130081301 \\
2630	8.13008130081301 \\
2640	8.13008130081301 \\
2650	8.13008130081301 \\
2660	8.13008130081301 \\
2670	8.13008130081301 \\
2680	8.13008130081301 \\
2690	8.13008130081301 \\
2700	8.13008130081301 \\
2710	8.13008130081301 \\
2720	8.13008130081301 \\
2730	8.13008130081301 \\
2740	8.13008130081301 \\
2750	8.13008130081301 \\
2760	8.13008130081301 \\
2770	8.13008130081301 \\
2780	8.13008130081301 \\
2790	8.13008130081301 \\
2800	8.13008130081301 \\
2810	8.13008130081301 \\
2820	8.13008130081301 \\
2830	8.13008130081301 \\
2840	8.13008130081301 \\
2850	8.13008130081301 \\
2860	8.13008130081301 \\
2870	8.13008130081301 \\
2880	8.13008130081301 \\
2890	8.13008130081301 \\
2900	8.13008130081301 \\
2910	8.13008130081301 \\
2920	8.13008130081301 \\
2930	8.13008130081301 \\
2940	8.13008130081301 \\
2950	8.13008130081301 \\
2960	8.13008130081301 \\
2970	8.13008130081301 \\
2980	8.13008130081301 \\
2990	8.13008130081301 \\
3000	8.13008130081301 \\
3010	8.13008130081301 \\
3020	8.13008130081301 \\
3030	8.13008130081301 \\
3040	8.13008130081301 \\
3050	8.13008130081301 \\
3060	8.13008130081301 \\
3070	8.13008130081301 \\
3080	8.13008130081301 \\
3090	8.13008130081301 \\
3100	8.13008130081301 \\
3110	8.13008130081301 \\
3120	8.13008130081301 \\
3130	8.13008130081301 \\
3140	8.13008130081301 \\
3150	8.13008130081301 \\
3160	8.13008130081301 \\
3170	8.13008130081301 \\
3180	8.13008130081301 \\
3190	8.13008130081301 \\
3200	8.13008130081301 \\
3210	8.13008130081301 \\
3220	8.13008130081301 \\
3230	8.13008130081301 \\
3240	8.13008130081301 \\
3250	8.13008130081301 \\
3260	8.13008130081301 \\
3270	8.13008130081301 \\
3280	8.13008130081301 \\
3290	8.13008130081301 \\
3300	8.13008130081301 \\
3310	8.13008130081301 \\
3320	8.13008130081301 \\
3330	8.13008130081301 \\
3340	8.13008130081301 \\
3350	8.13008130081301 \\
3360	8.13008130081301 \\
3370	8.13008130081301 \\
3380	8.13008130081301 \\
3390	8.13008130081301 \\
3400	8.13008130081301 \\
3410	8.13008130081301 \\
3420	8.13008130081301 \\
3430	8.13008130081301 \\
3440	8.13008130081301 \\
3450	8.13008130081301 \\
3460	8.13008130081301 \\
3470	8.13008130081301 \\
3480	8.13008130081301 \\
3490	8.13008130081301 \\
3500	8.13008130081301 \\
3510	8.13008130081301 \\
3520	8.13008130081301 \\
3530	8.13008130081301 \\
3540	8.13008130081301 \\
3550	8.13008130081301 \\
3560	8.13008130081301 \\
3570	8.13008130081301 \\
3580	8.13008130081301 \\
3590	8.13008130081301 \\
3600	8.13008130081301 \\
3610	8.13008130081301 \\
3620	8.13008130081301 \\
3630	8.13008130081301 \\
3640	8.13008130081301 \\
3650	8.13008130081301 \\
3660	8.13008130081301 \\
3670	8.13008130081301 \\
3680	8.13008130081301 \\
3690	8.13008130081301 \\
3700	8.13008130081301 \\
3710	8.13008130081301 \\
3720	8.13008130081301 \\
3730	8.13008130081301 \\
3740	8.13008130081301 \\
3750	8.13008130081301 \\
3760	8.13008130081301 \\
3770	8.13008130081301 \\
3780	8.13008130081301 \\
3790	8.13008130081301 \\
3800	8.13008130081301 \\
3810	8.13008130081301 \\
3820	8.13008130081301 \\
3830	8.13008130081301 \\
3840	8.13008130081301 \\
3850	8.13008130081301 \\
3860	8.13008130081301 \\
3870	8.13008130081301 \\
3880	8.13008130081301 \\
3890	8.13008130081301 \\
3900	8.13008130081301 \\
3910	8.13008130081301 \\
3920	8.13008130081301 \\
3930	8.13008130081301 \\
3940	8.13008130081301 \\
3950	8.13008130081301 \\
3960	8.13008130081301 \\
3970	8.13008130081301 \\
3980	8.13008130081301 \\
3990	8.13008130081301 \\
};
\addplot [blue,mark=*,mark repeat = 100]
table [row sep=\\]{%
0	95.1219512195122 \\
10	94.3089430894309 \\
20	91.869918699187 \\
30	91.0569105691057 \\
40	90.2439024390244 \\
50	86.9918699186992 \\
60	85.3658536585366 \\
70	86.1788617886179 \\
80	83.739837398374 \\
90	79.6747967479675 \\
100	77.2357723577236 \\
110	75.609756097561 \\
120	70.7317073170732 \\
130	70.7317073170732 \\
140	69.9186991869919 \\
150	68.2926829268293 \\
160	68.2926829268293 \\
170	62.6016260162602 \\
180	62.6016260162602 \\
190	60.1626016260163 \\
200	59.349593495935 \\
210	56.0975609756098 \\
220	52.0325203252033 \\
230	52.0325203252033 \\
240	52.8455284552846 \\
250	48.780487804878 \\
260	47.9674796747967 \\
270	47.9674796747967 \\
280	48.780487804878 \\
290	47.9674796747967 \\
300	47.1544715447154 \\
310	45.5284552845528 \\
320	46.3414634146341 \\
330	45.5284552845528 \\
340	43.0894308943089 \\
350	43.0894308943089 \\
360	41.4634146341463 \\
370	42.2764227642276 \\
380	41.4634146341463 \\
390	41.4634146341463 \\
400	40.650406504065 \\
410	42.2764227642276 \\
420	39.8373983739837 \\
430	39.8373983739837 \\
440	39.8373983739837 \\
450	39.0243902439024 \\
460	39.0243902439024 \\
470	38.2113821138211 \\
480	38.2113821138211 \\
490	38.2113821138211 \\
500	39.8373983739837 \\
510	39.0243902439024 \\
520	38.2113821138211 \\
530	35.7723577235772 \\
540	36.5853658536585 \\
550	35.7723577235772 \\
560	34.9593495934959 \\
570	31.7073170731707 \\
580	31.7073170731707 \\
590	31.7073170731707 \\
600	31.7073170731707 \\
610	30.0813008130081 \\
620	30.8943089430894 \\
630	29.2682926829268 \\
640	31.7073170731707 \\
650	31.7073170731707 \\
660	30.8943089430894 \\
670	30.0813008130081 \\
680	29.2682926829268 \\
690	30.0813008130081 \\
700	30.0813008130081 \\
710	30.0813008130081 \\
720	28.4552845528455 \\
730	27.6422764227642 \\
740	27.6422764227642 \\
750	26.8292682926829 \\
760	27.6422764227642 \\
770	28.4552845528455 \\
780	26.8292682926829 \\
790	26.8292682926829 \\
800	27.6422764227642 \\
810	22.7642276422764 \\
820	24.390243902439 \\
830	24.390243902439 \\
840	24.390243902439 \\
850	25.2032520325203 \\
860	24.390243902439 \\
870	23.5772357723577 \\
880	21.9512195121951 \\
890	21.9512195121951 \\
900	22.7642276422764 \\
910	21.9512195121951 \\
920	21.9512195121951 \\
930	21.9512195121951 \\
940	23.5772357723577 \\
950	23.5772357723577 \\
960	22.7642276422764 \\
970	25.2032520325203 \\
980	24.390243902439 \\
990	23.5772357723577 \\
1000	21.9512195121951 \\
1010	23.5772357723577 \\
1020	21.9512195121951 \\
1030	21.9512195121951 \\
1040	21.9512195121951 \\
1050	22.7642276422764 \\
1060	24.390243902439 \\
1070	21.9512195121951 \\
1080	20.3252032520325 \\
1090	22.7642276422764 \\
1100	21.1382113821138 \\
1110	21.9512195121951 \\
1120	20.3252032520325 \\
1130	19.5121951219512 \\
1140	20.3252032520325 \\
1150	20.3252032520325 \\
1160	19.5121951219512 \\
1170	20.3252032520325 \\
1180	20.3252032520325 \\
1190	18.6991869918699 \\
1200	20.3252032520325 \\
1210	18.6991869918699 \\
1220	18.6991869918699 \\
1230	20.3252032520325 \\
1240	20.3252032520325 \\
1250	20.3252032520325 \\
1260	20.3252032520325 \\
1270	20.3252032520325 \\
1280	20.3252032520325 \\
1290	20.3252032520325 \\
1300	20.3252032520325 \\
1310	20.3252032520325 \\
1320	18.6991869918699 \\
1330	20.3252032520325 \\
1340	20.3252032520325 \\
1350	20.3252032520325 \\
1360	20.3252032520325 \\
1370	20.3252032520325 \\
1380	18.6991869918699 \\
1390	20.3252032520325 \\
1400	18.6991869918699 \\
1410	18.6991869918699 \\
1420	17.0731707317073 \\
1430	18.6991869918699 \\
1440	18.6991869918699 \\
1450	17.0731707317073 \\
1460	17.0731707317073 \\
1470	17.0731707317073 \\
1480	17.0731707317073 \\
1490	17.0731707317073 \\
1500	17.0731707317073 \\
1510	17.0731707317073 \\
1520	17.0731707317073 \\
1530	17.0731707317073 \\
1540	18.6991869918699 \\
1550	18.6991869918699 \\
1560	17.0731707317073 \\
1570	18.6991869918699 \\
1580	18.6991869918699 \\
1590	17.0731707317073 \\
1600	18.6991869918699 \\
1610	18.6991869918699 \\
1620	18.6991869918699 \\
1630	18.6991869918699 \\
1640	18.6991869918699 \\
1650	18.6991869918699 \\
1660	18.6991869918699 \\
1670	18.6991869918699 \\
1680	19.5121951219512 \\
1690	18.6991869918699 \\
1700	17.8861788617886 \\
1710	17.8861788617886 \\
1720	17.8861788617886 \\
1730	18.6991869918699 \\
1740	18.6991869918699 \\
1750	18.6991869918699 \\
1760	18.6991869918699 \\
1770	18.6991869918699 \\
1780	18.6991869918699 \\
1790	17.0731707317073 \\
1800	16.260162601626 \\
1810	16.260162601626 \\
1820	16.260162601626 \\
1830	17.8861788617886 \\
1840	17.8861788617886 \\
1850	17.8861788617886 \\
1860	17.8861788617886 \\
1870	17.0731707317073 \\
1880	17.0731707317073 \\
1890	17.8861788617886 \\
1900	17.8861788617886 \\
1910	15.4471544715447 \\
1920	14.6341463414634 \\
1930	13.8211382113821 \\
1940	14.6341463414634 \\
1950	13.8211382113821 \\
1960	13.8211382113821 \\
1970	14.6341463414634 \\
1980	13.8211382113821 \\
1990	13.0081300813008 \\
2000	13.0081300813008 \\
2010	13.8211382113821 \\
2020	13.0081300813008 \\
2030	13.0081300813008 \\
2040	13.0081300813008 \\
2050	13.0081300813008 \\
2060	13.8211382113821 \\
2070	13.8211382113821 \\
2080	13.8211382113821 \\
2090	13.0081300813008 \\
2100	13.0081300813008 \\
2110	12.1951219512195 \\
2120	13.0081300813008 \\
2130	13.0081300813008 \\
2140	12.1951219512195 \\
2150	12.1951219512195 \\
2160	12.1951219512195 \\
2170	12.1951219512195 \\
2180	12.1951219512195 \\
2190	12.1951219512195 \\
2200	13.0081300813008 \\
2210	12.1951219512195 \\
2220	12.1951219512195 \\
2230	12.1951219512195 \\
2240	12.1951219512195 \\
2250	12.1951219512195 \\
2260	13.0081300813008 \\
2270	13.0081300813008 \\
2280	12.1951219512195 \\
2290	12.1951219512195 \\
2300	11.3821138211382 \\
2310	11.3821138211382 \\
2320	11.3821138211382 \\
2330	11.3821138211382 \\
2340	11.3821138211382 \\
2350	11.3821138211382 \\
2360	11.3821138211382 \\
2370	11.3821138211382 \\
2380	11.3821138211382 \\
2390	11.3821138211382 \\
2400	11.3821138211382 \\
2410	11.3821138211382 \\
2420	11.3821138211382 \\
2430	11.3821138211382 \\
2440	11.3821138211382 \\
2450	11.3821138211382 \\
2460	12.1951219512195 \\
2470	11.3821138211382 \\
2480	11.3821138211382 \\
2490	11.3821138211382 \\
2500	11.3821138211382 \\
2510	11.3821138211382 \\
2520	12.1951219512195 \\
2530	11.3821138211382 \\
2540	11.3821138211382 \\
2550	11.3821138211382 \\
2560	11.3821138211382 \\
2570	11.3821138211382 \\
2580	11.3821138211382 \\
2590	11.3821138211382 \\
2600	11.3821138211382 \\
2610	11.3821138211382 \\
2620	11.3821138211382 \\
2630	11.3821138211382 \\
2640	11.3821138211382 \\
2650	11.3821138211382 \\
2660	11.3821138211382 \\
2670	11.3821138211382 \\
2680	11.3821138211382 \\
2690	11.3821138211382 \\
2700	11.3821138211382 \\
2710	11.3821138211382 \\
2720	11.3821138211382 \\
2730	11.3821138211382 \\
2740	11.3821138211382 \\
2750	11.3821138211382 \\
2760	11.3821138211382 \\
2770	10.5691056910569 \\
2780	10.5691056910569 \\
2790	11.3821138211382 \\
2800	11.3821138211382 \\
2810	10.5691056910569 \\
2820	10.5691056910569 \\
2830	10.5691056910569 \\
2840	10.5691056910569 \\
2850	10.5691056910569 \\
2860	10.5691056910569 \\
2870	10.5691056910569 \\
2880	10.5691056910569 \\
2890	10.5691056910569 \\
2900	10.5691056910569 \\
2910	10.5691056910569 \\
2920	10.5691056910569 \\
2930	10.5691056910569 \\
2940	10.5691056910569 \\
2950	10.5691056910569 \\
2960	11.3821138211382 \\
2970	10.5691056910569 \\
2980	10.5691056910569 \\
2990	10.5691056910569 \\
3000	10.5691056910569 \\
3010	10.5691056910569 \\
3020	10.5691056910569 \\
3030	10.5691056910569 \\
3040	10.5691056910569 \\
3050	10.5691056910569 \\
3060	10.5691056910569 \\
3070	9.75609756097561 \\
3080	9.75609756097561 \\
3090	9.75609756097561 \\
3100	9.75609756097561 \\
3110	9.75609756097561 \\
3120	9.75609756097561 \\
3130	9.75609756097561 \\
3140	9.75609756097561 \\
3150	9.75609756097561 \\
3160	9.75609756097561 \\
3170	9.75609756097561 \\
3180	9.75609756097561 \\
3190	9.75609756097561 \\
3200	9.75609756097561 \\
3210	9.75609756097561 \\
3220	9.75609756097561 \\
3230	9.75609756097561 \\
3240	9.75609756097561 \\
3250	9.75609756097561 \\
3260	9.75609756097561 \\
3270	9.75609756097561 \\
3280	9.75609756097561 \\
3290	9.75609756097561 \\
3300	9.75609756097561 \\
3310	9.75609756097561 \\
3320	9.75609756097561 \\
3330	9.75609756097561 \\
3340	9.75609756097561 \\
3350	9.75609756097561 \\
3360	9.75609756097561 \\
3370	9.75609756097561 \\
3380	9.75609756097561 \\
3390	9.75609756097561 \\
3400	9.75609756097561 \\
3410	9.75609756097561 \\
3420	9.75609756097561 \\
3430	9.75609756097561 \\
3440	9.75609756097561 \\
3450	9.75609756097561 \\
3460	9.75609756097561 \\
3470	9.75609756097561 \\
3480	9.75609756097561 \\
3490	9.75609756097561 \\
3500	9.75609756097561 \\
3510	9.75609756097561 \\
3520	9.75609756097561 \\
3530	9.75609756097561 \\
3540	9.75609756097561 \\
3550	9.75609756097561 \\
3560	9.75609756097561 \\
3570	9.75609756097561 \\
3580	9.75609756097561 \\
3590	9.75609756097561 \\
3600	9.75609756097561 \\
3610	9.75609756097561 \\
3620	9.75609756097561 \\
3630	9.75609756097561 \\
3640	9.75609756097561 \\
3650	9.75609756097561 \\
3660	9.75609756097561 \\
3670	9.75609756097561 \\
3680	9.75609756097561 \\
3690	9.75609756097561 \\
3700	9.75609756097561 \\
3710	9.75609756097561 \\
3720	9.75609756097561 \\
3730	9.75609756097561 \\
3740	9.75609756097561 \\
3750	9.75609756097561 \\
3760	9.75609756097561 \\
3770	9.75609756097561 \\
3780	9.75609756097561 \\
3790	9.75609756097561 \\
3800	9.75609756097561 \\
3810	9.75609756097561 \\
3820	9.75609756097561 \\
3830	9.75609756097561 \\
3840	9.75609756097561 \\
3850	9.75609756097561 \\
3860	9.75609756097561 \\
3870	9.75609756097561 \\
3880	9.75609756097561 \\
3890	9.75609756097561 \\
3900	9.75609756097561 \\
3910	9.75609756097561 \\
3920	9.75609756097561 \\
3930	9.75609756097561 \\
3940	9.75609756097561 \\
3950	9.75609756097561 \\
3960	9.75609756097561 \\
3970	9.75609756097561 \\
3980	9.75609756097561 \\
3990	9.75609756097561 \\
};



\addplot [ red]
table [row sep=\\]{%
0	91.869918699187 \\
10	89.4308943089431 \\
20	83.739837398374 \\
30	78.0487804878049 \\
40	73.9837398373984 \\
50	70.7317073170732 \\
60	65.8536585365854 \\
70	65.8536585365854 \\
80	63.4146341463415 \\
90	60.1626016260163 \\
100	59.349593495935 \\
110	56.0975609756098 \\
120	52.0325203252033 \\
130	51.219512195122 \\
140	48.780487804878 \\
150	46.3414634146341 \\
160	42.2764227642276 \\
170	36.5853658536585 \\
180	34.1463414634146 \\
190	30.8943089430894 \\
200	29.2682926829268 \\
210	26.8292682926829 \\
220	26.8292682926829 \\
230	25.2032520325203 \\
240	26.8292682926829 \\
250	25.2032520325203 \\
260	22.7642276422764 \\
270	22.7642276422764 \\
280	20.3252032520325 \\
290	19.5121951219512 \\
300	19.5121951219512 \\
310	17.8861788617886 \\
320	18.6991869918699 \\
330	18.6991869918699 \\
340	18.6991869918699 \\
350	18.6991869918699 \\
360	17.8861788617886 \\
370	17.0731707317073 \\
380	17.8861788617886 \\
390	16.260162601626 \\
400	14.6341463414634 \\
410	13.8211382113821 \\
420	13.8211382113821 \\
430	13.8211382113821 \\
440	13.0081300813008 \\
450	13.0081300813008 \\
460	13.0081300813008 \\
470	13.0081300813008 \\
480	13.0081300813008 \\
490	12.1951219512195 \\
500	12.1951219512195 \\
510	12.1951219512195 \\
520	12.1951219512195 \\
530	12.1951219512195 \\
540	12.1951219512195 \\
550	12.1951219512195 \\
560	11.3821138211382 \\
570	10.5691056910569 \\
580	10.5691056910569 \\
590	10.5691056910569 \\
600	10.5691056910569 \\
610	10.5691056910569 \\
620	10.5691056910569 \\
630	10.5691056910569 \\
640	10.5691056910569 \\
650	10.5691056910569 \\
660	10.5691056910569 \\
670	10.5691056910569 \\
680	9.75609756097561 \\
690	9.75609756097561 \\
700	9.75609756097561 \\
710	9.75609756097561 \\
720	9.75609756097561 \\
730	9.75609756097561 \\
740	9.75609756097561 \\
750	9.75609756097561 \\
760	9.75609756097561 \\
770	9.75609756097561 \\
780	9.75609756097561 \\
790	9.75609756097561 \\
800	9.75609756097561 \\
810	9.75609756097561 \\
820	9.75609756097561 \\
830	9.75609756097561 \\
840	9.75609756097561 \\
850	9.75609756097561 \\
860	9.75609756097561 \\
870	9.75609756097561 \\
880	9.75609756097561 \\
890	9.75609756097561 \\
900	9.75609756097561 \\
910	9.75609756097561 \\
920	9.75609756097561 \\
930	9.75609756097561 \\
940	9.75609756097561 \\
950	9.75609756097561 \\
960	9.75609756097561 \\
970	9.75609756097561 \\
980	9.75609756097561 \\
990	9.75609756097561 \\
1000	9.75609756097561 \\
1010	9.75609756097561 \\
1020	9.75609756097561 \\
1030	9.75609756097561 \\
1040	9.75609756097561 \\
1050	9.75609756097561 \\
1060	9.75609756097561 \\
1070	9.75609756097561 \\
1080	9.75609756097561 \\
1090	9.75609756097561 \\
1100	9.75609756097561 \\
1110	9.75609756097561 \\
1120	9.75609756097561 \\
1130	9.75609756097561 \\
1140	9.75609756097561 \\
1150	9.75609756097561 \\
1160	9.75609756097561 \\
1170	9.75609756097561 \\
1180	9.75609756097561 \\
1190	9.75609756097561 \\
1200	9.75609756097561 \\
1210	8.94308943089431 \\
1220	8.94308943089431 \\
1230	8.94308943089431 \\
1240	8.94308943089431 \\
1250	8.94308943089431 \\
1260	8.94308943089431 \\
1270	8.94308943089431 \\
1280	8.94308943089431 \\
1290	8.94308943089431 \\
1300	8.94308943089431 \\
1310	8.94308943089431 \\
1320	8.94308943089431 \\
1330	8.94308943089431 \\
1340	8.94308943089431 \\
1350	8.94308943089431 \\
1360	8.13008130081301 \\
1370	8.13008130081301 \\
1380	8.13008130081301 \\
1390	8.13008130081301 \\
1400	8.13008130081301 \\
1410	8.13008130081301 \\
1420	8.13008130081301 \\
1430	8.13008130081301 \\
1440	8.13008130081301 \\
1450	8.13008130081301 \\
1460	8.13008130081301 \\
1470	8.13008130081301 \\
1480	8.13008130081301 \\
1490	8.13008130081301 \\
1500	8.13008130081301 \\
1510	8.13008130081301 \\
1520	8.13008130081301 \\
1530	8.13008130081301 \\
1540	8.13008130081301 \\
1550	8.13008130081301 \\
1560	8.13008130081301 \\
1570	8.13008130081301 \\
1580	8.13008130081301 \\
1590	8.13008130081301 \\
1600	8.13008130081301 \\
1610	8.13008130081301 \\
1620	8.13008130081301 \\
1630	8.13008130081301 \\
1640	8.13008130081301 \\
1650	8.13008130081301 \\
1660	8.13008130081301 \\
1670	8.13008130081301 \\
1680	8.13008130081301 \\
1690	8.13008130081301 \\
1700	8.13008130081301 \\
1710	8.13008130081301 \\
1720	8.13008130081301 \\
1730	8.13008130081301 \\
1740	8.13008130081301 \\
1750	8.13008130081301 \\
1760	8.13008130081301 \\
1770	8.13008130081301 \\
1780	8.13008130081301 \\
1790	8.13008130081301 \\
1800	8.13008130081301 \\
1810	8.13008130081301 \\
1820	8.13008130081301 \\
1830	8.13008130081301 \\
1840	8.13008130081301 \\
1850	8.13008130081301 \\
1860	8.13008130081301 \\
1870	8.13008130081301 \\
1880	8.13008130081301 \\
1890	8.13008130081301 \\
1900	8.13008130081301 \\
1910	8.13008130081301 \\
1920	8.13008130081301 \\
1930	8.13008130081301 \\
1940	8.13008130081301 \\
1950	8.13008130081301 \\
1960	8.13008130081301 \\
1970	8.13008130081301 \\
1980	8.13008130081301 \\
1990	8.13008130081301 \\
2000	8.13008130081301 \\
2010	8.13008130081301 \\
2020	8.13008130081301 \\
2030	8.13008130081301 \\
2040	8.13008130081301 \\
2050	8.13008130081301 \\
2060	8.13008130081301 \\
2070	8.13008130081301 \\
2080	8.13008130081301 \\
2090	8.13008130081301 \\
2100	8.13008130081301 \\
2110	8.13008130081301 \\
2120	8.13008130081301 \\
2130	8.13008130081301 \\
2140	8.13008130081301 \\
2150	8.13008130081301 \\
2160	8.13008130081301 \\
2170	8.13008130081301 \\
2180	8.13008130081301 \\
2190	8.13008130081301 \\
2200	8.13008130081301 \\
2210	8.13008130081301 \\
2220	8.13008130081301 \\
2230	8.13008130081301 \\
2240	8.13008130081301 \\
2250	8.13008130081301 \\
2260	8.13008130081301 \\
2270	8.13008130081301 \\
2280	8.13008130081301 \\
2290	8.13008130081301 \\
2300	8.13008130081301 \\
2310	8.13008130081301 \\
2320	8.13008130081301 \\
2330	8.13008130081301 \\
2340	8.13008130081301 \\
2350	8.13008130081301 \\
2360	8.13008130081301 \\
2370	8.13008130081301 \\
2380	8.13008130081301 \\
2390	8.13008130081301 \\
2400	8.13008130081301 \\
2410	8.13008130081301 \\
2420	8.13008130081301 \\
2430	8.13008130081301 \\
2440	8.13008130081301 \\
2450	8.13008130081301 \\
2460	8.13008130081301 \\
2470	8.13008130081301 \\
2480	8.13008130081301 \\
2490	8.13008130081301 \\
2500	8.13008130081301 \\
2510	8.13008130081301 \\
2520	8.13008130081301 \\
2530	8.13008130081301 \\
2540	8.13008130081301 \\
2550	8.13008130081301 \\
2560	8.13008130081301 \\
2570	8.13008130081301 \\
2580	8.13008130081301 \\
2590	8.13008130081301 \\
2600	8.13008130081301 \\
2610	8.13008130081301 \\
2620	8.13008130081301 \\
2630	8.13008130081301 \\
2640	8.13008130081301 \\
2650	8.13008130081301 \\
2660	8.13008130081301 \\
2670	8.13008130081301 \\
2680	8.13008130081301 \\
2690	8.13008130081301 \\
2700	8.13008130081301 \\
2710	8.13008130081301 \\
2720	8.13008130081301 \\
2730	8.13008130081301 \\
2740	8.13008130081301 \\
2750	8.13008130081301 \\
2760	8.13008130081301 \\
2770	8.13008130081301 \\
2780	8.13008130081301 \\
2790	8.13008130081301 \\
2800	8.13008130081301 \\
2810	8.13008130081301 \\
2820	8.13008130081301 \\
2830	8.13008130081301 \\
2840	8.13008130081301 \\
2850	8.13008130081301 \\
2860	8.13008130081301 \\
2870	8.13008130081301 \\
2880	8.13008130081301 \\
2890	8.13008130081301 \\
2900	8.13008130081301 \\
2910	8.13008130081301 \\
2920	8.13008130081301 \\
2930	8.13008130081301 \\
2940	8.13008130081301 \\
2950	8.13008130081301 \\
2960	8.13008130081301 \\
2970	8.13008130081301 \\
2980	8.13008130081301 \\
2990	8.13008130081301 \\
3000	8.13008130081301 \\
3010	8.13008130081301 \\
3020	8.13008130081301 \\
3030	8.13008130081301 \\
3040	8.13008130081301 \\
3050	8.13008130081301 \\
3060	8.13008130081301 \\
3070	8.13008130081301 \\
3080	8.13008130081301 \\
3090	8.13008130081301 \\
3100	8.13008130081301 \\
3110	8.13008130081301 \\
3120	8.13008130081301 \\
3130	8.13008130081301 \\
3140	8.13008130081301 \\
3150	8.13008130081301 \\
3160	8.13008130081301 \\
3170	8.13008130081301 \\
3180	8.13008130081301 \\
3190	8.13008130081301 \\
3200	8.13008130081301 \\
3210	8.13008130081301 \\
3220	8.13008130081301 \\
3230	8.13008130081301 \\
3240	8.13008130081301 \\
3250	8.13008130081301 \\
3260	8.13008130081301 \\
3270	8.13008130081301 \\
3280	8.13008130081301 \\
3290	8.13008130081301 \\
3300	8.13008130081301 \\
3310	8.13008130081301 \\
3320	8.13008130081301 \\
3330	8.13008130081301 \\
3340	8.13008130081301 \\
3350	8.13008130081301 \\
3360	8.13008130081301 \\
3370	8.13008130081301 \\
3380	8.13008130081301 \\
3390	8.13008130081301 \\
3400	8.13008130081301 \\
3410	8.13008130081301 \\
3420	8.13008130081301 \\
3430	8.13008130081301 \\
3440	8.13008130081301 \\
3450	8.13008130081301 \\
3460	8.13008130081301 \\
3470	8.13008130081301 \\
3480	8.13008130081301 \\
3490	8.13008130081301 \\
3500	8.13008130081301 \\
3510	8.13008130081301 \\
3520	8.13008130081301 \\
3530	8.13008130081301 \\
3540	8.13008130081301 \\
3550	8.13008130081301 \\
3560	8.13008130081301 \\
3570	8.13008130081301 \\
3580	8.13008130081301 \\
3590	8.13008130081301 \\
3600	8.13008130081301 \\
3610	8.13008130081301 \\
3620	8.13008130081301 \\
3630	8.13008130081301 \\
3640	8.13008130081301 \\
3650	8.13008130081301 \\
3660	8.13008130081301 \\
3670	8.13008130081301 \\
3680	8.13008130081301 \\
3690	8.13008130081301 \\
3700	8.13008130081301 \\
3710	8.13008130081301 \\
3720	8.13008130081301 \\
3730	8.13008130081301 \\
3740	8.13008130081301 \\
3750	8.13008130081301 \\
3760	8.13008130081301 \\
3770	8.13008130081301 \\
3780	8.13008130081301 \\
3790	8.13008130081301 \\
3800	8.13008130081301 \\
3810	8.13008130081301 \\
3820	8.13008130081301 \\
3830	8.13008130081301 \\
3840	8.13008130081301 \\
3850	8.13008130081301 \\
3860	8.13008130081301 \\
3870	8.13008130081301 \\
3880	8.13008130081301 \\
3890	8.13008130081301 \\
3900	8.13008130081301 \\
3910	8.13008130081301 \\
3920	8.13008130081301 \\
3930	8.13008130081301 \\
3940	8.13008130081301 \\
3950	8.13008130081301 \\
3960	8.13008130081301 \\
3970	8.13008130081301 \\
3980	8.13008130081301 \\
3990	8.13008130081301 \\
};


\addplot [red,mark=text, text mark={\tiny 10\%},mark repeat = 100]
table [row sep=\\]{%
0	94.3089430894309 \\
10	89.4308943089431 \\
20	79.6747967479675 \\
30	76.4227642276423 \\
40	69.9186991869919 \\
50	66.6666666666667 \\
60	63.4146341463415 \\
70	61.7886178861789 \\
80	62.6016260162602 \\
90	55.2845528455285 \\
100	50.4065040650407 \\
110	46.3414634146341 \\
120	44.7154471544715 \\
130	39.0243902439024 \\
140	39.0243902439024 \\
150	32.520325203252 \\
160	29.2682926829268 \\
170	30.0813008130081 \\
180	26.0162601626016 \\
190	25.2032520325203 \\
200	25.2032520325203 \\
210	22.7642276422764 \\
220	22.7642276422764 \\
230	21.9512195121951 \\
240	19.5121951219512 \\
250	18.6991869918699 \\
260	18.6991869918699 \\
270	18.6991869918699 \\
280	17.8861788617886 \\
290	17.8861788617886 \\
300	17.8861788617886 \\
310	17.8861788617886 \\
320	18.6991869918699 \\
330	17.8861788617886 \\
340	17.8861788617886 \\
350	17.8861788617886 \\
360	17.8861788617886 \\
370	17.8861788617886 \\
380	17.8861788617886 \\
390	17.0731707317073 \\
400	16.260162601626 \\
410	16.260162601626 \\
420	16.260162601626 \\
430	14.6341463414634 \\
440	14.6341463414634 \\
450	14.6341463414634 \\
460	13.8211382113821 \\
470	13.8211382113821 \\
480	13.8211382113821 \\
490	13.8211382113821 \\
500	13.8211382113821 \\
510	13.0081300813008 \\
520	13.0081300813008 \\
530	13.0081300813008 \\
540	13.0081300813008 \\
550	13.0081300813008 \\
560	13.0081300813008 \\
570	13.0081300813008 \\
580	13.0081300813008 \\
590	13.0081300813008 \\
600	13.0081300813008 \\
610	13.0081300813008 \\
620	12.1951219512195 \\
630	11.3821138211382 \\
640	11.3821138211382 \\
650	11.3821138211382 \\
660	11.3821138211382 \\
670	11.3821138211382 \\
680	11.3821138211382 \\
690	11.3821138211382 \\
700	10.5691056910569 \\
710	10.5691056910569 \\
720	9.75609756097561 \\
730	9.75609756097561 \\
740	9.75609756097561 \\
750	8.94308943089431 \\
760	8.94308943089431 \\
770	8.94308943089431 \\
780	8.94308943089431 \\
790	8.94308943089431 \\
800	8.94308943089431 \\
810	8.94308943089431 \\
820	8.94308943089431 \\
830	8.94308943089431 \\
840	8.94308943089431 \\
850	8.94308943089431 \\
860	8.94308943089431 \\
870	8.94308943089431 \\
880	8.94308943089431 \\
890	8.94308943089431 \\
900	8.94308943089431 \\
910	8.94308943089431 \\
920	8.94308943089431 \\
930	8.94308943089431 \\
940	8.94308943089431 \\
950	8.94308943089431 \\
960	8.94308943089431 \\
970	8.94308943089431 \\
980	8.94308943089431 \\
990	8.94308943089431 \\
1000	8.94308943089431 \\
1010	8.94308943089431 \\
1020	8.94308943089431 \\
1030	8.94308943089431 \\
1040	8.94308943089431 \\
1050	8.94308943089431 \\
1060	8.94308943089431 \\
1070	8.94308943089431 \\
1080	8.94308943089431 \\
1090	8.94308943089431 \\
1100	8.94308943089431 \\
1110	8.94308943089431 \\
1120	8.94308943089431 \\
1130	8.94308943089431 \\
1140	8.94308943089431 \\
1150	8.94308943089431 \\
1160	8.94308943089431 \\
1170	8.94308943089431 \\
1180	8.94308943089431 \\
1190	8.94308943089431 \\
1200	8.94308943089431 \\
1210	8.94308943089431 \\
1220	8.94308943089431 \\
1230	8.94308943089431 \\
1240	8.94308943089431 \\
1250	8.94308943089431 \\
1260	8.94308943089431 \\
1270	8.94308943089431 \\
1280	8.94308943089431 \\
1290	8.94308943089431 \\
1300	8.94308943089431 \\
1310	8.94308943089431 \\
1320	8.94308943089431 \\
1330	8.94308943089431 \\
1340	8.94308943089431 \\
1350	8.94308943089431 \\
1360	8.94308943089431 \\
1370	8.94308943089431 \\
1380	8.13008130081301 \\
1390	8.13008130081301 \\
1400	8.13008130081301 \\
1410	8.13008130081301 \\
1420	8.13008130081301 \\
1430	8.13008130081301 \\
1440	8.13008130081301 \\
1450	8.13008130081301 \\
1460	8.13008130081301 \\
1470	8.13008130081301 \\
1480	8.13008130081301 \\
1490	8.13008130081301 \\
1500	8.13008130081301 \\
1510	8.13008130081301 \\
1520	8.13008130081301 \\
1530	8.13008130081301 \\
1540	8.13008130081301 \\
1550	8.13008130081301 \\
1560	8.13008130081301 \\
1570	8.13008130081301 \\
1580	8.13008130081301 \\
1590	8.13008130081301 \\
1600	8.13008130081301 \\
1610	8.13008130081301 \\
1620	8.13008130081301 \\
1630	8.13008130081301 \\
1640	8.13008130081301 \\
1650	8.13008130081301 \\
1660	8.13008130081301 \\
1670	8.13008130081301 \\
1680	8.13008130081301 \\
1690	8.13008130081301 \\
1700	8.13008130081301 \\
1710	8.13008130081301 \\
1720	8.13008130081301 \\
1730	8.13008130081301 \\
1740	8.13008130081301 \\
1750	8.13008130081301 \\
1760	8.13008130081301 \\
1770	8.13008130081301 \\
1780	8.13008130081301 \\
1790	8.13008130081301 \\
1800	8.13008130081301 \\
1810	8.13008130081301 \\
1820	8.13008130081301 \\
1830	8.13008130081301 \\
1840	8.13008130081301 \\
1850	8.13008130081301 \\
1860	8.13008130081301 \\
1870	8.13008130081301 \\
1880	8.13008130081301 \\
1890	8.13008130081301 \\
1900	8.13008130081301 \\
1910	8.13008130081301 \\
1920	8.13008130081301 \\
1930	8.13008130081301 \\
1940	8.13008130081301 \\
1950	8.13008130081301 \\
1960	8.13008130081301 \\
1970	8.13008130081301 \\
1980	8.13008130081301 \\
1990	8.13008130081301 \\
2000	8.13008130081301 \\
2010	8.13008130081301 \\
2020	8.13008130081301 \\
2030	8.13008130081301 \\
2040	8.13008130081301 \\
2050	8.13008130081301 \\
2060	8.13008130081301 \\
2070	8.13008130081301 \\
2080	8.13008130081301 \\
2090	8.13008130081301 \\
2100	8.13008130081301 \\
2110	8.13008130081301 \\
2120	8.13008130081301 \\
2130	8.13008130081301 \\
2140	8.13008130081301 \\
2150	8.13008130081301 \\
2160	8.13008130081301 \\
2170	8.13008130081301 \\
2180	8.13008130081301 \\
2190	8.13008130081301 \\
2200	8.13008130081301 \\
2210	8.13008130081301 \\
2220	8.13008130081301 \\
2230	8.13008130081301 \\
2240	8.13008130081301 \\
2250	8.13008130081301 \\
2260	8.13008130081301 \\
2270	8.13008130081301 \\
2280	8.13008130081301 \\
2290	8.13008130081301 \\
2300	8.13008130081301 \\
2310	8.13008130081301 \\
2320	8.13008130081301 \\
2330	8.13008130081301 \\
2340	8.13008130081301 \\
2350	8.13008130081301 \\
2360	8.13008130081301 \\
2370	8.13008130081301 \\
2380	8.13008130081301 \\
2390	8.13008130081301 \\
2400	8.13008130081301 \\
2410	8.13008130081301 \\
2420	8.13008130081301 \\
2430	8.13008130081301 \\
2440	8.13008130081301 \\
2450	8.13008130081301 \\
2460	8.13008130081301 \\
2470	8.13008130081301 \\
2480	8.13008130081301 \\
2490	8.13008130081301 \\
2500	8.13008130081301 \\
2510	8.13008130081301 \\
2520	8.13008130081301 \\
2530	8.13008130081301 \\
2540	8.13008130081301 \\
2550	8.13008130081301 \\
2560	8.13008130081301 \\
2570	8.13008130081301 \\
2580	8.13008130081301 \\
2590	8.13008130081301 \\
2600	8.13008130081301 \\
2610	8.13008130081301 \\
2620	8.13008130081301 \\
2630	8.13008130081301 \\
2640	8.13008130081301 \\
2650	8.13008130081301 \\
2660	8.13008130081301 \\
2670	8.13008130081301 \\
2680	8.13008130081301 \\
2690	8.13008130081301 \\
2700	8.13008130081301 \\
2710	8.13008130081301 \\
2720	8.13008130081301 \\
2730	8.13008130081301 \\
2740	8.13008130081301 \\
2750	8.13008130081301 \\
2760	8.13008130081301 \\
2770	8.13008130081301 \\
2780	8.13008130081301 \\
2790	8.13008130081301 \\
2800	8.13008130081301 \\
2810	8.13008130081301 \\
2820	8.13008130081301 \\
2830	8.13008130081301 \\
2840	8.13008130081301 \\
2850	8.13008130081301 \\
2860	8.13008130081301 \\
2870	8.13008130081301 \\
2880	8.13008130081301 \\
2890	8.13008130081301 \\
2900	8.13008130081301 \\
2910	8.13008130081301 \\
2920	8.13008130081301 \\
2930	8.13008130081301 \\
2940	8.13008130081301 \\
2950	8.13008130081301 \\
2960	8.13008130081301 \\
2970	8.13008130081301 \\
2980	8.13008130081301 \\
2990	8.13008130081301 \\
3000	8.13008130081301 \\
3010	8.13008130081301 \\
3020	8.13008130081301 \\
3030	8.13008130081301 \\
3040	8.13008130081301 \\
3050	8.13008130081301 \\
3060	8.13008130081301 \\
3070	8.13008130081301 \\
3080	8.13008130081301 \\
3090	8.13008130081301 \\
3100	8.13008130081301 \\
3110	8.13008130081301 \\
3120	8.13008130081301 \\
3130	8.13008130081301 \\
3140	8.13008130081301 \\
3150	8.13008130081301 \\
3160	8.13008130081301 \\
3170	8.13008130081301 \\
3180	8.13008130081301 \\
3190	8.13008130081301 \\
3200	8.13008130081301 \\
3210	8.13008130081301 \\
3220	8.13008130081301 \\
3230	8.13008130081301 \\
3240	8.13008130081301 \\
3250	8.13008130081301 \\
3260	8.13008130081301 \\
3270	8.13008130081301 \\
3280	8.13008130081301 \\
3290	8.13008130081301 \\
3300	8.13008130081301 \\
3310	8.13008130081301 \\
3320	8.13008130081301 \\
3330	8.13008130081301 \\
3340	8.13008130081301 \\
3350	8.13008130081301 \\
3360	8.13008130081301 \\
3370	8.13008130081301 \\
3380	8.13008130081301 \\
3390	8.13008130081301 \\
3400	8.13008130081301 \\
3410	8.13008130081301 \\
3420	8.13008130081301 \\
3430	8.13008130081301 \\
3440	8.13008130081301 \\
3450	8.13008130081301 \\
3460	8.13008130081301 \\
3470	8.13008130081301 \\
3480	8.13008130081301 \\
3490	8.13008130081301 \\
3500	8.13008130081301 \\
3510	8.13008130081301 \\
3520	8.13008130081301 \\
3530	8.13008130081301 \\
3540	8.13008130081301 \\
3550	8.13008130081301 \\
3560	8.13008130081301 \\
3570	8.13008130081301 \\
3580	8.13008130081301 \\
3590	8.13008130081301 \\
3600	8.13008130081301 \\
3610	8.13008130081301 \\
3620	8.13008130081301 \\
3630	8.13008130081301 \\
3640	8.13008130081301 \\
3650	8.13008130081301 \\
3660	8.13008130081301 \\
3670	8.13008130081301 \\
3680	8.13008130081301 \\
3690	8.13008130081301 \\
3700	8.13008130081301 \\
3710	8.13008130081301 \\
3720	8.13008130081301 \\
3730	8.13008130081301 \\
3740	8.13008130081301 \\
3750	8.13008130081301 \\
3760	8.13008130081301 \\
3770	8.13008130081301 \\
3780	8.13008130081301 \\
3790	8.13008130081301 \\
3800	8.13008130081301 \\
3810	8.13008130081301 \\
3820	8.13008130081301 \\
3830	8.13008130081301 \\
3840	8.13008130081301 \\
3850	8.13008130081301 \\
3860	8.13008130081301 \\
3870	8.13008130081301 \\
3880	8.13008130081301 \\
3890	8.13008130081301 \\
3900	8.13008130081301 \\
3910	8.13008130081301 \\
3920	8.13008130081301 \\
3930	8.13008130081301 \\
3940	8.13008130081301 \\
3950	8.13008130081301 \\
3960	8.13008130081301 \\
3970	8.13008130081301 \\
3980	8.13008130081301 \\
3990	8.13008130081301 \\
};


\addplot [ dashed, red,mark=text, text mark={\tiny 20\%},mark repeat = 100]
table [row sep=\\]{%
0	95.1219512195122 \\
10	93.4959349593496 \\
20	91.0569105691057 \\
30	87.8048780487805 \\
40	85.3658536585366 \\
50	80.4878048780488 \\
60	78.0487804878049 \\
70	77.2357723577236 \\
80	74.7967479674797 \\
90	71.5447154471545 \\
100	68.2926829268293 \\
110	62.6016260162602 \\
120	60.1626016260163 \\
130	56.9105691056911 \\
140	54.4715447154472 \\
150	46.3414634146341 \\
160	46.3414634146341 \\
170	43.0894308943089 \\
180	39.0243902439024 \\
190	37.3983739837398 \\
200	34.1463414634146 \\
210	32.520325203252 \\
220	31.7073170731707 \\
230	30.0813008130081 \\
240	29.2682926829268 \\
250	29.2682926829268 \\
260	26.8292682926829 \\
270	25.2032520325203 \\
280	24.390243902439 \\
290	23.5772357723577 \\
300	22.7642276422764 \\
310	21.9512195121951 \\
320	22.7642276422764 \\
330	22.7642276422764 \\
340	21.1382113821138 \\
350	19.5121951219512 \\
360	19.5121951219512 \\
370	19.5121951219512 \\
380	19.5121951219512 \\
390	18.6991869918699 \\
400	17.8861788617886 \\
410	17.8861788617886 \\
420	17.8861788617886 \\
430	17.0731707317073 \\
440	17.0731707317073 \\
450	16.260162601626 \\
460	16.260162601626 \\
470	15.4471544715447 \\
480	15.4471544715447 \\
490	14.6341463414634 \\
500	13.8211382113821 \\
510	13.8211382113821 \\
520	13.8211382113821 \\
530	13.8211382113821 \\
540	13.8211382113821 \\
550	13.0081300813008 \\
560	12.1951219512195 \\
570	11.3821138211382 \\
580	10.5691056910569 \\
590	10.5691056910569 \\
600	10.5691056910569 \\
610	10.5691056910569 \\
620	10.5691056910569 \\
630	10.5691056910569 \\
640	10.5691056910569 \\
650	9.75609756097561 \\
660	9.75609756097561 \\
670	9.75609756097561 \\
680	9.75609756097561 \\
690	9.75609756097561 \\
700	9.75609756097561 \\
710	9.75609756097561 \\
720	9.75609756097561 \\
730	9.75609756097561 \\
740	9.75609756097561 \\
750	9.75609756097561 \\
760	9.75609756097561 \\
770	9.75609756097561 \\
780	9.75609756097561 \\
790	9.75609756097561 \\
800	9.75609756097561 \\
810	9.75609756097561 \\
820	9.75609756097561 \\
830	9.75609756097561 \\
840	9.75609756097561 \\
850	9.75609756097561 \\
860	9.75609756097561 \\
870	9.75609756097561 \\
880	9.75609756097561 \\
890	8.94308943089431 \\
900	8.94308943089431 \\
910	8.94308943089431 \\
920	8.94308943089431 \\
930	8.94308943089431 \\
940	8.94308943089431 \\
950	8.94308943089431 \\
960	8.94308943089431 \\
970	8.94308943089431 \\
980	8.94308943089431 \\
990	8.94308943089431 \\
1000	8.94308943089431 \\
1010	8.94308943089431 \\
1020	8.94308943089431 \\
1030	8.94308943089431 \\
1040	8.94308943089431 \\
1050	8.94308943089431 \\
1060	8.94308943089431 \\
1070	8.94308943089431 \\
1080	8.94308943089431 \\
1090	8.94308943089431 \\
1100	8.94308943089431 \\
1110	8.94308943089431 \\
1120	8.94308943089431 \\
1130	8.94308943089431 \\
1140	8.94308943089431 \\
1150	8.94308943089431 \\
1160	8.94308943089431 \\
1170	8.94308943089431 \\
1180	8.94308943089431 \\
1190	8.94308943089431 \\
1200	8.13008130081301 \\
1210	8.13008130081301 \\
1220	8.13008130081301 \\
1230	8.13008130081301 \\
1240	8.13008130081301 \\
1250	8.13008130081301 \\
1260	8.13008130081301 \\
1270	8.13008130081301 \\
1280	8.13008130081301 \\
1290	8.13008130081301 \\
1300	8.13008130081301 \\
1310	8.13008130081301 \\
1320	8.13008130081301 \\
1330	8.13008130081301 \\
1340	8.13008130081301 \\
1350	8.13008130081301 \\
1360	8.13008130081301 \\
1370	8.13008130081301 \\
1380	8.13008130081301 \\
1390	8.13008130081301 \\
1400	8.13008130081301 \\
1410	8.13008130081301 \\
1420	8.13008130081301 \\
1430	8.13008130081301 \\
1440	8.13008130081301 \\
1450	8.13008130081301 \\
1460	8.13008130081301 \\
1470	8.13008130081301 \\
1480	8.13008130081301 \\
1490	8.13008130081301 \\
1500	8.13008130081301 \\
1510	8.13008130081301 \\
1520	8.13008130081301 \\
1530	8.13008130081301 \\
1540	8.13008130081301 \\
1550	8.13008130081301 \\
1560	8.13008130081301 \\
1570	8.13008130081301 \\
1580	8.13008130081301 \\
1590	8.13008130081301 \\
1600	8.13008130081301 \\
1610	8.13008130081301 \\
1620	8.13008130081301 \\
1630	8.13008130081301 \\
1640	8.13008130081301 \\
1650	8.13008130081301 \\
1660	8.13008130081301 \\
1670	8.13008130081301 \\
1680	8.13008130081301 \\
1690	8.13008130081301 \\
1700	8.13008130081301 \\
1710	8.13008130081301 \\
1720	8.13008130081301 \\
1730	8.13008130081301 \\
1740	8.13008130081301 \\
1750	8.13008130081301 \\
1760	8.13008130081301 \\
1770	8.13008130081301 \\
1780	8.13008130081301 \\
1790	8.13008130081301 \\
1800	8.13008130081301 \\
1810	8.13008130081301 \\
1820	8.13008130081301 \\
1830	8.13008130081301 \\
1840	8.13008130081301 \\
1850	8.13008130081301 \\
1860	8.13008130081301 \\
1870	8.13008130081301 \\
1880	8.13008130081301 \\
1890	8.13008130081301 \\
1900	8.13008130081301 \\
1910	8.13008130081301 \\
1920	8.13008130081301 \\
1930	8.13008130081301 \\
1940	8.13008130081301 \\
1950	8.13008130081301 \\
1960	8.13008130081301 \\
1970	8.13008130081301 \\
1980	8.13008130081301 \\
1990	8.13008130081301 \\
2000	8.13008130081301 \\
2010	8.13008130081301 \\
2020	8.13008130081301 \\
2030	8.13008130081301 \\
2040	8.13008130081301 \\
2050	8.13008130081301 \\
2060	8.13008130081301 \\
2070	8.13008130081301 \\
2080	8.13008130081301 \\
2090	8.13008130081301 \\
2100	8.13008130081301 \\
2110	8.13008130081301 \\
2120	8.13008130081301 \\
2130	8.13008130081301 \\
2140	8.13008130081301 \\
2150	8.13008130081301 \\
2160	8.13008130081301 \\
2170	8.13008130081301 \\
2180	8.13008130081301 \\
2190	8.13008130081301 \\
2200	8.13008130081301 \\
2210	8.13008130081301 \\
2220	8.13008130081301 \\
2230	8.13008130081301 \\
2240	8.13008130081301 \\
2250	8.13008130081301 \\
2260	8.13008130081301 \\
2270	8.13008130081301 \\
2280	8.13008130081301 \\
2290	8.13008130081301 \\
2300	8.13008130081301 \\
2310	8.13008130081301 \\
2320	8.13008130081301 \\
2330	8.13008130081301 \\
2340	8.13008130081301 \\
2350	8.13008130081301 \\
2360	8.13008130081301 \\
2370	8.13008130081301 \\
2380	8.13008130081301 \\
2390	8.13008130081301 \\
2400	8.13008130081301 \\
2410	8.13008130081301 \\
2420	8.13008130081301 \\
2430	8.13008130081301 \\
2440	8.13008130081301 \\
2450	8.13008130081301 \\
2460	8.13008130081301 \\
2470	8.13008130081301 \\
2480	8.13008130081301 \\
2490	8.13008130081301 \\
2500	8.13008130081301 \\
2510	8.13008130081301 \\
2520	8.13008130081301 \\
2530	8.13008130081301 \\
2540	8.13008130081301 \\
2550	8.13008130081301 \\
2560	8.13008130081301 \\
2570	8.13008130081301 \\
2580	8.13008130081301 \\
2590	8.13008130081301 \\
2600	8.13008130081301 \\
2610	8.13008130081301 \\
2620	8.13008130081301 \\
2630	8.13008130081301 \\
2640	8.13008130081301 \\
2650	8.13008130081301 \\
2660	8.13008130081301 \\
2670	8.13008130081301 \\
2680	8.13008130081301 \\
2690	8.13008130081301 \\
2700	8.13008130081301 \\
2710	8.13008130081301 \\
2720	8.13008130081301 \\
2730	8.13008130081301 \\
2740	8.13008130081301 \\
2750	8.13008130081301 \\
2760	8.13008130081301 \\
2770	8.13008130081301 \\
2780	8.13008130081301 \\
2790	8.13008130081301 \\
2800	8.13008130081301 \\
2810	8.13008130081301 \\
2820	8.13008130081301 \\
2830	8.13008130081301 \\
2840	8.13008130081301 \\
2850	8.13008130081301 \\
2860	8.13008130081301 \\
2870	8.13008130081301 \\
2880	8.13008130081301 \\
2890	8.13008130081301 \\
2900	8.13008130081301 \\
2910	8.13008130081301 \\
2920	8.13008130081301 \\
2930	8.13008130081301 \\
2940	8.13008130081301 \\
2950	8.13008130081301 \\
2960	8.13008130081301 \\
2970	8.13008130081301 \\
2980	8.13008130081301 \\
2990	8.13008130081301 \\
3000	8.13008130081301 \\
3010	8.13008130081301 \\
3020	8.13008130081301 \\
3030	8.13008130081301 \\
3040	8.13008130081301 \\
3050	8.13008130081301 \\
3060	8.13008130081301 \\
3070	8.13008130081301 \\
3080	8.13008130081301 \\
3090	8.13008130081301 \\
3100	8.13008130081301 \\
3110	8.13008130081301 \\
3120	8.13008130081301 \\
3130	8.13008130081301 \\
3140	8.13008130081301 \\
3150	8.13008130081301 \\
3160	8.13008130081301 \\
3170	8.13008130081301 \\
3180	8.13008130081301 \\
3190	8.13008130081301 \\
3200	8.13008130081301 \\
3210	8.13008130081301 \\
3220	8.13008130081301 \\
3230	8.13008130081301 \\
3240	8.13008130081301 \\
3250	8.13008130081301 \\
3260	8.13008130081301 \\
3270	8.13008130081301 \\
3280	8.13008130081301 \\
3290	8.13008130081301 \\
3300	8.13008130081301 \\
3310	8.13008130081301 \\
3320	8.13008130081301 \\
3330	8.13008130081301 \\
3340	8.13008130081301 \\
3350	8.13008130081301 \\
3360	8.13008130081301 \\
3370	8.13008130081301 \\
3380	8.13008130081301 \\
3390	8.13008130081301 \\
3400	8.13008130081301 \\
3410	8.13008130081301 \\
3420	8.13008130081301 \\
3430	8.13008130081301 \\
3440	8.13008130081301 \\
3450	8.13008130081301 \\
3460	8.13008130081301 \\
3470	8.13008130081301 \\
3480	8.13008130081301 \\
3490	8.13008130081301 \\
3500	8.13008130081301 \\
3510	8.13008130081301 \\
3520	8.13008130081301 \\
3530	8.13008130081301 \\
3540	8.13008130081301 \\
3550	8.13008130081301 \\
3560	8.13008130081301 \\
3570	8.13008130081301 \\
3580	8.13008130081301 \\
3590	8.13008130081301 \\
3600	8.13008130081301 \\
3610	8.13008130081301 \\
3620	8.13008130081301 \\
3630	8.13008130081301 \\
3640	8.13008130081301 \\
3650	8.13008130081301 \\
3660	8.13008130081301 \\
3670	8.13008130081301 \\
3680	8.13008130081301 \\
3690	8.13008130081301 \\
3700	8.13008130081301 \\
3710	8.13008130081301 \\
3720	8.13008130081301 \\
3730	8.13008130081301 \\
3740	8.13008130081301 \\
3750	8.13008130081301 \\
3760	8.13008130081301 \\
3770	8.13008130081301 \\
3780	8.13008130081301 \\
3790	8.13008130081301 \\
3800	8.13008130081301 \\
3810	8.13008130081301 \\
3820	8.13008130081301 \\
3830	8.13008130081301 \\
3840	8.13008130081301 \\
3850	8.13008130081301 \\
3860	8.13008130081301 \\
3870	8.13008130081301 \\
3880	8.13008130081301 \\
3890	8.13008130081301 \\
3900	8.13008130081301 \\
3910	8.13008130081301 \\
3920	8.13008130081301 \\
3930	8.13008130081301 \\
3940	8.13008130081301 \\
3950	8.13008130081301 \\
3960	8.13008130081301 \\
3970	8.13008130081301 \\
3980	8.13008130081301 \\
3990	8.13008130081301 \\
};


\addplot [ dotted, red,mark=text, text mark={\tiny 50\%},mark repeat=100]
table [row sep=\\]{%
0	91.869918699187 \\
10	87.8048780487805 \\
20	82.9268292682927 \\
30	78.0487804878049 \\
40	75.609756097561 \\
50	74.7967479674797 \\
60	70.7317073170732 \\
70	69.1056910569106 \\
80	68.2926829268293 \\
90	64.2276422764228 \\
100	64.2276422764228 \\
110	60.9756097560976 \\
120	56.9105691056911 \\
130	50.4065040650407 \\
140	48.780487804878 \\
150	44.7154471544715 \\
160	42.2764227642276 \\
170	40.650406504065 \\
180	38.2113821138211 \\
190	37.3983739837398 \\
200	36.5853658536585 \\
210	34.9593495934959 \\
220	33.3333333333333 \\
230	30.8943089430894 \\
240	27.6422764227642 \\
250	27.6422764227642 \\
260	28.4552845528455 \\
270	27.6422764227642 \\
280	27.6422764227642 \\
290	26.0162601626016 \\
300	24.390243902439 \\
310	22.7642276422764 \\
320	21.1382113821138 \\
330	21.1382113821138 \\
340	19.5121951219512 \\
350	19.5121951219512 \\
360	18.6991869918699 \\
370	18.6991869918699 \\
380	17.8861788617886 \\
390	17.8861788617886 \\
400	17.8861788617886 \\
410	17.8861788617886 \\
420	17.8861788617886 \\
430	17.0731707317073 \\
440	17.0731707317073 \\
450	17.0731707317073 \\
460	16.260162601626 \\
470	14.6341463414634 \\
480	14.6341463414634 \\
490	14.6341463414634 \\
500	14.6341463414634 \\
510	14.6341463414634 \\
520	13.8211382113821 \\
530	13.8211382113821 \\
540	13.0081300813008 \\
550	12.1951219512195 \\
560	12.1951219512195 \\
570	12.1951219512195 \\
580	12.1951219512195 \\
590	12.1951219512195 \\
600	12.1951219512195 \\
610	12.1951219512195 \\
620	12.1951219512195 \\
630	12.1951219512195 \\
640	12.1951219512195 \\
650	12.1951219512195 \\
660	12.1951219512195 \\
670	12.1951219512195 \\
680	11.3821138211382 \\
690	10.5691056910569 \\
700	10.5691056910569 \\
710	10.5691056910569 \\
720	10.5691056910569 \\
730	10.5691056910569 \\
740	10.5691056910569 \\
750	10.5691056910569 \\
760	10.5691056910569 \\
770	10.5691056910569 \\
780	10.5691056910569 \\
790	10.5691056910569 \\
800	10.5691056910569 \\
810	10.5691056910569 \\
820	10.5691056910569 \\
830	10.5691056910569 \\
840	10.5691056910569 \\
850	10.5691056910569 \\
860	10.5691056910569 \\
870	9.75609756097561 \\
880	9.75609756097561 \\
890	9.75609756097561 \\
900	9.75609756097561 \\
910	9.75609756097561 \\
920	9.75609756097561 \\
930	9.75609756097561 \\
940	9.75609756097561 \\
950	9.75609756097561 \\
960	9.75609756097561 \\
970	9.75609756097561 \\
980	9.75609756097561 \\
990	9.75609756097561 \\
1000	9.75609756097561 \\
1010	9.75609756097561 \\
1020	9.75609756097561 \\
1030	9.75609756097561 \\
1040	9.75609756097561 \\
1050	9.75609756097561 \\
1060	9.75609756097561 \\
1070	9.75609756097561 \\
1080	9.75609756097561 \\
1090	9.75609756097561 \\
1100	9.75609756097561 \\
1110	9.75609756097561 \\
1120	9.75609756097561 \\
1130	9.75609756097561 \\
1140	9.75609756097561 \\
1150	9.75609756097561 \\
1160	9.75609756097561 \\
1170	9.75609756097561 \\
1180	9.75609756097561 \\
1190	9.75609756097561 \\
1200	9.75609756097561 \\
1210	9.75609756097561 \\
1220	9.75609756097561 \\
1230	9.75609756097561 \\
1240	9.75609756097561 \\
1250	9.75609756097561 \\
1260	9.75609756097561 \\
1270	9.75609756097561 \\
1280	9.75609756097561 \\
1290	9.75609756097561 \\
1300	9.75609756097561 \\
1310	9.75609756097561 \\
1320	9.75609756097561 \\
1330	9.75609756097561 \\
1340	9.75609756097561 \\
1350	9.75609756097561 \\
1360	9.75609756097561 \\
1370	8.94308943089431 \\
1380	8.94308943089431 \\
1390	8.94308943089431 \\
1400	8.94308943089431 \\
1410	8.94308943089431 \\
1420	8.94308943089431 \\
1430	8.94308943089431 \\
1440	8.94308943089431 \\
1450	8.94308943089431 \\
1460	8.94308943089431 \\
1470	8.94308943089431 \\
1480	8.94308943089431 \\
1490	8.94308943089431 \\
1500	8.94308943089431 \\
1510	8.94308943089431 \\
1520	8.94308943089431 \\
1530	8.94308943089431 \\
1540	8.94308943089431 \\
1550	8.13008130081301 \\
1560	8.13008130081301 \\
1570	8.13008130081301 \\
1580	8.13008130081301 \\
1590	8.13008130081301 \\
1600	8.13008130081301 \\
1610	8.13008130081301 \\
1620	8.13008130081301 \\
1630	8.13008130081301 \\
1640	8.13008130081301 \\
1650	8.13008130081301 \\
1660	8.13008130081301 \\
1670	8.13008130081301 \\
1680	8.13008130081301 \\
1690	8.13008130081301 \\
1700	8.13008130081301 \\
1710	8.13008130081301 \\
1720	8.13008130081301 \\
1730	8.13008130081301 \\
1740	8.13008130081301 \\
1750	8.13008130081301 \\
1760	8.13008130081301 \\
1770	8.13008130081301 \\
1780	8.13008130081301 \\
1790	8.13008130081301 \\
1800	8.13008130081301 \\
1810	8.13008130081301 \\
1820	8.13008130081301 \\
1830	8.13008130081301 \\
1840	8.13008130081301 \\
1850	8.13008130081301 \\
1860	8.13008130081301 \\
1870	8.13008130081301 \\
1880	8.13008130081301 \\
1890	8.13008130081301 \\
1900	8.13008130081301 \\
1910	8.13008130081301 \\
1920	8.13008130081301 \\
1930	8.13008130081301 \\
1940	8.13008130081301 \\
1950	8.13008130081301 \\
1960	8.13008130081301 \\
1970	8.13008130081301 \\
1980	8.13008130081301 \\
1990	8.13008130081301 \\
2000	8.13008130081301 \\
2010	8.13008130081301 \\
2020	8.13008130081301 \\
2030	8.13008130081301 \\
2040	8.13008130081301 \\
2050	8.13008130081301 \\
2060	8.13008130081301 \\
2070	8.13008130081301 \\
2080	8.13008130081301 \\
2090	8.13008130081301 \\
2100	8.13008130081301 \\
2110	8.13008130081301 \\
2120	8.13008130081301 \\
2130	8.13008130081301 \\
2140	8.13008130081301 \\
2150	8.13008130081301 \\
2160	8.13008130081301 \\
2170	8.13008130081301 \\
2180	8.13008130081301 \\
2190	8.13008130081301 \\
2200	8.13008130081301 \\
2210	8.13008130081301 \\
2220	8.13008130081301 \\
2230	8.13008130081301 \\
2240	8.13008130081301 \\
2250	8.13008130081301 \\
2260	8.13008130081301 \\
2270	8.13008130081301 \\
2280	8.13008130081301 \\
2290	8.13008130081301 \\
2300	8.13008130081301 \\
2310	8.13008130081301 \\
2320	8.13008130081301 \\
2330	8.13008130081301 \\
2340	8.13008130081301 \\
2350	8.13008130081301 \\
2360	8.13008130081301 \\
2370	8.13008130081301 \\
2380	8.13008130081301 \\
2390	8.13008130081301 \\
2400	8.13008130081301 \\
2410	8.13008130081301 \\
2420	8.13008130081301 \\
2430	8.13008130081301 \\
2440	8.13008130081301 \\
2450	8.13008130081301 \\
2460	8.13008130081301 \\
2470	8.13008130081301 \\
2480	8.13008130081301 \\
2490	8.13008130081301 \\
2500	8.13008130081301 \\
2510	8.13008130081301 \\
2520	8.13008130081301 \\
2530	8.13008130081301 \\
2540	8.13008130081301 \\
2550	8.13008130081301 \\
2560	8.13008130081301 \\
2570	8.13008130081301 \\
2580	8.13008130081301 \\
2590	8.13008130081301 \\
2600	8.13008130081301 \\
2610	8.13008130081301 \\
2620	8.13008130081301 \\
2630	8.13008130081301 \\
2640	8.13008130081301 \\
2650	8.13008130081301 \\
2660	8.13008130081301 \\
2670	8.13008130081301 \\
2680	8.13008130081301 \\
2690	8.13008130081301 \\
2700	8.13008130081301 \\
2710	8.13008130081301 \\
2720	8.13008130081301 \\
2730	8.13008130081301 \\
2740	8.13008130081301 \\
2750	8.13008130081301 \\
2760	8.13008130081301 \\
2770	8.13008130081301 \\
2780	8.13008130081301 \\
2790	8.13008130081301 \\
2800	8.13008130081301 \\
2810	8.13008130081301 \\
2820	8.13008130081301 \\
2830	8.13008130081301 \\
2840	8.13008130081301 \\
2850	8.13008130081301 \\
2860	8.13008130081301 \\
2870	8.13008130081301 \\
2880	8.13008130081301 \\
2890	8.13008130081301 \\
2900	8.13008130081301 \\
2910	8.13008130081301 \\
2920	8.13008130081301 \\
2930	8.13008130081301 \\
2940	8.13008130081301 \\
2950	8.13008130081301 \\
2960	8.13008130081301 \\
2970	8.13008130081301 \\
2980	8.13008130081301 \\
2990	8.13008130081301 \\
3000	8.13008130081301 \\
3010	8.13008130081301 \\
3020	8.13008130081301 \\
3030	8.13008130081301 \\
3040	8.13008130081301 \\
3050	8.13008130081301 \\
3060	8.13008130081301 \\
3070	8.13008130081301 \\
3080	8.13008130081301 \\
3090	8.13008130081301 \\
3100	8.13008130081301 \\
3110	8.13008130081301 \\
3120	8.13008130081301 \\
3130	8.13008130081301 \\
3140	8.13008130081301 \\
3150	8.13008130081301 \\
3160	8.13008130081301 \\
3170	8.13008130081301 \\
3180	8.13008130081301 \\
3190	8.13008130081301 \\
3200	8.13008130081301 \\
3210	8.13008130081301 \\
3220	8.13008130081301 \\
3230	8.13008130081301 \\
3240	8.13008130081301 \\
3250	8.13008130081301 \\
3260	8.13008130081301 \\
3270	8.13008130081301 \\
3280	8.13008130081301 \\
3290	8.13008130081301 \\
3300	8.13008130081301 \\
3310	8.13008130081301 \\
3320	8.13008130081301 \\
3330	8.13008130081301 \\
3340	8.13008130081301 \\
3350	8.13008130081301 \\
3360	8.13008130081301 \\
3370	8.13008130081301 \\
3380	8.13008130081301 \\
3390	8.13008130081301 \\
3400	8.13008130081301 \\
3410	8.13008130081301 \\
3420	8.13008130081301 \\
3430	8.13008130081301 \\
3440	8.13008130081301 \\
3450	8.13008130081301 \\
3460	8.13008130081301 \\
3470	8.13008130081301 \\
3480	8.13008130081301 \\
3490	8.13008130081301 \\
3500	8.13008130081301 \\
3510	8.13008130081301 \\
3520	8.13008130081301 \\
3530	8.13008130081301 \\
3540	8.13008130081301 \\
3550	8.13008130081301 \\
3560	8.13008130081301 \\
3570	8.13008130081301 \\
3580	8.13008130081301 \\
3590	8.13008130081301 \\
3600	8.13008130081301 \\
3610	8.13008130081301 \\
3620	8.13008130081301 \\
3630	8.13008130081301 \\
3640	8.13008130081301 \\
3650	8.13008130081301 \\
3660	8.13008130081301 \\
3670	8.13008130081301 \\
3680	8.13008130081301 \\
3690	8.13008130081301 \\
3700	8.13008130081301 \\
3710	8.13008130081301 \\
3720	8.13008130081301 \\
3730	8.13008130081301 \\
3740	8.13008130081301 \\
3750	8.13008130081301 \\
3760	8.13008130081301 \\
3770	8.13008130081301 \\
3780	8.13008130081301 \\
3790	8.13008130081301 \\
3800	8.13008130081301 \\
3810	8.13008130081301 \\
3820	8.13008130081301 \\
3830	8.13008130081301 \\
3840	8.13008130081301 \\
3850	8.13008130081301 \\
3860	8.13008130081301 \\
3870	8.13008130081301 \\
3880	8.13008130081301 \\
3890	8.13008130081301 \\
3900	8.13008130081301 \\
3910	8.13008130081301 \\
3920	8.13008130081301 \\
3930	8.13008130081301 \\
3940	8.13008130081301 \\
3950	8.13008130081301 \\
3960	8.13008130081301 \\
3970	8.13008130081301 \\
3980	8.13008130081301 \\
3990	8.13008130081301 \\
};
\end{axis}

\end{tikzpicture}
}
\scalebox{.85}{% This file was created by matplotlib2tikz v0.6.18.
\begin{tikzpicture}



\begin{axis}[
legend cell align={left},
legend columns=1,
legend entries={{\pgd},{20\% \algo},{1 \adaalgo},{10\% \adaalgo},{20\% \adaalgo},{50\% \adaalgo},},
legend style={at={(0.8,0.99)}, anchor=north},
tick align=outside,
tick pos=left,
xlabel={Iteration},
xmajorgrids,
xmin=0, xmax=4000,
ylabel={Suboptimality},
ymajorgrids,
ymin=1e-11, ymax=16.995627444391,
ymode=log
]

\addlegendimage{ black,thick,mark=square*,mark repeat = 100}
\addlegendimage{ blue,mark=*,mark repeat = 100}
\addlegendimage{ red }
\addlegendimage{ red,mark=text, text mark={\tiny 10\%},mark repeat = 100}
\addlegendimage{ dashed, red,mark=text, text mark={\tiny 20\%},mark repeat = 100}
\addlegendimage{ dotted, red,mark=text, text mark={\tiny 50\%},mark repeat = 100}

\addplot [black,thick,mark=square*,mark repeat = 100]
table [row sep=\\]{%
0	2.09092732507745 \\
10	1.76093138142936 \\
20	1.49402105644764 \\
30	1.27429766033831 \\
40	1.07856295304427 \\
50	0.914077018577608 \\
60	0.774128430961237 \\
70	0.65315917276041 \\
80	0.55235151898703 \\
90	0.469421417542497 \\
100	0.399565969711648 \\
110	0.340384500529901 \\
120	0.290966481605671 \\
130	0.254452910495519 \\
140	0.222074511935787 \\
150	0.193369326713135 \\
160	0.167736405728764 \\
170	0.146233887110576 \\
180	0.128078178237277 \\
190	0.112352735781869 \\
200	0.0987311110185153 \\
210	0.0861110194152439 \\
220	0.0754459207526227 \\
230	0.0663337992006726 \\
240	0.0580982016580068 \\
250	0.0510247428381324 \\
260	0.0444561890005148 \\
270	0.0385642106046931 \\
280	0.033797989263032 \\
290	0.0300703973658077 \\
300	0.0265915778895859 \\
310	0.0234003486001056 \\
320	0.0206923602811656 \\
330	0.0182493381457556 \\
340	0.0162850166388389 \\
350	0.0147866713964506 \\
360	0.0134074833186372 \\
370	0.0121298949505163 \\
380	0.0109527458654176 \\
390	0.00986179065062148 \\
400	0.00885681195369759 \\
410	0.00792776761794539 \\
420	0.00706526404777613 \\
430	0.00626399777149778 \\
440	0.0055299011233288 \\
450	0.00503109283811304 \\
460	0.00456901073358795 \\
470	0.00419672734801718 \\
480	0.00387583098585076 \\
490	0.00357820842365403 \\
500	0.00330195147639856 \\
510	0.00304943415105369 \\
520	0.00281786267459677 \\
530	0.00260267430953037 \\
540	0.00240261791396074 \\
550	0.0022165479809213 \\
560	0.00204341439673084 \\
570	0.00188225355133276 \\
580	0.00173218039386447 \\
590	0.00159238131081102 \\
600	0.00146210772883071 \\
610	0.00134067035835123 \\
620	0.00122816796339165 \\
630	0.00112351360716378 \\
640	0.00103090717352805 \\
650	0.000952149066066332 \\
660	0.000879636117740334 \\
670	0.000812624110271942 \\
680	0.000750822973817122 \\
690	0.00069384229944186 \\
700	0.000641123757514628 \\
710	0.000592328671151332 \\
720	0.00054714768444708 \\
730	0.000505297750742151 \\
740	0.000466519636208085 \\
750	0.000430575709381154 \\
760	0.000397247969035674 \\
770	0.000366336278039447 \\
780	0.000337656776815409 \\
790	0.000311040454418599 \\
800	0.000286331858683375 \\
810	0.000266852779093441 \\
820	0.000248879010914915 \\
830	0.000232131269458591 \\
840	0.00021651677476503 \\
850	0.000201955730633474 \\
860	0.000188374642196454 \\
870	0.000175705455451147 \\
880	0.000163885093026339 \\
890	0.000152855054270262 \\
900	0.000142561054439827 \\
910	0.000132952697078192 \\
920	0.000123983175581432 \\
930	0.000115609000652717 \\
940	0.000107789750863452 \\
950	0.000100487843939401 \\
960	9.36683267224736e-05 \\
970	8.72986820222854e-05 \\
980	8.13486507913463e-05 \\
990	7.57900682429313e-05 \\
1000	7.05967126820584e-05 \\
1010	6.5744165954118e-05 \\
1020	6.12096845253318e-05 \\
1030	5.69720803100804e-05 \\
1040	5.30116104432432e-05 \\
1050	4.9309875272241e-05 \\
1060	4.58497239077538e-05 \\
1070	4.26151667354824e-05 \\
1080	3.95912943382259e-05 \\
1090	3.67642023293424e-05 \\
1100	3.41209216419003e-05 \\
1110	3.16493538493612e-05 \\
1120	2.93382111394291e-05 \\
1130	2.71769605842409e-05 \\
1140	2.51557723846552e-05 \\
1150	2.3265471789824e-05 \\
1160	2.14974944197088e-05 \\
1170	1.98438447366889e-05 \\
1180	1.8297057434391e-05 \\
1190	1.68501615301908e-05 \\
1200	1.5496646963209e-05 \\
1210	1.43135342753342e-05 \\
1220	1.33502637134075e-05 \\
1230	1.24542127949434e-05 \\
1240	1.16405815556164e-05 \\
1250	1.08828423731611e-05 \\
1260	1.01767175476608e-05 \\
1270	9.5183681926847e-06 \\
1280	8.90429915556545e-06 \\
1290	8.3313150362474e-06 \\
1300	7.79648585291781e-06 \\
1310	7.2971183924242e-06 \\
1320	6.83073194385209e-06 \\
1330	6.39503751986847e-06 \\
1340	5.987919891981e-06 \\
1350	5.60742191885177e-06 \\
1360	5.25173075938135e-06 \\
1370	4.91916566075501e-06 \\
1380	4.60816706471245e-06 \\
1390	4.31728684019417e-06 \\
1400	4.04517947860672e-06 \\
1410	3.79059412553007e-06 \\
1420	3.5523673423965e-06 \\
1430	3.32941650965646e-06 \\
1440	3.12073379982181e-06 \\
1450	2.92538065799208e-06 \\
1460	2.74248273757216e-06 \\
1470	2.57122524438547e-06 \\
1480	2.41084865226782e-06 \\
1490	2.26064475361554e-06 \\
1500	2.11995301580004e-06 \\
1510	1.98815721508261e-06 \\
1520	1.86468232737935e-06 \\
1530	1.74899164900877e-06 \\
1540	1.64058413415491e-06 \\
1550	1.53899192695262e-06 \\
1560	1.44377807492768e-06 \\
1570	1.35453440808231e-06 \\
1580	1.27087957219052e-06 \\
1590	1.1924572025368e-06 \\
1600	1.11893422871656e-06 \\
1610	1.04999929934069e-06 \\
1620	9.85361318706079e-07 \\
1630	9.24748086716942e-07 \\
1640	8.67905032730931e-07 \\
1650	8.14594039777461e-07 \\
1660	7.64592348934112e-07 \\
1670	7.17691540030874e-07 \\
1680	6.73696582853545e-07 \\
1690	6.32424953184163e-07 \\
1700	5.93705809348588e-07 \\
1710	5.57379225385457e-07 \\
1720	5.23295475729491e-07 \\
1730	4.91314368411544e-07 \\
1740	4.61304624166381e-07 \\
1750	4.33143295675009e-07 \\
1760	4.0671522749669e-07 \\
1770	3.81912550917463e-07 \\
1780	3.58634213104558e-07 \\
1790	3.36785537402573e-07 \\
1800	3.16277812939525e-07 \\
1810	2.97027910933778e-07 \\
1820	2.78957926980183e-07 \\
1830	2.61994846262326e-07 \\
1840	2.4607023141332e-07 \\
1850	2.31119929805512e-07 \\
1860	2.17083801379303e-07 \\
1870	2.03905462903275e-07 \\
1880	1.91532049498289e-07 \\
1890	1.79913991871228e-07 \\
1900	1.69004807260009e-07 \\
1910	1.58760904478417e-07 \\
1920	1.49141401228903e-07 \\
1930	1.401079539054e-07 \\
1940	1.31624596999558e-07 \\
1950	1.23657593886772e-07 \\
1960	1.16175297049104e-07 \\
1970	1.0914801656936e-07 \\
1980	1.02547897840033e-07 \\
1990	9.63488065441886e-08 \\
2000	9.05262215189495e-08 \\
2010	8.5057133336619e-08 \\
2020	7.99199506018589e-08 \\
2030	7.50944113558916e-08 \\
2040	7.05615000873294e-08 \\
2050	6.630337084923e-08 \\
2060	6.23032742619323e-08 \\
2070	5.85454890678072e-08 \\
2080	5.50152589595676e-08 \\
2090	5.16987325727136e-08 \\
2100	4.85829074192701e-08 \\
2110	4.56555774852596e-08 \\
2120	4.29052838812893e-08 \\
2130	4.03212689348287e-08 \\
2140	3.78934322253777e-08 \\
2150	3.56122915601276e-08 \\
2160	3.34689431169544e-08 \\
2170	3.14550270830161e-08 \\
2180	2.95626933488613e-08 \\
2190	2.7784570921785e-08 \\
2200	2.61137382273624e-08 \\
2210	2.45436949652955e-08 \\
2220	2.30683371849061e-08 \\
2230	2.1681932083073e-08 \\
2240	2.03790960773276e-08 \\
2250	1.91547727679264e-08 \\
2260	1.80042133979263e-08 \\
2270	1.69229575353036e-08 \\
2280	1.59068160310305e-08 \\
2290	1.49518537551074e-08 \\
2300	1.4054374608552e-08 \\
2310	1.32109068684549e-08 \\
2320	1.24181894767261e-08 \\
2330	1.16731584398622e-08 \\
2340	1.09729363928501e-08 \\
2350	1.0314819554047e-08 \\
2360	9.69626867686202e-09 \\
2370	9.11489761445949e-09 \\
2380	8.56846477104156e-09 \\
2390	8.05486449761972e-09 \\
2400	7.57211826574178e-09 \\
2410	7.11836734001992e-09 \\
2420	6.69186517310294e-09 \\
2430	6.29097113291621e-09 \\
2440	5.91414317518968e-09 \\
2450	5.55993306949887e-09 \\
2460	5.22697929383753e-09 \\
2470	4.91400337088166e-09 \\
2480	4.61980337318479e-09 \\
2490	4.34325003739744e-09 \\
2500	4.08328221235266e-09 \\
2510	3.83890291777433e-09 \\
2520	3.60917512542969e-09 \\
2530	3.39321837294904e-09 \\
2540	3.19020504457868e-09 \\
2550	2.99935798420137e-09 \\
2560	2.81994616546655e-09 \\
2570	2.65128280441118e-09 \\
2580	2.49272263941336e-09 \\
2590	2.34365932216818e-09 \\
2600	2.20352253110789e-09 \\
2610	2.07177669464542e-09 \\
2620	1.94791849317255e-09 \\
2630	1.83147458310273e-09 \\
2640	1.7220002090923e-09 \\
2650	1.61907748319479e-09 \\
2660	1.52231327543717e-09 \\
2670	1.43133865870837e-09 \\
2680	1.34580646626858e-09 \\
2690	1.26539062561548e-09 \\
2700	1.18978404906045e-09 \\
2710	1.11869918884011e-09 \\
2720	1.05186453991379e-09 \\
2730	9.89026083253464e-10 \\
2740	9.2994395517465e-10 \\
2750	8.74393502048321e-10 \\
2760	8.22163115365981e-10 \\
2770	7.73054120717376e-10 \\
2780	7.26879501034006e-10 \\
2790	6.83463896589132e-10 \\
2800	6.42641828640933e-10 \\
2810	6.04258365566324e-10 \\
2820	5.68167513037565e-10 \\
2830	5.34232158511116e-10 \\
2840	5.02323127538062e-10 \\
2850	4.72319572342172e-10 \\
2860	4.44107250974213e-10 \\
2870	4.17579137934609e-10 \\
2880	3.92634647017331e-10 \\
2890	3.69179076198378e-10 \\
2900	3.47123274568872e-10 \\
2910	3.2638364233506e-10 \\
2920	3.06881631217948e-10 \\
2930	2.88543078319492e-10 \\
2940	2.71298594700653e-10 \\
2950	2.55082843736432e-10 \\
2960	2.39834208048961e-10 \\
2970	2.25495233596718e-10 \\
2980	2.12011297406889e-10 \\
2990	1.99331440242645e-10 \\
3000	1.87407644958171e-10 \\
3010	1.76194780987515e-10 \\
3020	1.65650271277684e-10 \\
3030	1.55734425355547e-10 \\
3040	1.46409606660569e-10 \\
3050	1.37640510100567e-10 \\
3060	1.29394051029408e-10 \\
3070	1.2163908769125e-10 \\
3080	1.14346088153638e-10 \\
3090	1.07487685419017e-10 \\
3100	1.01037900268608e-10 \\
3110	9.49722522847196e-11 \\
3120	8.92679818953468e-11 \\
3130	8.39034952626605e-11 \\
3140	7.88584753053101e-11 \\
3150	7.41139927207257e-11 \\
3160	6.96518953624548e-11 \\
3170	6.54554188628254e-11 \\
3180	6.15090200994928e-11 \\
3190	5.77974890170196e-11 \\
3200	5.43068923164469e-11 \\
3210	5.10240738549328e-11 \\
3220	4.79365991346015e-11 \\
3230	4.50329773471481e-11 \\
3240	4.23021617734776e-11 \\
3250	3.9733827339461e-11 \\
3260	3.73183706159352e-11 \\
3270	3.5046576751796e-11 \\
3280	3.29099525409049e-11 \\
3290	3.09003933551821e-11 \\
3300	2.90104607003627e-11 \\
3310	2.72329381267866e-11 \\
3320	2.55610532740036e-11 \\
3330	2.39886999153782e-11 \\
3340	2.25098828465775e-11 \\
3350	2.11188844190247e-11 \\
3360	1.98106531179576e-11 \\
3370	1.85802484509168e-11 \\
3380	1.74230074811987e-11 \\
3390	1.6334489316705e-11 \\
3400	1.53108081768494e-11 \\
3410	1.43478007252895e-11 \\
3420	1.34421918041028e-11 \\
3430	1.25904286996104e-11 \\
3440	1.17891807427384e-11 \\
3450	1.10355613536228e-11 \\
3460	1.03268504858534e-11 \\
3470	9.66010604841472e-12 \\
3480	9.03305208410643e-12 \\
3490	8.44324610227432e-12 \\
3500	7.88846765686912e-12 \\
3510	7.36660732414407e-12 \\
3520	6.87583323610852e-12 \\
3530	6.41420250246938e-12 \\
3540	5.97988325523602e-12 \\
3550	5.57148771562765e-12 \\
3560	5.18723952680489e-12 \\
3570	4.8258619322894e-12 \\
3580	4.48602266445164e-12 \\
3590	4.16627843335959e-12 \\
3600	3.86557452713987e-12 \\
3610	3.58268970046538e-12 \\
3620	3.31668026376519e-12 \\
3630	3.06626946056099e-12 \\
3640	2.83090217934046e-12 \\
3650	2.60941268592774e-12 \\
3660	2.40113484650806e-12 \\
3670	2.20512497151049e-12 \\
3680	2.02077243827148e-12 \\
3690	1.84746662412749e-12 \\
3700	1.68426383950759e-12 \\
3710	1.53094203980686e-12 \\
3720	1.38661304660559e-12 \\
3730	1.25088828184516e-12 \\
3740	1.1232681451645e-12 \\
3750	1.00308650274883e-12 \\
3760	8.90121309993219e-13 \\
3770	7.83761944234129e-13 \\
3780	6.83841872017865e-13 \\
3790	5.89750470680883e-13 \\
3800	5.01265695618258e-13 \\
3810	4.18109991073834e-13 \\
3820	3.39672734384067e-13 \\
3830	2.6612045900265e-13 \\
3840	1.96898053417272e-13 \\
3850	1.31672450720544e-13 \\
3860	7.03881397612349e-14 \\
3870	1.2712053631958e-14 \\
3880	-4.15778522722121e-14 \\
3890	-9.25926002537381e-14 \\
3900	-1.40609746068776e-13 \\
3910	-1.85740312019789e-13 \\
3920	-2.28261853862932e-13 \\
3930	-2.68174371598207e-13 \\
3940	-3.05755420981768e-13 \\
3950	-3.41227046618542e-13 \\
3960	-3.74478226206065e-13 \\
3970	-4.05619982046801e-13 \\
3980	-4.35207425653061e-13 \\
3990	-4.62907490117459e-13 \\
};

\addplot [blue,mark=*,mark repeat = 100]
table [row sep=\\]{%
0	1.16215662228905 \\
10	1.00428253393589 \\
20	0.924389652996051 \\
30	0.860199404522113 \\
40	0.800381218608045 \\
50	0.747097812544684 \\
60	0.693510584381706 \\
70	0.646359866592243 \\
80	0.600737195010842 \\
90	0.559001397962172 \\
100	0.51993229908195 \\
110	0.485620978416149 \\
120	0.453562233217362 \\
130	0.42599179628601 \\
140	0.396944458209628 \\
150	0.368309438763449 \\
160	0.34491084956204 \\
170	0.317901121247184 \\
180	0.29551213786032 \\
190	0.277131493553391 \\
200	0.260498104023332 \\
210	0.243551825840378 \\
220	0.231380232961997 \\
230	0.221210585650787 \\
240	0.211667746267815 \\
250	0.199001816769377 \\
260	0.187218135878817 \\
270	0.176715603427283 \\
280	0.170920754097678 \\
290	0.161573024638786 \\
300	0.154770592454409 \\
310	0.146647721748763 \\
320	0.141605849268692 \\
330	0.131116450898988 \\
340	0.125328192409247 \\
350	0.119186202739831 \\
360	0.111403821298566 \\
370	0.10611424909179 \\
380	0.0999312054381534 \\
390	0.0934940737762713 \\
400	0.0885493679744258 \\
410	0.0844833299114791 \\
420	0.0811402828286314 \\
430	0.0761720558518353 \\
440	0.0739668603469543 \\
450	0.0706049071342434 \\
460	0.0672204284114135 \\
470	0.0654500729371201 \\
480	0.0636174946047901 \\
490	0.0613114970178779 \\
500	0.058641862311199 \\
510	0.0565934021702331 \\
520	0.0547931511993578 \\
530	0.0516920762702148 \\
540	0.0488482668530374 \\
550	0.0467268475947321 \\
560	0.0449428218572198 \\
570	0.0434094302102165 \\
580	0.0418513058296786 \\
590	0.0399201814561859 \\
600	0.0388408778365645 \\
610	0.0377582161025896 \\
620	0.0370847751277119 \\
630	0.0360844445489145 \\
640	0.034335641650601 \\
650	0.0333971414046264 \\
660	0.0329114219648496 \\
670	0.0314629339342957 \\
680	0.0301265566701184 \\
690	0.0285098357145853 \\
700	0.0278251656729712 \\
710	0.0269755099369679 \\
720	0.0262002661684173 \\
730	0.025274356049916 \\
740	0.0248622658046704 \\
750	0.0238834648495216 \\
760	0.0229110857505238 \\
770	0.0222300524656775 \\
780	0.0217506912240895 \\
790	0.021400544373575 \\
800	0.0203476208837226 \\
810	0.0198112418011496 \\
820	0.019057752937924 \\
830	0.0183962131718027 \\
840	0.0181532477059995 \\
850	0.0178571614803595 \\
860	0.0175202893595809 \\
870	0.017161133223173 \\
880	0.0167479599724896 \\
890	0.0163154155663399 \\
900	0.0158278040833156 \\
910	0.0154641876862646 \\
920	0.0152476770299525 \\
930	0.014922429713923 \\
940	0.0145379366744003 \\
950	0.0142402484851094 \\
960	0.013851879833516 \\
970	0.0137008276387374 \\
980	0.0135275227034023 \\
990	0.013334318676045 \\
1000	0.0131044010811041 \\
1010	0.0125866175005585 \\
1020	0.012389540291766 \\
1030	0.0121423995140789 \\
1040	0.0119642443471491 \\
1050	0.0118482468435631 \\
1060	0.0116682762291604 \\
1070	0.0114737302106622 \\
1080	0.0112412001241596 \\
1090	0.0108963883836689 \\
1100	0.010698965938397 \\
1110	0.0103139534281668 \\
1120	0.0102214876843962 \\
1130	0.00982371495798162 \\
1140	0.00963867484875658 \\
1150	0.00949084409617262 \\
1160	0.00934418064755699 \\
1170	0.00914347410903416 \\
1180	0.00900928858910233 \\
1190	0.00882555225427389 \\
1200	0.00867139579835963 \\
1210	0.00848106459279535 \\
1220	0.00828811539502333 \\
1230	0.00805720927961762 \\
1240	0.00783154675799141 \\
1250	0.00777322793964003 \\
1260	0.00763163278926321 \\
1270	0.00746587538587223 \\
1280	0.00736864626665668 \\
1290	0.00730149509900868 \\
1300	0.00723894801561736 \\
1310	0.00717955809365828 \\
1320	0.00704102045310429 \\
1330	0.00677441470456369 \\
1340	0.00663129107355959 \\
1350	0.00653279157773279 \\
1360	0.00641188465395048 \\
1370	0.00634591733693829 \\
1380	0.00629007322459796 \\
1390	0.00619520712759919 \\
1400	0.00603085569746908 \\
1410	0.00598576934014899 \\
1420	0.00591953195381961 \\
1430	0.0058731508915561 \\
1440	0.00582855693115197 \\
1450	0.00575640017019674 \\
1460	0.00570060761909558 \\
1470	0.00562606782110697 \\
1480	0.0055725456332561 \\
1490	0.0054961888750602 \\
1500	0.00538391216681994 \\
1510	0.00532695994786314 \\
1520	0.00525993614918024 \\
1530	0.00504345760681946 \\
1540	0.00496237234109254 \\
1550	0.00493079011724074 \\
1560	0.00488401392739735 \\
1570	0.00483016635683076 \\
1580	0.00475273588394992 \\
1590	0.00469613012117359 \\
1600	0.00446811688047405 \\
1610	0.0044081748642128 \\
1620	0.00436151466360385 \\
1630	0.00427362753363175 \\
1640	0.00418812518500072 \\
1650	0.0041710647466523 \\
1660	0.00406381533772798 \\
1670	0.00399454724112897 \\
1680	0.0036892708223753 \\
1690	0.00361159740839928 \\
1700	0.00356805688472395 \\
1710	0.00349502518513406 \\
1720	0.00346740888139552 \\
1730	0.00339796547400584 \\
1740	0.00314615435397808 \\
1750	0.00306565730232328 \\
1760	0.00302821500385225 \\
1770	0.00299232156952473 \\
1780	0.00296838627791823 \\
1790	0.00286448977758069 \\
1800	0.00283187599353624 \\
1810	0.00276267746574782 \\
1820	0.00273137844298132 \\
1830	0.00268448438148661 \\
1840	0.0026429013257579 \\
1850	0.00259478414278547 \\
1860	0.00254563660796664 \\
1870	0.00248902074703011 \\
1880	0.00246668073962575 \\
1890	0.00241109540981199 \\
1900	0.00238267694394562 \\
1910	0.00229974863230015 \\
1920	0.00227576953329522 \\
1930	0.00228269619363408 \\
1940	0.00215529418568206 \\
1950	0.00213542925135329 \\
1960	0.00210878943431309 \\
1970	0.00201553181613146 \\
1980	0.00199781731729382 \\
1990	0.00195484400904289 \\
2000	0.00195295844677901 \\
2010	0.00188364044033179 \\
2020	0.00186196013039786 \\
2030	0.0018605249255606 \\
2040	0.00185078402264105 \\
2050	0.00181493499954033 \\
2060	0.00179629524202257 \\
2070	0.0017773628984713 \\
2080	0.0017609554556885 \\
2090	0.00175028326840637 \\
2100	0.00173763448871778 \\
2110	0.00170312455079941 \\
2120	0.00166693457273975 \\
2130	0.00160528026754564 \\
2140	0.0015968845081244 \\
2150	0.00158057164024988 \\
2160	0.00157224590030686 \\
2170	0.00156684438386112 \\
2180	0.00155539570706231 \\
2190	0.00154461977409093 \\
2200	0.00152954871454225 \\
2210	0.00152246097956632 \\
2220	0.00151429744827791 \\
2230	0.00150703176829237 \\
2240	0.0014889205360949 \\
2250	0.00148218242368059 \\
2260	0.00147183253330097 \\
2270	0.0014646741938063 \\
2280	0.00145957975659794 \\
2290	0.00145470894010064 \\
2300	0.00145017679377413 \\
2310	0.00144625603077414 \\
2320	0.0014107402458387 \\
2330	0.00138498613197924 \\
2340	0.00137925263354816 \\
2350	0.00137056312393274 \\
2360	0.00136577535485599 \\
2370	0.00135759747632075 \\
2380	0.00133039791114925 \\
2390	0.00132617146759106 \\
2400	0.00131733825876079 \\
2410	0.00131502998852012 \\
2420	0.00130777061252035 \\
2430	0.00129425047308557 \\
2440	0.00126772898422989 \\
2450	0.00126185377146676 \\
2460	0.00123342811488392 \\
2470	0.00122877911551339 \\
2480	0.00122263608504558 \\
2490	0.00121941040677948 \\
2500	0.00120613679448039 \\
2510	0.00118508039442489 \\
2520	0.00117469558004019 \\
2530	0.00117156854867118 \\
2540	0.00116842288343566 \\
2550	0.00116352489785565 \\
2560	0.00116004680623871 \\
2570	0.00113764065226091 \\
2580	0.00113646148135604 \\
2590	0.00112951265405115 \\
2600	0.00112581096944281 \\
2610	0.00112261632597521 \\
2620	0.00111837485655869 \\
2630	0.00111610353253777 \\
2640	0.00111353863820685 \\
2650	0.00110864146347217 \\
2660	0.00108300896561031 \\
2670	0.00107833686930298 \\
2680	0.00107101833744799 \\
2690	0.00106782750667217 \\
2700	0.00106513959982002 \\
2710	0.00106039447893819 \\
2720	0.00105890162127903 \\
2730	0.00105537474758072 \\
2740	0.00104634389881236 \\
2750	0.00104472557504515 \\
2760	0.00104124539459222 \\
2770	0.00103384389351757 \\
2780	0.00101082504395106 \\
2790	0.000989599400613272 \\
2800	0.000972216442321483 \\
2810	0.000968224320722189 \\
2820	0.000945243719447486 \\
2830	0.000927533905458688 \\
2840	0.000923988335640835 \\
2850	0.000921762283861205 \\
2860	0.00092112391562138 \\
2870	0.000918021890966403 \\
2880	0.000914545896808672 \\
2890	0.000913108849242283 \\
2900	0.000911293036714556 \\
2910	0.000909118040607726 \\
2920	0.000906437942235216 \\
2930	0.000902164363641489 \\
2940	0.000900048965439382 \\
2950	0.000896037050956366 \\
2960	0.00088704805963119 \\
2970	0.000866367919412359 \\
2980	0.000849585558542254 \\
2990	0.000847956006823858 \\
3000	0.000841462729487485 \\
3010	0.000837615129262281 \\
3020	0.000836504718731812 \\
3030	0.00083511954000437 \\
3040	0.000832817967951816 \\
3050	0.000831558572321756 \\
3060	0.000829910339526896 \\
3070	0.000827778066375795 \\
3080	0.000797092519662868 \\
3090	0.000791692175272307 \\
3100	0.000789072425096238 \\
3110	0.000785812280727993 \\
3120	0.000784464083113556 \\
3130	0.000782726476757312 \\
3140	0.000778083637455851 \\
3150	0.000776449174364868 \\
3160	0.000775497329342012 \\
3170	0.00073257605400967 \\
3180	0.000731611370084595 \\
3190	0.000703467617833631 \\
3200	0.000701436779111997 \\
3210	0.000699611873882178 \\
3220	0.00069827389137711 \\
3230	0.000698080815824065 \\
3240	0.000696559946478714 \\
3250	0.00069526184783425 \\
3260	0.00069364858993487 \\
3270	0.000681012006822468 \\
3280	0.000680750816102194 \\
3290	0.000678898056171773 \\
3300	0.000678272799104929 \\
3310	0.000676636650544127 \\
3320	0.000663635122529593 \\
3330	0.000662677345211693 \\
3340	0.000657047395649479 \\
3350	0.000655364519563828 \\
3360	0.000643378422833518 \\
3370	0.000641092298198032 \\
3380	0.000640065502158338 \\
3390	0.000637503611045864 \\
3400	0.000637473802264255 \\
3410	0.00063701982237313 \\
3420	0.000636773534601842 \\
3430	0.000635555147275813 \\
3440	0.000634891228435186 \\
3450	0.000634749624349895 \\
3460	0.000634062493641285 \\
3470	0.000633094517670518 \\
3480	0.00063100058283122 \\
3490	0.000630617862227068 \\
3500	0.000628740461308064 \\
3510	0.000628139760186497 \\
3520	0.000624957613428012 \\
3530	0.000611874845327542 \\
3540	0.000600142579023077 \\
3550	0.00059894958396256 \\
3560	0.000598908067941384 \\
3570	0.000598479269072139 \\
3580	0.000597993031749566 \\
3590	0.000597323061260646 \\
3600	0.000596687066369117 \\
3610	0.000592564668551565 \\
3620	0.000591884007059451 \\
3630	0.000591670609523032 \\
3640	0.000591190683365939 \\
3650	0.000590807957814854 \\
3660	0.000590081028905343 \\
3670	0.000589321076052418 \\
3680	0.000588662090566405 \\
3690	0.000587721200579039 \\
3700	0.000586750884875087 \\
3710	0.00058669354852986 \\
3720	0.000586203154680387 \\
3730	0.000583892994445723 \\
3740	0.000580517694229565 \\
3750	0.000569532784199966 \\
3760	0.000567803814439827 \\
3770	0.000567372825352186 \\
3780	0.000566717850514153 \\
3790	0.000566380805736455 \\
3800	0.000565568003373429 \\
3810	0.000565353893952347 \\
3820	0.000565019321212779 \\
3830	0.000564878535428692 \\
3840	0.00056474992761224 \\
3850	0.000563563892088559 \\
3860	0.000562245110088122 \\
3870	0.000561684136323681 \\
3880	0.000560988900251858 \\
3890	0.000560885054746885 \\
3900	0.00056079521915231 \\
3910	0.000560447127247177 \\
3920	0.0005602523237484 \\
3930	0.000548778231183378 \\
3940	0.000548180275123211 \\
3950	0.000547501443533494 \\
3960	0.0005470827120998 \\
3970	0.000526274435948604 \\
3980	0.000525579609402971 \\
3990	0.000514258166529558 \\
};


\addplot [red]
table [row sep=\\]{%
0	3.1805559289351 \\
10	2.64188544734182 \\
20	2.20403380748648 \\
30	1.83895313737967 \\
40	1.53105960388386 \\
50	1.26875598571308 \\
60	1.04102645391337 \\
70	0.849323436778866 \\
80	0.700238545857181 \\
90	0.581371393698809 \\
100	0.485598141235844 \\
110	0.404430636202285 \\
120	0.335905118668508 \\
130	0.279873896419759 \\
140	0.236080190649626 \\
150	0.201901474536513 \\
160	0.173237573530455 \\
170	0.149529068568246 \\
180	0.130711750113766 \\
190	0.115745971664834 \\
200	0.102497717087355 \\
210	0.0904611013135878 \\
220	0.0797058872376823 \\
230	0.0696914070735802 \\
240	0.061618035629484 \\
250	0.0545555259826275 \\
260	0.0482543623611812 \\
270	0.0425929974651959 \\
280	0.0373447068844376 \\
290	0.0328926105604252 \\
300	0.0289525878662524 \\
310	0.0255752324937894 \\
320	0.0226145850624336 \\
330	0.019862638207472 \\
340	0.0173024324491598 \\
350	0.0149187860868296 \\
360	0.0127153581919099 \\
370	0.0106765595470695 \\
380	0.00899583386564207 \\
390	0.007837431139539 \\
400	0.0068769006407059 \\
410	0.0061305818700898 \\
420	0.00555344510084332 \\
430	0.00501845754099123 \\
440	0.00458797441700687 \\
450	0.00420433950122068 \\
460	0.00384905115820333 \\
470	0.00351980520187389 \\
480	0.00321451464804606 \\
490	0.00293821479496115 \\
500	0.00269874722889035 \\
510	0.00247752842161642 \\
520	0.00227293292428515 \\
530	0.00208354039232767 \\
540	0.00190807820611227 \\
550	0.00174540068595502 \\
560	0.00160825729599806 \\
570	0.00149028336864609 \\
580	0.00138374057220836 \\
590	0.00128646035371505 \\
600	0.00119732182453913 \\
610	0.00111539992135951 \\
620	0.00103992056549429 \\
630	0.000970228235148807 \\
640	0.000905761675709382 \\
650	0.000846035505104481 \\
660	0.000790626151358664 \\
670	0.000739160996294952 \\
680	0.00069138374871186 \\
690	0.000647183430624976 \\
700	0.000605949032019104 \\
710	0.000567456698989 \\
720	0.000531504476571742 \\
730	0.000497909007740771 \\
740	0.000466503030747745 \\
750	0.000437133369854348 \\
760	0.000409659294894615 \\
770	0.000383951163567431 \\
780	0.000359889282846859 \\
790	0.000337362942023023 \\
800	0.000316269581719986 \\
810	0.000296514071955634 \\
820	0.000278008078753333 \\
830	0.000260669503592381 \\
840	0.000244421983543686 \\
850	0.000229194442593339 \\
860	0.000214920686656905 \\
870	0.000201539036294107 \\
880	0.000188991992283005 \\
890	0.000177225930089897 \\
900	0.000166190819954337 \\
910	0.000155839969831983 \\
920	0.000146129788863247 \\
930	0.000137019569361185 \\
940	0.000128471285582743 \\
950	0.000120449407766843 \\
960	0.000112920730099775 \\
970	0.000105854211420675 \\
980	9.92208276072715e-05 \\
990	9.29934346871608e-05 \\
1000	8.71466418173039e-05 \\
1010	8.16566933532492e-05 \\
1020	7.65013592995412e-05 \\
1030	7.16598334993268e-05 \\
1040	6.71126389746868e-05 \\
1050	6.28415398769566e-05 \\
1060	5.88294595577077e-05 \\
1070	5.50604043042524e-05 \\
1080	5.15193923243951e-05 \\
1090	4.81923875978452e-05 \\
1100	4.50662382415201e-05 \\
1110	4.21286190612213e-05 \\
1120	3.93679779924772e-05 \\
1130	3.67734861523306e-05 \\
1140	3.43349912447222e-05 \\
1150	3.20429740845452e-05 \\
1160	2.9888508020659e-05 \\
1170	2.78632210543028e-05 \\
1180	2.59592604680003e-05 \\
1190	2.41692597876542e-05 \\
1200	2.24863079189008e-05 \\
1210	2.09488757275667e-05 \\
1220	1.96647697905949e-05 \\
1230	1.84613917469933e-05 \\
1240	1.733317392838e-05 \\
1250	1.62750525265487e-05 \\
1260	1.52823750521613e-05 \\
1270	1.43508512726598e-05 \\
1280	1.34765141805993e-05 \\
1290	1.26556873504824e-05 \\
1300	1.18849572336988e-05 \\
1310	1.11611493725516e-05 \\
1320	1.04813077591581e-05 \\
1330	9.84267674597472e-06 \\
1340	9.24268505370085e-06 \\
1350	8.67893151940136e-06 \\
1360	8.16130177172525e-06 \\
1370	7.67512496874234e-06 \\
1380	7.21839115702672e-06 \\
1390	6.78925897396043e-06 \\
1400	6.38601170183106e-06 \\
1410	6.00704600783875e-06 \\
1420	5.65086332265929e-06 \\
1430	5.3160621504289e-06 \\
1440	5.00133109937462e-06 \\
1450	4.70544253033944e-06 \\
1460	4.42724674787343e-06 \\
1470	4.16566666860962e-06 \\
1480	3.91969291574368e-06 \\
1490	3.68837929259902e-06 \\
1500	3.47083859897346e-06 \\
1510	3.26623875801513e-06 \\
1520	3.07379922254158e-06 \\
1530	2.89278763992984e-06 \\
1540	2.7225167505418e-06 \\
1550	2.56234150158852e-06 \\
1560	2.41165635950225e-06 \\
1570	2.26989280560641e-06 \\
1580	2.13651699865203e-06 \\
1590	2.01102759461724e-06 \\
1600	1.89295371161302e-06 \\
1610	1.78185302673883e-06 \\
1620	1.67730999889315e-06 \\
1630	1.57893420749122e-06 \\
1640	1.48635879682057e-06 \\
1650	1.39923902170436e-06 \\
1660	1.31725088708956e-06 \\
1670	1.24008987228974e-06 \\
1680	1.16746973799486e-06 \\
1690	1.09912140822122e-06 \\
1700	1.03479192226086e-06 \\
1710	9.74243453688572e-07 \\
1720	9.17252389098788e-07 \\
1730	8.63608465906385e-07 \\
1740	8.13113963271661e-07 \\
1750	7.6558294281881e-07 \\
1760	7.20840536871936e-07 \\
1770	6.78722280711419e-07 \\
1780	6.39073485519948e-07 \\
1790	6.01748650741474e-07 \\
1800	5.66610910801568e-07 \\
1810	5.33531517910024e-07 \\
1820	5.02389354839483e-07 \\
1830	4.73070478179682e-07 \\
1840	4.45467688736656e-07 \\
1850	4.19480129076888e-07 \\
1860	3.95012904219616e-07 \\
1870	3.71976726754042e-07 \\
1880	3.50287581496467e-07 \\
1890	3.29866412851487e-07 \\
1900	3.10638828826537e-07 \\
1910	2.92534824086754e-07 \\
1920	2.75488519274614e-07 \\
1930	2.59437916205751e-07 \\
1940	2.44324667497686e-07 \\
1950	2.30093860020819e-07 \\
1960	2.16693811672108e-07 \\
1970	2.04075880139154e-07 \\
1980	1.92194282599978e-07 \\
1990	1.8100592724668e-07 \\
2000	1.70470253524346e-07 \\
2010	1.6054908330565e-07 \\
2020	1.51206479559463e-07 \\
2030	1.42408614733913e-07 \\
2040	1.34123645800788e-07 \\
2050	1.26321597238022e-07 \\
2060	1.18974251728243e-07 \\
2070	1.12055045686787e-07 \\
2080	1.05538972339225e-07 \\
2090	9.94024899059198e-08 \\
2100	9.3623435892809e-08 \\
2110	8.81809450459237e-08 \\
2120	8.30553739117335e-08 \\
2130	7.82282287836722e-08 \\
2140	7.36820980895558e-08 \\
2150	6.94005886647808e-08 \\
2160	6.5368266632948e-08 \\
2170	6.15705998963101e-08 \\
2180	5.79939066769342e-08 \\
2190	5.4625304501954e-08 \\
2200	5.14526639072699e-08 \\
2210	4.84645639731163e-08 \\
2220	4.56502510792767e-08 \\
2230	4.29995997142107e-08 \\
2240	4.05030755046276e-08 \\
2250	3.81517007985721e-08 \\
2260	3.59370221914013e-08 \\
2270	3.38510793285174e-08 \\
2280	3.18863765946809e-08 \\
2290	3.00358556359903e-08 \\
2300	2.82928698802642e-08 \\
2310	2.6651160334179e-08 \\
2320	2.5104833212275e-08 \\
2330	2.36483380655628e-08 \\
2340	2.22764483526205e-08 \\
2350	2.09842419551798e-08 \\
2360	1.97670838586461e-08 \\
2370	1.86206089436425e-08 \\
2380	1.7540706553909e-08 \\
2390	1.65235055082924e-08 \\
2400	1.5565359945402e-08 \\
2410	1.46628365560453e-08 \\
2420	1.38127019266854e-08 \\
2430	1.3011910826588e-08 \\
2440	1.22575955496806e-08 \\
2450	1.15470550343666e-08 \\
2460	1.08777456486742e-08 \\
2470	1.02472716423385e-08 \\
2480	9.65337682012901e-09 \\
2490	9.09393632619881e-09 \\
2500	8.56694859496798e-09 \\
2510	8.07052880080761e-09 \\
2520	7.60290175261247e-09 \\
2530	7.16239523246287e-09 \\
2540	6.74743472206529e-09 \\
2550	6.35653679692538e-09 \\
2560	5.98830446341125e-09 \\
2570	5.64142205172757e-09 \\
2580	5.31465038644541e-09 \\
2590	5.00682184600976e-09 \\
2600	4.71683730962624e-09 \\
2610	4.4436612722798e-09 \\
2620	4.18631823650983e-09 \\
2630	3.94388960378578e-09 \\
2640	3.71551017730454e-09 \\
2650	3.50036466478798e-09 \\
2660	3.29768556905918e-09 \\
2670	3.10674969083991e-09 \\
2680	2.92687596381569e-09 \\
2690	2.75742284561176e-09 \\
2700	2.5977859308135e-09 \\
2710	2.4473960080762e-09 \\
2720	2.30571711723471e-09 \\
2730	2.17224388476822e-09 \\
2740	2.04650074664414e-09 \\
2750	1.92803945031628e-09 \\
2760	1.81643822205757e-09 \\
2770	1.71129904691369e-09 \\
2780	1.61224755768075e-09 \\
2790	1.51893109201495e-09 \\
2800	1.43101691607583e-09 \\
2810	1.3481922245262e-09 \\
2820	1.27016219764187e-09 \\
2830	1.19664878006631e-09 \\
2840	1.12739051427724e-09 \\
2850	1.06214076422972e-09 \\
2860	1.00066743780047e-09 \\
2870	9.42751487986726e-10 \\
2880	8.8818719046202e-10 \\
2890	8.36780367219347e-10 \\
2900	7.88347942481948e-10 \\
2910	7.42717942703308e-10 \\
2920	6.99727886743773e-10 \\
2930	6.59224952404003e-10 \\
2940	6.21065143757704e-10 \\
2950	5.8511290257357e-10 \\
2960	5.51240386670315e-10 \\
2970	5.19326914805163e-10 \\
2980	4.89259410763054e-10 \\
2990	4.60930904555568e-10 \\
3000	4.34240698954369e-10 \\
3010	4.09094258468912e-10 \\
3020	3.85401710545352e-10 \\
3030	3.63079288856483e-10 \\
3040	3.42047556944891e-10 \\
3050	3.22231963334474e-10 \\
3060	3.03562064374319e-10 \\
3070	2.85971468727553e-10 \\
3080	2.69397948393646e-10 \\
3090	2.53782439507688e-10 \\
3100	2.39069763985356e-10 \\
3110	2.25207352766432e-10 \\
3120	2.12146356037834e-10 \\
3130	1.99840199943679e-10 \\
3140	1.88245530274855e-10 \\
3150	1.77320713667939e-10 \\
3160	1.67027391917429e-10 \\
3170	1.57328816641211e-10 \\
3180	1.48190848481278e-10 \\
3190	1.39580791369553e-10 \\
3200	1.31468225195164e-10 \\
3210	1.23824561715224e-10 \\
3220	1.16622433932179e-10 \\
3230	1.09836417738762e-10 \\
3240	1.03442532317644e-10 \\
3250	9.74180180968176e-11 \\
3260	9.17413367496067e-11 \\
3270	8.6392837328475e-11 \\
3280	8.13531464416428e-11 \\
3290	7.66046115430186e-11 \\
3300	7.21303017314767e-11 \\
3310	6.79144518400676e-11 \\
3320	6.39420738579588e-11 \\
3330	6.019917897504e-11 \\
3340	5.66723334927133e-11 \\
3350	5.33492139354053e-11 \\
3360	5.02180519390549e-11 \\
3370	4.72675787399623e-11 \\
3380	4.44876357974522e-11 \\
3390	4.18681200820004e-11 \\
3400	3.93998722536537e-11 \\
3410	3.70741215505177e-11 \\
3420	3.48824857887564e-11 \\
3430	3.28176930075585e-11 \\
3440	3.08718606234493e-11 \\
3450	2.90384938317345e-11 \\
3460	2.73109868054178e-11 \\
3470	2.56831778067124e-11 \\
3480	2.41491826535878e-11 \\
3490	2.27038943201308e-11 \\
3500	2.13420392469743e-11 \\
3510	2.00586769416589e-11 \\
3520	1.88494220232371e-11 \\
3530	1.77099446219131e-11 \\
3540	1.66363034459494e-11 \\
3550	1.56245572036084e-11 \\
3560	1.46712086923628e-11 \\
3570	1.37728162208361e-11 \\
3580	1.29263266757107e-11 \\
3590	1.21286869436688e-11 \\
3600	1.13771214671488e-11 \\
3610	1.06688546885891e-11 \\
3620	1.00015551396382e-11 \\
3630	9.37261379618803e-12 \\
3640	8.78003225679436e-12 \\
3650	8.22170109771037e-12 \\
3660	7.69539987288681e-12 \\
3670	7.19962978124045e-12 \\
3680	6.73233691017572e-12 \\
3690	6.29213348091184e-12 \\
3700	5.87729864776065e-12 \\
3710	5.4863336096389e-12 \\
3720	5.11785058776582e-12 \\
3730	4.77073935911676e-12 \\
3740	4.44366765606219e-12 \\
3750	4.13530321097255e-12 \\
3760	3.84497989003307e-12 \\
3770	3.57125440331174e-12 \\
3780	3.31323857238885e-12 \\
3790	3.07015524114718e-12 \\
3800	2.84111623116701e-12 \\
3810	2.62517785287741e-12 \\
3820	2.42195152821978e-12 \\
3830	2.23027152301825e-12 \\
3840	2.0497492592142e-12 \\
3850	1.87949655838793e-12 \\
3860	1.71912484248082e-12 \\
3870	1.56802348882934e-12 \\
3880	1.4255263636187e-12 \\
3890	1.29135591109275e-12 \\
3900	1.16484599743671e-12 \\
3910	1.0456635557432e-12 \\
3920	9.33364496802369e-13 \\
3930	8.27560242555592e-13 \\
3940	7.27806703793021e-13 \\
3950	6.33881835909733e-13 \\
3960	5.45230527393414e-13 \\
3970	4.61797267092834e-13 \\
3980	3.83193476949373e-13 \\
3990	3.09141601206875e-13 \\
};


\addplot [red,mark=text, text mark={\tiny 10\%},mark repeat = 100]
table [row sep=\\]{%
0	2.66979517505131 \\
10	2.02283246511696 \\
20	1.61519757858965 \\
30	1.31350915603608 \\
40	1.07897287260949 \\
50	0.883935256530944 \\
60	0.719488877515996 \\
70	0.577098505113612 \\
80	0.466935653936808 \\
90	0.377248864864245 \\
100	0.306300623661983 \\
110	0.247148369586576 \\
120	0.196354066009793 \\
130	0.156787861646511 \\
140	0.122306339570701 \\
150	0.095489335510445 \\
160	0.0740258540993126 \\
170	0.0588814039115829 \\
180	0.0462712305051751 \\
190	0.0399077581063943 \\
200	0.0349365713224245 \\
210	0.0308859295852145 \\
220	0.0274599848737583 \\
230	0.0243073896497898 \\
240	0.0218919616995246 \\
250	0.0199484698040907 \\
260	0.0181615462615447 \\
270	0.0164952169761741 \\
280	0.0149411120182689 \\
290	0.0134903214429895 \\
300	0.0121350739799063 \\
310	0.0108684368453455 \\
320	0.00968408782283064 \\
330	0.00857692509648228 \\
340	0.00755569140138851 \\
350	0.0066825953167795 \\
360	0.00594125789358579 \\
370	0.00524793321442096 \\
380	0.00459932382176198 \\
390	0.00405051805009421 \\
400	0.00365320876598763 \\
410	0.00329362121654514 \\
420	0.00296392393800965 \\
430	0.0026632042382857 \\
440	0.00248568842282731 \\
450	0.00232155467628625 \\
460	0.00216824936016174 \\
470	0.00202485432386906 \\
480	0.00189059035525924 \\
490	0.00176478526453067 \\
500	0.00164683509822672 \\
510	0.00153670305982329 \\
520	0.0014335738150012 \\
530	0.0013366632620605 \\
540	0.00124557261431935 \\
550	0.00115993442161894 \\
560	0.00107940862489669 \\
570	0.00100367947309105 \\
580	0.000932453020382751 \\
590	0.000865455056457132 \\
600	0.000802429365666824 \\
610	0.000743136239182041 \\
620	0.000688175715491957 \\
630	0.000638298144586102 \\
640	0.000592532042987981 \\
650	0.000549594380536977 \\
660	0.000509302026585445 \\
670	0.000471484884447426 \\
680	0.000435984777269915 \\
690	0.000402654469492836 \\
700	0.000374157199716341 \\
710	0.000348710943471608 \\
720	0.000325037157071495 \\
730	0.000303361084268317 \\
740	0.000283146747680929 \\
750	0.00026429279407264 \\
760	0.000246781815234465 \\
770	0.000230452626546307 \\
780	0.000215221745222482 \\
790	0.000201012461757799 \\
800	0.000187753998534324 \\
810	0.000175380915224899 \\
820	0.000163832608759351 \\
830	0.00015305288228401 \\
840	0.000142989569157204 \\
850	0.000133594201671361 \\
860	0.000124821716599444 \\
870	0.000116630191458522 \\
880	0.000108980606715547 \\
890	0.000101836630173768 \\
900	9.51644205409163e-05 \\
910	8.89324477622022e-05 \\
920	8.31113281444806e-05 \\
930	7.76736726424421e-05 \\
940	7.25939469440884e-05 \\
950	6.78483421974185e-05 \\
960	6.34146553920023e-05 \\
970	5.92721785348527e-05 \\
980	5.5401595869975e-05 \\
990	5.17848884789007e-05 \\
1000	4.84052456689588e-05 \\
1010	4.52469826221491e-05 \\
1020	4.22954638216733e-05 \\
1030	3.95370318261312e-05 \\
1040	3.69589409934257e-05 \\
1050	3.45492957929405e-05 \\
1060	3.22969933747586e-05 \\
1070	3.01916700901628e-05 \\
1080	2.82236516824241e-05 \\
1090	2.63839068887495e-05 \\
1100	2.46640042131929e-05 \\
1110	2.3056071647154e-05 \\
1120	2.15527591353459e-05 \\
1130	2.0147203591947e-05 \\
1140	1.88329962937939e-05 \\
1150	1.76041524838078e-05 \\
1160	1.64550830344945e-05 \\
1170	1.53805680268571e-05 \\
1180	1.43757321162119e-05 \\
1190	1.34360215600648e-05 \\
1200	1.2557182794859e-05 \\
1210	1.17352424566786e-05 \\
1220	1.09664887458782e-05 \\
1230	1.02474540447095e-05 \\
1240	9.57489870301309e-06 \\
1250	8.94579591148581e-06 \\
1260	8.35731758924707e-06 \\
1270	7.80682121725995e-06 \\
1280	7.29183755260321e-06 \\
1290	6.8100591651965e-06 \\
1300	6.35932973896969e-06 \\
1310	5.93763408895942e-06 \\
1320	5.54308884376242e-06 \\
1330	5.17393374893649e-06 \\
1340	4.82852354999341e-06 \\
1350	4.50532041540441e-06 \\
1360	4.20288686381332e-06 \\
1370	3.91987916115166e-06 \\
1380	3.668725892636e-06 \\
1390	3.43609684849699e-06 \\
1400	3.21846903500456e-06 \\
1410	3.01484062997792e-06 \\
1420	2.82428436942928e-06 \\
1430	2.64593827126181e-06 \\
1440	2.47900028738091e-06 \\
1450	2.32272365985331e-06 \\
1460	2.1764127249857e-06 \\
1470	2.03941909893146e-06 \\
1480	1.91113819725253e-06 \\
1490	1.79100605368632e-06 \\
1500	1.67849640164697e-06 \\
1510	1.57311799625637e-06 \\
1520	1.47441214709554e-06 \\
1530	1.38195044596667e-06 \\
1540	1.29533266884918e-06 \\
1550	1.21418483634006e-06 \\
1560	1.13815742003309e-06 \\
1570	1.06692368023742e-06 \\
1580	1.00017812626474e-06 \\
1590	9.37635086295519e-07 \\
1600	8.79027380551456e-07 \\
1610	8.24105088559346e-07 \\
1620	7.72634402290695e-07 \\
1630	7.24396560403129e-07 \\
1640	6.7918685531243e-07 \\
1650	6.36813709709028e-07 \\
1660	5.97097815413505e-07 \\
1670	5.59871331184958e-07 \\
1680	5.24977135707427e-07 \\
1690	4.92268129259621e-07 \\
1700	4.61606583901375e-07 \\
1710	4.32863536403705e-07 \\
1720	4.05918221202395e-07 \\
1730	3.8065754243144e-07 \\
1740	3.56975579207663e-07 \\
1750	3.34773124444077e-07 \\
1760	3.13957253916808e-07 \\
1770	2.94440923642725e-07 \\
1780	2.76142593291784e-07 \\
1790	2.58985874579398e-07 \\
1800	2.42899201585711e-07 \\
1810	2.27815524389552e-07 \\
1820	2.13672020410449e-07 \\
1830	2.00409825956704e-07 \\
1840	1.87973784260276e-07 \\
1850	1.76312209998475e-07 \\
1860	1.65376669414297e-07 \\
1870	1.55121773925959e-07 \\
1880	1.45504987114631e-07 \\
1890	1.36486444257677e-07 \\
1900	1.28028783574763e-07 \\
1910	1.20096987854534e-07 \\
1920	1.12658236128826e-07 \\
1930	1.0568176572745e-07 \\
1940	9.91387414939204e-08 \\
1950	9.30021352707477e-08 \\
1960	8.72466114354431e-08 \\
1970	8.18484199860414e-08 \\
1980	7.67852978422745e-08 \\
1990	7.20363743100805e-08 \\
2000	6.75820840956298e-08 \\
2010	6.34040847047324e-08 \\
2020	5.948518039256e-08 \\
2030	5.58092493885276e-08 \\
2040	5.23611771163779e-08 \\
2050	4.91267926339134e-08 \\
2060	4.60928094025981e-08 \\
2070	4.32467702760064e-08 \\
2080	4.05769950417856e-08 \\
2090	3.80725320714426e-08 \\
2100	3.57231124126223e-08 \\
2110	3.35191073230767e-08 \\
2120	3.14514879695693e-08 \\
2130	2.95117882909146e-08 \\
2140	2.76920694153304e-08 \\
2150	2.5984886686814e-08 \\
2160	2.43832589674753e-08 \\
2170	2.2880639605205e-08 \\
2180	2.14708895662774e-08 \\
2190	2.0148251345109e-08 \\
2200	1.89073257050865e-08 \\
2210	1.77430489745056e-08 \\
2220	1.6650672673979e-08 \\
2230	1.56257430883322e-08 \\
2240	1.46640831144573e-08 \\
2250	1.37617756634789e-08 \\
2260	1.29151463967858e-08 \\
2270	1.21207494041542e-08 \\
2280	1.13753528263594e-08 \\
2290	1.06759251439215e-08 \\
2300	1.00196230201632e-08 \\
2310	9.4037794218238e-09 \\
2320	8.82589323847327e-09 \\
2330	8.2836177917045e-09 \\
2340	7.77475239743808e-09 \\
2350	7.29723254044856e-09 \\
2360	6.84912182524755e-09 \\
2370	6.428603649411e-09 \\
2380	6.03397426468533e-09 \\
2390	5.66363494991506e-09 \\
2400	5.31608634890546e-09 \\
2410	4.98992197561776e-09 \\
2420	4.68382227447606e-09 \\
2430	4.39654929129674e-09 \\
2440	4.12694189932949e-09 \\
2450	3.87391047018681e-09 \\
2460	3.63643304357453e-09 \\
2470	3.41355088639972e-09 \\
2480	3.20436427392323e-09 \\
2490	3.00802921460175e-09 \\
2500	2.82375400839641e-09 \\
2510	2.65079569405913e-09 \\
2520	2.4884574401085e-09 \\
2530	2.33608521416073e-09 \\
2540	2.19306545146125e-09 \\
2550	2.05882250137179e-09 \\
2560	1.93281618487973e-09 \\
2570	1.81453985170776e-09 \\
2580	1.7035177157787e-09 \\
2590	1.5993037449924e-09 \\
2600	1.50147944077972e-09 \\
2610	1.40965167316764e-09 \\
2620	1.3234519591343e-09 \\
2630	1.24253429767407e-09 \\
2640	1.16657428161915e-09 \\
2650	1.09526721026043e-09 \\
2660	1.02832731219138e-09 \\
2670	9.65486413040395e-10 \\
2680	9.06492658714342e-10 \\
2690	8.51110015798184e-10 \\
2700	7.99116717242754e-10 \\
2710	7.5030492929784e-10 \\
2720	7.04479308222261e-10 \\
2730	6.61456778239256e-10 \\
2740	6.21065254780007e-10 \\
2750	5.83143477950188e-10 \\
2760	5.47540068840391e-10 \\
2770	5.14112807881162e-10 \\
2780	4.82728357287243e-10 \\
2790	4.53261539412608e-10 \\
2800	4.25594892661252e-10 \\
2810	3.99618116375677e-10 \\
2820	3.75227682258839e-10 \\
2830	3.52326556818383e-10 \\
2840	3.30823479721687e-10 \\
2850	3.10632741751249e-10 \\
2860	2.91674295826994e-10 \\
2870	2.73872535760944e-10 \\
2880	2.57156906879885e-10 \\
2890	2.41460629268886e-10 \\
2900	2.2672164146087e-10 \\
2910	2.12881379191288e-10 \\
2920	1.99884719886967e-10 \\
2930	1.8768037124417e-10 \\
2940	1.76219649983267e-10 \\
2950	1.65457314516004e-10 \\
2960	1.55350565744783e-10 \\
2970	1.45859491151867e-10 \\
2980	1.36946287643269e-10 \\
2990	1.2857587217141e-10 \\
3000	1.20714993556703e-10 \\
3010	1.13332621065609e-10 \\
3020	1.06399555832581e-10 \\
3030	9.98882643266086e-11 \\
3040	9.37730448846708e-11 \\
3050	8.80298611782848e-11 \\
3060	8.26360091465972e-11 \\
3070	7.75700614852326e-11 \\
3080	7.2812089690899e-11 \\
3090	6.83433865056315e-11 \\
3100	6.41461883610361e-11 \\
3110	6.02041194674996e-11 \\
3120	5.65014701692235e-11 \\
3130	5.30236965445852e-11 \\
3140	4.97572538726843e-11 \\
3150	4.66890970329814e-11 \\
3160	4.38071801056594e-11 \\
3170	4.11003453493208e-11 \\
3180	3.85577125783243e-11 \\
3190	3.61694008077507e-11 \\
3200	3.39259731418906e-11 \\
3210	3.18187143300008e-11 \\
3220	2.98391866770942e-11 \\
3230	2.79796186219983e-11 \\
3240	2.62329047373555e-11 \\
3250	2.4592106129262e-11 \\
3260	2.30507279930237e-11 \\
3270	2.16027751243075e-11 \\
3280	2.02425298745368e-11 \\
3290	1.89647186843445e-11 \\
3300	1.77643455501197e-11 \\
3310	1.66366365128567e-11 \\
3320	1.55772061916082e-11 \\
3330	1.45820022723342e-11 \\
3340	1.36468614186924e-11 \\
3350	1.27683974504578e-11 \\
3360	1.19432241874051e-11 \\
3370	1.11678999381581e-11 \\
3380	1.04394271005503e-11 \\
3390	9.75514113932263e-12 \\
3400	9.11215547461097e-12 \\
3410	8.50802761576119e-12 \\
3420	7.94042609442158e-12 \\
3430	7.40713046454289e-12 \\
3440	6.90608681352956e-12 \\
3450	6.43535225108849e-12 \\
3460	5.99303939807783e-12 \\
3470	5.57748291996063e-12 \\
3480	5.18701748219996e-12 \\
3490	4.82014428371258e-12 \\
3500	4.47530901226401e-12 \\
3510	4.15140144482962e-12 \\
3520	3.8469782914774e-12 \\
3530	3.56098484033396e-12 \\
3540	3.29225535722344e-12 \\
3550	3.03973513027245e-12 \\
3560	2.80248046991005e-12 \\
3570	2.57949217541409e-12 \\
3580	2.37004860181855e-12 \\
3590	2.17303952609882e-12 \\
3600	1.98796534789381e-12 \\
3610	1.81421544453997e-12 \\
3620	1.65067959301268e-12 \\
3630	1.49724677100949e-12 \\
3640	1.35291777780822e-12 \\
3650	1.21724852419902e-12 \\
3660	1.08990594327452e-12 \\
3670	9.70112878917462e-13 \\
3680	8.57591775371702e-13 \\
3690	7.51843032276156e-13 \\
3700	6.52422560420973e-13 \\
3710	5.59108315201229e-13 \\
3720	4.71178651650916e-13 \\
3730	3.88744592072499e-13 \\
3740	3.11251024953663e-13 \\
3750	2.3830937223579e-13 \\
3760	1.69753100465186e-13 \\
3770	1.05415676188159e-13 \\
3780	4.49640324973188e-14 \\
3790	-1.19348975147204e-14 \\
3800	-6.55031584528842e-14 \\
3810	-1.15574216863479e-13 \\
3820	-1.6286971771251e-13 \\
3830	-2.07223127546285e-13 \\
3840	-2.4891200212096e-13 \\
3850	-2.88213897192691e-13 \\
3860	-3.24906768156552e-13 \\
3870	-3.59545726524857e-13 \\
3880	-3.92186283448837e-13 \\
3890	-4.22828438928491e-13 \\
3900	-4.51583215266282e-13 \\
3910	-4.78561634764674e-13 \\
3920	-5.04041253179821e-13 \\
3930	-5.27855537058031e-13 \\
3940	-5.5033755330669e-13 \\
3950	-5.71431790774568e-13 \\
3960	-5.9136029406659e-13 \\
3970	-6.09956529729061e-13 \\
3980	-6.2755356466937e-13 \\
3990	-6.43984865433822e-13 \\
};


\addplot [ dashed, red,mark=text, text mark={\tiny 20\%},mark repeat = 100]
table [row sep=\\]{%
0	3.2362771223447 \\
10	2.70519029990559 \\
20	2.26831500697808 \\
30	1.90671043268573 \\
40	1.60508478698664 \\
50	1.34327204885578 \\
60	1.12431659412896 \\
70	0.941323331969572 \\
80	0.783151525877136 \\
90	0.654614632794805 \\
100	0.546751657653721 \\
110	0.453659920355868 \\
120	0.37051258651783 \\
130	0.297421474570098 \\
140	0.237644044082886 \\
150	0.189563578821049 \\
160	0.151511312722694 \\
170	0.120747753275547 \\
180	0.0982481020103385 \\
190	0.0797970665935637 \\
200	0.0653007072702782 \\
210	0.0535167221282357 \\
220	0.0451221647168469 \\
230	0.0380130772957125 \\
240	0.0321331876727168 \\
250	0.0273526450165275 \\
260	0.0230957896185156 \\
270	0.0196379879239438 \\
280	0.0168889939765222 \\
290	0.0149346535449601 \\
300	0.0132457947866588 \\
310	0.0117173656845562 \\
320	0.0103799462895621 \\
330	0.009146247244814 \\
340	0.00812093803872765 \\
350	0.00730937313795516 \\
360	0.00657945598451132 \\
370	0.00590922427560442 \\
380	0.00529317552225628 \\
390	0.00472769280450064 \\
400	0.00422230353550229 \\
410	0.00375876833221744 \\
420	0.0033311235593112 \\
430	0.00293893110563503 \\
440	0.00259297431409911 \\
450	0.00228286431736674 \\
460	0.00200880920107394 \\
470	0.00175692170846919 \\
480	0.00152611782627021 \\
490	0.0013310780319768 \\
500	0.0011681249096076 \\
510	0.00101887934143219 \\
520	0.000881366126202598 \\
530	0.000754488114585838 \\
540	0.000637328069973297 \\
550	0.000529316522761558 \\
560	0.000436630357601886 \\
570	0.00039093461713996 \\
580	0.000356121805988818 \\
590	0.000325776745113748 \\
600	0.000298249828152908 \\
610	0.000273224413199713 \\
620	0.000250429811908648 \\
630	0.000229638784701724 \\
640	0.000210660016016384 \\
650	0.000193389045993864 \\
660	0.000177688885701743 \\
670	0.000163316890091469 \\
680	0.000150154421987903 \\
690	0.000138089343545644 \\
700	0.0001270302366847 \\
710	0.000116875044159637 \\
720	0.000107552577184256 \\
730	9.89930399872274e-05 \\
740	9.11199494663117e-05 \\
750	8.38771910784963e-05 \\
760	7.72162183415581e-05 \\
770	7.1081372907511e-05 \\
780	6.54348036128183e-05 \\
790	6.02315474116377e-05 \\
800	5.54346564976549e-05 \\
810	5.10104104530784e-05 \\
820	4.69318564594867e-05 \\
830	4.31695302008372e-05 \\
840	3.96935098819773e-05 \\
850	3.64839367385517e-05 \\
860	3.35185227602341e-05 \\
870	3.07776940203786e-05 \\
880	2.82448084746312e-05 \\
890	2.59978846126052e-05 \\
900	2.41286882698422e-05 \\
910	2.24009971971917e-05 \\
920	2.08028808740823e-05 \\
930	1.93240693324159e-05 \\
940	1.79544990522862e-05 \\
950	1.66855060904081e-05 \\
960	1.55094522524823e-05 \\
970	1.44187676790475e-05 \\
980	1.34068382147134e-05 \\
990	1.24676006407753e-05 \\
1000	1.15956475335022e-05 \\
1010	1.07858077686429e-05 \\
1020	1.00334330608853e-05 \\
1030	9.33415797083059e-06 \\
1040	8.68413542548518e-06 \\
1050	8.07966133381521e-06 \\
1060	7.51741324023225e-06 \\
1070	6.99432429490665e-06 \\
1080	6.5075476188281e-06 \\
1090	6.05445636753243e-06 \\
1100	5.63262521552366e-06 \\
1110	5.23981359473735e-06 \\
1120	4.87395049009187e-06 \\
1130	4.53312062176359e-06 \\
1140	4.21555186957967e-06 \\
1150	3.91960381290746e-06 \\
1160	3.64375727363031e-06 \\
1170	3.38660476290054e-06 \\
1180	3.14684174673729e-06 \\
1190	2.92325865053344e-06 \\
1200	2.72246588517433e-06 \\
1210	2.54300735941237e-06 \\
1220	2.37569151706474e-06 \\
1230	2.21964928209806e-06 \\
1240	2.07408765795503e-06 \\
1250	1.93827438216276e-06 \\
1260	1.8115321248624e-06 \\
1270	1.69323379550779e-06 \\
1280	1.58279836487418e-06 \\
1290	1.47968710612068e-06 \\
1300	1.38340020061634e-06 \\
1310	1.29347366156773e-06 \\
1320	1.20947654036474e-06 \\
1330	1.13100838233793e-06 \\
1340	1.0576969051157e-06 \\
1350	9.89195875822357e-07 \\
1360	9.2518316757717e-07 \\
1370	8.65358975532526e-07 \\
1380	8.09444180349672e-07 \\
1390	7.57178841515049e-07 \\
1400	7.08320810782759e-07 \\
1410	6.62644454751948e-07 \\
1420	6.19939475920983e-07 \\
1430	5.80009824280303e-07 \\
1440	5.42672692338542e-07 \\
1450	5.07757584922164e-07 \\
1460	4.75105458752623e-07 \\
1470	4.44567926194406e-07 \\
1480	4.16006517345302e-07 \\
1490	3.89291995472885e-07 \\
1500	3.64303723021653e-07 \\
1510	3.40929073139318e-07 \\
1520	3.1906288228134e-07 \\
1530	2.98606943949231e-07 \\
1540	2.79469535957588e-07 \\
1550	2.61564982728668e-07 \\
1560	2.44813247896047e-07 \\
1570	2.29139554830393e-07 \\
1580	2.14474034310186e-07 \\
1590	2.00751396006726e-07 \\
1600	1.87910623061782e-07 \\
1610	1.75894687592937e-07 \\
1620	1.64650285239265e-07 \\
1630	1.54127588469777e-07 \\
1640	1.44280016489695e-07 \\
1650	1.35064020134745e-07 \\
1660	1.26438882641633e-07 \\
1670	1.18366532575465e-07 \\
1680	1.10811369635755e-07 \\
1690	1.03740102674887e-07 \\
1700	9.71215983747165e-08 \\
1710	9.09267394155755e-08 \\
1720	8.51282931924047e-08 \\
1730	7.97007880248835e-08 \\
1740	7.46203984713922e-08 \\
1750	6.98648381369793e-08 \\
1760	6.54132591426659e-08 \\
1770	6.12461587556901e-08 \\
1780	5.73452916818873e-08 \\
1790	5.36935886308321e-08 \\
1800	5.0275079710449e-08 \\
1810	4.70748232062057e-08 \\
1820	4.40788392452873e-08 \\
1830	4.12740470134842e-08 \\
1840	3.86482068015503e-08 \\
1850	3.61898659373416e-08 \\
1860	3.38883071049345e-08 \\
1870	3.17335016597475e-08 \\
1880	2.97160644424643e-08 \\
1890	2.78272133669155e-08 \\
1900	2.60587288414271e-08 \\
1910	2.44029192963957e-08 \\
1920	2.28525858791961e-08 \\
1930	2.14009912569146e-08 \\
1940	2.00418298068605e-08 \\
1950	1.8769199972013e-08 \\
1960	1.75775781152687e-08 \\
1970	1.64617942055578e-08 \\
1980	1.54170090582717e-08 \\
1990	1.44386934075591e-08 \\
2000	1.35226075892447e-08 \\
2010	1.26647832221494e-08 \\
2020	1.18615056665661e-08 \\
2030	1.11092975374483e-08 \\
2040	1.04049042159993e-08 \\
2050	9.74527863961683e-09 \\
2060	9.12756870086184e-09 \\
2070	8.54910414682664e-09 \\
2080	8.00738558792702e-09 \\
2090	7.50007328464974e-09 \\
2100	7.02497643390032e-09 \\
2110	6.58004462028572e-09 \\
2120	6.16335749104024e-09 \\
2130	5.773118316732e-09 \\
2140	5.40764388823334e-09 \\
2150	5.06535918765039e-09 \\
2160	4.74478911716147e-09 \\
2170	4.44455278136857e-09 \\
2180	4.1633573255595e-09 \\
2190	3.89999238459282e-09 \\
2200	3.65332464280499e-09 \\
2210	3.42229283800677e-09 \\
2220	3.20590370916918e-09 \\
2230	3.00322672286413e-09 \\
2240	2.8133910201511e-09 \\
2250	2.63558103119621e-09 \\
2260	2.46903320011427e-09 \\
2270	2.31303243225511e-09 \\
2280	2.16690904109029e-09 \\
2290	2.03003647225586e-09 \\
2300	1.90182730674948e-09 \\
2310	1.78173215070743e-09 \\
2320	1.66923608269087e-09 \\
2330	1.56385732141828e-09 \\
2340	1.46514422816324e-09 \\
2350	1.37267419653142e-09 \\
2360	1.2860514875257e-09 \\
2370	1.20490550870045e-09 \\
2380	1.12888903780473e-09 \\
2390	1.05767711255922e-09 \\
2400	9.90965198788274e-10 \\
2410	9.28468357752621e-10 \\
2420	8.69919580814837e-10 \\
2430	8.15068734727475e-10 \\
2440	7.63681839988095e-10 \\
2450	7.15539572038182e-10 \\
2460	6.70436484107029e-10 \\
2470	6.28180285566771e-10 \\
2480	5.8859073170936e-10 \\
2490	5.51499235168507e-10 \\
2500	5.16747811207807e-10 \\
2510	4.84188245053474e-10 \\
2520	4.5368225842779e-10 \\
2530	4.25099733192269e-10 \\
2540	3.98319044414563e-10 \\
2550	3.73226505256952e-10 \\
2560	3.49715423286767e-10 \\
2570	3.27685767409491e-10 \\
2580	3.07044001335299e-10 \\
2590	2.87702417445246e-10 \\
2600	2.69578970257811e-10 \\
2610	2.52596721317389e-10 \\
2620	2.3668361714968e-10 \\
2630	2.21772156194788e-10 \\
2640	2.07799166762612e-10 \\
2650	1.94705529477091e-10 \\
2660	1.82435455631236e-10 \\
2670	1.70937264343252e-10 \\
2680	1.60162161311206e-10 \\
2690	1.50064627391089e-10 \\
2700	1.40601918996452e-10 \\
2710	1.31733901564957e-10 \\
2720	1.23423216091822e-10 \\
2730	1.15634779529472e-10 \\
2740	1.08335562742923e-10 \\
2750	1.01494923576695e-10 \\
2760	9.50837741875432e-11 \\
2770	8.90752471782719e-11 \\
2780	8.34438629304657e-11 \\
2790	7.81659181825489e-11 \\
2800	7.32192084740291e-11 \\
2810	6.85828616120432e-11 \\
2820	6.42373376713579e-11 \\
2830	6.0164373483218e-11 \\
2840	5.63468716130444e-11 \\
2850	5.27686228046775e-11 \\
2860	4.94147500695874e-11 \\
2870	4.62709870419076e-11 \\
2880	4.33242330899475e-11 \\
2890	4.05621647381338e-11 \\
2900	3.79730136224055e-11 \\
2910	3.55460660905749e-11 \\
2920	3.3270997068513e-11 \\
2930	3.11384251716618e-11 \\
2940	2.91393020823705e-11 \\
2950	2.72654121502569e-11 \\
2960	2.55085397249388e-11 \\
2970	2.38616348902099e-11 \\
2980	2.23177587521661e-11 \\
2990	2.0870527528416e-11 \\
3000	1.95135574365679e-11 \\
3010	1.82415194061036e-11 \\
3020	1.70489733442025e-11 \\
3030	1.59309232472538e-11 \\
3040	1.48827061785539e-11 \\
3050	1.39000477794582e-11 \\
3060	1.29785071578681e-11 \\
3070	1.21148646670122e-11 \\
3080	1.13049014593969e-11 \\
3090	1.05453978882508e-11 \\
3100	9.83346737370994e-12 \\
3110	9.16600129130529e-12 \\
3120	8.5400020388704e-12 \\
3130	7.95313814805354e-12 \\
3140	7.40280059474685e-12 \\
3150	6.88665791059861e-12 \\
3160	6.40282271646697e-12 \\
3170	5.94918558860513e-12 \\
3180	5.52363710326631e-12 \\
3190	5.12478948166972e-12 \\
3200	4.75064432237104e-12 \\
3210	4.39986935774073e-12 \\
3220	4.07091027554429e-12 \\
3230	3.76237929700096e-12 \\
3240	3.47311068793488e-12 \\
3250	3.20177218071649e-12 \\
3260	2.94725355232117e-12 \\
3270	2.70872213548046e-12 \\
3280	2.48495668486726e-12 \\
3290	2.27501351091064e-12 \\
3300	2.07822647979583e-12 \\
3310	1.8936519019519e-12 \\
3320	1.72051262126161e-12 \\
3330	1.55819801506141e-12 \\
3340	1.40587541608284e-12 \\
3350	1.26293420166235e-12 \\
3360	1.12898579374132e-12 \\
3370	1.00325303620252e-12 \\
3380	8.85458373289794e-13 \\
3390	7.74935671188359e-13 \\
3400	6.71185329537138e-13 \\
3410	5.73929792579975e-13 \\
3420	4.82780482258249e-13 \\
3430	3.9718228705965e-13 \\
3440	3.16802140076788e-13 \\
3450	2.41584530158434e-13 \\
3460	1.70863323489812e-13 \\
3470	1.04638520070921e-13 \\
3480	4.25215418431435e-14 \\
3490	-1.58761892521397e-14 \\
3500	-7.05546732149287e-14 \\
3510	-1.21902488103842e-13 \\
3520	-1.70030656221343e-13 \\
3530	-2.15216733323587e-13 \\
3540	-2.57571741713036e-13 \\
3550	-2.97317725994617e-13 \\
3560	-3.3451019731956e-13 \\
3570	-3.69593244897715e-13 \\
3580	-4.02566868729082e-13 \\
3590	-4.33320046511199e-13 \\
3600	-4.62241356302684e-13 \\
3610	-4.89275286952306e-13 \\
3620	-5.14699394216223e-13 \\
3630	-5.38513678094432e-13 \\
3640	-5.60995694343092e-13 \\
3650	-5.81978909508507e-13 \\
3660	-6.01796390498066e-13 \\
3670	-6.20226092706844e-13 \\
3680	-6.37545571890996e-13 \\
3690	-6.53976872655448e-13 \\
3700	-6.69186928092813e-13 \\
3710	-6.8361982741294e-13 \\
3720	-6.97053526010905e-13 \\
3730	-7.09710068491631e-13 \\
3740	-7.2158945485512e-13 \\
3750	-7.3258066279891e-13 \\
3760	-7.43127781532849e-13 \\
3770	-7.5289774414955e-13 \\
3780	-7.62057084102707e-13 \\
3790	-7.70772334846015e-13 \\
3800	-7.78821451774547e-13 \\
3810	-7.8642647949323e-13 \\
3820	-7.93531906850831e-13 \\
3830	-8.00248756149813e-13 \\
3840	-8.06521516238945e-13 \\
3850	-8.12405698269458e-13 \\
3860	-8.17956813392584e-13 \\
3870	-8.23119350457091e-13 \\
3880	-8.28059842916673e-13 \\
3890	-8.32611757317636e-13 \\
3900	-8.36941627113674e-13 \\
3910	-8.40993941153556e-13 \\
3920	-8.4471318828605e-13 \\
3930	-8.48265901964851e-13 \\
3940	-8.51596571038726e-13 \\
3950	-8.54816217810139e-13 \\
3960	-8.57702797674165e-13 \\
3970	-8.60478355235728e-13 \\
3980	-8.63031868192365e-13 \\
3990	-8.65529869997772e-13 \\
};


\addplot [dotted, red,mark=text, text mark={\tiny 50\%},mark repeat = 100]
table [row sep=\\]{%
0	3.05754623134375 \\
10	2.51103589557646 \\
20	2.06811526162925 \\
30	1.72068865333226 \\
40	1.43004031705105 \\
50	1.18602667926735 \\
60	0.982892184964891 \\
70	0.814972806977224 \\
80	0.683185263628121 \\
90	0.577686798723891 \\
100	0.488611558424467 \\
110	0.412157843726483 \\
120	0.349340260380677 \\
130	0.298781496380663 \\
140	0.257438629526222 \\
150	0.222643827966988 \\
160	0.192426715511325 \\
170	0.167123885742266 \\
180	0.146580911830579 \\
190	0.128668248785467 \\
200	0.112521576949724 \\
210	0.098740889135461 \\
220	0.0867392612379725 \\
230	0.0779252312441432 \\
240	0.0699538421515053 \\
250	0.0627363752396216 \\
260	0.0571232216300618 \\
270	0.0522872084002348 \\
280	0.0477471807741496 \\
290	0.0435030523978138 \\
300	0.0397185940670108 \\
310	0.0362760323853878 \\
320	0.0332054206577786 \\
330	0.0304666827253359 \\
340	0.0279277450350094 \\
350	0.0255765817481997 \\
360	0.0234547815320261 \\
370	0.0215203006858838 \\
380	0.0197057225670911 \\
390	0.0180048715314601 \\
400	0.01640806854798 \\
410	0.0149626362886374 \\
420	0.013669769518957 \\
430	0.0125316097807399 \\
440	0.0115637207372394 \\
450	0.0106550299232834 \\
460	0.00980385167826886 \\
470	0.00907077080717472 \\
480	0.00842195688616576 \\
490	0.00781238933235368 \\
500	0.00723959349981718 \\
510	0.0067012671904858 \\
520	0.00619636275087243 \\
530	0.00572290049792429 \\
540	0.00527899356365263 \\
550	0.00486859478604235 \\
560	0.00448339250967711 \\
570	0.0041211742288691 \\
580	0.00378051290384285 \\
590	0.00346008045123131 \\
600	0.00315863830780344 \\
610	0.00287502981839188 \\
620	0.00260817363494509 \\
630	0.00235705790261731 \\
640	0.00212073508627908 \\
650	0.00189831732740536 \\
660	0.00168897224685566 \\
670	0.00149191912777275 \\
680	0.0013603685852604 \\
690	0.00126419615453838 \\
700	0.00118078887728162 \\
710	0.00110269661281565 \\
720	0.00102951713399574 \\
730	0.000960906244845627 \\
740	0.000896551562783232 \\
750	0.000836167182567793 \\
760	0.000779490323171428 \\
770	0.000726278656555279 \\
780	0.00067630810747854 \\
790	0.000629371013065982 \\
800	0.000585274562320492 \\
810	0.000543839456283723 \\
820	0.00050489874438292 \\
830	0.000468296803364354 \\
840	0.000433888433198415 \\
850	0.000401538050234074 \\
860	0.000371118962259553 \\
870	0.000345069997834457 \\
880	0.000323132317298414 \\
890	0.000302627694577673 \\
900	0.000283451745664631 \\
910	0.000265510349358511 \\
920	0.000248717820764022 \\
930	0.000232995743352338 \\
940	0.000218272052982138 \\
950	0.000204480297691323 \\
960	0.000191559030343402 \\
970	0.000179451303149503 \\
980	0.000168104241096423 \\
990	0.000157468677151218 \\
1000	0.000147498836407078 \\
1010	0.000138152059497576 \\
1020	0.000129388557939147 \\
1030	0.000121171195794356 \\
1040	0.000113465293328063 \\
1050	0.000106238449292362 \\
1060	9.94603791851789e-05 \\
1070	9.31027673738982e-05 \\
1080	8.7139131378644e-05 \\
1090	8.15446969246181e-05 \\
1100	7.62962826094116e-05 \\
1110	7.13721932176781e-05 \\
1120	6.67521208619348e-05 \\
1130	6.2417053241226e-05 \\
1140	5.83491884025844e-05 \\
1150	5.45318554640573e-05 \\
1160	5.09494408236222e-05 \\
1170	4.75873194201726e-05 \\
1180	4.44317906674319e-05 \\
1190	4.14700187093553e-05 \\
1200	3.86899766816606e-05 \\
1210	3.6080394691218e-05 \\
1220	3.36307112471235e-05 \\
1230	3.13310279027035e-05 \\
1240	2.91720668838535e-05 \\
1250	2.71451314968751e-05 \\
1260	2.52420691256305e-05 \\
1270	2.34552366428775e-05 \\
1280	2.17774680700278e-05 \\
1290	2.02020443350048e-05 \\
1300	1.87226649881467e-05 \\
1310	1.73334217439258e-05 \\
1320	1.60287737278053e-05 \\
1330	1.48035243157651e-05 \\
1340	1.36527994591962e-05 \\
1350	1.25720274004038e-05 \\
1360	1.15569196839083e-05 \\
1370	1.07076029184361e-05 \\
1380	9.99345134872209e-06 \\
1390	9.32826453253766e-06 \\
1400	8.70833755850509e-06 \\
1410	8.13031745378057e-06 \\
1420	7.59114784260051e-06 \\
1430	7.08803261867397e-06 \\
1440	6.61840640880929e-06 \\
1450	6.17990986240269e-06 \\
1460	5.77036871629577e-06 \\
1470	5.38777586428507e-06 \\
1480	5.03027584269899e-06 \\
1490	4.69615128273482e-06 \\
1500	4.38381098033513e-06 \\
1510	4.09177931198768e-06 \\
1520	3.81868678223096e-06 \\
1530	3.56326153549968e-06 \\
1540	3.3243216949197e-06 \\
1550	3.10321699154015e-06 \\
1560	2.90571622280078e-06 \\
1570	2.72121266564618e-06 \\
1580	2.54879175715095e-06 \\
1590	2.3876230570985e-06 \\
1600	2.2369387088772e-06 \\
1610	2.09602722534008e-06 \\
1620	1.96422874559898e-06 \\
1630	1.84093083283043e-06 \\
1640	1.72556468897023e-06 \\
1650	1.61760173322811e-06 \\
1660	1.5165504962944e-06 \\
1670	1.42195379249088e-06 \\
1680	1.33338613972356e-06 \\
1690	1.25045139842683e-06 \\
1700	1.17278060829396e-06 \\
1710	1.10003000231007e-06 \\
1720	1.03187917976921e-06 \\
1730	9.68029425785222e-07 \\
1740	9.08202161253868e-07 \\
1750	8.5213751249702e-07 \\
1760	7.99592991762577e-07 \\
1770	7.50342275979143e-07 \\
1780	7.0417407888046e-07 \\
1790	6.60891105230821e-07 \\
1800	6.20309084153892e-07 \\
1810	5.82255871794946e-07 \\
1820	5.46570620596487e-07 \\
1830	5.13103007804272e-07 \\
1840	4.81712519873057e-07 \\
1850	4.52267788497718e-07 \\
1860	4.24645972385562e-07 \\
1870	3.98732185435957e-07 \\
1880	3.74418961890388e-07 \\
1890	3.51605762283214e-07 \\
1900	3.30198510867241e-07 \\
1910	3.10109168233375e-07 \\
1920	2.91255330520102e-07 \\
1930	2.73559858210426e-07 \\
1940	2.56950529964328e-07 \\
1950	2.41359719710399e-07 \\
1960	2.26724096497044e-07 \\
1970	2.12984343495037e-07 \\
1980	2.00084897261643e-07 \\
1990	1.87973704324218e-07 \\
2000	1.76601992918357e-07 \\
2010	1.65924062156542e-07 \\
2020	1.55897083020662e-07 \\
2030	1.46480914453573e-07 \\
2040	1.37637930108792e-07 \\
2050	1.29332858034292e-07 \\
2060	1.21532629571153e-07 \\
2070	1.1420623929892e-07 \\
2080	1.07324613252135e-07 \\
2090	1.00860486629273e-07 \\
2100	9.47882885515838e-08 \\
2110	8.90840353706679e-08 \\
2120	8.37252299712432e-08 \\
2130	7.86907679573012e-08 \\
2140	7.39608503330658e-08 \\
2150	6.95169007358665e-08 \\
2160	6.53414893858617e-08 \\
2170	6.14182606994973e-08 \\
2180	5.77318661210136e-08 \\
2190	5.42679013393332e-08 \\
2200	5.10128475017524e-08 \\
2210	4.79540153697222e-08 \\
2220	4.50794940820565e-08 \\
2230	4.23781029712522e-08 \\
2240	3.98393458778123e-08 \\
2250	3.74533686842149e-08 \\
2260	3.52109197909733e-08 \\
2270	3.31033125910984e-08 \\
2280	3.11223907756286e-08 \\
2290	2.92604951379616e-08 \\
2300	2.75104334312992e-08 \\
2310	2.58654516138712e-08 \\
2320	2.43192063709152e-08 \\
2330	2.28657400236365e-08 \\
2340	2.14994572700355e-08 \\
2350	2.02151024808472e-08 \\
2360	1.90077390493926e-08 \\
2370	1.78727298516534e-08 \\
2380	1.68057187610593e-08 \\
2390	1.58026135510525e-08 \\
2400	1.4859569963388e-08 \\
2410	1.39729762760332e-08 \\
2420	1.31394391478246e-08 \\
2430	1.2355770628858e-08 \\
2440	1.16189752819018e-08 \\
2450	1.09262383585218e-08 \\
2460	1.02749153629844e-08 \\
2470	9.66252106104903e-09 \\
2480	9.08672054267257e-09 \\
2490	8.54531906346878e-09 \\
2500	8.03625455070289e-09 \\
2510	7.55758899906311e-09 \\
2520	7.1075011431887e-09 \\
2530	6.68427929673143e-09 \\
2540	6.28631446897288e-09 \\
2550	5.91209492473155e-09 \\
2560	5.56019957853593e-09 \\
2570	5.22929261004279e-09 \\
2580	4.91811857905589e-09 \\
2590	4.62549737401119e-09 \\
2600	4.35031982659595e-09 \\
2610	4.09154327085659e-09 \\
2620	3.84818760190697e-09 \\
2630	3.6193312791255e-09 \\
2640	3.40410849508643e-09 \\
2650	3.20170479017889e-09 \\
2660	3.01135488767201e-09 \\
2670	2.83233919651238e-09 \\
2680	2.66398131332224e-09 \\
2690	2.50564519133079e-09 \\
2700	2.35673291992811e-09 \\
2710	2.21668233768568e-09 \\
2720	2.08496503395494e-09 \\
2730	1.96108407291007e-09 \\
2740	1.8445723837246e-09 \\
2750	1.73499042910308e-09 \\
2760	1.63192548363611e-09 \\
2770	1.53498913579853e-09 \\
2780	1.44381606670407e-09 \\
2790	1.35806327294929e-09 \\
2800	1.27740773514518e-09 \\
2810	1.20154597382793e-09 \\
2820	1.13019260616909e-09 \\
2830	1.06307912473014e-09 \\
2840	9.99953231328732e-10 \\
2850	9.40577560282208e-10 \\
2860	8.84729012273766e-10 \\
2870	8.32197644129451e-10 \\
2880	7.82786169217786e-10 \\
2890	7.36308958249055e-10 \\
2900	6.92591484163785e-10 \\
2910	6.51469655998937e-10 \\
2920	6.12789097242938e-10 \\
2930	5.76404812768772e-10 \\
2940	5.42180134122106e-10 \\
2950	5.09986719521294e-10 \\
2960	4.79703776701257e-10 \\
2970	4.51217951891181e-10 \\
2980	4.24422275102643e-10 \\
2990	3.99216493196519e-10 \\
3000	3.7550590414881e-10 \\
3010	3.53202023184451e-10 \\
3020	3.32221083976236e-10 \\
3030	3.12484482734021e-10 \\
3040	2.93918389626668e-10 \\
3050	2.76453360203988e-10 \\
3060	2.60023946818677e-10 \\
3070	2.44568809648626e-10 \\
3080	2.30029884029648e-10 \\
3090	2.16352769033534e-10 \\
3100	2.03486616445758e-10 \\
3110	1.91383131564749e-10 \\
3120	1.79996961779949e-10 \\
3130	1.69285641060668e-10 \\
3140	1.59209145866868e-10 \\
3150	1.49729728615711e-10 \\
3160	1.40812084215014e-10 \\
3170	1.32422905974039e-10 \\
3180	1.24530608047735e-10 \\
3190	1.17105936059403e-10 \\
3200	1.10121189944579e-10 \\
3210	1.0355011292873e-10 \\
3220	9.73682801053144e-11 \\
3230	9.15526543465717e-11 \\
3240	8.60814197700677e-11 \\
3250	8.09341482721493e-11 \\
3260	7.60918550390954e-11 \\
3270	7.15361658798486e-11 \\
3280	6.72503719378881e-11 \\
3290	6.32182639570544e-11 \\
3300	5.94248539265152e-11 \\
3310	5.58560420138576e-11 \\
3320	5.24985055427862e-11 \\
3330	4.9339754504274e-11 \\
3340	4.63679650231086e-11 \\
3350	4.3572090380195e-11 \\
3360	4.09416944791019e-11 \\
3370	3.846689633491e-11 \\
3380	3.61387031411198e-11 \\
3390	3.39481220912319e-11 \\
3400	3.18873261129227e-11 \\
3410	2.99484326227173e-11 \\
3420	2.81242251709557e-11 \\
3430	2.64079869083389e-11 \\
3440	2.47932785413241e-11 \\
3450	2.3274049354427e-11 \\
3460	2.18448037436758e-11 \\
3470	2.05000461050986e-11 \\
3480	1.92347804350845e-11 \\
3490	1.80443993080814e-11 \\
3500	1.69244618319908e-11 \\
3510	1.58708046704703e-11 \\
3520	1.487937550948e-11 \\
3530	1.394656612419e-11 \\
3540	1.30689348232238e-11 \\
3550	1.22432064486588e-11 \\
3560	1.14664389094798e-11 \\
3570	1.07354125589154e-11 \\
3580	1.00476849063114e-11 \\
3590	9.40064692755982e-12 \\
3600	8.79180062085538e-12 \\
3610	8.21898105130003e-12 \\
3620	7.68007879514698e-12 \\
3630	7.17298442864944e-12 \\
3640	6.69581057266555e-12 \\
3650	6.24689189265837e-12 \\
3660	5.82461856524219e-12 \\
3670	5.42721423357762e-12 \\
3680	5.0531800965814e-12 \\
3690	4.70140593122892e-12 \\
3700	4.37028191413447e-12 \\
3710	4.05891986687834e-12 \\
3720	3.7658209883773e-12 \\
3730	3.49009710021164e-12 \\
3740	3.23063797935674e-12 \\
3750	2.9865554473929e-12 \\
3760	2.75679479244673e-12 \\
3770	2.54068988070344e-12 \\
3780	2.33735253374334e-12 \\
3790	2.1460055954492e-12 \\
3800	1.965927420855e-12 \\
3810	1.79661840959966e-12 \\
3820	1.63719038326349e-12 \\
3830	1.48725476378786e-12 \\
3840	1.346145417358e-12 \\
3850	1.2133072324616e-12 \\
3860	1.08840714219127e-12 \\
3870	9.70834523883468e-13 \\
3880	8.60256310630803e-13 \\
3890	7.56172902072194e-13 \\
3900	6.58195720149024e-13 \\
3910	5.66047209105136e-13 \\
3920	4.7928327973068e-13 \\
3930	3.97737398571962e-13 \\
3940	3.20909965267901e-13 \\
3950	2.48745468667266e-13 \\
3960	1.80799819560207e-13 \\
3970	1.16739951039335e-13 \\
3980	5.6621374255883e-14 \\
3990	0 \\
};
\end{axis}

\end{tikzpicture}
}
\scalebox{.85}{% This file was created by matplotlib2tikz v0.6.18.
\begin{tikzpicture}



\begin{axis}[
legend cell align={left},
legend columns=1,
legend entries={{\pgd},{20\% \algo},{1 \adaalgo},{10\% \adaalgo},{20\% \adaalgo},{50\% \adaalgo}},
legend style={at={(0.8,0.99)}, anchor=north},
tick align=outside,
tick pos=left,
xlabel={Number of Subspaces explored},
xmajorgrids,
xmin=0, xmax=400000,
ylabel={Suboptimality},
ymajorgrids,
ymin=1e-11, ymax=16.995627444391,
ymode=log
]


\addlegendimage{ black,thick,mark=square*,mark repeat = 100}
\addlegendimage{ blue,mark=*,mark repeat = 100}
\addlegendimage{ red }
\addlegendimage{ red,mark=text, text mark={\tiny 10\%} }
\addlegendimage{ dashed, red,mark=text, text mark={\tiny 20\%}}
\addlegendimage{ dotted, red,mark=text, text mark={\tiny 50\%}}


\addplot [black,thick,mark=square*,mark repeat = 100]
table [row sep=\\]{%
1190	2.09092732507745 \\
2380	1.76093138142936 \\
3570	1.49402105644764 \\
4760	1.27429766033831 \\
5950	1.07856295304427 \\
7140	0.914077018577608 \\
8330	0.774128430961237 \\
9520	0.65315917276041 \\
10710	0.55235151898703 \\
11900	0.469421417542497 \\
13090	0.399565969711648 \\
14280	0.340384500529901 \\
15470	0.290966481605671 \\
16660	0.254452910495519 \\
17850	0.222074511935787 \\
19040	0.193369326713135 \\
20230	0.167736405728764 \\
21420	0.146233887110576 \\
22610	0.128078178237277 \\
23800	0.112352735781869 \\
24990	0.0987311110185153 \\
26180	0.0861110194152439 \\
27370	0.0754459207526227 \\
28560	0.0663337992006726 \\
29750	0.0580982016580068 \\
30940	0.0510247428381324 \\
32130	0.0444561890005148 \\
33320	0.0385642106046931 \\
34510	0.033797989263032 \\
35700	0.0300703973658077 \\
36890	0.0265915778895859 \\
38080	0.0234003486001056 \\
39270	0.0206923602811656 \\
40460	0.0182493381457556 \\
41650	0.0162850166388389 \\
42840	0.0147866713964506 \\
44030	0.0134074833186372 \\
45220	0.0121298949505163 \\
46410	0.0109527458654176 \\
47600	0.00986179065062148 \\
48790	0.00885681195369759 \\
49980	0.00792776761794539 \\
51170	0.00706526404777613 \\
52360	0.00626399777149778 \\
53550	0.0055299011233288 \\
54740	0.00503109283811304 \\
55930	0.00456901073358795 \\
57120	0.00419672734801718 \\
58310	0.00387583098585076 \\
59500	0.00357820842365403 \\
60690	0.00330195147639856 \\
61880	0.00304943415105369 \\
63070	0.00281786267459677 \\
64260	0.00260267430953037 \\
65450	0.00240261791396074 \\
66640	0.0022165479809213 \\
67830	0.00204341439673084 \\
69020	0.00188225355133276 \\
70210	0.00173218039386447 \\
71400	0.00159238131081102 \\
72590	0.00146210772883071 \\
73780	0.00134067035835123 \\
74970	0.00122816796339165 \\
76160	0.00112351360716378 \\
77350	0.00103090717352805 \\
78540	0.000952149066066332 \\
79730	0.000879636117740334 \\
80920	0.000812624110271942 \\
82110	0.000750822973817122 \\
83300	0.00069384229944186 \\
84490	0.000641123757514628 \\
85680	0.000592328671151332 \\
86870	0.00054714768444708 \\
88060	0.000505297750742151 \\
89250	0.000466519636208085 \\
90440	0.000430575709381154 \\
91630	0.000397247969035674 \\
92820	0.000366336278039447 \\
94010	0.000337656776815409 \\
95200	0.000311040454418599 \\
96390	0.000286331858683375 \\
97580	0.000266852779093441 \\
98770	0.000248879010914915 \\
99960	0.000232131269458591 \\
101150	0.00021651677476503 \\
102340	0.000201955730633474 \\
103530	0.000188374642196454 \\
104720	0.000175705455451147 \\
105910	0.000163885093026339 \\
107100	0.000152855054270262 \\
108290	0.000142561054439827 \\
109480	0.000132952697078192 \\
110670	0.000123983175581432 \\
111860	0.000115609000652717 \\
113050	0.000107789750863452 \\
114240	0.000100487843939401 \\
115430	9.36683267224736e-05 \\
116620	8.72986820222854e-05 \\
117810	8.13486507913463e-05 \\
119000	7.57900682429313e-05 \\
120190	7.05967126820584e-05 \\
121380	6.5744165954118e-05 \\
122570	6.12096845253318e-05 \\
123760	5.69720803100804e-05 \\
124950	5.30116104432432e-05 \\
126140	4.9309875272241e-05 \\
127330	4.58497239077538e-05 \\
128520	4.26151667354824e-05 \\
129710	3.95912943382259e-05 \\
130900	3.67642023293424e-05 \\
132090	3.41209216419003e-05 \\
133280	3.16493538493612e-05 \\
134470	2.93382111394291e-05 \\
135660	2.71769605842409e-05 \\
136850	2.51557723846552e-05 \\
138040	2.3265471789824e-05 \\
139230	2.14974944197088e-05 \\
140420	1.98438447366889e-05 \\
141610	1.8297057434391e-05 \\
142800	1.68501615301908e-05 \\
143990	1.5496646963209e-05 \\
145180	1.43135342753342e-05 \\
146370	1.33502637134075e-05 \\
147560	1.24542127949434e-05 \\
148750	1.16405815556164e-05 \\
149940	1.08828423731611e-05 \\
151130	1.01767175476608e-05 \\
152320	9.5183681926847e-06 \\
153510	8.90429915556545e-06 \\
154700	8.3313150362474e-06 \\
155890	7.79648585291781e-06 \\
157080	7.2971183924242e-06 \\
158270	6.83073194385209e-06 \\
159460	6.39503751986847e-06 \\
160650	5.987919891981e-06 \\
161840	5.60742191885177e-06 \\
163030	5.25173075938135e-06 \\
164220	4.91916566075501e-06 \\
165410	4.60816706471245e-06 \\
166600	4.31728684019417e-06 \\
167790	4.04517947860672e-06 \\
168980	3.79059412553007e-06 \\
170170	3.5523673423965e-06 \\
171360	3.32941650965646e-06 \\
172550	3.12073379982181e-06 \\
173740	2.92538065799208e-06 \\
174930	2.74248273757216e-06 \\
176120	2.57122524438547e-06 \\
177310	2.41084865226782e-06 \\
178500	2.26064475361554e-06 \\
179690	2.11995301580004e-06 \\
180880	1.98815721508261e-06 \\
182070	1.86468232737935e-06 \\
183260	1.74899164900877e-06 \\
184450	1.64058413415491e-06 \\
185640	1.53899192695262e-06 \\
186830	1.44377807492768e-06 \\
188020	1.35453440808231e-06 \\
189210	1.27087957219052e-06 \\
190400	1.1924572025368e-06 \\
191590	1.11893422871656e-06 \\
192780	1.04999929934069e-06 \\
193970	9.85361318706079e-07 \\
195160	9.24748086716942e-07 \\
196350	8.67905032730931e-07 \\
197540	8.14594039777461e-07 \\
198730	7.64592348934112e-07 \\
199920	7.17691540030874e-07 \\
201110	6.73696582853545e-07 \\
202300	6.32424953184163e-07 \\
203490	5.93705809348588e-07 \\
204680	5.57379225385457e-07 \\
205870	5.23295475729491e-07 \\
207060	4.91314368411544e-07 \\
208250	4.61304624166381e-07 \\
209440	4.33143295675009e-07 \\
210630	4.0671522749669e-07 \\
211820	3.81912550917463e-07 \\
213010	3.58634213104558e-07 \\
214200	3.36785537402573e-07 \\
215390	3.16277812939525e-07 \\
216580	2.97027910933778e-07 \\
217770	2.78957926980183e-07 \\
218960	2.61994846262326e-07 \\
220150	2.4607023141332e-07 \\
221340	2.31119929805512e-07 \\
222530	2.17083801379303e-07 \\
223720	2.03905462903275e-07 \\
224910	1.91532049498289e-07 \\
226100	1.79913991871228e-07 \\
227290	1.69004807260009e-07 \\
228480	1.58760904478417e-07 \\
229670	1.49141401228903e-07 \\
230860	1.401079539054e-07 \\
232050	1.31624596999558e-07 \\
233240	1.23657593886772e-07 \\
234430	1.16175297049104e-07 \\
235620	1.0914801656936e-07 \\
236810	1.02547897840033e-07 \\
238000	9.63488065441886e-08 \\
239190	9.05262215189495e-08 \\
240380	8.5057133336619e-08 \\
241570	7.99199506018589e-08 \\
242760	7.50944113558916e-08 \\
243950	7.05615000873294e-08 \\
245140	6.630337084923e-08 \\
246330	6.23032742619323e-08 \\
247520	5.85454890678072e-08 \\
248710	5.50152589595676e-08 \\
249900	5.16987325727136e-08 \\
251090	4.85829074192701e-08 \\
252280	4.56555774852596e-08 \\
253470	4.29052838812893e-08 \\
254660	4.03212689348287e-08 \\
255850	3.78934322253777e-08 \\
257040	3.56122915601276e-08 \\
258230	3.34689431169544e-08 \\
259420	3.14550270830161e-08 \\
260610	2.95626933488613e-08 \\
261800	2.7784570921785e-08 \\
262990	2.61137382273624e-08 \\
264180	2.45436949652955e-08 \\
265370	2.30683371849061e-08 \\
266560	2.1681932083073e-08 \\
267750	2.03790960773276e-08 \\
268940	1.91547727679264e-08 \\
270130	1.80042133979263e-08 \\
271320	1.69229575353036e-08 \\
272510	1.59068160310305e-08 \\
273700	1.49518537551074e-08 \\
274890	1.4054374608552e-08 \\
276080	1.32109068684549e-08 \\
277270	1.24181894767261e-08 \\
278460	1.16731584398622e-08 \\
279650	1.09729363928501e-08 \\
280840	1.0314819554047e-08 \\
282030	9.69626867686202e-09 \\
283220	9.11489761445949e-09 \\
284410	8.56846477104156e-09 \\
285600	8.05486449761972e-09 \\
286790	7.57211826574178e-09 \\
287980	7.11836734001992e-09 \\
289170	6.69186517310294e-09 \\
290360	6.29097113291621e-09 \\
291550	5.91414317518968e-09 \\
292740	5.55993306949887e-09 \\
293930	5.22697929383753e-09 \\
295120	4.91400337088166e-09 \\
296310	4.61980337318479e-09 \\
297500	4.34325003739744e-09 \\
298690	4.08328221235266e-09 \\
299880	3.83890291777433e-09 \\
301070	3.60917512542969e-09 \\
302260	3.39321837294904e-09 \\
303450	3.19020504457868e-09 \\
304640	2.99935798420137e-09 \\
305830	2.81994616546655e-09 \\
307020	2.65128280441118e-09 \\
308210	2.49272263941336e-09 \\
309400	2.34365932216818e-09 \\
310590	2.20352253110789e-09 \\
311780	2.07177669464542e-09 \\
312970	1.94791849317255e-09 \\
314160	1.83147458310273e-09 \\
315350	1.7220002090923e-09 \\
316540	1.61907748319479e-09 \\
317730	1.52231327543717e-09 \\
318920	1.43133865870837e-09 \\
320110	1.34580646626858e-09 \\
321300	1.26539062561548e-09 \\
322490	1.18978404906045e-09 \\
323680	1.11869918884011e-09 \\
324870	1.05186453991379e-09 \\
326060	9.89026083253464e-10 \\
327250	9.2994395517465e-10 \\
328440	8.74393502048321e-10 \\
329630	8.22163115365981e-10 \\
330820	7.73054120717376e-10 \\
332010	7.26879501034006e-10 \\
333200	6.83463896589132e-10 \\
334390	6.42641828640933e-10 \\
335580	6.04258365566324e-10 \\
336770	5.68167513037565e-10 \\
337960	5.34232158511116e-10 \\
339150	5.02323127538062e-10 \\
340340	4.72319572342172e-10 \\
341530	4.44107250974213e-10 \\
342720	4.17579137934609e-10 \\
343910	3.92634647017331e-10 \\
345100	3.69179076198378e-10 \\
346290	3.47123274568872e-10 \\
347480	3.2638364233506e-10 \\
348670	3.06881631217948e-10 \\
349860	2.88543078319492e-10 \\
351050	2.71298594700653e-10 \\
352240	2.55082843736432e-10 \\
353430	2.39834208048961e-10 \\
354620	2.25495233596718e-10 \\
355810	2.12011297406889e-10 \\
357000	1.99331440242645e-10 \\
358190	1.87407644958171e-10 \\
359380	1.76194780987515e-10 \\
360570	1.65650271277684e-10 \\
361760	1.55734425355547e-10 \\
362950	1.46409606660569e-10 \\
364140	1.37640510100567e-10 \\
365330	1.29394051029408e-10 \\
366520	1.2163908769125e-10 \\
367710	1.14346088153638e-10 \\
368900	1.07487685419017e-10 \\
370090	1.01037900268608e-10 \\
371280	9.49722522847196e-11 \\
372470	8.92679818953468e-11 \\
373660	8.39034952626605e-11 \\
374850	7.88584753053101e-11 \\
376040	7.41139927207257e-11 \\
377230	6.96518953624548e-11 \\
378420	6.54554188628254e-11 \\
379610	6.15090200994928e-11 \\
380800	5.77974890170196e-11 \\
381990	5.43068923164469e-11 \\
383180	5.10240738549328e-11 \\
384370	4.79365991346015e-11 \\
385560	4.50329773471481e-11 \\
386750	4.23021617734776e-11 \\
387940	3.9733827339461e-11 \\
389130	3.73183706159352e-11 \\
390320	3.5046576751796e-11 \\
391510	3.29099525409049e-11 \\
392700	3.09003933551821e-11 \\
393890	2.90104607003627e-11 \\
395080	2.72329381267866e-11 \\
396270	2.55610532740036e-11 \\
397460	2.39886999153782e-11 \\
398650	2.25098828465775e-11 \\
399840	2.11188844190247e-11 \\
401030	1.98106531179576e-11 \\
402220	1.85802484509168e-11 \\
403410	1.74230074811987e-11 \\
404600	1.6334489316705e-11 \\
405790	1.53108081768494e-11 \\
406980	1.43478007252895e-11 \\
408170	1.34421918041028e-11 \\
409360	1.25904286996104e-11 \\
410550	1.17891807427384e-11 \\
411740	1.10355613536228e-11 \\
412930	1.03268504858534e-11 \\
414120	9.66010604841472e-12 \\
415310	9.03305208410643e-12 \\
416500	8.44324610227432e-12 \\
417690	7.88846765686912e-12 \\
418880	7.36660732414407e-12 \\
420070	6.87583323610852e-12 \\
421260	6.41420250246938e-12 \\
422450	5.97988325523602e-12 \\
423640	5.57148771562765e-12 \\
424830	5.18723952680489e-12 \\
426020	4.8258619322894e-12 \\
427210	4.48602266445164e-12 \\
428400	4.16627843335959e-12 \\
429590	3.86557452713987e-12 \\
430780	3.58268970046538e-12 \\
431970	3.31668026376519e-12 \\
433160	3.06626946056099e-12 \\
434350	2.83090217934046e-12 \\
435540	2.60941268592774e-12 \\
436730	2.40113484650806e-12 \\
437920	2.20512497151049e-12 \\
439110	2.02077243827148e-12 \\
440300	1.84746662412749e-12 \\
441490	1.68426383950759e-12 \\
442680	1.53094203980686e-12 \\
443870	1.38661304660559e-12 \\
445060	1.25088828184516e-12 \\
446250	1.1232681451645e-12 \\
447440	1.00308650274883e-12 \\
448630	8.90121309993219e-13 \\
449820	7.83761944234129e-13 \\
451010	6.83841872017865e-13 \\
452200	5.89750470680883e-13 \\
453390	5.01265695618258e-13 \\
454580	4.18109991073834e-13 \\
455770	3.39672734384067e-13 \\
456960	2.6612045900265e-13 \\
458150	1.96898053417272e-13 \\
459340	1.31672450720544e-13 \\
460530	7.03881397612349e-14 \\
461720	1.2712053631958e-14 \\
462910	-4.15778522722121e-14 \\
464100	-9.25926002537381e-14 \\
465290	-1.40609746068776e-13 \\
466480	-1.85740312019789e-13 \\
467670	-2.28261853862932e-13 \\
468860	-2.68174371598207e-13 \\
470050	-3.05755420981768e-13 \\
471240	-3.41227046618542e-13 \\
472430	-3.74478226206065e-13 \\
473620	-4.05619982046801e-13 \\
474810	-4.35207425653061e-13 \\
476000	-4.62907490117459e-13 \\
};
\addplot [blue,mark=*,mark repeat = 100]
table [row sep=\\]{%
260	1.16215662228905 \\
520	1.00428253393589 \\
780	0.924389652996051 \\
1040	0.860199404522113 \\
1300	0.800381218608045 \\
1560	0.747097812544684 \\
1820	0.693510584381706 \\
2080	0.646359866592243 \\
2340	0.600737195010842 \\
2600	0.559001397962172 \\
2860	0.51993229908195 \\
3120	0.485620978416149 \\
3380	0.453562233217362 \\
3640	0.42599179628601 \\
3900	0.396944458209628 \\
4160	0.368309438763449 \\
4420	0.34491084956204 \\
4680	0.317901121247184 \\
4940	0.29551213786032 \\
5200	0.277131493553391 \\
5460	0.260498104023332 \\
5720	0.243551825840378 \\
5980	0.231380232961997 \\
6240	0.221210585650787 \\
6500	0.211667746267815 \\
6760	0.199001816769377 \\
7020	0.187218135878817 \\
7280	0.176715603427283 \\
7540	0.170920754097678 \\
7800	0.161573024638786 \\
8060	0.154770592454409 \\
8320	0.146647721748763 \\
8580	0.141605849268692 \\
8840	0.131116450898988 \\
9100	0.125328192409247 \\
9360	0.119186202739831 \\
9620	0.111403821298566 \\
9880	0.10611424909179 \\
10140	0.0999312054381534 \\
10400	0.0934940737762713 \\
10660	0.0885493679744258 \\
10920	0.0844833299114791 \\
11180	0.0811402828286314 \\
11440	0.0761720558518353 \\
11700	0.0739668603469543 \\
11960	0.0706049071342434 \\
12220	0.0672204284114135 \\
12480	0.0654500729371201 \\
12740	0.0636174946047901 \\
13000	0.0613114970178779 \\
13260	0.058641862311199 \\
13520	0.0565934021702331 \\
13780	0.0547931511993578 \\
14040	0.0516920762702148 \\
14300	0.0488482668530374 \\
14560	0.0467268475947321 \\
14820	0.0449428218572198 \\
15080	0.0434094302102165 \\
15340	0.0418513058296786 \\
15600	0.0399201814561859 \\
15860	0.0388408778365645 \\
16120	0.0377582161025896 \\
16380	0.0370847751277119 \\
16640	0.0360844445489145 \\
16900	0.034335641650601 \\
17160	0.0333971414046264 \\
17420	0.0329114219648496 \\
17680	0.0314629339342957 \\
17940	0.0301265566701184 \\
18200	0.0285098357145853 \\
18460	0.0278251656729712 \\
18720	0.0269755099369679 \\
18980	0.0262002661684173 \\
19240	0.025274356049916 \\
19500	0.0248622658046704 \\
19760	0.0238834648495216 \\
20020	0.0229110857505238 \\
20280	0.0222300524656775 \\
20540	0.0217506912240895 \\
20800	0.021400544373575 \\
21060	0.0203476208837226 \\
21320	0.0198112418011496 \\
21580	0.019057752937924 \\
21840	0.0183962131718027 \\
22100	0.0181532477059995 \\
22360	0.0178571614803595 \\
22620	0.0175202893595809 \\
22880	0.017161133223173 \\
23140	0.0167479599724896 \\
23400	0.0163154155663399 \\
23660	0.0158278040833156 \\
23920	0.0154641876862646 \\
24180	0.0152476770299525 \\
24440	0.014922429713923 \\
24700	0.0145379366744003 \\
24960	0.0142402484851094 \\
25220	0.013851879833516 \\
25480	0.0137008276387374 \\
25740	0.0135275227034023 \\
26000	0.013334318676045 \\
26260	0.0131044010811041 \\
26520	0.0125866175005585 \\
26780	0.012389540291766 \\
27040	0.0121423995140789 \\
27300	0.0119642443471491 \\
27560	0.0118482468435631 \\
27820	0.0116682762291604 \\
28080	0.0114737302106622 \\
28340	0.0112412001241596 \\
28600	0.0108963883836689 \\
28860	0.010698965938397 \\
29120	0.0103139534281668 \\
29380	0.0102214876843962 \\
29640	0.00982371495798162 \\
29900	0.00963867484875658 \\
30160	0.00949084409617262 \\
30420	0.00934418064755699 \\
30680	0.00914347410903416 \\
30940	0.00900928858910233 \\
31200	0.00882555225427389 \\
31460	0.00867139579835963 \\
31720	0.00848106459279535 \\
31980	0.00828811539502333 \\
32240	0.00805720927961762 \\
32500	0.00783154675799141 \\
32760	0.00777322793964003 \\
33020	0.00763163278926321 \\
33280	0.00746587538587223 \\
33540	0.00736864626665668 \\
33800	0.00730149509900868 \\
34060	0.00723894801561736 \\
34320	0.00717955809365828 \\
34580	0.00704102045310429 \\
34840	0.00677441470456369 \\
35100	0.00663129107355959 \\
35360	0.00653279157773279 \\
35620	0.00641188465395048 \\
35880	0.00634591733693829 \\
36140	0.00629007322459796 \\
36400	0.00619520712759919 \\
36660	0.00603085569746908 \\
36920	0.00598576934014899 \\
37180	0.00591953195381961 \\
37440	0.0058731508915561 \\
37700	0.00582855693115197 \\
37960	0.00575640017019674 \\
38220	0.00570060761909558 \\
38480	0.00562606782110697 \\
38740	0.0055725456332561 \\
39000	0.0054961888750602 \\
39260	0.00538391216681994 \\
39520	0.00532695994786314 \\
39780	0.00525993614918024 \\
40040	0.00504345760681946 \\
40300	0.00496237234109254 \\
40560	0.00493079011724074 \\
40820	0.00488401392739735 \\
41080	0.00483016635683076 \\
41340	0.00475273588394992 \\
41600	0.00469613012117359 \\
41860	0.00446811688047405 \\
42120	0.0044081748642128 \\
42380	0.00436151466360385 \\
42640	0.00427362753363175 \\
42900	0.00418812518500072 \\
43160	0.0041710647466523 \\
43420	0.00406381533772798 \\
43680	0.00399454724112897 \\
43940	0.0036892708223753 \\
44200	0.00361159740839928 \\
44460	0.00356805688472395 \\
44720	0.00349502518513406 \\
44980	0.00346740888139552 \\
45240	0.00339796547400584 \\
45500	0.00314615435397808 \\
45760	0.00306565730232328 \\
46020	0.00302821500385225 \\
46280	0.00299232156952473 \\
46540	0.00296838627791823 \\
46800	0.00286448977758069 \\
47060	0.00283187599353624 \\
47320	0.00276267746574782 \\
47580	0.00273137844298132 \\
47840	0.00268448438148661 \\
48100	0.0026429013257579 \\
48360	0.00259478414278547 \\
48620	0.00254563660796664 \\
48880	0.00248902074703011 \\
49140	0.00246668073962575 \\
49400	0.00241109540981199 \\
49660	0.00238267694394562 \\
49920	0.00229974863230015 \\
50180	0.00227576953329522 \\
50440	0.00228269619363408 \\
50700	0.00215529418568206 \\
50960	0.00213542925135329 \\
51220	0.00210878943431309 \\
51480	0.00201553181613146 \\
51740	0.00199781731729382 \\
52000	0.00195484400904289 \\
52260	0.00195295844677901 \\
52520	0.00188364044033179 \\
52780	0.00186196013039786 \\
53040	0.0018605249255606 \\
53300	0.00185078402264105 \\
53560	0.00181493499954033 \\
53820	0.00179629524202257 \\
54080	0.0017773628984713 \\
54340	0.0017609554556885 \\
54600	0.00175028326840637 \\
54860	0.00173763448871778 \\
55120	0.00170312455079941 \\
55380	0.00166693457273975 \\
55640	0.00160528026754564 \\
55900	0.0015968845081244 \\
56160	0.00158057164024988 \\
56420	0.00157224590030686 \\
56680	0.00156684438386112 \\
56940	0.00155539570706231 \\
57200	0.00154461977409093 \\
57460	0.00152954871454225 \\
57720	0.00152246097956632 \\
57980	0.00151429744827791 \\
58240	0.00150703176829237 \\
58500	0.0014889205360949 \\
58760	0.00148218242368059 \\
59020	0.00147183253330097 \\
59280	0.0014646741938063 \\
59540	0.00145957975659794 \\
59800	0.00145470894010064 \\
60060	0.00145017679377413 \\
60320	0.00144625603077414 \\
60580	0.0014107402458387 \\
60840	0.00138498613197924 \\
61100	0.00137925263354816 \\
61360	0.00137056312393274 \\
61620	0.00136577535485599 \\
61880	0.00135759747632075 \\
62140	0.00133039791114925 \\
62400	0.00132617146759106 \\
62660	0.00131733825876079 \\
62920	0.00131502998852012 \\
63180	0.00130777061252035 \\
63440	0.00129425047308557 \\
63700	0.00126772898422989 \\
63960	0.00126185377146676 \\
64220	0.00123342811488392 \\
64480	0.00122877911551339 \\
64740	0.00122263608504558 \\
65000	0.00121941040677948 \\
65260	0.00120613679448039 \\
65520	0.00118508039442489 \\
65780	0.00117469558004019 \\
66040	0.00117156854867118 \\
66300	0.00116842288343566 \\
66560	0.00116352489785565 \\
66820	0.00116004680623871 \\
67080	0.00113764065226091 \\
67340	0.00113646148135604 \\
67600	0.00112951265405115 \\
67860	0.00112581096944281 \\
68120	0.00112261632597521 \\
68380	0.00111837485655869 \\
68640	0.00111610353253777 \\
68900	0.00111353863820685 \\
69160	0.00110864146347217 \\
69420	0.00108300896561031 \\
69680	0.00107833686930298 \\
69940	0.00107101833744799 \\
70200	0.00106782750667217 \\
70460	0.00106513959982002 \\
70720	0.00106039447893819 \\
70980	0.00105890162127903 \\
71240	0.00105537474758072 \\
71500	0.00104634389881236 \\
71760	0.00104472557504515 \\
72020	0.00104124539459222 \\
72280	0.00103384389351757 \\
72540	0.00101082504395106 \\
72800	0.000989599400613272 \\
73060	0.000972216442321483 \\
73320	0.000968224320722189 \\
73580	0.000945243719447486 \\
73840	0.000927533905458688 \\
74100	0.000923988335640835 \\
74360	0.000921762283861205 \\
74620	0.00092112391562138 \\
74880	0.000918021890966403 \\
75140	0.000914545896808672 \\
75400	0.000913108849242283 \\
75660	0.000911293036714556 \\
75920	0.000909118040607726 \\
76180	0.000906437942235216 \\
76440	0.000902164363641489 \\
76700	0.000900048965439382 \\
76960	0.000896037050956366 \\
77220	0.00088704805963119 \\
77480	0.000866367919412359 \\
77740	0.000849585558542254 \\
78000	0.000847956006823858 \\
78260	0.000841462729487485 \\
78520	0.000837615129262281 \\
78780	0.000836504718731812 \\
79040	0.00083511954000437 \\
79300	0.000832817967951816 \\
79560	0.000831558572321756 \\
79820	0.000829910339526896 \\
80080	0.000827778066375795 \\
80340	0.000797092519662868 \\
80600	0.000791692175272307 \\
80860	0.000789072425096238 \\
81120	0.000785812280727993 \\
81380	0.000784464083113556 \\
81640	0.000782726476757312 \\
81900	0.000778083637455851 \\
82160	0.000776449174364868 \\
82420	0.000775497329342012 \\
82680	0.00073257605400967 \\
82940	0.000731611370084595 \\
83200	0.000703467617833631 \\
83460	0.000701436779111997 \\
83720	0.000699611873882178 \\
83980	0.00069827389137711 \\
84240	0.000698080815824065 \\
84500	0.000696559946478714 \\
84760	0.00069526184783425 \\
85020	0.00069364858993487 \\
85280	0.000681012006822468 \\
85540	0.000680750816102194 \\
85800	0.000678898056171773 \\
86060	0.000678272799104929 \\
86320	0.000676636650544127 \\
86580	0.000663635122529593 \\
86840	0.000662677345211693 \\
87100	0.000657047395649479 \\
87360	0.000655364519563828 \\
87620	0.000643378422833518 \\
87880	0.000641092298198032 \\
88140	0.000640065502158338 \\
88400	0.000637503611045864 \\
88660	0.000637473802264255 \\
88920	0.00063701982237313 \\
89180	0.000636773534601842 \\
89440	0.000635555147275813 \\
89700	0.000634891228435186 \\
89960	0.000634749624349895 \\
90220	0.000634062493641285 \\
90480	0.000633094517670518 \\
90740	0.00063100058283122 \\
91000	0.000630617862227068 \\
91260	0.000628740461308064 \\
91520	0.000628139760186497 \\
91780	0.000624957613428012 \\
92040	0.000611874845327542 \\
92300	0.000600142579023077 \\
92560	0.00059894958396256 \\
92820	0.000598908067941384 \\
93080	0.000598479269072139 \\
93340	0.000597993031749566 \\
93600	0.000597323061260646 \\
93860	0.000596687066369117 \\
94120	0.000592564668551565 \\
94380	0.000591884007059451 \\
94640	0.000591670609523032 \\
94900	0.000591190683365939 \\
95160	0.000590807957814854 \\
95420	0.000590081028905343 \\
95680	0.000589321076052418 \\
95940	0.000588662090566405 \\
96200	0.000587721200579039 \\
96460	0.000586750884875087 \\
96720	0.00058669354852986 \\
96980	0.000586203154680387 \\
97240	0.000583892994445723 \\
97500	0.000580517694229565 \\
97760	0.000569532784199966 \\
98020	0.000567803814439827 \\
98280	0.000567372825352186 \\
98540	0.000566717850514153 \\
98800	0.000566380805736455 \\
99060	0.000565568003373429 \\
99320	0.000565353893952347 \\
99580	0.000565019321212779 \\
99840	0.000564878535428692 \\
100100	0.00056474992761224 \\
100360	0.000563563892088559 \\
100620	0.000562245110088122 \\
100880	0.000561684136323681 \\
101140	0.000560988900251858 \\
101400	0.000560885054746885 \\
101660	0.00056079521915231 \\
101920	0.000560447127247177 \\
102180	0.0005602523237484 \\
102440	0.000548778231183378 \\
102700	0.000548180275123211 \\
102960	0.000547501443533494 \\
103220	0.0005470827120998 \\
103480	0.000526274435948604 \\
103740	0.000525579609402971 \\
104000	0.000514258166529558 \\
};

\addplot [ red]
table [row sep=\\]{%
1178	3.1805559289351 \\
2348	2.64188544734182 \\
3518	2.20403380748648 \\
4688	1.83895313737967 \\
5858	1.53105960388386 \\
7028	1.26875598571308 \\
8198	1.04102645391337 \\
9368	0.849323436778866 \\
10538	0.700238545857181 \\
11708	0.581371393698809 \\
12878	0.485598141235844 \\
14048	0.404430636202285 \\
15218	0.335905118668508 \\
16388	0.279873896419759 \\
17558	0.236080190649626 \\
18728	0.201901474536513 \\
19898	0.173237573530455 \\
21068	0.149529068568246 \\
22238	0.130711750113766 \\
23408	0.115745971664834 \\
24578	0.102497717087355 \\
25748	0.0904611013135878 \\
26918	0.0797058872376823 \\
28088	0.0696914070735802 \\
29258	0.061618035629484 \\
30428	0.0545555259826275 \\
31598	0.0482543623611812 \\
32768	0.0425929974651959 \\
33938	0.0373447068844376 \\
35108	0.0328926105604252 \\
36278	0.0289525878662524 \\
37448	0.0255752324937894 \\
38618	0.0226145850624336 \\
39788	0.019862638207472 \\
40958	0.0173024324491598 \\
42128	0.0149187860868296 \\
43298	0.0127153581919099 \\
44468	0.0106765595470695 \\
45638	0.00899583386564207 \\
46808	0.007837431139539 \\
47978	0.0068769006407059 \\
49148	0.0061305818700898 \\
50318	0.00555344510084332 \\
51488	0.00501845754099123 \\
52658	0.00458797441700687 \\
53828	0.00420433950122068 \\
54998	0.00384905115820333 \\
56168	0.00351980520187389 \\
57338	0.00321451464804606 \\
58508	0.00293821479496115 \\
59678	0.00269874722889035 \\
60848	0.00247752842161642 \\
62018	0.00227293292428515 \\
63188	0.00208354039232767 \\
64358	0.00190807820611227 \\
65528	0.00174540068595502 \\
66698	0.00160825729599806 \\
67868	0.00149028336864609 \\
69038	0.00138374057220836 \\
70208	0.00128646035371505 \\
71378	0.00119732182453913 \\
72548	0.00111539992135951 \\
73718	0.00103992056549429 \\
74888	0.000970228235148807 \\
76058	0.000905761675709382 \\
77228	0.000846035505104481 \\
78398	0.000790626151358664 \\
79568	0.000739160996294952 \\
80738	0.00069138374871186 \\
81908	0.000647183430624976 \\
83078	0.000605949032019104 \\
84248	0.000567456698989 \\
85418	0.000531504476571742 \\
86588	0.000497909007740771 \\
87758	0.000466503030747745 \\
88928	0.000437133369854348 \\
90098	0.000409659294894615 \\
91268	0.000383951163567431 \\
92438	0.000359889282846859 \\
93608	0.000337362942023023 \\
94778	0.000316269581719986 \\
95948	0.000296514071955634 \\
97118	0.000278008078753333 \\
98288	0.000260669503592381 \\
99458	0.000244421983543686 \\
100628	0.000229194442593339 \\
101798	0.000214920686656905 \\
102968	0.000201539036294107 \\
104138	0.000188991992283005 \\
105308	0.000177225930089897 \\
106478	0.000166190819954337 \\
107648	0.000155839969831983 \\
108818	0.000146129788863247 \\
109988	0.000137019569361185 \\
111158	0.000128471285582743 \\
112328	0.000120449407766843 \\
113498	0.000112920730099775 \\
114668	0.000105854211420675 \\
115838	9.92208276072715e-05 \\
117008	9.29934346871608e-05 \\
118178	8.71466418173039e-05 \\
119348	8.16566933532492e-05 \\
120518	7.65013592995412e-05 \\
121688	7.16598334993268e-05 \\
122858	6.71126389746868e-05 \\
124028	6.28415398769566e-05 \\
125198	5.88294595577077e-05 \\
126368	5.50604043042524e-05 \\
127538	5.15193923243951e-05 \\
128708	4.81923875978452e-05 \\
129878	4.50662382415201e-05 \\
131048	4.21286190612213e-05 \\
132218	3.93679779924772e-05 \\
133388	3.67734861523306e-05 \\
134558	3.43349912447222e-05 \\
135728	3.20429740845452e-05 \\
136898	2.9888508020659e-05 \\
138068	2.78632210543028e-05 \\
139238	2.59592604680003e-05 \\
140408	2.41692597876542e-05 \\
141578	2.24863079189008e-05 \\
142748	2.09488757275667e-05 \\
143918	1.96647697905949e-05 \\
145088	1.84613917469933e-05 \\
146258	1.733317392838e-05 \\
147428	1.62750525265487e-05 \\
148598	1.52823750521613e-05 \\
149768	1.43508512726598e-05 \\
150938	1.34765141805993e-05 \\
152108	1.26556873504824e-05 \\
153278	1.18849572336988e-05 \\
154448	1.11611493725516e-05 \\
155618	1.04813077591581e-05 \\
156788	9.84267674597472e-06 \\
157958	9.24268505370085e-06 \\
159128	8.67893151940136e-06 \\
160298	8.16130177172525e-06 \\
161468	7.67512496874234e-06 \\
162638	7.21839115702672e-06 \\
163808	6.78925897396043e-06 \\
164978	6.38601170183106e-06 \\
166148	6.00704600783875e-06 \\
167318	5.65086332265929e-06 \\
168488	5.3160621504289e-06 \\
169658	5.00133109937462e-06 \\
170828	4.70544253033944e-06 \\
171998	4.42724674787343e-06 \\
173168	4.16566666860962e-06 \\
174338	3.91969291574368e-06 \\
175508	3.68837929259902e-06 \\
176678	3.47083859897346e-06 \\
177848	3.26623875801513e-06 \\
179018	3.07379922254158e-06 \\
180188	2.89278763992984e-06 \\
181358	2.7225167505418e-06 \\
182528	2.56234150158852e-06 \\
183698	2.41165635950225e-06 \\
184868	2.26989280560641e-06 \\
186038	2.13651699865203e-06 \\
187208	2.01102759461724e-06 \\
188378	1.89295371161302e-06 \\
189548	1.78185302673883e-06 \\
190718	1.67730999889315e-06 \\
191888	1.57893420749122e-06 \\
193058	1.48635879682057e-06 \\
194228	1.39923902170436e-06 \\
195398	1.31725088708956e-06 \\
196568	1.24008987228974e-06 \\
197738	1.16746973799486e-06 \\
198908	1.09912140822122e-06 \\
200078	1.03479192226086e-06 \\
201248	9.74243453688572e-07 \\
202418	9.17252389098788e-07 \\
203588	8.63608465906385e-07 \\
204758	8.13113963271661e-07 \\
205928	7.6558294281881e-07 \\
207098	7.20840536871936e-07 \\
208268	6.78722280711419e-07 \\
209438	6.39073485519948e-07 \\
210608	6.01748650741474e-07 \\
211778	5.66610910801568e-07 \\
212948	5.33531517910024e-07 \\
214118	5.02389354839483e-07 \\
215288	4.73070478179682e-07 \\
216458	4.45467688736656e-07 \\
217628	4.19480129076888e-07 \\
218798	3.95012904219616e-07 \\
219968	3.71976726754042e-07 \\
221138	3.50287581496467e-07 \\
222308	3.29866412851487e-07 \\
223478	3.10638828826537e-07 \\
224648	2.92534824086754e-07 \\
225818	2.75488519274614e-07 \\
226988	2.59437916205751e-07 \\
228158	2.44324667497686e-07 \\
229328	2.30093860020819e-07 \\
230498	2.16693811672108e-07 \\
231668	2.04075880139154e-07 \\
232838	1.92194282599978e-07 \\
234008	1.8100592724668e-07 \\
235178	1.70470253524346e-07 \\
236348	1.6054908330565e-07 \\
237518	1.51206479559463e-07 \\
238688	1.42408614733913e-07 \\
239858	1.34123645800788e-07 \\
241028	1.26321597238022e-07 \\
242198	1.18974251728243e-07 \\
243368	1.12055045686787e-07 \\
244538	1.05538972339225e-07 \\
245708	9.94024899059198e-08 \\
246878	9.3623435892809e-08 \\
248048	8.81809450459237e-08 \\
249218	8.30553739117335e-08 \\
250388	7.82282287836722e-08 \\
251558	7.36820980895558e-08 \\
252728	6.94005886647808e-08 \\
253898	6.5368266632948e-08 \\
255068	6.15705998963101e-08 \\
256238	5.79939066769342e-08 \\
257408	5.4625304501954e-08 \\
258578	5.14526639072699e-08 \\
259748	4.84645639731163e-08 \\
260918	4.56502510792767e-08 \\
262088	4.29995997142107e-08 \\
263258	4.05030755046276e-08 \\
264428	3.81517007985721e-08 \\
265598	3.59370221914013e-08 \\
266768	3.38510793285174e-08 \\
267938	3.18863765946809e-08 \\
269108	3.00358556359903e-08 \\
270278	2.82928698802642e-08 \\
271133	2.6651160334179e-08 \\
271253	2.5104833212275e-08 \\
271373	2.36483380655628e-08 \\
271493	2.22764483526205e-08 \\
271613	2.09842419551798e-08 \\
271733	1.97670838586461e-08 \\
271853	1.86206089436425e-08 \\
271973	1.7540706553909e-08 \\
272093	1.65235055082924e-08 \\
272213	1.5565359945402e-08 \\
272333	1.46628365560453e-08 \\
272453	1.38127019266854e-08 \\
272573	1.3011910826588e-08 \\
272693	1.22575955496806e-08 \\
272813	1.15470550343666e-08 \\
272933	1.08777456486742e-08 \\
273053	1.02472716423385e-08 \\
273173	9.65337682012901e-09 \\
273293	9.09393632619881e-09 \\
273413	8.56694859496798e-09 \\
273533	8.07052880080761e-09 \\
273653	7.60290175261247e-09 \\
273773	7.16239523246287e-09 \\
273893	6.74743472206529e-09 \\
274013	6.35653679692538e-09 \\
274133	5.98830446341125e-09 \\
274253	5.64142205172757e-09 \\
274373	5.31465038644541e-09 \\
274493	5.00682184600976e-09 \\
274613	4.71683730962624e-09 \\
274733	4.4436612722798e-09 \\
274853	4.18631823650983e-09 \\
274973	3.94388960378578e-09 \\
275093	3.71551017730454e-09 \\
275213	3.50036466478798e-09 \\
275333	3.29768556905918e-09 \\
275453	3.10674969083991e-09 \\
275573	2.92687596381569e-09 \\
275693	2.75742284561176e-09 \\
275813	2.5977859308135e-09 \\
275933	2.4473960080762e-09 \\
276053	2.30571711723471e-09 \\
276173	2.17224388476822e-09 \\
276293	2.04650074664414e-09 \\
276413	1.92803945031628e-09 \\
276533	1.81643822205757e-09 \\
276653	1.71129904691369e-09 \\
276773	1.61224755768075e-09 \\
276893	1.51893109201495e-09 \\
277013	1.43101691607583e-09 \\
277133	1.3481922245262e-09 \\
277253	1.27016219764187e-09 \\
277373	1.19664878006631e-09 \\
277493	1.12739051427724e-09 \\
277613	1.06214076422972e-09 \\
277733	1.00066743780047e-09 \\
277853	9.42751487986726e-10 \\
277973	8.8818719046202e-10 \\
278093	8.36780367219347e-10 \\
278213	7.88347942481948e-10 \\
278333	7.42717942703308e-10 \\
278453	6.99727886743773e-10 \\
278573	6.59224952404003e-10 \\
278693	6.21065143757704e-10 \\
278813	5.8511290257357e-10 \\
278933	5.51240386670315e-10 \\
279053	5.19326914805163e-10 \\
279173	4.89259410763054e-10 \\
279293	4.60930904555568e-10 \\
279413	4.34240698954369e-10 \\
279533	4.09094258468912e-10 \\
279653	3.85401710545352e-10 \\
279773	3.63079288856483e-10 \\
279893	3.42047556944891e-10 \\
280013	3.22231963334474e-10 \\
280133	3.03562064374319e-10 \\
280253	2.85971468727553e-10 \\
280373	2.69397948393646e-10 \\
280493	2.53782439507688e-10 \\
280613	2.39069763985356e-10 \\
280733	2.25207352766432e-10 \\
280853	2.12146356037834e-10 \\
280973	1.99840199943679e-10 \\
281093	1.88245530274855e-10 \\
281213	1.77320713667939e-10 \\
281333	1.67027391917429e-10 \\
281453	1.57328816641211e-10 \\
281573	1.48190848481278e-10 \\
281693	1.39580791369553e-10 \\
281813	1.31468225195164e-10 \\
281933	1.23824561715224e-10 \\
282053	1.16622433932179e-10 \\
282173	1.09836417738762e-10 \\
282293	1.03442532317644e-10 \\
282413	9.74180180968176e-11 \\
282533	9.17413367496067e-11 \\
282653	8.6392837328475e-11 \\
282773	8.13531464416428e-11 \\
282893	7.66046115430186e-11 \\
283013	7.21303017314767e-11 \\
283133	6.79144518400676e-11 \\
283253	6.39420738579588e-11 \\
283373	6.019917897504e-11 \\
283493	5.66723334927133e-11 \\
283613	5.33492139354053e-11 \\
283733	5.02180519390549e-11 \\
283853	4.72675787399623e-11 \\
283973	4.44876357974522e-11 \\
284093	4.18681200820004e-11 \\
284213	3.93998722536537e-11 \\
284333	3.70741215505177e-11 \\
284453	3.48824857887564e-11 \\
284573	3.28176930075585e-11 \\
284693	3.08718606234493e-11 \\
284813	2.90384938317345e-11 \\
284933	2.73109868054178e-11 \\
285053	2.56831778067124e-11 \\
285173	2.41491826535878e-11 \\
285293	2.27038943201308e-11 \\
285413	2.13420392469743e-11 \\
285533	2.00586769416589e-11 \\
285653	1.88494220232371e-11 \\
285773	1.77099446219131e-11 \\
285893	1.66363034459494e-11 \\
286013	1.56245572036084e-11 \\
286133	1.46712086923628e-11 \\
286253	1.37728162208361e-11 \\
286373	1.29263266757107e-11 \\
286493	1.21286869436688e-11 \\
286613	1.13771214671488e-11 \\
286733	1.06688546885891e-11 \\
286853	1.00015551396382e-11 \\
286973	9.37261379618803e-12 \\
287093	8.78003225679436e-12 \\
287213	8.22170109771037e-12 \\
287333	7.69539987288681e-12 \\
287453	7.19962978124045e-12 \\
287573	6.73233691017572e-12 \\
287693	6.29213348091184e-12 \\
287813	5.87729864776065e-12 \\
287933	5.4863336096389e-12 \\
288053	5.11785058776582e-12 \\
288173	4.77073935911676e-12 \\
288293	4.44366765606219e-12 \\
288413	4.13530321097255e-12 \\
288533	3.84497989003307e-12 \\
288653	3.57125440331174e-12 \\
288773	3.31323857238885e-12 \\
288893	3.07015524114718e-12 \\
289013	2.84111623116701e-12 \\
289133	2.62517785287741e-12 \\
289253	2.42195152821978e-12 \\
289373	2.23027152301825e-12 \\
289493	2.0497492592142e-12 \\
289613	1.87949655838793e-12 \\
289733	1.71912484248082e-12 \\
289853	1.56802348882934e-12 \\
289973	1.4255263636187e-12 \\
290093	1.29135591109275e-12 \\
290213	1.16484599743671e-12 \\
290333	1.0456635557432e-12 \\
290453	9.33364496802369e-13 \\
290573	8.27560242555592e-13 \\
290693	7.27806703793021e-13 \\
290813	6.33881835909733e-13 \\
290933	5.45230527393414e-13 \\
291053	4.61797267092834e-13 \\
291173	3.83193476949373e-13 \\
291293	3.09141601206875e-13 \\
};

\addplot [red,mark=text, text mark={ \tiny 10\% } ,mark repeat=100]
table [row sep=\\]{%
1190	2.66979517505131 \\
2380	2.02283246511696 \\
3561	1.61519757858965 \\
4741	1.31350915603608 \\
5921	1.07897287260949 \\
7101	0.883935256530944 \\
8281	0.719488877515996 \\
9461	0.577098505113612 \\
10641	0.466935653936808 \\
11821	0.377248864864245 \\
13001	0.306300623661983 \\
14181	0.247148369586576 \\
15361	0.196354066009793 \\
16541	0.156787861646511 \\
17721	0.122306339570701 \\
18901	0.095489335510445 \\
20081	0.0740258540993126 \\
21261	0.0588814039115829 \\
22441	0.0462712305051751 \\
23621	0.0399077581063943 \\
24801	0.0349365713224245 \\
25981	0.0308859295852145 \\
27161	0.0274599848737583 \\
28341	0.0243073896497898 \\
29521	0.0218919616995246 \\
30701	0.0199484698040907 \\
31881	0.0181615462615447 \\
33061	0.0164952169761741 \\
34241	0.0149411120182689 \\
35421	0.0134903214429895 \\
36601	0.0121350739799063 \\
37781	0.0108684368453455 \\
38961	0.00968408782283064 \\
40141	0.00857692509648228 \\
41321	0.00755569140138851 \\
42501	0.0066825953167795 \\
43681	0.00594125789358579 \\
44861	0.00524793321442096 \\
46041	0.00459932382176198 \\
47221	0.00405051805009421 \\
48401	0.00365320876598763 \\
49581	0.00329362121654514 \\
50761	0.00296392393800965 \\
51941	0.0026632042382857 \\
53121	0.00248568842282731 \\
54301	0.00232155467628625 \\
55481	0.00216824936016174 \\
56661	0.00202485432386906 \\
57841	0.00189059035525924 \\
59021	0.00176478526453067 \\
60201	0.00164683509822672 \\
61381	0.00153670305982329 \\
62561	0.0014335738150012 \\
63741	0.0013366632620605 \\
64921	0.00124557261431935 \\
66101	0.00115993442161894 \\
67281	0.00107940862489669 \\
68461	0.00100367947309105 \\
69641	0.000932453020382751 \\
70821	0.000865455056457132 \\
72001	0.000802429365666824 \\
73181	0.000743136239182041 \\
74361	0.000688175715491957 \\
75541	0.000638298144586102 \\
76721	0.000592532042987981 \\
77901	0.000549594380536977 \\
79081	0.000509302026585445 \\
80261	0.000471484884447426 \\
81441	0.000435984777269915 \\
82621	0.000402654469492836 \\
83801	0.000374157199716341 \\
84981	0.000348710943471608 \\
86161	0.000325037157071495 \\
87341	0.000303361084268317 \\
88521	0.000283146747680929 \\
89701	0.00026429279407264 \\
90881	0.000246781815234465 \\
92061	0.000230452626546307 \\
93241	0.000215221745222482 \\
94421	0.000201012461757799 \\
95601	0.000187753998534324 \\
96781	0.000175380915224899 \\
97961	0.000163832608759351 \\
99141	0.00015305288228401 \\
100321	0.000142989569157204 \\
101501	0.000133594201671361 \\
102681	0.000124821716599444 \\
103861	0.000116630191458522 \\
105041	0.000108980606715547 \\
106221	0.000101836630173768 \\
107401	9.51644205409163e-05 \\
108581	8.89324477622022e-05 \\
109761	8.31113281444806e-05 \\
110941	7.76736726424421e-05 \\
112121	7.25939469440884e-05 \\
113301	6.78483421974185e-05 \\
114481	6.34146553920023e-05 \\
115661	5.92721785348527e-05 \\
116841	5.5401595869975e-05 \\
117457	5.17848884789007e-05 \\
117697	4.84052456689588e-05 \\
117937	4.52469826221491e-05 \\
118177	4.22954638216733e-05 \\
118417	3.95370318261312e-05 \\
118657	3.69589409934257e-05 \\
118897	3.45492957929405e-05 \\
119137	3.22969933747586e-05 \\
119377	3.01916700901628e-05 \\
119617	2.82236516824241e-05 \\
119857	2.63839068887495e-05 \\
120097	2.46640042131929e-05 \\
120337	2.3056071647154e-05 \\
120577	2.15527591353459e-05 \\
120817	2.0147203591947e-05 \\
121057	1.88329962937939e-05 \\
121297	1.76041524838078e-05 \\
121537	1.64550830344945e-05 \\
121777	1.53805680268571e-05 \\
122017	1.43757321162119e-05 \\
122257	1.34360215600648e-05 \\
122497	1.2557182794859e-05 \\
122737	1.17352424566786e-05 \\
122977	1.09664887458782e-05 \\
123217	1.02474540447095e-05 \\
123457	9.57489870301309e-06 \\
123697	8.94579591148581e-06 \\
123937	8.35731758924707e-06 \\
124177	7.80682121725995e-06 \\
124417	7.29183755260321e-06 \\
124657	6.8100591651965e-06 \\
124897	6.35932973896969e-06 \\
125137	5.93763408895942e-06 \\
125377	5.54308884376242e-06 \\
125617	5.17393374893649e-06 \\
125857	4.82852354999341e-06 \\
126097	4.50532041540441e-06 \\
126337	4.20288686381332e-06 \\
126577	3.91987916115166e-06 \\
126817	3.668725892636e-06 \\
127057	3.43609684849699e-06 \\
127297	3.21846903500456e-06 \\
127537	3.01484062997792e-06 \\
127777	2.82428436942928e-06 \\
128017	2.64593827126181e-06 \\
128257	2.47900028738091e-06 \\
128497	2.32272365985331e-06 \\
128737	2.1764127249857e-06 \\
128977	2.03941909893146e-06 \\
129217	1.91113819725253e-06 \\
129457	1.79100605368632e-06 \\
129697	1.67849640164697e-06 \\
129937	1.57311799625637e-06 \\
130177	1.47441214709554e-06 \\
130417	1.38195044596667e-06 \\
130657	1.29533266884918e-06 \\
130897	1.21418483634006e-06 \\
131137	1.13815742003309e-06 \\
131377	1.06692368023742e-06 \\
131617	1.00017812626474e-06 \\
131857	9.37635086295519e-07 \\
132097	8.79027380551456e-07 \\
132337	8.24105088559346e-07 \\
132577	7.72634402290695e-07 \\
132817	7.24396560403129e-07 \\
133057	6.7918685531243e-07 \\
133297	6.36813709709028e-07 \\
133537	5.97097815413505e-07 \\
133777	5.59871331184958e-07 \\
134017	5.24977135707427e-07 \\
134257	4.92268129259621e-07 \\
134497	4.61606583901375e-07 \\
134737	4.32863536403705e-07 \\
134977	4.05918221202395e-07 \\
135217	3.8065754243144e-07 \\
135457	3.56975579207663e-07 \\
135697	3.34773124444077e-07 \\
135937	3.13957253916808e-07 \\
136177	2.94440923642725e-07 \\
136417	2.76142593291784e-07 \\
136657	2.58985874579398e-07 \\
136897	2.42899201585711e-07 \\
137137	2.27815524389552e-07 \\
137377	2.13672020410449e-07 \\
137617	2.00409825956704e-07 \\
137857	1.87973784260276e-07 \\
138097	1.76312209998475e-07 \\
138337	1.65376669414297e-07 \\
138577	1.55121773925959e-07 \\
138817	1.45504987114631e-07 \\
139057	1.36486444257677e-07 \\
139297	1.28028783574763e-07 \\
139537	1.20096987854534e-07 \\
139777	1.12658236128826e-07 \\
140017	1.0568176572745e-07 \\
140257	9.91387414939204e-08 \\
140497	9.30021352707477e-08 \\
140737	8.72466114354431e-08 \\
140977	8.18484199860414e-08 \\
141217	7.67852978422745e-08 \\
141457	7.20363743100805e-08 \\
141697	6.75820840956298e-08 \\
141937	6.34040847047324e-08 \\
142177	5.948518039256e-08 \\
142417	5.58092493885276e-08 \\
142657	5.23611771163779e-08 \\
142897	4.91267926339134e-08 \\
143137	4.60928094025981e-08 \\
143377	4.32467702760064e-08 \\
143617	4.05769950417856e-08 \\
143857	3.80725320714426e-08 \\
144097	3.57231124126223e-08 \\
144337	3.35191073230767e-08 \\
144577	3.14514879695693e-08 \\
144817	2.95117882909146e-08 \\
145057	2.76920694153304e-08 \\
145297	2.5984886686814e-08 \\
145537	2.43832589674753e-08 \\
145777	2.2880639605205e-08 \\
146017	2.14708895662774e-08 \\
146257	2.0148251345109e-08 \\
146497	1.89073257050865e-08 \\
146737	1.77430489745056e-08 \\
146977	1.6650672673979e-08 \\
147217	1.56257430883322e-08 \\
147457	1.46640831144573e-08 \\
147697	1.37617756634789e-08 \\
147937	1.29151463967858e-08 \\
148177	1.21207494041542e-08 \\
148417	1.13753528263594e-08 \\
148657	1.06759251439215e-08 \\
148897	1.00196230201632e-08 \\
149137	9.4037794218238e-09 \\
149377	8.82589323847327e-09 \\
149617	8.2836177917045e-09 \\
149857	7.77475239743808e-09 \\
150097	7.29723254044856e-09 \\
150337	6.84912182524755e-09 \\
150577	6.428603649411e-09 \\
150817	6.03397426468533e-09 \\
151057	5.66363494991506e-09 \\
151297	5.31608634890546e-09 \\
151537	4.98992197561776e-09 \\
151777	4.68382227447606e-09 \\
152017	4.39654929129674e-09 \\
152257	4.12694189932949e-09 \\
152497	3.87391047018681e-09 \\
152737	3.63643304357453e-09 \\
152977	3.41355088639972e-09 \\
153217	3.20436427392323e-09 \\
153457	3.00802921460175e-09 \\
153697	2.82375400839641e-09 \\
153937	2.65079569405913e-09 \\
154177	2.4884574401085e-09 \\
154417	2.33608521416073e-09 \\
154657	2.19306545146125e-09 \\
154897	2.05882250137179e-09 \\
155137	1.93281618487973e-09 \\
155377	1.81453985170776e-09 \\
155617	1.7035177157787e-09 \\
155857	1.5993037449924e-09 \\
156097	1.50147944077972e-09 \\
156337	1.40965167316764e-09 \\
156577	1.3234519591343e-09 \\
156817	1.24253429767407e-09 \\
157057	1.16657428161915e-09 \\
157297	1.09526721026043e-09 \\
157537	1.02832731219138e-09 \\
157777	9.65486413040395e-10 \\
158017	9.06492658714342e-10 \\
158257	8.51110015798184e-10 \\
158497	7.99116717242754e-10 \\
158737	7.5030492929784e-10 \\
158977	7.04479308222261e-10 \\
159217	6.61456778239256e-10 \\
159457	6.21065254780007e-10 \\
159697	5.83143477950188e-10 \\
159937	5.47540068840391e-10 \\
160177	5.14112807881162e-10 \\
160417	4.82728357287243e-10 \\
160657	4.53261539412608e-10 \\
160897	4.25594892661252e-10 \\
161137	3.99618116375677e-10 \\
161377	3.75227682258839e-10 \\
161617	3.52326556818383e-10 \\
161857	3.30823479721687e-10 \\
162097	3.10632741751249e-10 \\
162337	2.91674295826994e-10 \\
162577	2.73872535760944e-10 \\
162817	2.57156906879885e-10 \\
163057	2.41460629268886e-10 \\
163297	2.2672164146087e-10 \\
163537	2.12881379191288e-10 \\
163777	1.99884719886967e-10 \\
164017	1.8768037124417e-10 \\
164257	1.76219649983267e-10 \\
164497	1.65457314516004e-10 \\
164737	1.55350565744783e-10 \\
164977	1.45859491151867e-10 \\
165217	1.36946287643269e-10 \\
165457	1.2857587217141e-10 \\
165697	1.20714993556703e-10 \\
165937	1.13332621065609e-10 \\
166177	1.06399555832581e-10 \\
166417	9.98882643266086e-11 \\
166657	9.37730448846708e-11 \\
166897	8.80298611782848e-11 \\
167137	8.26360091465972e-11 \\
167377	7.75700614852326e-11 \\
167617	7.2812089690899e-11 \\
167857	6.83433865056315e-11 \\
168097	6.41461883610361e-11 \\
168337	6.02041194674996e-11 \\
168577	5.65014701692235e-11 \\
168817	5.30236965445852e-11 \\
169057	4.97572538726843e-11 \\
169297	4.66890970329814e-11 \\
169537	4.38071801056594e-11 \\
169777	4.11003453493208e-11 \\
170017	3.85577125783243e-11 \\
170257	3.61694008077507e-11 \\
170497	3.39259731418906e-11 \\
170737	3.18187143300008e-11 \\
170977	2.98391866770942e-11 \\
171217	2.79796186219983e-11 \\
171457	2.62329047373555e-11 \\
171697	2.4592106129262e-11 \\
171937	2.30507279930237e-11 \\
172177	2.16027751243075e-11 \\
172417	2.02425298745368e-11 \\
172657	1.89647186843445e-11 \\
172897	1.77643455501197e-11 \\
173137	1.66366365128567e-11 \\
173377	1.55772061916082e-11 \\
173617	1.45820022723342e-11 \\
173857	1.36468614186924e-11 \\
174097	1.27683974504578e-11 \\
174337	1.19432241874051e-11 \\
174577	1.11678999381581e-11 \\
174817	1.04394271005503e-11 \\
175057	9.75514113932263e-12 \\
175297	9.11215547461097e-12 \\
175537	8.50802761576119e-12 \\
175777	7.94042609442158e-12 \\
176017	7.40713046454289e-12 \\
176257	6.90608681352956e-12 \\
176497	6.43535225108849e-12 \\
176737	5.99303939807783e-12 \\
176977	5.57748291996063e-12 \\
177217	5.18701748219996e-12 \\
177457	4.82014428371258e-12 \\
177697	4.47530901226401e-12 \\
177937	4.15140144482962e-12 \\
178177	3.8469782914774e-12 \\
178417	3.56098484033396e-12 \\
178657	3.29225535722344e-12 \\
178897	3.03973513027245e-12 \\
179137	2.80248046991005e-12 \\
179377	2.57949217541409e-12 \\
179617	2.37004860181855e-12 \\
179857	2.17303952609882e-12 \\
180097	1.98796534789381e-12 \\
180337	1.81421544453997e-12 \\
180577	1.65067959301268e-12 \\
180817	1.49724677100949e-12 \\
181057	1.35291777780822e-12 \\
181297	1.21724852419902e-12 \\
181537	1.08990594327452e-12 \\
181777	9.70112878917462e-13 \\
182017	8.57591775371702e-13 \\
182257	7.51843032276156e-13 \\
182497	6.52422560420973e-13 \\
182737	5.59108315201229e-13 \\
182977	4.71178651650916e-13 \\
183217	3.88744592072499e-13 \\
183457	3.11251024953663e-13 \\
183697	2.3830937223579e-13 \\
183937	1.69753100465186e-13 \\
184177	1.05415676188159e-13 \\
184417	4.49640324973188e-14 \\
184657	-1.19348975147204e-14 \\
184897	-6.55031584528842e-14 \\
185137	-1.15574216863479e-13 \\
185377	-1.6286971771251e-13 \\
185617	-2.07223127546285e-13 \\
185857	-2.4891200212096e-13 \\
186097	-2.88213897192691e-13 \\
186337	-3.24906768156552e-13 \\
186577	-3.59545726524857e-13 \\
186817	-3.92186283448837e-13 \\
187057	-4.22828438928491e-13 \\
187297	-4.51583215266282e-13 \\
187537	-4.78561634764674e-13 \\
187777	-5.04041253179821e-13 \\
188017	-5.27855537058031e-13 \\
188257	-5.5033755330669e-13 \\
188497	-5.71431790774568e-13 \\
188737	-5.9136029406659e-13 \\
188977	-6.09956529729061e-13 \\
189217	-6.2755356466937e-13 \\
189457	-6.43984865433822e-13 \\
};

\addplot [dashed, red,mark=text, text mark={\tiny 20\%}, mark repeat = 100]
table [row sep=\\]{%
1190	3.2362771223447 \\
2380	2.70519029990559 \\
3570	2.26831500697808 \\
4760	1.90671043268573 \\
5950	1.60508478698664 \\
7140	1.34327204885578 \\
8330	1.12431659412896 \\
9520	0.941323331969572 \\
10710	0.783151525877136 \\
11890	0.654614632794805 \\
13070	0.546751657653721 \\
14250	0.453659920355868 \\
15430	0.37051258651783 \\
16610	0.297421474570098 \\
17790	0.237644044082886 \\
18970	0.189563578821049 \\
20150	0.151511312722694 \\
21330	0.120747753275547 \\
22510	0.0982481020103385 \\
23690	0.0797970665935637 \\
24870	0.0653007072702782 \\
26050	0.0535167221282357 \\
27230	0.0451221647168469 \\
28410	0.0380130772957125 \\
29590	0.0321331876727168 \\
30770	0.0273526450165275 \\
31950	0.0230957896185156 \\
33130	0.0196379879239438 \\
34310	0.0168889939765222 \\
35490	0.0149346535449601 \\
36670	0.0132457947866588 \\
37850	0.0117173656845562 \\
39030	0.0103799462895621 \\
40210	0.009146247244814 \\
41390	0.00812093803872765 \\
42570	0.00730937313795516 \\
43750	0.00657945598451132 \\
44930	0.00590922427560442 \\
46110	0.00529317552225628 \\
47290	0.00472769280450064 \\
48470	0.00422230353550229 \\
49650	0.00375876833221744 \\
50830	0.0033311235593112 \\
52010	0.00293893110563503 \\
53190	0.00259297431409911 \\
54370	0.00228286431736674 \\
55550	0.00200880920107394 \\
56730	0.00175692170846919 \\
57910	0.00152611782627021 \\
59090	0.0013310780319768 \\
60270	0.0011681249096076 \\
61450	0.00101887934143219 \\
62630	0.000881366126202598 \\
63810	0.000754488114585838 \\
64990	0.000637328069973297 \\
66170	0.000529316522761558 \\
67350	0.000436630357601886 \\
68530	0.00039093461713996 \\
69710	0.000356121805988818 \\
70890	0.000325776745113748 \\
72070	0.000298249828152908 \\
73250	0.000273224413199713 \\
74430	0.000250429811908648 \\
75610	0.000229638784701724 \\
76790	0.000210660016016384 \\
77970	0.000193389045993864 \\
79150	0.000177688885701743 \\
80330	0.000163316890091469 \\
81510	0.000150154421987903 \\
82690	0.000138089343545644 \\
83870	0.0001270302366847 \\
85050	0.000116875044159637 \\
86230	0.000107552577184256 \\
87410	9.89930399872274e-05 \\
88590	9.11199494663117e-05 \\
89770	8.38771910784963e-05 \\
90950	7.72162183415581e-05 \\
92130	7.1081372907511e-05 \\
93310	6.54348036128183e-05 \\
94490	6.02315474116377e-05 \\
95670	5.54346564976549e-05 \\
96850	5.10104104530784e-05 \\
98030	4.69318564594867e-05 \\
99210	4.31695302008372e-05 \\
100390	3.96935098819773e-05 \\
101570	3.64839367385517e-05 \\
102750	3.35185227602341e-05 \\
103930	3.07776940203786e-05 \\
105110	2.82448084746312e-05 \\
106290	2.59978846126052e-05 \\
107470	2.41286882698422e-05 \\
108650	2.24009971971917e-05 \\
109830	2.08028808740823e-05 \\
111010	1.93240693324159e-05 \\
112190	1.79544990522862e-05 \\
113370	1.66855060904081e-05 \\
114550	1.55094522524823e-05 \\
115730	1.44187676790475e-05 \\
116910	1.34068382147134e-05 \\
118090	1.24676006407753e-05 \\
119270	1.15956475335022e-05 \\
120450	1.07858077686429e-05 \\
121630	1.00334330608853e-05 \\
122810	9.33415797083059e-06 \\
123990	8.68413542548518e-06 \\
125170	8.07966133381521e-06 \\
125783	7.51741324023225e-06 \\
126153	6.99432429490665e-06 \\
126523	6.5075476188281e-06 \\
126893	6.05445636753243e-06 \\
127263	5.63262521552366e-06 \\
127633	5.23981359473735e-06 \\
128003	4.87395049009187e-06 \\
128373	4.53312062176359e-06 \\
128743	4.21555186957967e-06 \\
129113	3.91960381290746e-06 \\
129483	3.64375727363031e-06 \\
129853	3.38660476290054e-06 \\
130223	3.14684174673729e-06 \\
130593	2.92325865053344e-06 \\
130963	2.72246588517433e-06 \\
131333	2.54300735941237e-06 \\
131703	2.37569151706474e-06 \\
132073	2.21964928209806e-06 \\
132443	2.07408765795503e-06 \\
132813	1.93827438216276e-06 \\
133183	1.8115321248624e-06 \\
133553	1.69323379550779e-06 \\
133923	1.58279836487418e-06 \\
134293	1.47968710612068e-06 \\
134663	1.38340020061634e-06 \\
135033	1.29347366156773e-06 \\
135403	1.20947654036474e-06 \\
135773	1.13100838233793e-06 \\
136143	1.0576969051157e-06 \\
136513	9.89195875822357e-07 \\
136883	9.2518316757717e-07 \\
137253	8.65358975532526e-07 \\
137623	8.09444180349672e-07 \\
137993	7.57178841515049e-07 \\
138363	7.08320810782759e-07 \\
138733	6.62644454751948e-07 \\
139103	6.19939475920983e-07 \\
139473	5.80009824280303e-07 \\
139843	5.42672692338542e-07 \\
140213	5.07757584922164e-07 \\
140583	4.75105458752623e-07 \\
140953	4.44567926194406e-07 \\
141323	4.16006517345302e-07 \\
141693	3.89291995472885e-07 \\
142063	3.64303723021653e-07 \\
142433	3.40929073139318e-07 \\
142803	3.1906288228134e-07 \\
143173	2.98606943949231e-07 \\
143543	2.79469535957588e-07 \\
143913	2.61564982728668e-07 \\
144283	2.44813247896047e-07 \\
144653	2.29139554830393e-07 \\
145023	2.14474034310186e-07 \\
145393	2.00751396006726e-07 \\
145763	1.87910623061782e-07 \\
146133	1.75894687592937e-07 \\
146503	1.64650285239265e-07 \\
146873	1.54127588469777e-07 \\
147243	1.44280016489695e-07 \\
147613	1.35064020134745e-07 \\
147983	1.26438882641633e-07 \\
148353	1.18366532575465e-07 \\
148723	1.10811369635755e-07 \\
149093	1.03740102674887e-07 \\
149463	9.71215983747165e-08 \\
149833	9.09267394155755e-08 \\
150203	8.51282931924047e-08 \\
150573	7.97007880248835e-08 \\
150943	7.46203984713922e-08 \\
151313	6.98648381369793e-08 \\
151683	6.54132591426659e-08 \\
152053	6.12461587556901e-08 \\
152423	5.73452916818873e-08 \\
152793	5.36935886308321e-08 \\
153163	5.0275079710449e-08 \\
153533	4.70748232062057e-08 \\
153903	4.40788392452873e-08 \\
154273	4.12740470134842e-08 \\
154643	3.86482068015503e-08 \\
155013	3.61898659373416e-08 \\
155383	3.38883071049345e-08 \\
155753	3.17335016597475e-08 \\
156123	2.97160644424643e-08 \\
156493	2.78272133669155e-08 \\
156863	2.60587288414271e-08 \\
157233	2.44029192963957e-08 \\
157603	2.28525858791961e-08 \\
157973	2.14009912569146e-08 \\
158343	2.00418298068605e-08 \\
158713	1.8769199972013e-08 \\
159083	1.75775781152687e-08 \\
159453	1.64617942055578e-08 \\
159823	1.54170090582717e-08 \\
160193	1.44386934075591e-08 \\
160563	1.35226075892447e-08 \\
160933	1.26647832221494e-08 \\
161303	1.18615056665661e-08 \\
161673	1.11092975374483e-08 \\
162043	1.04049042159993e-08 \\
162413	9.74527863961683e-09 \\
162783	9.12756870086184e-09 \\
163153	8.54910414682664e-09 \\
163523	8.00738558792702e-09 \\
163893	7.50007328464974e-09 \\
164263	7.02497643390032e-09 \\
164633	6.58004462028572e-09 \\
165003	6.16335749104024e-09 \\
165373	5.773118316732e-09 \\
165743	5.40764388823334e-09 \\
166113	5.06535918765039e-09 \\
166483	4.74478911716147e-09 \\
166853	4.44455278136857e-09 \\
167223	4.1633573255595e-09 \\
167593	3.89999238459282e-09 \\
167963	3.65332464280499e-09 \\
168333	3.42229283800677e-09 \\
168703	3.20590370916918e-09 \\
169073	3.00322672286413e-09 \\
169443	2.8133910201511e-09 \\
169813	2.63558103119621e-09 \\
170183	2.46903320011427e-09 \\
170553	2.31303243225511e-09 \\
170923	2.16690904109029e-09 \\
171293	2.03003647225586e-09 \\
171663	1.90182730674948e-09 \\
172033	1.78173215070743e-09 \\
172403	1.66923608269087e-09 \\
172773	1.56385732141828e-09 \\
173143	1.46514422816324e-09 \\
173513	1.37267419653142e-09 \\
173883	1.2860514875257e-09 \\
174253	1.20490550870045e-09 \\
174623	1.12888903780473e-09 \\
174993	1.05767711255922e-09 \\
175363	9.90965198788274e-10 \\
175733	9.28468357752621e-10 \\
176103	8.69919580814837e-10 \\
176473	8.15068734727475e-10 \\
176843	7.63681839988095e-10 \\
177213	7.15539572038182e-10 \\
177583	6.70436484107029e-10 \\
177953	6.28180285566771e-10 \\
178323	5.8859073170936e-10 \\
178693	5.51499235168507e-10 \\
179063	5.16747811207807e-10 \\
179433	4.84188245053474e-10 \\
179803	4.5368225842779e-10 \\
180173	4.25099733192269e-10 \\
180543	3.98319044414563e-10 \\
180913	3.73226505256952e-10 \\
181283	3.49715423286767e-10 \\
181653	3.27685767409491e-10 \\
182023	3.07044001335299e-10 \\
182393	2.87702417445246e-10 \\
182763	2.69578970257811e-10 \\
183133	2.52596721317389e-10 \\
183503	2.3668361714968e-10 \\
183873	2.21772156194788e-10 \\
184243	2.07799166762612e-10 \\
184613	1.94705529477091e-10 \\
184983	1.82435455631236e-10 \\
185353	1.70937264343252e-10 \\
185723	1.60162161311206e-10 \\
186093	1.50064627391089e-10 \\
186463	1.40601918996452e-10 \\
186833	1.31733901564957e-10 \\
187203	1.23423216091822e-10 \\
187573	1.15634779529472e-10 \\
187943	1.08335562742923e-10 \\
188313	1.01494923576695e-10 \\
188683	9.50837741875432e-11 \\
189053	8.90752471782719e-11 \\
189423	8.34438629304657e-11 \\
189793	7.81659181825489e-11 \\
190163	7.32192084740291e-11 \\
190533	6.85828616120432e-11 \\
190903	6.42373376713579e-11 \\
191273	6.0164373483218e-11 \\
191643	5.63468716130444e-11 \\
192013	5.27686228046775e-11 \\
192383	4.94147500695874e-11 \\
192753	4.62709870419076e-11 \\
193123	4.33242330899475e-11 \\
193493	4.05621647381338e-11 \\
193863	3.79730136224055e-11 \\
194233	3.55460660905749e-11 \\
194603	3.3270997068513e-11 \\
194973	3.11384251716618e-11 \\
195343	2.91393020823705e-11 \\
195713	2.72654121502569e-11 \\
196083	2.55085397249388e-11 \\
196453	2.38616348902099e-11 \\
196823	2.23177587521661e-11 \\
197193	2.0870527528416e-11 \\
197563	1.95135574365679e-11 \\
197933	1.82415194061036e-11 \\
198303	1.70489733442025e-11 \\
198673	1.59309232472538e-11 \\
199043	1.48827061785539e-11 \\
199413	1.39000477794582e-11 \\
199783	1.29785071578681e-11 \\
200153	1.21148646670122e-11 \\
200523	1.13049014593969e-11 \\
200893	1.05453978882508e-11 \\
201263	9.83346737370994e-12 \\
201633	9.16600129130529e-12 \\
202003	8.5400020388704e-12 \\
202373	7.95313814805354e-12 \\
202743	7.40280059474685e-12 \\
203113	6.88665791059861e-12 \\
203483	6.40282271646697e-12 \\
203853	5.94918558860513e-12 \\
204223	5.52363710326631e-12 \\
204593	5.12478948166972e-12 \\
204963	4.75064432237104e-12 \\
205333	4.39986935774073e-12 \\
205703	4.07091027554429e-12 \\
206073	3.76237929700096e-12 \\
206443	3.47311068793488e-12 \\
206813	3.20177218071649e-12 \\
207183	2.94725355232117e-12 \\
207553	2.70872213548046e-12 \\
207923	2.48495668486726e-12 \\
208293	2.27501351091064e-12 \\
208663	2.07822647979583e-12 \\
209033	1.8936519019519e-12 \\
209403	1.72051262126161e-12 \\
209773	1.55819801506141e-12 \\
210143	1.40587541608284e-12 \\
210513	1.26293420166235e-12 \\
210883	1.12898579374132e-12 \\
211253	1.00325303620252e-12 \\
211623	8.85458373289794e-13 \\
211993	7.74935671188359e-13 \\
212363	6.71185329537138e-13 \\
212733	5.73929792579975e-13 \\
213103	4.82780482258249e-13 \\
213473	3.9718228705965e-13 \\
213843	3.16802140076788e-13 \\
214213	2.41584530158434e-13 \\
214583	1.70863323489812e-13 \\
214953	1.04638520070921e-13 \\
215323	4.25215418431435e-14 \\
215693	-1.58761892521397e-14 \\
216063	-7.05546732149287e-14 \\
216433	-1.21902488103842e-13 \\
216803	-1.70030656221343e-13 \\
217173	-2.15216733323587e-13 \\
217543	-2.57571741713036e-13 \\
217913	-2.97317725994617e-13 \\
218283	-3.3451019731956e-13 \\
218653	-3.69593244897715e-13 \\
219023	-4.02566868729082e-13 \\
219393	-4.33320046511199e-13 \\
219763	-4.62241356302684e-13 \\
220133	-4.89275286952306e-13 \\
220503	-5.14699394216223e-13 \\
220873	-5.38513678094432e-13 \\
221243	-5.60995694343092e-13 \\
221613	-5.81978909508507e-13 \\
221983	-6.01796390498066e-13 \\
222353	-6.20226092706844e-13 \\
222723	-6.37545571890996e-13 \\
223093	-6.53976872655448e-13 \\
223463	-6.69186928092813e-13 \\
223833	-6.8361982741294e-13 \\
224203	-6.97053526010905e-13 \\
224573	-7.09710068491631e-13 \\
224943	-7.2158945485512e-13 \\
225313	-7.3258066279891e-13 \\
225683	-7.43127781532849e-13 \\
226053	-7.5289774414955e-13 \\
226423	-7.62057084102707e-13 \\
226793	-7.70772334846015e-13 \\
227163	-7.78821451774547e-13 \\
227533	-7.8642647949323e-13 \\
227903	-7.93531906850831e-13 \\
228273	-8.00248756149813e-13 \\
228643	-8.06521516238945e-13 \\
229013	-8.12405698269458e-13 \\
229383	-8.17956813392584e-13 \\
229753	-8.23119350457091e-13 \\
230123	-8.28059842916673e-13 \\
230493	-8.32611757317636e-13 \\
230863	-8.36941627113674e-13 \\
231233	-8.40993941153556e-13 \\
231603	-8.4471318828605e-13 \\
231973	-8.48265901964851e-13 \\
232343	-8.51596571038726e-13 \\
232713	-8.54816217810139e-13 \\
233083	-8.57702797674165e-13 \\
233453	-8.60478355235728e-13 \\
233823	-8.63031868192365e-13 \\
234193	-8.65529869997772e-13 \\
};

\addplot [dotted, red,mark=text, text mark={\tiny 50\%}, mark repeat = 100]
table [row sep=\\]{%
1190	3.05754623134375 \\
2380	2.51103589557646 \\
3570	2.06811526162925 \\
4760	1.72068865333226 \\
5950	1.43004031705105 \\
7140	1.18602667926735 \\
8330	0.982892184964891 \\
9520	0.814972806977224 \\
10710	0.683185263628121 \\
11900	0.577686798723891 \\
13090	0.488611558424467 \\
14280	0.412157843726483 \\
15470	0.349340260380677 \\
16660	0.298781496380663 \\
17850	0.257438629526222 \\
19035	0.222643827966988 \\
20215	0.192426715511325 \\
21395	0.167123885742266 \\
22575	0.146580911830579 \\
23755	0.128668248785467 \\
24935	0.112521576949724 \\
26115	0.098740889135461 \\
27295	0.0867392612379725 \\
28475	0.0779252312441432 \\
29655	0.0699538421515053 \\
30835	0.0627363752396216 \\
32015	0.0571232216300618 \\
33195	0.0522872084002348 \\
34375	0.0477471807741496 \\
35555	0.0435030523978138 \\
36735	0.0397185940670108 \\
37915	0.0362760323853878 \\
39095	0.0332054206577786 \\
40275	0.0304666827253359 \\
41455	0.0279277450350094 \\
42635	0.0255765817481997 \\
43815	0.0234547815320261 \\
44995	0.0215203006858838 \\
46175	0.0197057225670911 \\
47355	0.0180048715314601 \\
48535	0.01640806854798 \\
49715	0.0149626362886374 \\
50895	0.013669769518957 \\
52075	0.0125316097807399 \\
53255	0.0115637207372394 \\
54435	0.0106550299232834 \\
55615	0.00980385167826886 \\
56795	0.00907077080717472 \\
57975	0.00842195688616576 \\
59155	0.00781238933235368 \\
60335	0.00723959349981718 \\
61515	0.0067012671904858 \\
62695	0.00619636275087243 \\
63875	0.00572290049792429 \\
65055	0.00527899356365263 \\
66235	0.00486859478604235 \\
67415	0.00448339250967711 \\
68595	0.0041211742288691 \\
69775	0.00378051290384285 \\
70955	0.00346008045123131 \\
72135	0.00315863830780344 \\
73315	0.00287502981839188 \\
74495	0.00260817363494509 \\
75675	0.00235705790261731 \\
76855	0.00212073508627908 \\
78035	0.00189831732740536 \\
79215	0.00168897224685566 \\
80395	0.00149191912777275 \\
81575	0.0013603685852604 \\
82755	0.00126419615453838 \\
83935	0.00118078887728162 \\
85115	0.00110269661281565 \\
86295	0.00102951713399574 \\
87475	0.000960906244845627 \\
88655	0.000896551562783232 \\
89835	0.000836167182567793 \\
91015	0.000779490323171428 \\
92195	0.000726278656555279 \\
93375	0.00067630810747854 \\
94555	0.000629371013065982 \\
95735	0.000585274562320492 \\
96915	0.000543839456283723 \\
98095	0.00050489874438292 \\
99275	0.000468296803364354 \\
100455	0.000433888433198415 \\
101635	0.000401538050234074 \\
102815	0.000371118962259553 \\
103995	0.000345069997834457 \\
105175	0.000323132317298414 \\
106355	0.000302627694577673 \\
107535	0.000283451745664631 \\
108715	0.000265510349358511 \\
109895	0.000248717820764022 \\
111075	0.000232995743352338 \\
112255	0.000218272052982138 \\
113435	0.000204480297691323 \\
114615	0.000191559030343402 \\
115795	0.000179451303149503 \\
116975	0.000168104241096423 \\
118155	0.000157468677151218 \\
119335	0.000147498836407078 \\
120515	0.000138152059497576 \\
121695	0.000129388557939147 \\
122875	0.000121171195794356 \\
124055	0.000113465293328063 \\
125235	0.000106238449292362 \\
126415	9.94603791851789e-05 \\
127595	9.31027673738982e-05 \\
128775	8.7139131378644e-05 \\
129955	8.15446969246181e-05 \\
131135	7.62962826094116e-05 \\
132315	7.13721932176781e-05 \\
133405	6.67521208619348e-05 \\
134135	6.2417053241226e-05 \\
134865	5.83491884025844e-05 \\
135595	5.45318554640573e-05 \\
136325	5.09494408236222e-05 \\
137055	4.75873194201726e-05 \\
137785	4.44317906674319e-05 \\
138515	4.14700187093553e-05 \\
139245	3.86899766816606e-05 \\
139975	3.6080394691218e-05 \\
140705	3.36307112471235e-05 \\
141435	3.13310279027035e-05 \\
142165	2.91720668838535e-05 \\
142895	2.71451314968751e-05 \\
143625	2.52420691256305e-05 \\
144355	2.34552366428775e-05 \\
145085	2.17774680700278e-05 \\
145815	2.02020443350048e-05 \\
146545	1.87226649881467e-05 \\
147275	1.73334217439258e-05 \\
148005	1.60287737278053e-05 \\
148735	1.48035243157651e-05 \\
149465	1.36527994591962e-05 \\
150195	1.25720274004038e-05 \\
150925	1.15569196839083e-05 \\
151655	1.07076029184361e-05 \\
152385	9.99345134872209e-06 \\
153115	9.32826453253766e-06 \\
153845	8.70833755850509e-06 \\
154575	8.13031745378057e-06 \\
155305	7.59114784260051e-06 \\
156035	7.08803261867397e-06 \\
156765	6.61840640880929e-06 \\
157495	6.17990986240269e-06 \\
158225	5.77036871629577e-06 \\
158955	5.38777586428507e-06 \\
159685	5.03027584269899e-06 \\
160415	4.69615128273482e-06 \\
161145	4.38381098033513e-06 \\
161875	4.09177931198768e-06 \\
162605	3.81868678223096e-06 \\
163335	3.56326153549968e-06 \\
164065	3.3243216949197e-06 \\
164795	3.10321699154015e-06 \\
165525	2.90571622280078e-06 \\
166255	2.72121266564618e-06 \\
166985	2.54879175715095e-06 \\
167715	2.3876230570985e-06 \\
168445	2.2369387088772e-06 \\
169175	2.09602722534008e-06 \\
169905	1.96422874559898e-06 \\
170635	1.84093083283043e-06 \\
171365	1.72556468897023e-06 \\
172095	1.61760173322811e-06 \\
172825	1.5165504962944e-06 \\
173555	1.42195379249088e-06 \\
174285	1.33338613972356e-06 \\
175015	1.25045139842683e-06 \\
175745	1.17278060829396e-06 \\
176475	1.10003000231007e-06 \\
177205	1.03187917976921e-06 \\
177935	9.68029425785222e-07 \\
178665	9.08202161253868e-07 \\
179395	8.5213751249702e-07 \\
180125	7.99592991762577e-07 \\
180855	7.50342275979143e-07 \\
181585	7.0417407888046e-07 \\
182315	6.60891105230821e-07 \\
183045	6.20309084153892e-07 \\
183775	5.82255871794946e-07 \\
184505	5.46570620596487e-07 \\
185235	5.13103007804272e-07 \\
185965	4.81712519873057e-07 \\
186695	4.52267788497718e-07 \\
187425	4.24645972385562e-07 \\
188155	3.98732185435957e-07 \\
188885	3.74418961890388e-07 \\
189615	3.51605762283214e-07 \\
190345	3.30198510867241e-07 \\
191075	3.10109168233375e-07 \\
191805	2.91255330520102e-07 \\
192535	2.73559858210426e-07 \\
193265	2.56950529964328e-07 \\
193995	2.41359719710399e-07 \\
194725	2.26724096497044e-07 \\
195455	2.12984343495037e-07 \\
196185	2.00084897261643e-07 \\
196915	1.87973704324218e-07 \\
197645	1.76601992918357e-07 \\
198375	1.65924062156542e-07 \\
199105	1.55897083020662e-07 \\
199835	1.46480914453573e-07 \\
200565	1.37637930108792e-07 \\
201295	1.29332858034292e-07 \\
202025	1.21532629571153e-07 \\
202755	1.1420623929892e-07 \\
203485	1.07324613252135e-07 \\
204215	1.00860486629273e-07 \\
204945	9.47882885515838e-08 \\
205675	8.90840353706679e-08 \\
206405	8.37252299712432e-08 \\
207135	7.86907679573012e-08 \\
207865	7.39608503330658e-08 \\
208595	6.95169007358665e-08 \\
209325	6.53414893858617e-08 \\
210055	6.14182606994973e-08 \\
210785	5.77318661210136e-08 \\
211515	5.42679013393332e-08 \\
212245	5.10128475017524e-08 \\
212975	4.79540153697222e-08 \\
213705	4.50794940820565e-08 \\
214435	4.23781029712522e-08 \\
215165	3.98393458778123e-08 \\
215895	3.74533686842149e-08 \\
216625	3.52109197909733e-08 \\
217355	3.31033125910984e-08 \\
218085	3.11223907756286e-08 \\
218815	2.92604951379616e-08 \\
219545	2.75104334312992e-08 \\
220275	2.58654516138712e-08 \\
221005	2.43192063709152e-08 \\
221735	2.28657400236365e-08 \\
222465	2.14994572700355e-08 \\
223195	2.02151024808472e-08 \\
223925	1.90077390493926e-08 \\
224655	1.78727298516534e-08 \\
225385	1.68057187610593e-08 \\
226115	1.58026135510525e-08 \\
226845	1.4859569963388e-08 \\
227575	1.39729762760332e-08 \\
228305	1.31394391478246e-08 \\
229035	1.2355770628858e-08 \\
229765	1.16189752819018e-08 \\
230495	1.09262383585218e-08 \\
231225	1.02749153629844e-08 \\
231955	9.66252106104903e-09 \\
232685	9.08672054267257e-09 \\
233415	8.54531906346878e-09 \\
234145	8.03625455070289e-09 \\
234875	7.55758899906311e-09 \\
235605	7.1075011431887e-09 \\
236335	6.68427929673143e-09 \\
237065	6.28631446897288e-09 \\
237795	5.91209492473155e-09 \\
238525	5.56019957853593e-09 \\
239255	5.22929261004279e-09 \\
239985	4.91811857905589e-09 \\
240715	4.62549737401119e-09 \\
241445	4.35031982659595e-09 \\
242175	4.09154327085659e-09 \\
242905	3.84818760190697e-09 \\
243635	3.6193312791255e-09 \\
244365	3.40410849508643e-09 \\
245095	3.20170479017889e-09 \\
245825	3.01135488767201e-09 \\
246555	2.83233919651238e-09 \\
247285	2.66398131332224e-09 \\
248015	2.50564519133079e-09 \\
248745	2.35673291992811e-09 \\
249475	2.21668233768568e-09 \\
250205	2.08496503395494e-09 \\
250935	1.96108407291007e-09 \\
251665	1.8445723837246e-09 \\
252395	1.73499042910308e-09 \\
253125	1.63192548363611e-09 \\
253855	1.53498913579853e-09 \\
254585	1.44381606670407e-09 \\
255315	1.35806327294929e-09 \\
256045	1.27740773514518e-09 \\
256775	1.20154597382793e-09 \\
257505	1.13019260616909e-09 \\
258235	1.06307912473014e-09 \\
258965	9.99953231328732e-10 \\
259695	9.40577560282208e-10 \\
260425	8.84729012273766e-10 \\
261155	8.32197644129451e-10 \\
261885	7.82786169217786e-10 \\
262615	7.36308958249055e-10 \\
263345	6.92591484163785e-10 \\
264075	6.51469655998937e-10 \\
264805	6.12789097242938e-10 \\
265535	5.76404812768772e-10 \\
266265	5.42180134122106e-10 \\
266995	5.09986719521294e-10 \\
267725	4.79703776701257e-10 \\
268455	4.51217951891181e-10 \\
269185	4.24422275102643e-10 \\
269915	3.99216493196519e-10 \\
270645	3.7550590414881e-10 \\
271375	3.53202023184451e-10 \\
272105	3.32221083976236e-10 \\
272835	3.12484482734021e-10 \\
273565	2.93918389626668e-10 \\
274295	2.76453360203988e-10 \\
275025	2.60023946818677e-10 \\
275755	2.44568809648626e-10 \\
276485	2.30029884029648e-10 \\
277215	2.16352769033534e-10 \\
277945	2.03486616445758e-10 \\
278675	1.91383131564749e-10 \\
279405	1.79996961779949e-10 \\
280135	1.69285641060668e-10 \\
280865	1.59209145866868e-10 \\
281595	1.49729728615711e-10 \\
282325	1.40812084215014e-10 \\
283055	1.32422905974039e-10 \\
283785	1.24530608047735e-10 \\
284515	1.17105936059403e-10 \\
285245	1.10121189944579e-10 \\
285975	1.0355011292873e-10 \\
286705	9.73682801053144e-11 \\
287435	9.15526543465717e-11 \\
288165	8.60814197700677e-11 \\
288895	8.09341482721493e-11 \\
289625	7.60918550390954e-11 \\
290355	7.15361658798486e-11 \\
291085	6.72503719378881e-11 \\
291815	6.32182639570544e-11 \\
292545	5.94248539265152e-11 \\
293275	5.58560420138576e-11 \\
294005	5.24985055427862e-11 \\
294735	4.9339754504274e-11 \\
295465	4.63679650231086e-11 \\
296195	4.3572090380195e-11 \\
296925	4.09416944791019e-11 \\
297655	3.846689633491e-11 \\
298385	3.61387031411198e-11 \\
299115	3.39481220912319e-11 \\
299845	3.18873261129227e-11 \\
300575	2.99484326227173e-11 \\
301305	2.81242251709557e-11 \\
302035	2.64079869083389e-11 \\
302765	2.47932785413241e-11 \\
303495	2.3274049354427e-11 \\
304225	2.18448037436758e-11 \\
304955	2.05000461050986e-11 \\
305685	1.92347804350845e-11 \\
306415	1.80443993080814e-11 \\
307145	1.69244618319908e-11 \\
307875	1.58708046704703e-11 \\
308605	1.487937550948e-11 \\
309335	1.394656612419e-11 \\
310065	1.30689348232238e-11 \\
310795	1.22432064486588e-11 \\
311525	1.14664389094798e-11 \\
312255	1.07354125589154e-11 \\
312985	1.00476849063114e-11 \\
313715	9.40064692755982e-12 \\
314445	8.79180062085538e-12 \\
315175	8.21898105130003e-12 \\
315905	7.68007879514698e-12 \\
316635	7.17298442864944e-12 \\
317365	6.69581057266555e-12 \\
318095	6.24689189265837e-12 \\
318825	5.82461856524219e-12 \\
319555	5.42721423357762e-12 \\
320285	5.0531800965814e-12 \\
321015	4.70140593122892e-12 \\
321745	4.37028191413447e-12 \\
322475	4.05891986687834e-12 \\
323205	3.7658209883773e-12 \\
323935	3.49009710021164e-12 \\
324665	3.23063797935674e-12 \\
325395	2.9865554473929e-12 \\
326125	2.75679479244673e-12 \\
326855	2.54068988070344e-12 \\
327585	2.33735253374334e-12 \\
328315	2.1460055954492e-12 \\
329045	1.965927420855e-12 \\
329775	1.79661840959966e-12 \\
330505	1.63719038326349e-12 \\
331235	1.48725476378786e-12 \\
331965	1.346145417358e-12 \\
332695	1.2133072324616e-12 \\
333425	1.08840714219127e-12 \\
334155	9.70834523883468e-13 \\
334885	8.60256310630803e-13 \\
335615	7.56172902072194e-13 \\
336345	6.58195720149024e-13 \\
337075	5.66047209105136e-13 \\
337805	4.7928327973068e-13 \\
338535	3.97737398571962e-13 \\
339265	3.20909965267901e-13 \\
339995	2.48745468667266e-13 \\
340725	1.80799819560207e-13 \\
341455	1.16739951039335e-13 \\
342185	5.6621374255883e-14 \\
342915	0 \\
};
\end{axis}

\end{tikzpicture}
}
\end{center}
\caption{1D-TV-regularized logistic regression \eqref{eq:logtv}}
\label{fig:a11}
\end{figure}

% ========================================================================
% ========================================================================
%\section{Additional experiments on adaptive selection}
% ========================================================================
% ========================================================================

Finally, Figure~\ref{fig:tvcomp} displays 20 runs of 1 and 20\% \adaalgo~as well as the median of the runs in bold. We notice that more than 50\% of the time, a low-dimensional structure is quickly identified (after the third adaptation) resulting in a dramatic speed increase in terms of subspaces explored. However, this adaptation to the lower-dimensional subspace might take some more time (either because of poor identification in the first iterates or because a first heavy adaptation was made early and a pessimistic bound on the rate prevents a new adaptation in theory). Yet, one can notice that these adaptations are more stable for the 20\% than for the 1 \adaalgo, illustrating the ``speed versus stability'' tradeoff in the selection. 


\begin{figure}[H]
\begin{center}
\scalebox{.9}{% This file was created by matplotlib2tikz v0.6.18.
\begin{tikzpicture}

\begin{axis}[
legend cell align={left},
legend entries={{1 \adaalgo }},
legend style={at={(0.5,1.05)}, anchor=north},
tick align=outside,
tick pos=left,
xlabel={Number of Subspaces explored},
xmajorgrids,
xmin=0, xmax=400000,
ylabel={Suboptimality},
ymajorgrids,
ymin=1e-12, ymax=18.3325629648828,
ymode=log
]
\addlegendimage{no markers, red}
\addplot [line width=0.01pt, red, forget plot]
table [row sep=\\]{%
1172	3.35150350606517 \\
2342	2.70073306759077 \\
3512	2.21486783445578 \\
4682	1.82381937460277 \\
5852	1.51762960039436 \\
7022	1.26414336420533 \\
8192	1.04981356540928 \\
9362	0.881115774146242 \\
10532	0.738859472334074 \\
11702	0.625241132140057 \\
12872	0.531169593365919 \\
14042	0.451247046034789 \\
15212	0.381256608390881 \\
16382	0.322245688022483 \\
17552	0.273489377180174 \\
18722	0.236468253786931 \\
19892	0.20542146382878 \\
21062	0.177150907341218 \\
22232	0.151629196332942 \\
23402	0.133374673337798 \\
24572	0.117483258927176 \\
25742	0.104176250285456 \\
26912	0.0932580334722057 \\
28082	0.0831991216847496 \\
29252	0.0747150900718986 \\
30422	0.0675072554749215 \\
31592	0.0609062215027531 \\
32762	0.0551259305022782 \\
33932	0.0497541630509611 \\
35102	0.0447178444359171 \\
36272	0.040488310702503 \\
37442	0.0365516989377759 \\
38612	0.0328767624837716 \\
39782	0.0294358433502053 \\
40952	0.0262907242383209 \\
42122	0.0234123446597893 \\
43292	0.0208566626511138 \\
44462	0.0186753764961016 \\
45632	0.0166252777830706 \\
46802	0.0146980002170139 \\
47972	0.0129613539064473 \\
49142	0.0116486507339414 \\
50312	0.010553537211083 \\
51482	0.00957071850719166 \\
52652	0.00867221205485003 \\
53822	0.00783323749143755 \\
54992	0.00705099479262217 \\
56162	0.00632223734425824 \\
57332	0.00583418206286301 \\
58502	0.00537699402929376 \\
59672	0.00494746439285637 \\
60842	0.0045437640390727 \\
62012	0.00416425478243232 \\
63182	0.00380741747987384 \\
64352	0.00354096370658397 \\
65522	0.00329259979920765 \\
66692	0.00305940995419263 \\
67862	0.00284041213127767 \\
69032	0.00263469900416247 \\
70202	0.00244153549497456 \\
71372	0.00226143800580314 \\
72542	0.00210290429469678 \\
73712	0.00196485306857663 \\
74882	0.00183555363701832 \\
76052	0.00171435763698891 \\
77222	0.00160066012595939 \\
78392	0.00149494914931936 \\
79562	0.00139793060311022 \\
80732	0.00130680098400721 \\
81902	0.00122254498573593 \\
83072	0.00114342749742979 \\
84242	0.00106911111715918 \\
85412	0.000999284962628577 \\
86582	0.000933661429775967 \\
87752	0.000871973870987586 \\
88922	0.000813974629276182 \\
90092	0.000762597044788482 \\
91262	0.000715326765125646 \\
92432	0.000671046023688948 \\
93602	0.000629551259918482 \\
94772	0.00059066808205599 \\
95942	0.000554233691718087 \\
97112	0.000520076010353976 \\
98282	0.00048804700154409 \\
99452	0.000458009210277943 \\
100622	0.000429834804339668 \\
101792	0.000403404755531078 \\
102962	0.000378608122080515 \\
104132	0.000355341413012689 \\
105302	0.000333508020347817 \\
106472	0.000313017708221064 \\
107642	0.00029378615038117 \\
108812	0.000275734509307901 \\
109982	0.000258789051530706 \\
111152	0.000242880794752098 \\
112322	0.000227945183164935 \\
113492	0.000213921787956428 \\
114662	0.000200754030464434 \\
115832	0.0001883889258259 \\
117002	0.000176776845250859 \\
118172	0.000165871295299713 \\
119342	0.000155628712730616 \\
120512	0.00014600827365352 \\
121682	0.000136971715854806 \\
122852	0.000128483173276628 \\
124022	0.000120509021730775 \\
125192	0.000113017735010601 \\
126362	0.000105979750641583 \\
127532	9.9367344575052e-05 \\
128702	9.31545141830692e-05 \\
129872	8.73168689750625e-05 \\
131042	8.18315284893401e-05 \\
132212	7.66770268660344e-05 \\
133382	7.18332236390729e-05 \\
134552	6.72812203232342e-05 \\
135722	6.30032824014415e-05 \\
136892	5.89827663465825e-05 \\
138062	5.5204051341351e-05 \\
139232	5.16524753790271e-05 \\
140402	4.83142754553745e-05 \\
141572	4.51765315786501e-05 \\
142742	4.2227114346316e-05 \\
143912	3.94546358544745e-05 \\
145082	3.69566034197355e-05 \\
146252	3.47020609693938e-05 \\
147422	3.25877046704059e-05 \\
148592	3.06043365629338e-05 \\
149762	2.87434553234434e-05 \\
150932	2.69971755562093e-05 \\
152102	2.53581698597682e-05 \\
153272	2.38196195836426e-05 \\
154442	2.23751719878318e-05 \\
155612	2.10189025671159e-05 \\
156782	1.97452816169119e-05 \\
157952	1.85491443293051e-05 \\
159122	1.74256638621473e-05 \\
160292	1.63703269424609e-05 \\
161462	1.5378911654651e-05 \\
162632	1.44474671311956e-05 \\
163802	1.35722949183292e-05 \\
164972	1.27499318244295e-05 \\
166142	1.19771340973407e-05 \\
167312	1.12508627921892e-05 \\
168482	1.05682702225551e-05 \\
169652	9.92668738936109e-06 \\
170822	9.32361230832068e-06 \\
171992	8.75669915617605e-06 \\
173162	8.22374817222027e-06 \\
174332	7.72269625198829e-06 \\
175502	7.25160818382253e-06 \\
176672	6.80866847474482e-06 \\
177842	6.4050985036368e-06 \\
179012	6.02674252325741e-06 \\
180182	5.67106164206521e-06 \\
181352	5.33665724622434e-06 \\
182522	5.02222413900633e-06 \\
183692	4.72654141997486e-06 \\
184862	4.44846631436624e-06 \\
186032	4.1869286740992e-06 \\
187202	3.94092595423778e-06 \\
188372	3.70951860062529e-06 \\
189542	3.49182580428042e-06 \\
190712	3.28702158508554e-06 \\
191882	3.09433117101632e-06 \\
193052	2.91302764626744e-06 \\
194222	2.74242884240605e-06 \\
195392	2.58189445195844e-06 \\
196562	2.43082334538958e-06 \\
197732	2.28865107465559e-06 \\
198902	2.15484754756412e-06 \\
200072	2.02891486078549e-06 \\
201242	1.91038527841414e-06 \\
202412	1.79881934542214e-06 \\
203582	1.69380412540221e-06 \\
204752	1.59495155471756e-06 \\
205922	1.50189690373281e-06 \\
207092	1.41429733718779e-06 \\
208262	1.33183056849617e-06 \\
209432	1.25419359914281e-06 \\
210602	1.18110153884965e-06 \\
211772	1.11228650134798e-06 \\
212942	1.04749656831826e-06 \\
214112	9.86494820609529e-07 \\
215282	9.29058428134066e-07 \\
216452	8.74977797993282e-07 \\
217622	8.24055775283661e-07 \\
218792	7.76106892863559e-07 \\
219962	7.30956668915095e-07 \\
221132	6.8844094608389e-07 \\
222302	6.48405272751784e-07 \\
223472	6.10704321779565e-07 \\
224642	5.75201343888665e-07 \\
225812	5.41767656458969e-07 \\
226982	5.10282161358955e-07 \\
228152	4.80630892918388e-07 \\
229322	4.52706594322727e-07 \\
230492	4.26408317766303e-07 \\
231662	4.01641050307155e-07 \\
232832	3.78315361537762e-07 \\
234002	3.56347072461016e-07 \\
235172	3.35656944905338e-07 \\
236342	3.16170388814374e-07 \\
237512	2.97817187855376e-07 \\
238682	2.80531240237636e-07 \\
239852	2.64250316683867e-07 \\
241022	2.48915831724261e-07 \\
242192	2.34472628646287e-07 \\
243362	2.20868777989214e-07 \\
244532	2.08055386974326e-07 \\
245702	1.95986421425154e-07 \\
246872	1.8461853751317e-07 \\
248042	1.73910923273457e-07 \\
249212	1.63825150556463e-07 \\
250382	1.54325034806835e-07 \\
251552	1.45376503113415e-07 \\
252722	1.3694747097448e-07 \\
253892	1.29007725113706e-07 \\
255062	1.21528814733818e-07 \\
256232	1.14483947044608e-07 \\
257402	1.07847891561708e-07 \\
258572	1.01596887680522e-07 \\
259742	9.57085590225226e-08 \\
260912	9.01618329440801e-08 \\
262082	8.49368633759617e-08 \\
263252	8.00149600466149e-08 \\
264422	7.53785204810065e-08 \\
265592	7.10109668289327e-08 \\
266762	6.68966853578645e-08 \\
267932	6.30209715524188e-08 \\
269102	5.93699751583188e-08 \\
270272	5.59306521097369e-08 \\
270812	5.26907165121493e-08 \\
270932	4.96385969550595e-08 \\
271052	4.67633943790347e-08 \\
271172	4.40548432179e-08 \\
271292	4.15032748168898e-08 \\
271412	3.90995816279549e-08 \\
271532	3.68351859569849e-08 \\
271652	3.47020078783622e-08 \\
271772	3.26924367577419e-08 \\
271892	3.07993041070986e-08 \\
272012	2.90158578275523e-08 \\
272132	2.73357374513949e-08 \\
272252	2.57529521041633e-08 \\
272372	2.42618583556897e-08 \\
272492	2.2857140680177e-08 \\
272612	2.15337918052505e-08 \\
272732	2.02870951704348e-08 \\
272852	1.91126076631853e-08 \\
272972	1.80061440757662e-08 \\
273092	1.69637623392838e-08 \\
273212	1.59817483136315e-08 \\
273332	1.50566041301481e-08 \\
273452	1.41850342028071e-08 \\
273572	1.33639343480318e-08 \\
273692	1.25903803493976e-08 \\
273812	1.18616174105135e-08 \\
273932	1.1175050607104e-08 \\
274052	1.05282355611358e-08 \\
274172	9.91886933698893e-09 \\
274292	9.34478266989558e-09 \\
274412	8.80393219437892e-09 \\
274532	8.29439267269194e-09 \\
274652	7.81435055552393e-09 \\
274772	7.36209754270689e-09 \\
274892	6.93602408841087e-09 \\
275012	6.53461446065151e-09 \\
275132	6.15643985790726e-09 \\
275252	5.80015502293918e-09 \\
275372	5.46449235860891e-09 \\
275492	5.14825782005346e-09 \\
275612	4.85032647379313e-09 \\
275732	4.56963838990632e-09 \\
275852	4.30519531136042e-09 \\
275972	4.05605682374244e-09 \\
276092	3.82133663601181e-09 \\
276212	3.60020030454322e-09 \\
276332	3.39186101427913e-09 \\
276452	3.19557835748441e-09 \\
276572	3.01065422592117e-09 \\
276692	2.83643103449194e-09 \\
276812	2.67228950079357e-09 \\
276932	2.51764592507087e-09 \\
277052	2.37195069141549e-09 \\
277172	2.2346856032307e-09 \\
277292	2.10536305056408e-09 \\
277412	1.98352317903883e-09 \\
277532	1.86873277963073e-09 \\
277652	1.7605840119117e-09 \\
277772	1.65869262769291e-09 \\
277892	1.56269613915683e-09 \\
278012	1.47225365232373e-09 \\
278132	1.38704381313914e-09 \\
278252	1.30676353071735e-09 \\
278372	1.23112769978562e-09 \\
278492	1.15986764637199e-09 \\
278612	1.09273001758226e-09 \\
278732	1.02947628199956e-09 \\
278852	9.69881897017189e-10 \\
278972	9.13734865548577e-10 \\
279092	8.60835791538506e-10 \\
279212	8.10996769740058e-10 \\
279332	7.64040775091956e-10 \\
279452	7.19800996584752e-10 \\
279572	6.78120282149308e-10 \\
279692	6.38850361500687e-10 \\
279812	6.01852068182751e-10 \\
279932	5.66993785255931e-10 \\
280052	5.34151889386436e-10 \\
280172	5.03209529600923e-10 \\
280292	4.74056904842257e-10 \\
280412	4.46590320279938e-10 \\
280532	4.2071257588816e-10 \\
280652	3.96331412133577e-10 \\
280772	3.73360453664873e-10 \\
280892	3.51718099089737e-10 \\
281012	3.31327243419111e-10 \\
281132	3.12115888689846e-10 \\
281252	2.94015589652474e-10 \\
281372	2.76962119905022e-10 \\
281492	2.60894916781496e-10 \\
281612	2.45756859307278e-10 \\
281732	2.3149432371028e-10 \\
281852	2.18056628309427e-10 \\
281972	2.0539603351466e-10 \\
282092	1.93467519782331e-10 \\
282212	1.82228843126353e-10 \\
282332	1.71640202051293e-10 \\
282452	1.61663848974314e-10 \\
282572	1.52264478803232e-10 \\
282692	1.43408562802705e-10 \\
282812	1.35064737172286e-10 \\
282932	1.2720346997952e-10 \\
283052	1.19796728093036e-10 \\
283172	1.12818199227149e-10 \\
283292	1.06243402964168e-10 \\
283412	1.00048580531364e-10 \\
283532	9.42121936020612e-11 \\
283652	8.87130924276391e-11 \\
283772	8.35320146386209e-11 \\
283892	7.86504750216466e-11 \\
284012	7.40513206309856e-11 \\
284132	6.97180646547224e-11 \\
284252	6.56352749928146e-11 \\
284372	6.17886297682446e-11 \\
284492	5.81644177266583e-11 \\
284612	5.47497047698187e-11 \\
284732	5.15323894667574e-11 \\
284852	4.85011475426234e-11 \\
284972	4.56450433006239e-11 \\
285092	4.29541402446887e-11 \\
285212	4.04188904568059e-11 \\
285332	3.80301345970224e-11 \\
285452	3.57795459926535e-11 \\
285572	3.36590200156195e-11 \\
285692	3.16611181716553e-11 \\
285812	2.97787905445546e-11 \\
285932	2.80051537515646e-11 \\
286052	2.63340460548989e-11 \\
286172	2.47596942948292e-11 \\
286292	2.32762698004763e-11 \\
286412	2.18785545236244e-11 \\
286532	2.0561830016419e-11 \\
286652	1.93210447640979e-11 \\
286772	1.81520909414701e-11 \\
286892	1.70506941898907e-11 \\
287012	1.60128577064711e-11 \\
287132	1.5035139799835e-11 \\
287252	1.4113876734001e-11 \\
287372	1.3245904373349e-11 \\
287492	1.24281140934102e-11 \\
287612	1.16576193143203e-11 \\
287732	1.09315889673667e-11 \\
287852	1.02476360730464e-11 \\
287972	9.60304058494899e-12 \\
288092	8.99591512393272e-12 \\
288212	8.42381719934338e-12 \\
288332	7.88474840973663e-12 \\
288452	7.37676586481939e-12 \\
288572	6.89825974120595e-12 \\
288692	6.44734265975444e-12 \\
288812	6.02262684168409e-12 \\
288932	5.62228041900426e-12 \\
289052	5.24519316869032e-12 \\
289172	4.88992180081027e-12 \\
289292	4.55507853658332e-12 \\
289412	4.23960866413609e-12 \\
289532	3.94240196044393e-12 \\
289652	3.6624592247847e-12 \\
289772	3.3985592118313e-12 \\
289892	3.14998027661773e-12 \\
290012	2.91572321842182e-12 \\
290132	2.69506639227757e-12 \\
290252	2.48717713091651e-12 \\
290372	2.29116725591894e-12 \\
290492	2.10670370037747e-12 \\
290612	1.9327317524187e-12 \\
290732	1.76891834513526e-12 \\
290852	1.61437530010744e-12 \\
290972	1.46899159503278e-12 \\
};
\addplot [line width=0.01pt, red, forget plot]
table [row sep=\\]{%
1181	2.72202728514471 \\
2361	2.16462163629181 \\
3541	1.73910112153469 \\
4721	1.41100781462238 \\
5901	1.15626612898288 \\
7081	0.954740389252827 \\
8261	0.794127678455655 \\
9441	0.661214231838067 \\
10621	0.549988291994691 \\
11801	0.453344615851676 \\
12981	0.372590570494666 \\
14161	0.309528486485886 \\
15341	0.254931604671682 \\
16521	0.208628891668166 \\
17701	0.170064660084681 \\
18881	0.14047919045082 \\
20061	0.120330244004731 \\
21241	0.102398500303974 \\
22421	0.0860649640790264 \\
23601	0.0723921310909172 \\
24781	0.06283770355276 \\
25961	0.0562258463271095 \\
27141	0.0505528434086254 \\
28321	0.0459804211549474 \\
29501	0.041779875452133 \\
30681	0.037924463779551 \\
31861	0.0343390915342334 \\
33041	0.0309900069137304 \\
34221	0.0278571828607526 \\
35401	0.0250252063921902 \\
36581	0.0226770885934706 \\
37761	0.0204930032128109 \\
38941	0.0185570177487178 \\
40121	0.0169064115705236 \\
41301	0.0154250915906946 \\
42481	0.0140420093063562 \\
43661	0.0127446247861565 \\
44841	0.0115265787497898 \\
46021	0.010382352481729 \\
47201	0.0093069570581541 \\
48381	0.0082958496385247 \\
49561	0.00734486954420094 \\
50741	0.00645018812885667 \\
51921	0.00560826877472465 \\
53101	0.00481583443197392 \\
54281	0.00412022716645027 \\
55461	0.00356224855940268 \\
56641	0.00312944016495903 \\
57821	0.00273003392963816 \\
59001	0.0023668636711377 \\
60181	0.00202749745495151 \\
61361	0.00171856795585706 \\
62541	0.00156384208590915 \\
63721	0.0014512216484866 \\
64901	0.00134724164467231 \\
66081	0.00125490944055823 \\
67261	0.001172425501278 \\
68441	0.00109607112749444 \\
69621	0.00102529607357077 \\
70801	0.000959612749904526 \\
71981	0.000898586732445805 \\
73161	0.000841829436975583 \\
74341	0.000788992070058447 \\
75521	0.000739760569719194 \\
76701	0.000693851339324425 \\
77881	0.000651007624189404 \\
79061	0.000610996413726195 \\
80241	0.000573605777041575 \\
81421	0.000538642558941538 \\
82601	0.000505930377900365 \\
83781	0.000475307878802067 \\
84961	0.000446627202020822 \\
86141	0.000419752637276005 \\
87321	0.000394559436131758 \\
88501	0.000370932761352527 \\
89681	0.000348766754815821 \\
90861	0.00032796370852578 \\
92041	0.000308433325589275 \\
93221	0.000290092059934965 \\
94401	0.000272862525148965 \\
95581	0.000256672964133231 \\
96761	0.000241456772412163 \\
97941	0.000227152068867642 \\
99121	0.000213701308485081 \\
100301	0.000201050932390034 \\
101481	0.000189153946314913 \\
102661	0.000177962212726257 \\
103841	0.000167429662166818 \\
105021	0.000157516045357076 \\
106201	0.000148183749532349 \\
107381	0.000139397602520352 \\
108561	0.000131124694962015 \\
109741	0.000123334217944027 \\
110921	0.000115997314361571 \\
112101	0.000109086942590209 \\
113281	0.000102577751231347 \\
114461	9.64459638440607e-05 \\
115011	9.06692727065117e-05 \\
115141	8.52267407627227e-05 \\
115271	8.00987112745677e-05 \\
115401	7.52667221727843e-05 \\
115531	7.07134329085979e-05 \\
115661	6.64224608882824e-05 \\
115791	6.23786080195532e-05 \\
115921	5.85674698226346e-05 \\
116051	5.49756212721575e-05 \\
116181	5.15900754735066e-05 \\
116311	4.83989643256244e-05 \\
116441	4.53909737454516e-05 \\
116571	4.25556401901228e-05 \\
116701	3.98826522936546e-05 \\
116831	3.73631486180193e-05 \\
116961	3.49879764676153e-05 \\
117091	3.27485353801005e-05 \\
117221	3.06375684830384e-05 \\
117351	2.86470782261405e-05 \\
117481	2.6771059923214e-05 \\
117611	2.50019980985172e-05 \\
117741	2.33340459337983e-05 \\
117871	2.17611856078159e-05 \\
118001	2.02783453775446e-05 \\
118131	1.8880136119781e-05 \\
118261	1.75620518123032e-05 \\
118391	1.63184678744765e-05 \\
118521	1.51463150889763e-05 \\
118651	1.40405941714827e-05 \\
118781	1.29996165639246e-05 \\
118911	1.20171450149198e-05 \\
119041	1.12139041528736e-05 \\
119171	1.05152739880432e-05 \\
119301	9.86336135111454e-06 \\
119431	9.25361751191733e-06 \\
119561	8.68302919349029e-06 \\
119691	8.14887542327503e-06 \\
119821	7.64862460184146e-06 \\
119951	7.17999634042288e-06 \\
120081	6.74086658564166e-06 \\
120211	6.32927830374053e-06 \\
120341	5.94334526077267e-06 \\
120471	5.58148785140267e-06 \\
120601	5.24214983849136e-06 \\
120731	4.92385375960858e-06 \\
120861	4.62524957300925e-06 \\
120991	4.34511611563382e-06 \\
121121	4.08221176279033e-06 \\
121251	3.83547513171933e-06 \\
121381	3.60737637733477e-06 \\
121511	3.38888194217901e-06 \\
121641	3.18586125325204e-06 \\
121771	2.9941969905134e-06 \\
121901	2.81087346559739e-06 \\
122031	2.64189244231217e-06 \\
122161	2.48319586454437e-06 \\
122291	2.33550267619842e-06 \\
122421	2.19460135186811e-06 \\
122551	2.0627113719307e-06 \\
122681	1.93916707152963e-06 \\
122811	1.82311385049649e-06 \\
122941	1.71395972231592e-06 \\
123071	1.61148573063485e-06 \\
123201	1.51519378915888e-06 \\
123331	1.42470595204669e-06 \\
123461	1.33966724324974e-06 \\
123591	1.25974547221519e-06 \\
123721	1.18462889947546e-06 \\
123851	1.11402514246794e-06 \\
123981	1.04765993885758e-06 \\
124111	9.8527599096121e-07 \\
124241	9.26631888609197e-07 \\
124371	8.71501102339689e-07 \\
124501	8.19671041929482e-07 \\
124631	7.70942179706413e-07 \\
124761	7.25127227430011e-07 \\
124891	6.82050369793519e-07 \\
125021	6.4154654538795e-07 \\
125151	6.03460775683295e-07 \\
125281	5.67647536753313e-07 \\
125411	5.33970170635278e-07 \\
125541	5.03592515022078e-07 \\
125671	4.72522988237412e-07 \\
125801	4.50975854493496e-07 \\
125931	4.19432090614791e-07 \\
126061	3.96113998490932e-07 \\
126191	3.72493759226433e-07 \\
126321	3.50434364138064e-07 \\
126451	3.27589143866991e-07 \\
126581	3.11953415266242e-07 \\
126711	2.89970165767706e-07 \\
126841	2.72813552804063e-07 \\
126971	2.56676379373211e-07 \\
127101	2.41496832031896e-07 \\
127231	2.27217750781161e-07 \\
127361	2.1378546610773e-07 \\
127491	2.01149533973766e-07 \\
127621	1.8926253547713e-07 \\
127751	1.78079894608274e-07 \\
127881	1.67559705221976e-07 \\
128011	1.5766256977745e-07 \\
128141	1.48351719686524e-07 \\
128271	1.39591747938539e-07 \\
128401	1.31350239740424e-07 \\
128531	1.2359638801307e-07 \\
128661	1.16301240804528e-07 \\
128791	1.09437574558058e-07 \\
128921	1.02979788862978e-07 \\
129051	1.00061895880188e-07 \\
129181	9.43316062484989e-08 \\
129311	8.58126091851652e-08 \\
129441	8.07507597455626e-08 \\
129571	7.59884176138037e-08 \\
129701	7.15074298107687e-08 \\
129831	6.94394268752063e-08 \\
129961	6.39326835361054e-08 \\
130091	6.0633432386048e-08 \\
130221	5.91159947105346e-08 \\
130351	5.30481990557163e-08 \\
130481	5.02176679684219e-08 \\
130611	4.87243059787978e-08 \\
130741	4.39855329381622e-08 \\
130871	4.13937666365882e-08 \\
131001	3.89549149049095e-08 \\
131131	3.66599257062283e-08 \\
131261	3.65617466302481e-08 \\
131391	3.40713369739731e-08 \\
131521	3.08584615638097e-08 \\
131651	2.92560352610494e-08 \\
131781	2.70637524613093e-08 \\
131911	2.54699318857909e-08 \\
132041	2.39700666537779e-08 \\
132171	2.25586053725912e-08 \\
132301	2.12303302160599e-08 \\
132431	1.99803315004132e-08 \\
132561	1.88039906423576e-08 \\
132691	1.76969615073297e-08 \\
132821	1.66551554769967e-08 \\
132951	1.56747249069333e-08 \\
133081	1.47520500259901e-08 \\
133211	1.38837246144163e-08 \\
133341	1.30665440134514e-08 \\
133471	1.69236082925295e-08 \\
133601	1.4082880250843e-08 \\
133731	1.08943775667392e-08 \\
133861	1.02530678947055e-08 \\
133991	9.94142679289212e-09 \\
134121	9.08187830495066e-09 \\
134251	1.0377549986007e-08 \\
134381	8.04487015892974e-09 \\
134511	7.57146229046768e-09 \\
134641	7.12595360496948e-09 \\
134771	6.70667188451546e-09 \\
134901	6.31207064394346e-09 \\
135031	5.94069532455777e-09 \\
135161	5.5911782981255e-09 \\
135291	5.26223165042694e-09 \\
135421	4.95264412814223e-09 \\
135551	4.66127569875852e-09 \\
135681	7.50301859531177e-09 \\
135811	4.12938322424949e-09 \\
135941	3.88635595927056e-09 \\
136071	3.65772323540625e-09 \\
136201	3.44254907913566e-09 \\
136331	3.24003507357062e-09 \\
136461	3.04943614803932e-09 \\
136591	2.87005058607903e-09 \\
136721	2.70121830459047e-09 \\
136851	2.5423180782802e-09 \\
136981	2.39276537472577e-09 \\
137111	2.25201007841846e-09 \\
137241	2.11953438133961e-09 \\
137371	5.44561712390035e-09 \\
137501	5.24595611395284e-09 \\
137631	4.43266029437339e-09 \\
137761	1.66476860075093e-09 \\
137891	1.56597401712588e-09 \\
138021	1.47378970138945e-09 \\
138151	1.38707750840794e-09 \\
138281	4.09869693740461e-09 \\
138411	2.87374940510787e-09 \\
138541	1.15698267633135e-09 \\
138671	1.08861281100658e-09 \\
138801	1.0245367887407e-09 \\
138931	9.64246460455342e-10 \\
139061	9.07503405755961e-10 \\
139191	8.54097514935148e-10 \\
139321	8.03832611584454e-10 \\
139451	7.56523732547976e-10 \\
139581	7.11996739344301e-10 \\
139711	6.70088262655355e-10 \\
139841	6.30643981480716e-10 \\
139971	5.93519122737973e-10 \\
140101	5.58577128995097e-10 \\
140231	5.25689602959289e-10 \\
140361	4.19557100173051e-09 \\
140491	1.71942715621043e-09 \\
140621	8.93653706590669e-10 \\
140751	8.14873668542049e-10 \\
140881	3.88327980882508e-10 \\
141011	3.65415087077992e-10 \\
141141	3.43882700093445e-10 \\
141271	3.23614635089342e-10 \\
141401	3.04539338191745e-10 \\
141531	4.71997496642729e-10 \\
141661	4.07331335328109e-10 \\
141791	2.53795096050169e-10 \\
141921	2.38823572029645e-10 \\
142051	2.24733454068371e-10 \\
142181	4.34485503131299e-10 \\
142311	1.99199379213866e-10 \\
142441	1.87293625053542e-10 \\
142571	2.22848961506372e-10 \\
142701	1.65812585883884e-10 \\
142831	1.56014146046601e-10 \\
142961	1.46791745425645e-10 \\
143091	1.38111522218765e-10 \\
143221	1.29941557514002e-10 \\
143351	1.2225181977854e-10 \\
143481	6.59958976356734e-10 \\
143611	1.08211051230711e-10 \\
143741	4.52251458504804e-10 \\
143871	9.57653401023606e-11 \\
144001	2.11497525048898e-09 \\
144131	1.17842297031601e-09 \\
144261	7.98388577472053e-11 \\
144391	7.50206008426346e-11 \\
144521	7.05536185030553e-11 \\
144651	6.63534227562934e-11 \\
144781	6.2400584699418e-11 \\
144911	5.86805604108065e-11 \\
145041	5.51788059688363e-11 \\
145171	5.1985249438502e-11 \\
145301	4.88039053614386e-11 \\
145431	4.58766358235607e-11 \\
145561	4.31261693023544e-11 \\
145691	4.05385724988605e-11 \\
145821	2.10423756197287e-09 \\
145951	3.59223761847716e-11 \\
146081	1.38323158482834e-09 \\
146211	3.16811576972498e-11 \\
146341	2.97494806567045e-11 \\
146471	2.7946422953562e-11 \\
146601	2.62503352388421e-11 \\
146731	2.46543341297922e-11 \\
146861	2.31520913551719e-11 \\
146991	2.17381668221606e-11 \\
147121	2.04075090159961e-11 \\
147251	1.1703436553212e-09 \\
147381	1.7988610601094e-11 \\
147511	3.24234084025932e-10 \\
147641	1.58285606843833e-11 \\
147771	1.48446255288093e-11 \\
147901	1.3919199126633e-11 \\
148031	1.30481181415121e-11 \\
148161	1.22284959935826e-11 \\
148291	1.14577236587365e-11 \\
148421	1.07309161556657e-11 \\
148551	1.00471853059503e-11 \\
148681	9.40370004087754e-12 \\
148811	8.82355299935966e-12 \\
148941	8.23202617183938e-12 \\
149071	7.69401209410603e-12 \\
149201	7.18880510675035e-12 \\
149331	6.71318556300093e-12 \\
149461	6.26571017292576e-12 \\
149591	5.84460257968544e-12 \\
149721	5.4481419375918e-12 \\
149851	5.07499597901528e-12 \\
149981	4.72372141402388e-12 \\
150111	4.39304148613928e-12 \\
150241	4.08195699463931e-12 \\
150371	3.78913567189443e-12 \\
150501	3.51341178372877e-12 \\
150631	3.25395266287387e-12 \\
150761	3.0098146197588e-12 \\
150891	2.78105316553479e-12 \\
151021	2.56383803076687e-12 \\
151151	4.8158937948628e-10 \\
151281	2.17192930307419e-12 \\
151411	1.99723571014943e-12 \\
151541	1.82026616002418e-12 \\
151671	1.65983893296584e-12 \\
151801	1.50923717967544e-12 \\
151931	1.36751721058204e-12 \\
152061	1.23417942532456e-12 \\
152191	1.10855769008822e-12 \\
152321	9.90429960268102e-13 \\
152451	8.7913010204943e-13 \\
152581	2.41694220193267e-10 \\
152711	6.76403377752877e-13 \\
152841	5.8342219944052e-13 \\
152971	4.9604764740252e-13 \\
153101	4.13891143580258e-13 \\
153231	3.36453087612654e-13 \\
153361	2.63788990650937e-13 \\
153491	1.95177207729103e-13 \\
153621	1.30784272300843e-13 \\
153751	6.99995617026161e-14 \\
153881	1.28785870856518e-14 \\
154011	-4.10227407598995e-14 \\
154141	-9.16489106828067e-14 \\
154271	-1.39221967287995e-13 \\
};
\addplot [line width=0.01pt, red, forget plot]
table [row sep=\\]{%
1184	2.35509599069627 \\
2354	2.00656879842106 \\
3524	1.71553640270006 \\
4694	1.4581832205054 \\
5864	1.22627417724631 \\
7034	1.02465382150507 \\
8204	0.853293797265428 \\
9374	0.711756921561274 \\
10544	0.599974968238468 \\
11714	0.505203391131232 \\
12884	0.427556292428862 \\
14054	0.361605991254195 \\
15224	0.308159889742635 \\
16394	0.26141114899716 \\
17564	0.219246385117515 \\
18734	0.184820559981043 \\
19904	0.154916548953479 \\
21074	0.128759657110732 \\
22244	0.110256841794519 \\
23414	0.0960732312817882 \\
24584	0.0834872172038232 \\
25754	0.0723560199936595 \\
26924	0.0630785870249239 \\
28094	0.0546685166063753 \\
29264	0.0476625716227141 \\
30434	0.041838622370105 \\
31604	0.0364470879484766 \\
32774	0.0315363216402557 \\
33944	0.027162144398886 \\
35114	0.023480898871876 \\
36284	0.0202673462638649 \\
37454	0.0174181070241132 \\
38624	0.0149920187827308 \\
39794	0.0134063739802232 \\
40964	0.0119882834464592 \\
42134	0.0107414334700328 \\
43304	0.00971218077776176 \\
44474	0.00882639665713791 \\
45644	0.00800496744546214 \\
46814	0.00724044590972461 \\
47984	0.00657047072048988 \\
49154	0.00597732598025258 \\
50324	0.00548500562886456 \\
51494	0.00502710737573908 \\
52664	0.00460053789059028 \\
53834	0.00420265896037753 \\
55004	0.0038383624246025 \\
56174	0.00353139128009033 \\
57344	0.00324767173091689 \\
58514	0.00299560048669612 \\
59684	0.00276438890645753 \\
60854	0.00256347969150061 \\
62024	0.00237720538706293 \\
63194	0.00220435364596222 \\
64364	0.00204383433876826 \\
65534	0.00189466222331564 \\
66704	0.00175594417003044 \\
67874	0.00162686848044113 \\
69044	0.00150669582282914 \\
70214	0.00139475146809914 \\
71384	0.00129271492050231 \\
72554	0.00120196421814939 \\
73724	0.00111782554567458 \\
74894	0.00103973969976451 \\
76064	0.000967232295308873 \\
77234	0.000900054503590486 \\
78404	0.000837659318269068 \\
79574	0.0007796522376462 \\
80744	0.000725699487211073 \\
81914	0.000675494965716117 \\
83084	0.000628757710365302 \\
84254	0.000585229645667151 \\
85424	0.000544673554592423 \\
86594	0.000506871246123419 \\
87764	0.000471621897967567 \\
88934	0.000438740555953332 \\
90104	0.000408056773871179 \\
91274	0.00038121546029235 \\
92444	0.000356737085644043 \\
93614	0.00033392281249367 \\
94784	0.000312643397559542 \\
95954	0.000292786364820385 \\
97124	0.00027424864919634 \\
98294	0.000256935410557491 \\
99464	0.000240759290620673 \\
100634	0.000225639761533725 \\
101804	0.000211502538036257 \\
102974	0.000198279045306393 \\
104144	0.000185905936582298 \\
105314	0.000174324652431557 \\
106484	0.000163481000392762 \\
107654	0.000153324815687472 \\
108824	0.00014380964848143 \\
109994	0.00013489246125381 \\
111164	0.000126533353051073 \\
112334	0.000118695308206684 \\
113504	0.000111343967173883 \\
114674	0.000104447417355436 \\
115844	9.79760020393683e-05 \\
117014	9.19021457319591e-05 \\
118184	8.62001943593427e-05 \\
119354	8.08462689572131e-05 \\
120524	7.58181316087847e-05 \\
121694	7.10950625114615e-05 \\
122864	6.66714008086822e-05 \\
124034	6.26509622200233e-05 \\
125204	5.88796537093761e-05 \\
126374	5.53411202468546e-05 \\
127544	5.20202328384567e-05 \\
128714	4.89029471852365e-05 \\
129884	4.59762080705639e-05 \\
131054	4.32278668229258e-05 \\
132224	4.06466081322998e-05 \\
133394	3.82218846179949e-05 \\
134564	3.59438579632387e-05 \\
135734	3.38033456823705e-05 \\
136904	3.17917727669093e-05 \\
138074	2.99011275980421e-05 \\
139244	2.8123921622325e-05 \\
140414	2.64531523683775e-05 \\
141584	2.48822694528528e-05 \\
142754	2.34051432701499e-05 \\
143924	2.2016275600778e-05 \\
145094	2.07233534888496e-05 \\
146264	1.95073507565646e-05 \\
147434	1.83635335478938e-05 \\
148604	1.728750308988e-05 \\
149774	1.62755415332172e-05 \\
150944	1.53236893234432e-05 \\
152114	1.44281441780136e-05 \\
153284	1.35855088047587e-05 \\
154454	1.27925982274513e-05 \\
155624	1.20464255893293e-05 \\
156794	1.13441890716715e-05 \\
157964	1.06832597809858e-05 \\
159134	1.00611705137754e-05 \\
160304	9.47560532221781e-06 \\
161474	8.92438981481281e-06 \\
162644	8.4054821279933e-06 \\
163814	7.91696452029367e-06 \\
164984	7.45703553539645e-06 \\
166154	7.02400269386727e-06 \\
167324	6.61627567294376e-06 \\
168494	6.23235993679749e-06 \\
169664	5.87085078779248e-06 \\
170834	5.53042780448987e-06 \\
172004	5.20984964269555e-06 \\
173174	4.90794917107351e-06 \\
174344	4.62362891961998e-06 \\
175514	4.35585681990425e-06 \\
176684	4.10366221476055e-06 \\
177854	3.86613212260967e-06 \\
179024	3.64240773614855e-06 \\
180194	3.43168114141923e-06 \\
181364	3.23319224254659e-06 \\
182534	3.04622587848913e-06 \\
183704	2.87010911725893e-06 \\
184874	2.70420872094945e-06 \\
186044	2.54792876430709e-06 \\
187214	2.4007084019062e-06 \\
188384	2.26201977165941e-06 \\
189554	2.1313660264477e-06 \\
190724	2.00827948676485e-06 \\
191894	1.89231990571637e-06 \\
193064	1.78307283921209e-06 \\
194234	1.68014811452455e-06 \\
195404	1.58317839343836e-06 \\
196574	1.49181781966545e-06 \\
197744	1.40574074952715e-06 \\
198914	1.32464055874193e-06 \\
200084	1.24822851976791e-06 \\
201254	1.17623274853429e-06 \\
202424	1.10839721145783e-06 \\
203594	1.04448079452091e-06 \\
204764	9.84256425973218e-07 \\
205934	9.27510252157759e-07 \\
207104	8.74040862186565e-07 \\
208274	8.23658558135598e-07 \\
209444	7.76184670259195e-07 \\
210614	7.31450909674525e-07 \\
211784	6.89298763401069e-07 \\
212954	6.49578921319005e-07 \\
214124	6.12150739209838e-07 \\
215294	5.76881732938794e-07 \\
216464	5.43647102613232e-07 \\
217634	5.12329283941515e-07 \\
218804	4.82817527125423e-07 \\
219974	4.55007499788884e-07 \\
221144	4.28800912888327e-07 \\
222314	4.04105170215274e-07 \\
223484	3.80833035995565e-07 \\
224654	3.58902323915888e-07 \\
225824	3.38235603913883e-07 \\
226994	3.18759925177492e-07 \\
228164	3.00406556741351e-07 \\
229334	2.83110741849946e-07 \\
230504	2.66811468141448e-07 \\
231674	2.51451249766443e-07 \\
232844	2.36975923495475e-07 \\
234014	2.2333445526268e-07 \\
235184	2.10478760032107e-07 \\
236354	1.98363529768653e-07 \\
237524	1.86946073565952e-07 \\
238694	1.76186165101733e-07 \\
239864	1.66045900973355e-07 \\
241034	1.56489565694695e-07 \\
242204	1.47483505075208e-07 \\
243374	1.3899600731504e-07 \\
244544	1.3099719059495e-07 \\
245714	1.23458896772455e-07 \\
246884	1.16354592349932e-07 \\
248054	1.09659274605267e-07 \\
249224	1.03349382163387e-07 \\
250394	9.74027132283339e-08 \\
251564	9.17983455916982e-08 \\
252734	8.65165639685195e-08 \\
253904	8.1538789276081e-08 \\
255074	7.6847513463818e-08 \\
256244	7.24262378404283e-08 \\
257414	6.82594146761417e-08 \\
258584	6.43323921356576e-08 \\
259754	6.0631362874819e-08 \\
260924	5.71433150242662e-08 \\
262094	5.38559867258037e-08 \\
263264	5.07578223341021e-08 \\
264434	4.78379318380462e-08 \\
265604	4.50860528355967e-08 \\
266774	4.24925139519416e-08 \\
267944	4.00482008111602e-08 \\
269114	3.77445242283336e-08 \\
270284	3.55733893453447e-08 \\
271454	3.35271677642801e-08 \\
271574	3.15986704024773e-08 \\
271694	2.9781121901884e-08 \\
271814	2.80681371478408e-08 \\
271934	2.64536988980879e-08 \\
272054	2.49321359668819e-08 \\
272174	2.34981037405824e-08 \\
272294	2.2146565692438e-08 \\
272414	2.08727748418625e-08 \\
272534	1.96722576562003e-08 \\
272654	1.8540797896982e-08 \\
272774	1.74744224090695e-08 \\
272894	1.64693862991783e-08 \\
273014	1.55221604458688e-08 \\
273134	1.46294188430041e-08 \\
273254	1.37880269424073e-08 \\
273374	1.29950303295878e-08 \\
273494	1.2247645009289e-08 \\
273614	1.15432470249033e-08 \\
273734	1.08793631325987e-08 \\
273854	1.02536630297578e-08 \\
273974	9.66394991808173e-09 \\
274094	9.10815362020756e-09 \\
274214	8.58432319672531e-09 \\
274334	8.09061995177274e-09 \\
274454	7.62531066067496e-09 \\
274574	7.18676224087389e-09 \\
274694	6.7734353681459e-09 \\
274814	6.3838793140647e-09 \\
274934	6.01672672795317e-09 \\
275054	5.67068897394662e-09 \\
275174	5.34455174561188e-09 \\
275294	5.0371702364771e-09 \\
275414	4.74746575385154e-09 \\
275534	4.47442161100042e-09 \\
275654	4.21707946340888e-09 \\
275774	3.97453642220214e-09 \\
275894	3.74594127938721e-09 \\
276014	3.53049173229536e-09 \\
276134	3.32743183006912e-09 \\
276254	3.13604914259358e-09 \\
276374	2.95567181840539e-09 \\
276494	2.78566714140283e-09 \\
276614	2.62543847773244e-09 \\
276734	2.47442361045458e-09 \\
276854	2.33209268563073e-09 \\
276974	2.19794632494441e-09 \\
277094	2.07151368281089e-09 \\
277214	1.95235139166527e-09 \\
277334	1.84004128600535e-09 \\
277454	1.73418929216851e-09 \\
277574	1.63442387401957e-09 \\
277694	1.54039525579464e-09 \\
277814	1.45177331267732e-09 \\
277934	1.36824734875418e-09 \\
278054	1.28952432065788e-09 \\
278174	1.21532767183297e-09 \\
278294	1.14539749906939e-09 \\
278414	1.0794883320564e-09 \\
278534	1.01736896684912e-09 \\
278654	9.58821411156663e-10 \\
278774	9.03640329230626e-10 \\
278894	8.51632153686666e-10 \\
279014	8.02614363859533e-10 \\
279134	7.56415097225016e-10 \\
279254	7.12872094688066e-10 \\
279374	6.718328116051e-10 \\
279494	6.33153363072125e-10 \\
279614	5.96697746768626e-10 \\
279734	5.62338287046771e-10 \\
279854	5.29954469197236e-10 \\
279974	4.99432717404602e-10 \\
280094	4.70665784124691e-10 \\
280214	4.43552750084564e-10 \\
280334	4.17998857749069e-10 \\
280454	3.93914179053212e-10 \\
280574	3.7121428153597e-10 \\
280694	3.49819562206477e-10 \\
280814	3.29654914477118e-10 \\
280934	3.10649672652374e-10 \\
281054	2.9273722335077e-10 \\
281174	2.75854616926807e-10 \\
281294	2.59942678493275e-10 \\
281414	2.44945674854335e-10 \\
281534	2.3081087041632e-10 \\
281654	2.17488749232331e-10 \\
281774	2.04932681935333e-10 \\
281894	1.93098315115492e-10 \\
282014	1.81944570520898e-10 \\
282134	1.71432090745327e-10 \\
282254	1.61523960873211e-10 \\
282374	1.52185597457333e-10 \\
282494	1.43383971362709e-10 \\
282614	1.35088551456164e-10 \\
282734	1.27270027849846e-10 \\
282854	1.19900922523897e-10 \\
282974	1.12955589326447e-10 \\
283094	1.06409603350954e-10 \\
283214	1.00239927469659e-10 \\
283334	9.44249123335794e-11 \\
283454	8.89443518836686e-11 \\
283574	8.37788172169951e-11 \\
283694	7.89102116982576e-11 \\
283814	7.43216599374819e-11 \\
283934	6.99967861450546e-11 \\
284054	6.59206023101433e-11 \\
284174	6.20788420668816e-11 \\
284294	5.84579606943691e-11 \\
284414	5.50451351166714e-11 \\
284534	5.18285969697274e-11 \\
284654	4.87969664675347e-11 \\
284774	4.59396409802082e-11 \\
284894	4.32465174782237e-11 \\
285014	4.07083811104769e-11 \\
285134	3.83161280481659e-11 \\
285254	3.60613205963034e-11 \\
285374	3.39361871937172e-11 \\
285494	3.19332893461421e-11 \\
285614	3.00454661150695e-11 \\
285734	2.8266167184654e-11 \\
285854	2.6589230817109e-11 \\
285974	2.50086062969501e-11 \\
286094	2.35190200648105e-11 \\
286214	2.21150320278696e-11 \\
286334	2.07916461825164e-11 \\
286454	1.95444216366525e-11 \\
286574	1.83689174981794e-11 \\
286694	1.72609149196035e-11 \\
286814	1.62166946537923e-11 \\
286934	1.52324264313108e-11 \\
287054	1.43048350942365e-11 \\
287174	1.34305344623442e-11 \\
287294	1.26065269334674e-11 \\
287414	1.18298704165909e-11 \\
287534	1.10978448653043e-11 \\
287654	1.0407896766651e-11 \\
287774	9.75763914112804e-12 \\
287894	9.14479603153495e-12 \\
288014	8.56709148067125e-12 \\
288134	8.02269362054631e-12 \\
288254	7.50960404971579e-12 \\
288374	7.02593538903784e-12 \\
288494	6.57007781512675e-12 \\
288614	6.14047701574805e-12 \\
288734	5.73557867866725e-12 \\
288854	5.35393951395235e-12 \\
288974	4.99417174282257e-12 \\
289094	4.65522065340451e-12 \\
289214	4.33569846691739e-12 \\
289334	4.03455047148782e-12 \\
289454	3.75061093293994e-12 \\
289574	3.48304718400527e-12 \\
289694	3.2309155351129e-12 \\
289814	2.99327229669188e-12 \\
289934	2.76934031262499e-12 \\
290054	2.55823140449252e-12 \\
290174	2.35916841617723e-12 \\
290294	2.17170725846927e-12 \\
290414	1.99495975294894e-12 \\
290534	1.82842629925517e-12 \\
290654	1.67132974127071e-12 \\
290774	1.52333701208818e-12 \\
290894	1.38389300019526e-12 \\
291014	1.25244259407964e-12 \\
291134	1.12848619338024e-12 \\
291254	1.01169073118967e-12 \\
291374	9.01556607146858e-13 \\
291494	7.97972798949331e-13 \\
291614	7.00106639328624e-13 \\
};
\addplot [line width=0.01pt, red, forget plot]
table [row sep=\\]{%
1174	2.63166256830464 \\
2344	1.97912764363193 \\
3514	1.56276040320389 \\
4684	1.24180706542601 \\
5854	0.983672614370838 \\
7024	0.784344116928874 \\
8194	0.620600885941624 \\
9364	0.49400817729168 \\
10534	0.39079189800162 \\
11704	0.303998897350643 \\
12874	0.237755296395716 \\
14044	0.18784829382467 \\
15214	0.147623815482068 \\
16384	0.11939916811051 \\
17554	0.0970756160258332 \\
18724	0.0794464732187124 \\
19894	0.0651232822360621 \\
21064	0.0528220853399139 \\
22234	0.0431338562279367 \\
23404	0.0349423189162325 \\
24574	0.0279985394424191 \\
25744	0.0219253623902926 \\
26914	0.0175969220998612 \\
28084	0.0140435761018118 \\
29254	0.0113726338684743 \\
30424	0.00935326972524836 \\
31594	0.00800970793100314 \\
32764	0.00695071330081548 \\
33934	0.00598731726880936 \\
35104	0.00510350686460714 \\
36274	0.00447106764577654 \\
37444	0.0039513648554152 \\
38614	0.003480511360336 \\
39784	0.00308113496721873 \\
40954	0.00271837623052862 \\
42124	0.00244448707230249 \\
43294	0.00219937331131859 \\
44464	0.0019724562354711 \\
45634	0.00176219845585562 \\
46804	0.00156754455416486 \\
47974	0.0013961393987777 \\
49144	0.00123701507325341 \\
50314	0.00112532539618865 \\
51484	0.00104338990168157 \\
52654	0.000976143454336498 \\
53824	0.000915235262331116 \\
54994	0.000858364454429605 \\
56164	0.000805230577381577 \\
57334	0.00075556081160677 \\
58504	0.000709106124122671 \\
59674	0.000665638606148711 \\
60844	0.000624949220585147 \\
62014	0.000586845855902907 \\
63184	0.000551151634401892 \\
64354	0.000517703434335159 \\
65524	0.000486350592857121 \\
66694	0.000456953762637713 \\
67864	0.000429383899719138 \\
69034	0.000403521364012949 \\
70204	0.000379255116932153 \\
71374	0.00035648200317917 \\
72544	0.000335106105769589 \\
73714	0.000315038165061843 \\
74884	0.000296195053958892 \\
76054	0.000278499302597268 \\
77224	0.000261878666801829 \\
78394	0.000246265735378615 \\
79564	0.000231597571990871 \\
80734	0.000217815387927478 \\
81904	0.000204864242547398 \\
83074	0.000192692768590685 \\
84244	0.000181252919887742 \\
85414	0.000170499739297669 \\
86584	0.00016039114495664 \\
87754	0.00015088773313493 \\
88924	0.000141952596194594 \\
90094	0.000133551154298639 \\
91264	0.000125650999669269 \\
92434	0.000118221752315406 \\
93604	0.000111234926262249 \\
94774	0.000104663805408023 \\
95944	9.84833282232089e-05 \\
97114	9.26699805781084e-05 \\
98284	8.72016960554167e-05 \\
99454	8.20577631611696e-05 \\
100624	7.72187389021517e-05 \\
101794	7.2666368244545e-05 \\
102964	6.83835090110607e-05 \\
104134	6.43540618126548e-05 \\
105304	6.0562904643735e-05 \\
106474	5.69958318029085e-05 \\
107644	5.36394968266296e-05 \\
108814	5.04813591509778e-05 \\
109984	4.75096342373305e-05 \\
111154	4.47132469207911e-05 \\
112324	4.20817877561097e-05 \\
113494	3.96054721576444e-05 \\
114664	3.72751021395223e-05 \\
115834	3.5082030482414e-05 \\
117004	3.30181271628316e-05 \\
118174	3.10761683104954e-05 \\
119344	2.92503895317764e-05 \\
120514	2.75321412240292e-05 \\
121684	2.59150618084347e-05 \\
122854	2.43931700978917e-05 \\
124024	2.29608413487292e-05 \\
125194	2.1612785609193e-05 \\
126364	2.03440274832434e-05 \\
127534	1.91498871673357e-05 \\
128704	1.80259626666457e-05 \\
129874	1.6968113113025e-05 \\
131044	1.59724431100239e-05 \\
132214	1.50352880377014e-05 \\
133384	1.41532002557176e-05 \\
134554	1.33229361463649e-05 \\
135724	1.2541443945191e-05 \\
136894	1.18058523078668e-05 \\
138064	1.1113459568779e-05 \\
139234	1.0461723647992e-05 \\
140404	9.84825256489241e-06 \\
141574	9.27079552387466e-06 \\
142744	8.72723453310131e-06 \\
143914	8.2155765284142e-06 \\
145084	7.73394596603749e-06 \\
146254	7.28057786070213e-06 \\
147424	6.85381123810558e-06 \\
148594	6.45208297972433e-06 \\
149764	6.07392203261226e-06 \\
150934	5.71794396653225e-06 \\
152104	5.38284585355209e-06 \\
153274	5.0674014524521e-06 \\
154444	4.77045668029197e-06 \\
155614	4.49092535154128e-06 \\
156784	4.22778517206179e-06 \\
157954	3.98007396851252e-06 \\
159124	3.74688614301899e-06 \\
160294	3.52736933773024e-06 \\
161464	3.32072129627381e-06 \\
162634	3.12618691117317e-06 \\
163804	2.94305544668028e-06 \\
164974	2.77065792542164e-06 \\
166144	2.60836466886571e-06 \\
167314	2.4555829843953e-06 \\
168484	2.3117549883267e-06 \\
169654	2.1763555573262e-06 \\
170824	2.04889040128498e-06 \\
171994	1.92889424971421e-06 \\
173164	1.81592914444417e-06 \\
174334	1.70958283474132e-06 \\
175504	1.60946726479594e-06 \\
176674	1.51521715235914e-06 \\
177844	1.42648864959183e-06 \\
179014	1.34295808423834e-06 \\
180184	1.26432077357519e-06 \\
181354	1.19028990946957e-06 \\
182524	1.12059550738675e-06 \\
183694	1.05498341967936e-06 \\
184864	9.93214404443332e-07 \\
186034	9.35063250551149e-07 \\
187204	8.80317954476961e-07 \\
188374	8.28778944084174e-07 \\
189544	7.80258349486473e-07 \\
190714	7.34579315875283e-07 \\
191884	6.91575357369967e-07 \\
193054	6.51089748338052e-07 \\
194224	6.1297495046464e-07 \\
195394	5.77092074294239e-07 \\
196564	5.4331037135924e-07 \\
197734	5.11506756728508e-07 \\
198904	4.81565359311542e-07 \\
200074	4.5337709958515e-07 \\
201244	4.26839289524583e-07 \\
202414	4.0185525979064e-07 \\
203584	3.78334005235459e-07 \\
204754	3.56189853722988e-07 \\
205924	3.35342152713025e-07 \\
207094	3.15714975496206e-07 \\
208264	2.97236843915805e-07 \\
209434	2.79840467576342e-07 \\
210604	2.63462498206746e-07 \\
211774	2.48043298900491e-07 \\
212944	2.33526726345357e-07 \\
214114	2.19859925931765e-07 \\
215284	2.06993139739708e-07 \\
216454	1.94879523685021e-07 \\
217624	1.83474978043829e-07 \\
218794	1.7273798613715e-07 \\
219964	1.62629462729935e-07 \\
221134	1.53112611978035e-07 \\
222304	1.44152793035701e-07 \\
223474	1.35717393712209e-07 \\
224644	1.27775712066569e-07 \\
225814	1.2029884371989e-07 \\
226984	1.13259577494418e-07 \\
228154	1.0663229510488e-07 \\
229324	1.00392879065492e-07 \\
230494	9.45186236500639e-08 \\
231664	8.89881530130587e-08 \\
232834	8.37813426968204e-08 \\
234004	7.88792470229893e-08 \\
235174	7.4264029148452e-08 \\
236344	6.99188968944497e-08 \\
237514	6.58280411847123e-08 \\
238684	6.19765785914161e-08 \\
239854	5.83504969342563e-08 \\
241024	5.49366045987654e-08 \\
242194	5.17224821305895e-08 \\
243364	4.86964370494114e-08 \\
244534	4.58474618270088e-08 \\
245704	4.31651927534915e-08 \\
246874	4.06398732999413e-08 \\
248044	3.82623178696306e-08 \\
249214	3.60238786023537e-08 \\
250384	3.39164140106263e-08 \\
251554	3.19322588926418e-08 \\
252724	3.00641969097626e-08 \\
253894	2.83054342187228e-08 \\
255064	2.66495743805883e-08 \\
256234	2.50905957122072e-08 \\
257404	2.36228288041929e-08 \\
258574	2.22409360373099e-08 \\
259744	2.09398923201043e-08 \\
260914	1.97149663816454e-08 \\
262084	1.85617045622699e-08 \\
263254	1.747591288348e-08 \\
264424	1.64536438362894e-08 \\
265594	1.5491180171967e-08 \\
266764	1.45850225230504e-08 \\
267934	1.37318765247585e-08 \\
269104	1.2928639991916e-08 \\
270274	1.21723929824569e-08 \\
270919	1.14603860290607e-08 \\
271039	1.07900304802122e-08 \\
271159	1.01588896184168e-08 \\
271279	9.56466889023844e-09 \\
271399	9.00520835678265e-09 \\
271519	8.47847414497949e-09 \\
271639	7.98255161971184e-09 \\
271759	7.51563811185463e-09 \\
271879	7.07603647898125e-09 \\
271999	6.66214877709237e-09 \\
272119	6.27247082052307e-09 \\
272239	5.90558613122738e-09 \\
272359	5.56016127584158e-09 \\
272479	5.23494075865827e-09 \\
272599	4.92874269175658e-09 \\
272719	4.64045424308779e-09 \\
272839	4.36902769518355e-09 \\
272959	4.11347661488648e-09 \\
273079	3.87287230063649e-09 \\
273199	3.64634045180168e-09 \\
273319	3.43305794903159e-09 \\
273439	3.23224952358814e-09 \\
273559	3.04318553689953e-09 \\
273679	2.86517926051388e-09 \\
273799	2.69758354543015e-09 \\
273919	2.53978960085277e-09 \\
274039	2.39122405210068e-09 \\
274159	2.25134710873931e-09 \\
274279	2.11965051066798e-09 \\
274399	1.99565591829653e-09 \\
274519	1.87891263658813e-09 \\
274639	1.76899656034735e-09 \\
274759	1.66550856439684e-09 \\
274879	1.56807250517588e-09 \\
274999	1.47633438807304e-09 \\
275119	1.38996103515865e-09 \\
275239	1.30863864189479e-09 \\
275359	1.23207183344576e-09 \\
275479	1.15998227689929e-09 \\
275599	1.09210851473307e-09 \\
275719	1.02820357783528e-09 \\
275839	9.68035540616086e-10 \\
275959	9.11385744650772e-10 \\
276079	8.58048576635184e-10 \\
276199	8.07830247140373e-10 \\
276319	7.60548346523393e-10 \\
276439	7.16031178793486e-10 \\
276559	6.74116928944812e-10 \\
276679	6.34653662956453e-10 \\
276799	5.97497606946717e-10 \\
276919	5.62514201885023e-10 \\
277039	5.295763827462e-10 \\
277159	4.98564189932438e-10 \\
277279	4.69365324384796e-10 \\
277399	4.41873815315574e-10 \\
277519	4.15989520607951e-10 \\
277639	3.91618737438648e-10 \\
277759	3.68672814499149e-10 \\
277879	3.47068318529153e-10 \\
277999	3.26727089827727e-10 \\
278119	3.07575021007978e-10 \\
278239	2.89542723130864e-10 \\
278359	2.72564693037936e-10 \\
278479	2.5657914681787e-10 \\
278599	2.41528241851086e-10 \\
278719	2.27357299653619e-10 \\
278839	2.14014639343674e-10 \\
278959	2.01452188264284e-10 \\
279079	1.89624149715684e-10 \\
279199	1.78487502555669e-10 \\
279319	1.68001890177294e-10 \\
279439	1.5812945397542e-10 \\
279559	1.48833945168292e-10 \\
279679	1.40081835020567e-10 \\
279799	1.31841482176043e-10 \\
279919	1.2408268856845e-10 \\
280039	1.16777532088719e-10 \\
280159	1.09899478406561e-10 \\
280279	1.03423380970469e-10 \\
280399	9.73258140746225e-11 \\
280519	9.15847397919833e-11 \\
280639	8.61793414408396e-11 \\
280759	8.10898015402017e-11 \\
280879	7.62977458990122e-11 \\
280999	7.17858550380868e-11 \\
281119	6.75375866343586e-11 \\
281239	6.35378416546928e-11 \\
281359	5.97716875994081e-11 \\
281479	5.6225801792209e-11 \\
281599	5.2887083601405e-11 \\
281719	4.97435981294814e-11 \\
281839	4.67836880346795e-11 \\
281959	4.39969727317191e-11 \\
282079	4.13730161241688e-11 \\
282199	3.89024923386216e-11 \\
282319	3.65762975462758e-11 \\
282439	3.43861050744465e-11 \\
282559	3.23238658062053e-11 \\
282679	3.03821967584383e-11 \\
282799	2.85540480149393e-11 \\
282919	2.68327027264093e-11 \\
283039	2.52119436439102e-11 \\
283159	2.36859976077142e-11 \\
283279	2.22490359469418e-11 \\
283399	2.08963402137385e-11 \\
283519	1.96224148041324e-11 \\
283639	1.84230408706298e-11 \\
283759	1.72938330322836e-11 \\
283879	1.62305169304489e-11 \\
283999	1.52294288291444e-11 \\
284119	1.42867939700864e-11 \\
284239	1.3399281684201e-11 \\
284359	1.25635613024144e-11 \\
284479	1.17767462448626e-11 \\
284599	1.10358944205302e-11 \\
284719	1.03383412941582e-11 \\
284839	9.68153335278998e-12 \\
284959	9.06313912807377e-12 \\
285079	8.48082715165788e-12 \\
285199	7.93254351094674e-12 \\
285319	7.41628980449605e-12 \\
285439	6.93028967546638e-12 \\
285559	6.47260023356466e-12 \\
285679	6.04177818885887e-12 \\
285799	5.63599167335838e-12 \\
285919	5.25396393058486e-12 \\
286039	4.89425167060631e-12 \\
286159	4.55563364809564e-12 \\
286279	4.23666657312083e-12 \\
286399	3.93646226726219e-12 \\
286519	3.65368846289016e-12 \\
286639	3.38756800388751e-12 \\
286759	3.13682413377592e-12 \\
286879	2.90084622989184e-12 \\
286999	2.67857958036188e-12 \\
287119	2.46935805137127e-12 \\
287239	2.27240448680277e-12 \\
287359	2.08699724169037e-12 \\
287479	1.91224813761437e-12 \\
287599	1.74782410766738e-12 \\
287719	1.59294799573217e-12 \\
287839	1.44712020144766e-12 \\
287959	1.30995214675522e-12 \\
288079	1.18061116438639e-12 \\
288199	1.05887520973624e-12 \\
288319	9.44355704746158e-13 \\
288439	8.36386515601362e-13 \\
288559	7.3480110884816e-13 \\
288679	6.39155395276703e-13 \\
288799	5.49060796828371e-13 \\
288919	4.6429526889824e-13 \\
289039	3.8441472227646e-13 \\
289159	3.09197112358106e-13 \\
289279	2.38420394538252e-13 \\
289399	1.71751501909512e-13 \\
289519	1.08968389866959e-13 \\
289639	4.99045249569008e-14 \\
289759	-5.6621374255883e-15 \\
289879	-5.81756864903582e-14 \\
289999	-1.07469588783715e-13 \\
290119	-1.5404344466674e-13 \\
290239	-1.9773072068574e-13 \\
290359	-2.38975506050565e-13 \\
290479	-2.7761126730752e-13 \\
290599	-3.14137604817688e-13 \\
290719	-3.48610029732299e-13 \\
290839	-3.80917519748891e-13 \\
290959	-4.11504164077314e-13 \\
291079	-4.40147918112643e-13 \\
};
\addplot [line width=0.01pt, red, forget plot]
table [row sep=\\]{%
1171	2.37127690262287 \\
2341	1.91527733166002 \\
3511	1.55547088874662 \\
4681	1.25675401692562 \\
5851	1.00227735831463 \\
7021	0.800255015092139 \\
8191	0.631971690664359 \\
9361	0.50316249842948 \\
10531	0.398682551180777 \\
11701	0.317996069246329 \\
12871	0.252014407274243 \\
14041	0.203831726446543 \\
15211	0.166288526113563 \\
16381	0.141980042689638 \\
17551	0.126903389155369 \\
18721	0.113264191572948 \\
19891	0.101165687278717 \\
21061	0.0909790129602416 \\
22231	0.0819505227172377 \\
23401	0.0736272056922063 \\
24571	0.0659228530620488 \\
25741	0.058765253950014 \\
26911	0.0525948239286785 \\
28081	0.047086502786212 \\
29251	0.042318236527151 \\
30421	0.0380557698061912 \\
31591	0.034301524258216 \\
32761	0.0308179990996076 \\
33931	0.027552406678552 \\
35101	0.0244895964598092 \\
36271	0.0219152567460203 \\
37441	0.019849089363673 \\
38611	0.0179542670570619 \\
39781	0.0161806091562872 \\
40951	0.01454313637576 \\
42121	0.0132326198978723 \\
43291	0.0122415542368687 \\
44461	0.0113235149165253 \\
45631	0.0104717760637042 \\
46801	0.00968051090320393 \\
47971	0.00894459527756702 \\
49141	0.00825947937862476 \\
50311	0.00762109907196445 \\
51481	0.00702580500830474 \\
52651	0.00647030524662356 \\
53821	0.00595695355625853 \\
54991	0.0054819619174834 \\
56161	0.00503749456597669 \\
57331	0.00462142722073461 \\
58501	0.00423539442052795 \\
59671	0.00387691195385903 \\
60841	0.00354420442711556 \\
62011	0.00323233906596271 \\
63181	0.00294082192616812 \\
64351	0.00266793224463041 \\
65521	0.0024729147865038 \\
66691	0.00229538115709377 \\
67861	0.00212929536798862 \\
69031	0.00197387173479391 \\
70201	0.00182875013143269 \\
71371	0.0016969546900848 \\
72541	0.00157518558561448 \\
73711	0.00146131874914784 \\
74881	0.00135481066019566 \\
76051	0.00125516025909783 \\
77221	0.00116190456240578 \\
78391	0.00107461500818334 \\
79561	0.000992894310843428 \\
80731	0.00091637372343284 \\
81901	0.000844710631655698 \\
83071	0.000777586421477183 \\
84241	0.00071470457519307 \\
85411	0.000655788960607195 \\
86581	0.00060058228531179 \\
87751	0.000548844693632422 \\
88921	0.000501197356830019 \\
90091	0.000464110009131269 \\
91261	0.000429789432501104 \\
92431	0.000397906670241088 \\
93601	0.000368272507615341 \\
94771	0.000340718126899564 \\
95941	0.000315088513228146 \\
97111	0.000291241027891187 \\
98281	0.000269044325455547 \\
99451	0.00024837738581035 \\
100621	0.000229128639255505 \\
101791	0.000211195173665246 \\
102961	0.000194482008354924 \\
104131	0.000178901213794047 \\
105301	0.000164371340099612 \\
106471	0.000150817214876586 \\
107641	0.000139091187550344 \\
108811	0.000130291988111264 \\
109981	0.000122096599481536 \\
111151	0.000114427485215396 \\
112321	0.000107246200543754 \\
113491	0.00010051925558513 \\
114661	9.42158371234347e-05 \\
115831	8.83074573596043e-05 \\
117001	8.27677426034623e-05 \\
118171	7.75722514778088e-05 \\
119341	7.26983128723213e-05 \\
120511	6.82012552040145e-05 \\
121681	6.40291690157668e-05 \\
122851	6.01208343907755e-05 \\
124021	5.64585244414162e-05 \\
125191	5.30258439616071e-05 \\
126361	4.98076028345729e-05 \\
127531	4.67897099279968e-05 \\
128701	4.39590797917111e-05 \\
129871	4.13035497711656e-05 \\
131041	3.88118058788356e-05 \\
132211	3.64733161259401e-05 \\
133381	3.42782702735245e-05 \\
134551	3.22175251654677e-05 \\
135721	3.02825549503005e-05 \\
136891	2.84654056248446e-05 \\
138061	2.6758653418113e-05 \\
139231	2.51553666110182e-05 \\
140401	2.36490704448311e-05 \\
141571	2.22337148159113e-05 \\
142741	2.09036444980826e-05 \\
143911	1.96535716585067e-05 \\
145081	1.84785504676599e-05 \\
146251	1.73739536176165e-05 \\
147421	1.63468678621714e-05 \\
148591	1.53816492950987e-05 \\
149761	1.44741899145151e-05 \\
150931	1.36209310965252e-05 \\
152101	1.28185460362418e-05 \\
153271	1.2063921773553e-05 \\
154441	1.13541442403053e-05 \\
155611	1.0686752971889e-05 \\
156781	1.00595601004727e-05 \\
157951	9.46962935205464e-06 \\
159121	8.91470193054023e-06 \\
160291	8.39266080016055e-06 \\
161461	7.90152108992581e-06 \\
162631	7.43942137787501e-06 \\
163801	7.00461561581323e-06 \\
164971	6.59546563297475e-06 \\
166141	6.21043416654254e-06 \\
167311	5.8480783771131e-06 \\
168481	5.5070438126914e-06 \\
169651	5.18605878291334e-06 \\
170821	4.88392911496272e-06 \\
171991	4.59953326198415e-06 \\
173161	4.33181773668023e-06 \\
174331	4.07979284816617e-06 \\
175501	3.84252871954427e-06 \\
176671	3.6191515657702e-06 \\
177841	3.40884021404753e-06 \\
179011	3.21082284981955e-06 \\
180181	3.02437397259325e-06 \\
181351	2.84881154688499e-06 \\
182521	2.68349433513171e-06 \\
183691	2.52781940152103e-06 \\
184861	2.38121977241823e-06 \\
186031	2.24316224617382e-06 \\
187201	2.11314533943296e-06 \\
188371	1.99069736345203e-06 \\
189541	1.87537462015275e-06 \\
190711	1.76675971058637e-06 \\
191881	1.6644599483695e-06 \\
193051	1.56810587248479e-06 \\
194221	1.4773498505094e-06 \\
195391	1.39186476910691e-06 \\
196561	1.31134280467737e-06 \\
197731	1.23549426861436e-06 \\
198901	1.16404652500401e-06 \\
200071	1.09674297243956e-06 \\
201241	1.03334208922945e-06 \\
202411	9.73616535837607e-07 \\
203581	9.17352312834741e-07 \\
204751	8.64347967866053e-07 \\
205921	8.14413852856521e-07 \\
207091	7.67371424570396e-07 \\
208261	7.23052587525697e-07 \\
209431	6.81299076876751e-07 \\
210601	6.41961877545505e-07 \\
211771	6.04900678990994e-07 \\
212941	5.69983362452842e-07 \\
214111	5.3708551783771e-07 \\
215281	5.06089991081371e-07 \\
216451	4.76886457045911e-07 \\
217621	4.49371018784728e-07 \\
218791	4.23445830066704e-07 \\
219961	3.99018740382395e-07 \\
221131	3.76002961488542e-07 \\
222301	3.54316752881889e-07 \\
223471	3.33883126757417e-07 \\
224641	3.14629569675429e-07 \\
225811	2.96487781548116e-07 \\
226981	2.79393428392893e-07 \\
228151	2.63285911905609e-07 \\
229321	2.48108150691095e-07 \\
230491	2.33806375316004e-07 \\
231661	2.20329935463059e-07 \\
232831	2.07631117743468e-07 \\
234001	1.95664974556031e-07 \\
235171	1.84389163770948e-07 \\
236341	1.73763796129656e-07 \\
237511	1.63751293358327e-07 \\
238681	1.54316253220266e-07 \\
239851	1.45425323450077e-07 \\
241021	1.37047082016117e-07 \\
242191	1.2915192598717e-07 \\
243361	1.21711964895521e-07 \\
244531	1.14700921594046e-07 \\
245701	1.08094039441564e-07 \\
246871	1.01867992430282e-07 \\
248041	9.60008044170735e-08 \\
249211	9.04717683547496e-08 \\
250381	8.52613750157438e-08 \\
251551	8.03512421043706e-08 \\
252721	7.57240496418454e-08 \\
253891	7.13634780713512e-08 \\
255061	6.72541507484858e-08 \\
256231	6.33815787076664e-08 \\
257401	5.97321099804482e-08 \\
258571	5.62928808567342e-08 \\
259741	5.3051770088075e-08 \\
260911	4.9997356310616e-08 \\
262081	4.71188770778674e-08 \\
263251	4.44061916127225e-08 \\
264421	4.18497436704968e-08 \\
265591	3.94405287873489e-08 \\
266761	3.71700618062576e-08 \\
267931	3.50303474561109e-08 \\
269101	3.30138510973299e-08 \\
270271	3.11134729646945e-08 \\
270811	2.93225226322136e-08 \\
270931	2.76346956984419e-08 \\
271051	2.60440511934412e-08 \\
271171	2.45449904290318e-08 \\
271291	2.31322378474452e-08 \\
271411	2.18008220920218e-08 \\
271531	2.05460576885308e-08 \\
271651	1.93635301126704e-08 \\
271771	1.82490787481449e-08 \\
271891	1.71987829533649e-08 \\
272011	1.62089479616156e-08 \\
272131	1.52760919469586e-08 \\
272251	1.43969336452443e-08 \\
272371	1.35683811408605e-08 \\
272491	1.27875202093897e-08 \\
272611	1.2051604880714e-08 \\
272731	1.13580472804742e-08 \\
272851	1.0704408581752e-08 \\
272971	1.00883904563531e-08 \\
273091	9.50782708120101e-09 \\
273211	8.96067697819802e-09 \\
273331	8.44501679697629e-09 \\
273451	7.95903376538121e-09 \\
273571	7.50101919466672e-09 \\
273691	7.06936326144714e-09 \\
273811	6.66254856840354e-09 \\
273931	6.27914531481366e-09 \\
274051	5.91780574543677e-09 \\
274171	5.57726004268844e-09 \\
274291	5.25631116410352e-09 \\
274411	4.95383117860015e-09 \\
274531	4.66875721416571e-09 \\
274651	4.40008735003161e-09 \\
274771	4.14687734151542e-09 \\
274891	3.9082375669075e-09 \\
275011	3.68332958577966e-09 \\
275131	3.47136275280491e-09 \\
275251	3.27159271895638e-09 \\
275371	3.08331726817102e-09 \\
275491	2.90587498508188e-09 \\
275611	2.73864247946065e-09 \\
275731	2.5810321102604e-09 \\
275851	2.43249032028103e-09 \\
275971	2.29249508265639e-09 \\
276091	2.1605546240977e-09 \\
276211	2.03620553751449e-09 \\
276331	1.91901089463542e-09 \\
276451	1.80855908027411e-09 \\
276571	1.70446207148345e-09 \\
276691	1.60635410528798e-09 \\
276811	1.51389067948315e-09 \\
276931	1.42674699832313e-09 \\
277051	1.3446168622977e-09 \\
277171	1.26721194648738e-09 \\
277291	1.19426019073998e-09 \\
277411	1.12550563313718e-09 \\
277531	1.06070646710421e-09 \\
277651	9.99635374476782e-10 \\
277771	9.42077749144232e-10 \\
277891	8.87831419493779e-10 \\
278011	8.36705760232093e-10 \\
278131	7.88521525851849e-10 \\
278251	7.43109240808337e-10 \\
278371	7.00309421564071e-10 \\
278491	6.59971965966122e-10 \\
278611	6.21955043023092e-10 \\
278731	5.86125092905121e-10 \\
278851	5.52356327343517e-10 \\
278971	5.20530341052705e-10 \\
279091	4.905351125295e-10 \\
279211	4.6226550365347e-10 \\
279331	4.3562214946391e-10 \\
279451	4.10511458159846e-10 \\
279571	3.86845333544272e-10 \\
279691	3.64540675423797e-10 \\
279811	3.4351910205288e-10 \\
279931	3.23706839111537e-10 \\
280051	3.0503416459382e-10 \\
280171	2.87435741874731e-10 \\
280291	2.70849620509495e-10 \\
280411	2.55217791345075e-10 \\
280531	2.40484909763694e-10 \\
280651	2.26599738972766e-10 \\
280771	2.13513262625753e-10 \\
280891	2.01179628511738e-10 \\
281011	1.89555315888157e-10 \\
281131	1.78599857125761e-10 \\
281251	1.68274449929839e-10 \\
281371	1.58543123074395e-10 \\
281491	1.49371515156815e-10 \\
281611	1.40727485220538e-10 \\
281731	1.32580835199292e-10 \\
281851	1.24902588272136e-10 \\
281971	1.17666154597629e-10 \\
282091	1.10845999046205e-10 \\
282211	1.04417974267079e-10 \\
282331	9.83598757997584e-11 \\
282451	9.26502208287161e-11 \\
282571	8.7268969828358e-11 \\
282691	8.21972490072653e-11 \\
282811	7.74173503081954e-11 \\
282931	7.29123428300227e-11 \\
283051	6.8666350383495e-11 \\
283171	6.46648290469898e-11 \\
283291	6.08934014323381e-11 \\
283411	5.73387448632445e-11 \\
283531	5.39888134198918e-11 \\
283651	5.08313946490091e-11 \\
283771	4.78556638761063e-11 \\
283891	4.50510739824495e-11 \\
284011	4.24079105165731e-11 \\
284131	3.99166810716167e-11 \\
284251	3.75687259079882e-11 \\
284371	3.53558848864566e-11 \\
284491	3.3270275423547e-11 \\
284611	3.13047365807506e-11 \\
284731	2.94521629307098e-11 \\
284851	2.77061706910331e-11 \\
284971	2.60605981239337e-11 \\
285091	2.45096720696836e-11 \\
285211	2.30479524354621e-11 \\
285331	2.16704432176584e-11 \\
285451	2.03719263680568e-11 \\
285571	1.91482940614662e-11 \\
285691	1.79949943834856e-11 \\
285811	1.69080305312264e-11 \\
285931	1.5883461212951e-11 \\
286051	1.49180667818882e-11 \\
286171	1.40080724797542e-11 \\
286291	1.31503696820801e-11 \\
286411	1.23420718090017e-11 \\
286531	1.15802367695039e-11 \\
286651	1.08621445171764e-11 \\
286771	1.01855190948186e-11 \\
286891	9.54775147832265e-12 \\
287011	8.94662122163936e-12 \\
287131	8.38012992332438e-12 \\
287251	7.84627918193337e-12 \\
287371	7.34290406256832e-12 \\
287491	6.86867229759969e-12 \\
287611	6.4216965078856e-12 \\
287731	6.00036687004035e-12 \\
287851	5.60340662758563e-12 \\
287971	5.22915044598449e-12 \\
288091	4.87648810221231e-12 \\
288211	4.5440318174883e-12 \\
288331	4.23067136878785e-12 \\
288451	3.93546306654002e-12 \\
288571	3.65707464311527e-12 \\
288691	3.39495098700127e-12 \\
288811	3.14764880826601e-12 \\
288931	2.91472401769965e-12 \\
289051	2.69501088112634e-12 \\
289171	2.4882318427899e-12 \\
289291	2.29322116851449e-12 \\
289411	2.10931272448533e-12 \\
289531	1.93606242149258e-12 \\
289651	1.77285963687268e-12 \\
289771	1.6189272145084e-12 \\
289891	1.4738210651899e-12 \\
290011	1.33720812200977e-12 \\
290131	1.20831122885079e-12 \\
290251	1.08690834110803e-12 \\
290371	9.72388836117943e-13 \\
290491	8.64586180426841e-13 \\
290611	7.63000773673639e-13 \\
290731	6.67021993194794e-13 \\
290851	5.76816372444e-13 \\
290971	4.91717777606482e-13 \\
};
\addplot [line width=0.01pt, red, forget plot]
table [row sep=\\]{%
1181	2.70378329558139 \\
2361	2.27181564784682 \\
3541	1.90004673255514 \\
4721	1.58185519399633 \\
5901	1.31167182879842 \\
7081	1.07340916634479 \\
8261	0.87408632555493 \\
9441	0.714877263671169 \\
10621	0.584608740524067 \\
11801	0.481842203317448 \\
12981	0.396314374760245 \\
14161	0.324619894291651 \\
15341	0.264220513039252 \\
16521	0.217558479335384 \\
17701	0.182251011389673 \\
18881	0.155112020270469 \\
20061	0.13248203982072 \\
21241	0.113509298645696 \\
22421	0.0972670540490144 \\
23601	0.0835664863953139 \\
24781	0.0712947665048114 \\
25961	0.0611356089769976 \\
27141	0.0525172958065042 \\
28321	0.0446908969271422 \\
29501	0.037779288952123 \\
30681	0.0314215169648445 \\
31861	0.0258297406320409 \\
33041	0.0213451862745568 \\
34221	0.0195264637678843 \\
35401	0.0178479054057457 \\
36581	0.0162919488443676 \\
37761	0.0148471490155105 \\
38941	0.0135038050210891 \\
40121	0.0123041053806816 \\
41301	0.0111909788706311 \\
42481	0.0101536224132565 \\
43661	0.00918606668118516 \\
44841	0.00838004040979007 \\
46021	0.00765618696081427 \\
47201	0.00698716487953083 \\
48381	0.00636645749214776 \\
49561	0.00578996556296363 \\
50741	0.00525919948742365 \\
51921	0.00477381117416387 \\
53101	0.00432063803052007 \\
54281	0.00389736895308807 \\
55461	0.00350463765798198 \\
56641	0.00316000996339988 \\
57821	0.00284081410912523 \\
59001	0.00254435719087498 \\
60181	0.00227347500085545 \\
61361	0.00202523496002616 \\
62541	0.00179401384701755 \\
63721	0.00157839706547214 \\
64901	0.00137715242366199 \\
66081	0.00118917794764184 \\
67261	0.0010993353645582 \\
68441	0.00102647285335505 \\
69621	0.000958652720291997 \\
70801	0.000895466014277757 \\
71981	0.000836549303327605 \\
73161	0.000781574643602456 \\
74341	0.000730244919429412 \\
75521	0.00068229009949683 \\
76701	0.000637464081022998 \\
77881	0.00059747895090223 \\
79061	0.000561346563101084 \\
80241	0.000527461929360096 \\
81421	0.00049566257793926 \\
82601	0.000465813276013805 \\
83781	0.000437789553533408 \\
84961	0.000411475822787721 \\
86141	0.00038676450887698 \\
87321	0.000363555347361599 \\
88501	0.000341754776981895 \\
89681	0.000321275406747978 \\
90861	0.000302035543611467 \\
92041	0.000283958770228965 \\
93221	0.00026697356467642 \\
94401	0.000251012955744756 \\
95581	0.000236014208780888 \\
96761	0.000221918538045429 \\
97941	0.000208670842334058 \\
99121	0.000196219461200853 \\
100301	0.000184515949574304 \\
101481	0.000173514868920643 \\
102661	0.000163173593381138 \\
103841	0.00015345212953527 \\
105021	0.000144312948621328 \\
106201	0.000135720830184649 \\
107381	0.000127642716249543 \\
108561	0.000120047575200011 \\
109741	0.000112906274645608 \\
110921	0.000106191462614313 \\
112101	9.98774564724414e-05 \\
113281	9.39401390291361e-05 \\
114461	8.83568613272434e-05 \\
115011	8.31063516575914e-05 \\
115141	7.81686303797935e-05 \\
115271	7.35249301530017e-05 \\
115401	6.91576212220602e-05 \\
115531	6.50501414202753e-05 \\
115661	6.11869305786605e-05 \\
115791	5.75533690571617e-05 \\
115921	5.41357201233605e-05 \\
116051	5.09210759365164e-05 \\
116181	4.78973068980815e-05 \\
116311	4.50530141561933e-05 \\
116441	4.23774850609182e-05 \\
116571	3.98606513818955e-05 \\
116701	3.74930501123028e-05 \\
116831	3.52657866959927e-05 \\
116961	3.31705005225946e-05 \\
117091	3.11993325505244e-05 \\
117221	2.93448949196251e-05 \\
117351	2.76138037437201e-05 \\
117481	2.59910190369972e-05 \\
117611	2.44640297887844e-05 \\
117741	2.3027120128527e-05 \\
117871	2.16749294550955e-05 \\
118001	2.040242303053e-05 \\
118131	1.92048707935433e-05 \\
118261	1.80778280741145e-05 \\
118391	1.70171176872569e-05 \\
118521	1.6018813265628e-05 \\
118651	1.50792237210706e-05 \\
118781	1.41948787437673e-05 \\
118911	1.33625152576866e-05 \\
119041	1.25790647572122e-05 \\
119171	1.18416414603417e-05 \\
119301	1.11475312190579e-05 \\
119431	1.04941811316928e-05 \\
119561	9.87918981026903e-06 \\
119691	9.30029825552126e-06 \\
119821	8.75538129946341e-06 \\
119951	8.24243957875392e-06 \\
120081	7.75959200094434e-06 \\
120211	7.3050686744125e-06 \\
120341	6.87720426967298e-06 \\
120471	6.4744317859744e-06 \\
120601	6.0952766964828e-06 \\
120731	5.73835144757062e-06 \\
120861	5.40235029211633e-06 \\
120991	5.08604443233418e-06 \\
121121	4.78827745759025e-06 \\
121251	4.50796105389006e-06 \\
121381	4.24407097171509e-06 \\
121511	3.9956432361099e-06 \\
121641	3.76177058108995e-06 \\
121771	3.54159909948804e-06 \\
121901	3.33432509103115e-06 \\
122031	3.13919209810054e-06 \\
122161	2.95548811901636e-06 \\
122291	2.78254298563541e-06 \\
122421	2.61972589726822e-06 \\
122551	2.4664431009791e-06 \\
122681	2.32213570960926e-06 \\
122811	2.1862776479753e-06 \\
122941	2.05837372174722e-06 \\
123071	1.93795780056849e-06 \\
123201	1.82459110736888e-06 \\
123331	1.717860610706e-06 \\
123461	1.61737750942192e-06 \\
123591	1.52277580989235e-06 \\
123721	1.43371098437761e-06 \\
123851	1.34985871025339e-06 \\
123981	1.27091368290477e-06 \\
124111	1.1965884985643e-06 \\
124241	1.12661260309732e-06 \\
124371	1.06073130273776e-06 \\
124501	9.98704832722019e-07 \\
124631	9.40307481489544e-07 \\
124761	8.85326765065475e-07 \\
124891	8.33562651791908e-07 \\
125021	7.8482683113501e-07 \\
125151	7.38942025846345e-07 \\
125281	6.95741345368983e-07 \\
125411	6.55067676158527e-07 \\
125541	6.16773108474966e-07 \\
125671	5.80718396481217e-07 \\
125801	5.46772450538136e-07 \\
125931	5.14811858476349e-07 \\
126061	4.84720436066954e-07 \\
126191	4.56388803360408e-07 \\
126321	4.29713985006241e-07 \\
126451	4.04599035663811e-07 \\
126581	3.8095268672933e-07 \\
126711	3.58689012935898e-07 \\
126841	3.3772711960367e-07 \\
126971	3.17990848153205e-07 \\
127101	2.99408498272147e-07 \\
127231	2.81912567068332e-07 \\
127361	2.6543950243374e-07 \\
127491	2.49929472562194e-07 \\
127621	2.35326147457471e-07 \\
127751	2.21576493819597e-07 \\
127881	2.08630582365643e-07 \\
128011	1.96441405586611e-07 \\
128141	1.84964706939628e-07 \\
128271	1.74158819532533e-07 \\
128401	1.63984514800486e-07 \\
128531	1.54404859398216e-07 \\
128661	1.45385080807525e-07 \\
128791	1.36892441160441e-07 \\
128921	1.28896118112287e-07 \\
129051	1.21367092154046e-07 \\
129181	1.14278041585258e-07 \\
129311	1.07603243093557e-07 \\
129441	1.01318477940815e-07 \\
129571	9.54009443110415e-08 \\
129701	8.9829173655076e-08 \\
129831	8.45829536966214e-08 \\
129961	7.96432536587233e-08 \\
130091	7.49921565401657e-08 \\
130221	7.0612792724134e-08 \\
130351	6.64892791935046e-08 \\
130481	6.26066618547583e-08 \\
130611	5.89508611370526e-08 \\
130741	5.5508620755429e-08 \\
130871	5.22674595271333e-08 \\
131001	4.92156259079835e-08 \\
131131	4.63420559149164e-08 \\
131261	4.36363321032474e-08 \\
131391	4.10886462076654e-08 \\
131521	3.86897632265182e-08 \\
131651	3.64309882816549e-08 \\
131781	3.43041348105366e-08 \\
131911	3.23014946457256e-08 \\
132041	3.04158103148211e-08 \\
132171	2.86402484506176e-08 \\
132301	2.69683754217098e-08 \\
132431	2.53941332961638e-08 \\
132561	2.39118185807463e-08 \\
132691	2.25160607936203e-08 \\
132821	2.12018035905537e-08 \\
132951	1.99642857801052e-08 \\
133081	1.87990250033465e-08 \\
133211	1.77018004143825e-08 \\
133341	1.6668637747852e-08 \\
133471	1.56957954966508e-08 \\
133601	1.47797503680103e-08 \\
133731	1.39171849600217e-08 \\
133861	1.31049757157164e-08 \\
133991	1.23401819873692e-08 \\
134121	1.16200343791562e-08 \\
134251	1.09419256988375e-08 \\
134381	1.03034007992164e-08 \\
134511	9.70214814044468e-09 \\
134641	9.13599129681586e-09 \\
134771	8.60288085213767e-09 \\
134901	8.10088696123756e-09 \\
135031	7.62819302169149e-09 \\
135161	7.18308795777389e-09 \\
135291	6.7639611134318e-09 \\
135421	6.36929553543553e-09 \\
135551	5.99766319941963e-09 \\
135681	5.64771923672325e-09 \\
135811	5.31819743798678e-09 \\
135941	5.00790570123755e-09 \\
136071	4.71572120241959e-09 \\
136201	4.44058695370231e-09 \\
136331	4.18150786218874e-09 \\
136461	3.93754623351228e-09 \\
136591	3.70781999547987e-09 \\
136721	3.49149797962411e-09 \\
136851	3.28779797831302e-09 \\
136981	3.095983081014e-09 \\
137111	2.91535973140356e-09 \\
137241	2.74527450772055e-09 \\
137371	2.5851126239651e-09 \\
137501	2.43429482127411e-09 \\
137631	2.29227570258672e-09 \\
137761	2.15854178975405e-09 \\
137891	2.03260946962658e-09 \\
138021	1.9140236617865e-09 \\
138151	1.80235543156826e-09 \\
138281	1.6972013239247e-09 \\
138411	1.59818103195875e-09 \\
138541	1.50493673078955e-09 \\
138671	1.4171315232403e-09 \\
138801	1.33444766348134e-09 \\
138931	1.25658661254135e-09 \\
139061	1.18326687337245e-09 \\
139191	1.11422349124979e-09 \\
139321	1.0492069435486e-09 \\
139451	9.87982529121467e-10 \\
139581	9.30328591941532e-10 \\
139711	8.76037076213976e-10 \\
139841	8.249118610415e-10 \\
139971	7.76768149801654e-10 \\
140101	7.3143191503533e-10 \\
140231	6.88739454357545e-10 \\
140361	6.48536613301331e-10 \\
140491	6.10678230206219e-10 \\
140621	5.75027303550968e-10 \\
140751	5.41455325020479e-10 \\
140881	5.09840836215858e-10 \\
141011	4.80069761721325e-10 \\
141141	4.52034520925793e-10 \\
141271	4.25633861489416e-10 \\
141401	4.00772637298985e-10 \\
141531	3.77360753756051e-10 \\
141661	3.55314000444196e-10 \\
141791	3.34552441305647e-10 \\
141921	3.15001358330846e-10 \\
142051	2.96590141335429e-10 \\
142181	2.79252232449068e-10 \\
142311	2.62925070604325e-10 \\
142441	2.47549702958594e-10 \\
142571	2.33070618360642e-10 \\
142701	2.19435636328313e-10 \\
142831	2.06595351937011e-10 \\
142961	1.94503690931214e-10 \\
143091	1.83116743990297e-10 \\
143221	1.72393599395804e-10 \\
143351	1.62295454853023e-10 \\
143481	1.52785950557899e-10 \\
143611	1.43830725107819e-10 \\
143741	1.35397304479312e-10 \\
143871	1.27455601628412e-10 \\
144001	1.19976639734176e-10 \\
144131	1.1293366242171e-10 \\
144261	1.06301134561448e-10 \\
144391	1.00055075336059e-10 \\
144521	9.41730582404432e-11 \\
144651	8.86339335259834e-11 \\
144781	8.34175506447821e-11 \\
144911	7.85050913165719e-11 \\
145041	7.38789585064126e-11 \\
145171	6.95224433577835e-11 \\
145301	6.54196696814324e-11 \\
145431	6.15560380445856e-11 \\
145561	5.79175041259816e-11 \\
145691	5.44911338273835e-11 \\
145821	5.12643261174617e-11 \\
145951	4.82254236544577e-11 \\
146081	4.53637127861839e-11 \\
146211	4.26687019050576e-11 \\
146341	4.01307875819157e-11 \\
146471	3.77406439433514e-11 \\
146601	3.54898888055288e-11 \\
146731	3.33700844734608e-11 \\
146861	3.13739034751848e-11 \\
146991	2.94939073164358e-11 \\
147121	2.77236011925197e-11 \\
147251	2.60562682541376e-11 \\
147381	2.44861908527128e-11 \\
147511	2.30075403173657e-11 \\
147641	2.16150430887296e-11 \\
147771	2.03035921408912e-11 \\
147901	1.90685800482981e-11 \\
148031	1.79055659188521e-11 \\
148161	1.68101643716057e-11 \\
148291	1.57786006482752e-11 \\
148421	1.48072665240306e-11 \\
148551	1.38924427517395e-11 \\
148681	1.30308541734792e-11 \\
148811	1.22193921647806e-11 \\
148941	1.14553366792336e-11 \\
149071	1.0735801136974e-11 \\
149201	1.00580099804404e-11 \\
149331	9.41974276358337e-12 \\
149461	8.81877904035377e-12 \\
149591	8.25267632009741e-12 \\
149721	7.71965824597487e-12 \\
149851	7.21750437193691e-12 \\
149981	6.74477140805152e-12 \\
150111	6.29957197517683e-12 \\
150241	5.88012971647345e-12 \\
150371	5.48522338661428e-12 \\
150501	5.11329867336485e-12 \\
150631	4.76302330909562e-12 \\
150761	4.43317604847948e-12 \\
150891	4.12253564618936e-12 \\
151021	3.8299363680494e-12 \\
151151	3.55443452448867e-12 \\
151281	3.29497540363377e-12 \\
151411	3.05061531591377e-12 \\
151541	2.82046608290898e-12 \\
151671	2.60375054850215e-12 \\
151801	2.39958053427358e-12 \\
151931	2.20723439525727e-12 \\
152061	2.02626804224337e-12 \\
152191	1.85573778566095e-12 \\
152321	1.69525504745138e-12 \\
152451	1.54393164919497e-12 \\
152581	1.4015455462868e-12 \\
152711	1.26748611606331e-12 \\
152841	1.14114273586097e-12 \\
152971	1.02218233877238e-12 \\
153101	9.10049813285241e-13 \\
153231	8.04634137097082e-13 \\
153361	7.05158154090668e-13 \\
153491	6.11621864265999e-13 \\
153621	5.2341464495953e-13 \\
153751	4.404254738688e-13 \\
153881	3.62210261783957e-13 \\
154011	2.88657986402541e-13 \\
154141	2.193245585147e-13 \\
154271	1.53987933515509e-13 \\
};
\addplot [line width=0.01pt, red, forget plot]
table [row sep=\\]{%
1185	2.29761828882597 \\
2365	1.86826345544712 \\
3545	1.55232578637427 \\
4725	1.30383839339407 \\
5905	1.09469408137257 \\
7085	0.916462887648499 \\
8265	0.765639866219667 \\
9445	0.638901228498857 \\
10625	0.5350402169963 \\
11805	0.450410102232462 \\
12985	0.37746344706485 \\
14165	0.315550749264919 \\
15345	0.260067457752341 \\
16525	0.215891451019263 \\
17705	0.178573539758913 \\
18885	0.149154586100371 \\
20065	0.124368382315805 \\
21245	0.103738745558412 \\
22425	0.0878256604498436 \\
23605	0.0741865822759915 \\
24785	0.062518407127198 \\
25965	0.0528584516026895 \\
27145	0.0440093471512455 \\
28325	0.0364006410348819 \\
29505	0.0302501932467751 \\
30685	0.0254671515737391 \\
31865	0.0215222484687583 \\
33045	0.0181453237911945 \\
34225	0.0150847701927692 \\
35405	0.0124536776637541 \\
36585	0.0103935043812408 \\
37765	0.00887363526330148 \\
38945	0.00791670198803202 \\
40125	0.00710559335058264 \\
41305	0.00648882355423747 \\
42485	0.00593136874545863 \\
43665	0.00541879616115898 \\
44845	0.00494693299023691 \\
46025	0.00452472325921444 \\
47205	0.00414225270091545 \\
48385	0.00379519653772314 \\
49565	0.0034789412582551 \\
50745	0.00318906237663635 \\
51925	0.00293812943380994 \\
53105	0.00270787160822572 \\
54285	0.00249617054505757 \\
55465	0.00230134155773587 \\
56645	0.0021221493970594 \\
57825	0.00195861642324924 \\
59005	0.00180792357523374 \\
60185	0.0016689322539768 \\
61365	0.00154064188294728 \\
62545	0.0014221482726921 \\
63725	0.00131263186171571 \\
64905	0.00121134879866214 \\
66085	0.00111762315861175 \\
67265	0.00103084006453952 \\
68445	0.000950439589521568 \\
69625	0.000877610640415527 \\
70805	0.000811622449338634 \\
71985	0.000750379428407666 \\
73165	0.000693492212004432 \\
74345	0.000640611311091999 \\
75525	0.000591420929017261 \\
76705	0.00054563475285746 \\
77885	0.000502992479620379 \\
79065	0.000463256879171703 \\
80245	0.000426211286392142 \\
81425	0.000391657442398885 \\
82605	0.00035941362272629 \\
83785	0.000329313003775256 \\
84965	0.00030120222896407 \\
86145	0.000274940143720259 \\
87325	0.00025039667436022 \\
88505	0.000227451830490888 \\
89685	0.000205994814146515 \\
90865	0.000185923221700246 \\
92045	0.000167142326843062 \\
93225	0.000149564434739291 \\
94405	0.000133108298936147 \\
95585	0.000117698593814219 \\
96765	0.000103885219005451 \\
97945	9.59524771187992e-05 \\
99125	8.86654732458192e-05 \\
100305	8.19578241245122e-05 \\
101485	7.5773731276485e-05 \\
102665	7.0064843634754e-05 \\
103845	6.47888398537799e-05 \\
105025	5.99083636076969e-05 \\
106205	5.53901986951599e-05 \\
107385	5.12046224619422e-05 \\
108565	4.73248934912962e-05 \\
109745	4.37268410685143e-05 \\
110925	4.03885323242981e-05 \\
112105	3.72899991307185e-05 \\
113285	3.44130113571484e-05 \\
114465	3.17408864345947e-05 \\
115437	2.92583276486624e-05 \\
115577	2.6951285408805e-05 \\
115717	2.4806837098823e-05 \\
115857	2.28130821287587e-05 \\
115997	2.09590495627299e-05 \\
116137	1.92346162691259e-05 \\
116277	1.76304339676303e-05 \\
116417	1.613786387622e-05 \\
116557	1.47489179080362e-05 \\
116697	1.34562055649767e-05 \\
116837	1.2252885821129e-05 \\
116977	1.1132623404464e-05 \\
117117	1.0089548985126e-05 \\
117257	9.11822284360708e-06 \\
117397	8.21360165875751e-06 \\
117537	7.37100810016544e-06 \\
117677	6.58610295123374e-06 \\
117817	5.85485952359077e-06 \\
117957	5.17354014806237e-06 \\
118097	4.53867455568746e-06 \\
118237	4.02424323187045e-06 \\
118377	3.68266289563213e-06 \\
118517	3.37017057655542e-06 \\
118657	3.08397307552388e-06 \\
118797	2.82162731990443e-06 \\
118937	2.58096521221773e-06 \\
119077	2.36005168041675e-06 \\
119217	2.15715195983623e-06 \\
119357	1.9707050079032e-06 \\
119497	1.79930161903208e-06 \\
119637	1.64166621391404e-06 \\
119777	1.49664152704343e-06 \\
119917	1.36317560439636e-06 \\
120057	1.240310656625e-06 \\
120197	1.12717342304336e-06 \\
120337	1.02296678400338e-06 \\
120477	9.33499284649919e-07 \\
120617	8.58678409931812e-07 \\
120757	7.9012099302167e-07 \\
120897	7.27264238331582e-07 \\
121037	6.69608364811669e-07 \\
121177	6.16701501421346e-07 \\
121317	5.68133998291032e-07 \\
121457	5.23533950302912e-07 \\
121597	4.82563318693341e-07 \\
121737	4.44914530162155e-07 \\
121877	4.10307476661487e-07 \\
122017	3.78486857965932e-07 \\
122157	3.4921981900693e-07 \\
122297	3.22293839283283e-07 \\
122437	2.97514845426505e-07 \\
122577	2.74705516445373e-07 \\
122717	2.53703759167667e-07 \\
122857	2.34361334838784e-07 \\
122997	2.16542618003412e-07 \\
123137	2.00123475180281e-07 \\
123277	1.84990249174621e-07 \\
123417	1.71038839424931e-07 \\
123557	1.58173865227873e-07 \\
123697	1.46307909065779e-07 \\
123837	1.35360825714859e-07 \\
123977	1.2525911596839e-07 \\
124117	1.15935356148622e-07 \\
124257	1.07327678300351e-07 \\
124397	9.93792966252904e-08 \\
124537	9.20380759383832e-08 \\
124677	8.52561381492478e-08 \\
124817	7.89895036046318e-08 \\
124957	7.31977622958624e-08 \\
125097	6.78437760415207e-08 \\
125237	6.28934038737761e-08 \\
125377	5.83152530153619e-08 \\
125517	5.40804505622106e-08 \\
125657	5.01624344839691e-08 \\
125797	4.65367632207503e-08 \\
125937	4.31809409340289e-08 \\
126077	4.00742574679924e-08 \\
126217	3.71976424662357e-08 \\
126357	3.45335312568196e-08 \\
126497	3.20657427832494e-08 \\
126637	2.97793660286594e-08 \\
126777	2.76606588189843e-08 \\
126917	2.56969517886674e-08 \\
127057	2.38765631710436e-08 \\
127197	2.21887187512593e-08 \\
127337	2.06234794797311e-08 \\
127477	1.91716747477422e-08 \\
127617	1.78248409365978e-08 \\
127757	1.65751654068735e-08 \\
127897	1.5415434484467e-08 \\
128037	1.43389868867416e-08 \\
128177	1.33396693136056e-08 \\
128317	1.24117972011284e-08 \\
128457	1.15501170849797e-08 \\
128597	1.07497741264062e-08 \\
128737	1.00062794161637e-08 \\
128877	9.3154829960973e-09 \\
129017	8.67354643663276e-09 \\
129157	8.07691935555965e-09 \\
129297	7.52231688050387e-09 \\
129437	7.00669922082398e-09 \\
129577	6.52725273830868e-09 \\
129717	6.0813729607645e-09 \\
129857	5.6666482062262e-09 \\
129997	5.28084487250169e-09 \\
130137	4.92189428102918e-09 \\
130277	4.58787974277897e-09 \\
130417	4.27702523397855e-09 \\
130557	3.98768468246047e-09 \\
130697	3.71833247525544e-09 \\
130837	3.46755352209627e-09 \\
130977	3.23403653856857e-09 \\
131117	3.01656460921507e-09 \\
131257	2.81400913682006e-09 \\
131397	2.62532312556019e-09 \\
131537	2.44953474171083e-09 \\
131677	2.28574231764256e-09 \\
131817	2.13310852315018e-09 \\
131957	1.99085625762763e-09 \\
132097	1.85826359855312e-09 \\
132237	1.73466063735361e-09 \\
132377	1.61942448340113e-09 \\
132517	1.5119766549887e-09 \\
132657	1.41177985968355e-09 \\
132797	1.3183344971246e-09 \\
132937	1.23117638306525e-09 \\
133077	1.14987380728238e-09 \\
133217	1.07402559068603e-09 \\
133357	1.0032587538511e-09 \\
133497	9.37225907993167e-10 \\
133637	8.75604533323582e-10 \\
133777	8.18094314514184e-10 \\
133917	7.64415863940826e-10 \\
134057	7.1430911185999e-10 \\
134197	6.67532140674609e-10 \\
134337	6.23859630621837e-10 \\
134477	5.83082027105775e-10 \\
134617	5.45004097407542e-10 \\
134757	5.09444042506857e-10 \\
134897	4.7623238685901e-10 \\
135037	4.4521170083911e-10 \\
135177	4.16234824385242e-10 \\
135317	3.89165255576529e-10 \\
135457	3.63875041209383e-10 \\
135597	3.40245720487076e-10 \\
135737	3.18166437640599e-10 \\
135877	2.97533886417511e-10 \\
136017	2.78252032526183e-10 \\
136157	2.60230947901618e-10 \\
136297	2.43387088261215e-10 \\
136437	2.27642404926343e-10 \\
136577	2.12924289311189e-10 \\
136717	1.99164740255497e-10 \\
136857	1.86300419535712e-10 \\
136997	1.74272485331528e-10 \\
137137	1.63025759558622e-10 \\
137277	1.52508783379801e-10 \\
137417	1.42673595160403e-10 \\
137557	1.33475563934837e-10 \\
137697	1.24872778783924e-10 \\
137837	1.16826381901802e-10 \\
137977	1.09299735928658e-10 \\
138117	1.02259034573393e-10 \\
138257	9.56724699463507e-11 \\
138397	8.95103990927737e-11 \\
138537	8.37452884816514e-11 \\
138677	7.83511033830564e-11 \\
138817	7.33037519573543e-11 \\
138957	6.85807521882964e-11 \\
139097	6.41611208607173e-11 \\
139237	6.00248739601739e-11 \\
139377	5.61539148513646e-11 \\
139517	5.25310350774078e-11 \\
139657	4.91400253821439e-11 \\
139797	4.59660642881943e-11 \\
139937	4.29950519631461e-11 \\
140077	4.02138322641576e-11 \\
140217	3.76102482491092e-11 \\
140357	3.51728091096959e-11 \\
140497	3.28907456825789e-11 \\
140637	3.07542324939902e-11 \\
140777	2.87538881593719e-11 \\
140917	2.68807753833755e-11 \\
141057	2.51268450490727e-11 \\
141197	2.34844366175935e-11 \\
141337	2.19463336392778e-11 \\
141477	2.05060968205828e-11 \\
141617	1.91570648233608e-11 \\
141757	1.78937420436398e-11 \\
141897	1.67105218551455e-11 \\
142037	1.56022417208135e-11 \\
142177	1.45642942150914e-11 \\
142317	1.35919608901247e-11 \\
142457	1.26811339207222e-11 \\
142597	1.18278720151466e-11 \\
142737	1.10285669485677e-11 \\
142877	1.02798325407605e-11 \\
143017	9.57833812265108e-12 \\
143157	8.92108609207298e-12 \\
143297	8.3054119137671e-12 \\
143437	7.72848451902064e-12 \\
143577	7.18786141717942e-12 \\
143717	6.68137767334542e-12 \\
143857	6.20675733031817e-12 \\
143997	5.76200198665333e-12 \\
144137	5.34516875205782e-12 \\
144277	4.95464780314592e-12 \\
144417	4.58871829422947e-12 \\
144557	4.24565937962029e-12 \\
144697	3.92424981399131e-12 \\
144837	3.62299079625927e-12 \\
144977	3.34071659224833e-12 \\
145117	3.07592840087523e-12 \\
145257	2.8277935548715e-12 \\
145397	2.595312853515e-12 \\
145537	2.37737607378108e-12 \\
145677	2.17309503725005e-12 \\
145817	1.98158156550221e-12 \\
145957	1.80211401357155e-12 \\
146097	1.63380420303838e-12 \\
146237	1.47609702239038e-12 \\
146377	1.32815980435907e-12 \\
146517	1.18943743743216e-12 \\
146657	1.05943032124856e-12 \\
146797	9.37638855447176e-13 \\
146937	8.23285883910785e-13 \\
147077	7.16149362034457e-13 \\
147217	6.15729689457112e-13 \\
147357	5.21527265817667e-13 \\
147497	4.33153513057505e-13 \\
147637	3.50219853118006e-13 \\
147777	2.7272628599917e-13 \\
147917	1.99840144432528e-13 \\
148057	1.31616939569312e-13 \\
148197	6.75015598972095e-14 \\
148337	7.49400541621981e-15 \\
148477	-4.87943019322756e-14 \\
148617	-1.01751940206896e-13 \\
148757	-1.51267887105178e-13 \\
148897	-1.9773072068574e-13 \\
149037	-2.41362485553509e-13 \\
149177	-2.82163181708484e-13 \\
149317	-3.20465876058051e-13 \\
149457	-3.56492613207138e-13 \\
149597	-3.90187882004511e-13 \\
149737	-4.21884749357559e-13 \\
149877	-4.51527704115051e-13 \\
150017	-4.79338790881911e-13 \\
150157	-5.05484543111834e-13 \\
150297	-5.29964960804818e-13 \\
150437	-5.52891066263328e-13 \\
150577	-5.74484904092287e-13 \\
150717	-5.94746474291696e-13 \\
150857	-6.13620265710324e-13 \\
150997	-6.3155036755802e-13 \\
151137	-6.48203712927398e-13 \\
151277	-6.63857857574612e-13 \\
151417	-6.78679334953358e-13 \\
151557	-6.92390589307479e-13 \\
151697	-7.05269176393131e-13 \\
151837	-7.17370607361545e-13 \\
151977	-7.28861415666415e-13 \\
152117	-7.39519556702817e-13 \\
152257	-7.49622586226906e-13 \\
152397	-7.59003970784988e-13 \\
152537	-7.67774732679527e-13 \\
152677	-7.76045894212984e-13 \\
152817	-7.83983988839054e-13 \\
152957	-7.91200438499118e-13 \\
153097	-7.98028310100563e-13 \\
153237	-8.04412092492157e-13 \\
153377	-8.10462807976364e-13 \\
153517	-8.16124945401953e-13 \\
153657	-8.21454015920153e-13 \\
153797	-8.26394508379735e-13 \\
153937	-8.31112956234392e-13 \\
154077	-8.35553848332893e-13 \\
154217	-8.39606162372775e-13 \\
154357	-8.43491942958963e-13 \\
154497	-8.47155678940226e-13 \\
154637	-8.50486348014101e-13 \\
154777	-8.53705994785514e-13 \\
154917	-8.56703596952002e-13 \\
155057	-8.59645687967259e-13 \\
155197	-8.62310223226359e-13 \\
155337	-8.64697202729303e-13 \\
155477	-8.66917648778553e-13 \\
155617	-8.69304628281498e-13 \\
155757	-8.71247518574592e-13 \\
155897	-8.73245920018917e-13 \\
156037	-8.74911254555855e-13 \\
156177	-8.76632100244024e-13 \\
156317	-8.7824192362973e-13 \\
156457	-8.79851747015437e-13 \\
156597	-8.81128503493755e-13 \\
156737	-8.82516282274537e-13 \\
156877	-8.83793038752856e-13 \\
157017	-8.85014284079944e-13 \\
157157	-8.86068995953337e-13 \\
157297	-8.870681966755e-13 \\
157437	-8.879563750952e-13 \\
157577	-8.88955575817363e-13 \\
157717	-8.897327319346e-13 \\
};
\addplot [line width=0.01pt, red, forget plot]
table [row sep=\\]{%
1181	2.79768876315512 \\
2361	2.32576521648755 \\
3541	1.93270502037213 \\
4721	1.60525700756662 \\
5901	1.33758569341974 \\
7081	1.10978227403889 \\
8261	0.923423304033014 \\
9441	0.764248487577208 \\
10621	0.634539565402948 \\
11801	0.536636480049229 \\
12981	0.457923966390105 \\
14161	0.38928274008926 \\
15341	0.331262316650026 \\
16521	0.282643018290445 \\
17701	0.241316203830048 \\
18881	0.203680113831454 \\
20061	0.17241991641083 \\
21241	0.144230154803789 \\
22421	0.120867149972773 \\
23601	0.103259521748066 \\
24781	0.0880423370964351 \\
25961	0.0754271956961616 \\
27141	0.0647141608659087 \\
28321	0.0565620868654851 \\
29501	0.0494715680109212 \\
30681	0.0428627100930207 \\
31861	0.0371051701677116 \\
33041	0.0320512833874384 \\
34221	0.0273345052966706 \\
35401	0.02380043013784 \\
36581	0.020659196924361 \\
37761	0.0184815527912085 \\
38941	0.0166298255985704 \\
40121	0.0150652180901379 \\
41301	0.0136466535833128 \\
42481	0.0123193088666612 \\
43661	0.011077071722333 \\
44841	0.0101104025524302 \\
46021	0.00920583100705596 \\
47201	0.00835753413463686 \\
48381	0.00756210642308569 \\
49561	0.00681600977686075 \\
50741	0.00611582786279136 \\
51921	0.00545830030399513 \\
53101	0.00484066735026439 \\
54281	0.00426031411067418 \\
55461	0.00371486282579653 \\
56641	0.00320574412885055 \\
57821	0.00277755548607084 \\
59001	0.00243643626430518 \\
60181	0.00215387389103444 \\
61361	0.00190368429251375 \\
62541	0.00167164527560371 \\
63721	0.00153326125566194 \\
64901	0.00140751474356754 \\
66081	0.00129148518320449 \\
67261	0.00118877625170105 \\
68441	0.00109620344276695 \\
69621	0.00101163315677061 \\
70801	0.000933648128640774 \\
71981	0.000861691884335769 \\
73161	0.000795272361798494 \\
74341	0.000733939989421395 \\
75521	0.000677282364008214 \\
76701	0.000624924719999309 \\
77881	0.000576519819633192 \\
79061	0.00053175550865997 \\
80241	0.000490350218698199 \\
81421	0.000452027393770482 \\
82601	0.000416563453812135 \\
83781	0.000383720120574804 \\
84961	0.000353292069201094 \\
86141	0.000325096811355952 \\
87321	0.000298967898185731 \\
88501	0.000274746607164644 \\
89681	0.000252281515716712 \\
90861	0.000231639827897534 \\
92041	0.000215229566460362 \\
93221	0.000200070332090796 \\
94401	0.000186121339745993 \\
95581	0.000173120456306686 \\
96761	0.000161193867266707 \\
97941	0.000149928063358229 \\
99121	0.000139450739085389 \\
100301	0.000129889169380892 \\
101481	0.000121043630280648 \\
102661	0.00011268517400298 \\
103841	0.000104952744827858 \\
105021	9.78692882718235e-05 \\
106201	9.11514187202922e-05 \\
107381	8.49028589615664e-05 \\
108561	7.89891696930667e-05 \\
109741	7.35531602737427e-05 \\
110921	6.85064867554375e-05 \\
112101	6.38257566427103e-05 \\
113281	5.94085854384585e-05 \\
114461	5.52818811657141e-05 \\
115005	5.16094807116452e-05 \\
115125	4.82327857141418e-05 \\
115245	4.50869316014901e-05 \\
115365	4.2155081277917e-05 \\
115485	3.94217672892716e-05 \\
115605	3.68727494551302e-05 \\
115725	3.4494900592319e-05 \\
115845	3.22761064285859e-05 \\
115965	3.02051768545364e-05 \\
116085	2.82717667542953e-05 \\
116205	2.64663050341674e-05 \\
116325	2.47799307345753e-05 \\
116445	2.32044353059568e-05 \\
116565	2.17322102958328e-05 \\
116685	2.03561998109447e-05 \\
116805	1.90698572195003e-05 \\
116925	1.78671056389468e-05 \\
117045	1.67423018150314e-05 \\
117165	1.56902030539197e-05 \\
117285	1.4705936908499e-05 \\
117405	1.37849733589657e-05 \\
117525	1.29230992553797e-05 \\
117645	1.21163948187952e-05 \\
117765	1.13612120151152e-05 \\
117885	1.06541546394667e-05 \\
118005	9.99205996243813e-06 \\
118125	9.37198180411825e-06 \\
118245	8.79117491781045e-06 \\
118365	8.24708057128865e-06 \\
118485	7.73731322806226e-06 \\
118605	7.25964823961034e-06 \\
118725	6.81201046359714e-06 \\
118845	6.39246373607127e-06 \\
118965	5.99920112748231e-06 \\
119085	5.63053592145257e-06 \\
119205	5.28489325885007e-06 \\
119325	4.96080239342778e-06 \\
119445	4.65688951456489e-06 \\
119565	4.37187108953685e-06 \\
119685	4.10454768812185e-06 \\
119805	3.85379825040832e-06 \\
119925	3.61857476616212e-06 \\
120045	3.39789733189155e-06 \\
120165	3.19084955996418e-06 \\
120285	2.99657431118705e-06 \\
120405	2.81426972753573e-06 \\
120525	2.64318554338283e-06 \\
120645	2.48261965302143e-06 \\
120765	2.33191491738616e-06 \\
120885	2.19045619087588e-06 \\
121005	2.05925636120874e-06 \\
121125	1.93300859713252e-06 \\
121245	1.81597663756472e-06 \\
121365	1.70609745392403e-06 \\
121485	1.60292794820771e-06 \\
121605	1.50605300325291e-06 \\
121725	1.41615130833106e-06 \\
121845	1.32965417076747e-06 \\
121965	1.24942554957652e-06 \\
122085	1.17407630845223e-06 \\
122205	1.10330630154198e-06 \\
122325	1.03712900911823e-06 \\
122445	9.74465632130883e-07 \\
122565	9.15743698759464e-07 \\
122685	8.60647085543231e-07 \\
122805	8.08887631209565e-07 \\
122925	7.60261278376362e-07 \\
123045	7.14576617932838e-07 \\
123165	6.71654088790774e-07 \\
123285	6.31325230648905e-07 \\
123405	5.93431985607129e-07 \\
123525	5.57826044245147e-07 \\
123645	5.24368234111616e-07 \\
123765	4.92927363704432e-07 \\
123885	4.63382022586334e-07 \\
124005	4.35616606175859e-07 \\
124125	4.09523090383335e-07 \\
124245	3.85000099967225e-07 \\
124365	3.61952468330706e-07 \\
124485	3.40290850942004e-07 \\
124605	3.19931364400894e-07 \\
124725	3.00795247987207e-07 \\
124845	2.82808546414604e-07 \\
124965	2.6590181406716e-07 \\
125085	2.50009644264804e-07 \\
125205	2.35071182796975e-07 \\
125325	2.21028711278048e-07 \\
125445	2.07828204146221e-07 \\
125565	1.95418919102242e-07 \\
125685	1.83753194216152e-07 \\
125805	1.72786258911817e-07 \\
125925	1.62476056442262e-07 \\
126045	1.52783078022356e-07 \\
126165	1.43670207564117e-07 \\
126285	1.36920300264354e-07 \\
126405	1.27046653197738e-07 \\
126525	1.19472981563451e-07 \\
126645	1.12352330661203e-07 \\
126765	1.06921825060446e-07 \\
126885	1.00419574600163e-07 \\
127005	9.34416696241058e-08 \\
127125	8.78762444478376e-08 \\
127245	8.33086546414563e-08 \\
127365	7.77221887959811e-08 \\
127485	7.30952965377618e-08 \\
127605	6.87445466107661e-08 \\
127725	6.46533725445764e-08 \\
127845	6.08062363793849e-08 \\
127965	5.82508782875912e-08 \\
128085	5.37864137117872e-08 \\
128205	5.05872401190821e-08 \\
128325	4.75787556575291e-08 \\
128445	4.47495592248437e-08 \\
128565	4.20889272878533e-08 \\
128685	3.95868076097372e-08 \\
128805	3.72337233267572e-08 \\
128925	3.50207742250142e-08 \\
129045	3.2939593885839e-08 \\
129165	3.09822242305913e-08 \\
129285	2.91414251618605e-08 \\
129405	2.74102201491822e-08 \\
129525	2.57820381155405e-08 \\
129645	2.42507330883157e-08 \\
129765	2.28105282440438e-08 \\
129885	2.14559918165769e-08 \\
130005	2.0182017335113e-08 \\
130125	1.89838034736489e-08 \\
130245	1.78568119180156e-08 \\
130365	1.73012054682786e-08 \\
130485	1.57998654159996e-08 \\
130605	1.48621686069106e-08 \\
130725	1.39802006637879e-08 \\
130845	1.31506429079664e-08 \\
130965	1.23703754462134e-08 \\
131085	1.16364644031641e-08 \\
131205	1.09461507080688e-08 \\
131325	1.02968402138082e-08 \\
131445	9.68609342733018e-09 \\
131565	9.11161607275446e-09 \\
131685	8.57125087572186e-09 \\
131805	9.69454166943606e-09 \\
131925	7.58475565687533e-09 \\
132045	7.13501652205295e-09 \\
132165	6.71199129609334e-09 \\
132285	6.31407398588024e-09 \\
132405	5.93977139695667e-09 \\
132525	5.58767992986375e-09 \\
132645	5.25647975146981e-09 \\
132765	4.94492946589986e-09 \\
132885	4.65186184017696e-09 \\
133005	4.37617869719631e-09 \\
133125	4.116847529545e-09 \\
133245	3.872897447188e-09 \\
133365	3.64340529968032e-09 \\
133485	4.56757542899311e-09 \\
133605	3.520121749645e-09 \\
133725	3.03324104722336e-09 \\
133845	2.85352286244489e-09 \\
133965	2.68447686302409e-09 \\
134085	2.52545467871457e-09 \\
134205	2.37585856543987e-09 \\
134325	2.23512885977328e-09 \\
134445	2.10273964906804e-09 \\
134565	1.97819599589977e-09 \\
134685	1.86103232824308e-09 \\
134805	1.75081116271514e-09 \\
134925	1.64712038452919e-09 \\
135045	1.54955309694671e-09 \\
135165	1.45777451221463e-09 \\
135285	1.37144301470826e-09 \\
135405	1.29022592609829e-09 \\
135525	1.21381937834286e-09 \\
135645	1.14193821065811e-09 \\
135765	1.07431391560553e-09 \\
135885	1.01069164148981e-09 \\
136005	9.50839407209969e-10 \\
136125	8.94531337891635e-10 \\
136245	8.41557268316251e-10 \\
136365	7.91719578785433e-10 \\
136485	7.44832306942556e-10 \\
136605	7.00720870216998e-10 \\
136725	6.59220456000753e-10 \\
136845	6.20176743293399e-10 \\
136965	5.83444015322954e-10 \\
137085	5.48885270568178e-10 \\
137205	5.16371889691669e-10 \\
137325	4.85782525316836e-10 \\
137445	4.57003490605956e-10 \\
137565	4.29927315970247e-10 \\
137685	4.04453248670222e-10 \\
137805	3.80486586681883e-10 \\
137925	3.57937790518292e-10 \\
138045	3.36723204874545e-10 \\
138165	3.16763504315531e-10 \\
138285	2.97984525943207e-10 \\
138405	2.80316436729322e-10 \\
138525	2.63693233915063e-10 \\
138645	2.48053411144866e-10 \\
138765	2.33338737221089e-10 \\
138885	2.19494256104014e-10 \\
139005	2.0646861997875e-10 \\
139125	1.94213145565669e-10 \\
139245	1.82682535765366e-10 \\
139365	1.71833713924485e-10 \\
139485	1.61626378947233e-10 \\
139605	1.52022561206167e-10 \\
139725	1.42986567031045e-10 \\
139845	1.34484923197675e-10 \\
139965	1.264857107941e-10 \\
140085	1.18959453399015e-10 \\
140205	1.11878117881048e-10 \\
140325	1.05215447465667e-10 \\
140445	9.89466286682728e-11 \\
140565	9.3048180271893e-11 \\
140685	8.74983974163968e-11 \\
140805	8.22767409758285e-11 \\
140925	7.73635600026523e-11 \\
141045	7.27408688838693e-11 \\
141165	6.83912371179929e-11 \\
141285	6.42987330046196e-11 \\
141405	9.24230136867266e-11 \\
141525	5.68236568909697e-11 \\
141645	5.34144950492532e-11 \\
141765	5.02069497088087e-11 \\
141885	4.71888084163652e-11 \\
142005	4.4349079963979e-11 \\
142125	4.16771062106136e-11 \\
142245	3.91629506601987e-11 \\
142365	3.67973429504787e-11 \\
142485	3.45714568084077e-11 \\
142605	3.24771320947548e-11 \\
142725	3.05064307148939e-11 \\
142845	2.8652136219165e-11 \\
142965	2.69073097136641e-11 \\
143085	2.52656784383021e-11 \\
143205	2.37210251441411e-11 \\
143325	2.22672991156969e-11 \\
143445	2.08997819051149e-11 \\
143565	1.96129223972719e-11 \\
143685	1.84018911220107e-11 \\
143805	1.72624692318379e-11 \\
143925	1.61904378792599e-11 \\
144045	1.5181578216783e-11 \\
144165	1.42322820195773e-11 \\
144285	1.33391075962663e-11 \\
144405	1.24985577443226e-11 \\
144525	1.17078013950334e-11 \\
144645	1.09636744127783e-11 \\
144765	1.02634567511473e-11 \\
144885	9.60453938603223e-12 \\
145005	8.98453533793031e-12 \\
145125	8.40111313848979e-12 \\
145245	7.85216336396388e-12 \\
145365	7.33568761290826e-12 \\
145485	6.84957646157613e-12 \\
145605	1.06797237720002e-10 \\
145725	5.9615645753297e-12 \\
145845	5.55649970479521e-12 \\
145965	5.17547116274386e-12 \\
146085	4.8168136146387e-12 \\
146205	4.47947234860635e-12 \\
146325	4.16194856356356e-12 \\
146445	3.86313203648569e-12 \\
146565	6.040656863604e-11 \\
146685	5.89941984152631e-11 \\
146805	5.87448423239323e-11 \\
146925	2.83389978150694e-12 \\
147045	2.61340948881639e-12 \\
147165	2.40601982781641e-12 \\
147285	2.21089813123854e-12 \\
147405	2.02732275411677e-12 \\
147525	1.8544055180314e-12 \\
147645	1.69181335607504e-12 \\
147765	1.5388246232817e-12 \\
147885	1.39482869698782e-12 \\
148005	1.25927046568108e-12 \\
148125	1.13176135130288e-12 \\
148245	1.01180175349214e-12 \\
148365	8.98947583038989e-13 \\
148485	7.92699239582362e-13 \\
148605	6.92779167366098e-13 \\
148725	5.98632254877884e-13 \\
148845	5.10091968664028e-13 \\
148965	4.26825241817141e-13 \\
149085	3.48443496278605e-13 \\
149205	2.74669176292264e-13 \\
149325	2.05280237253191e-13 \\
149445	1.3988810110277e-13 \\
149565	7.83817455385361e-14 \\
149685	2.07056594092592e-14 \\
149805	-3.38618022510673e-14 \\
149925	-8.5043083686287e-14 \\
150045	-1.33393296408713e-13 \\
150165	-1.78690395813419e-13 \\
150285	5.64206459330308e-11 \\
150405	-2.61402011147993e-13 \\
150525	-2.99371638590173e-13 \\
150645	-3.35009797680641e-13 \\
150765	-3.68538533024321e-13 \\
150885	-3.99957844621213e-13 \\
151005	-4.29600799378704e-13 \\
151125	-4.57522908448027e-13 \\
151245	-4.83779682980412e-13 \\
};
\addplot [line width=0.01pt, red, forget plot]
table [row sep=\\]{%
1171	3.01403245507076 \\
2341	2.51467124622582 \\
3511	2.11351234868671 \\
4681	1.78047756225305 \\
5851	1.49556948119767 \\
7021	1.24857460326888 \\
8191	1.03773872323782 \\
9361	0.862113895662823 \\
10531	0.71917383215328 \\
11701	0.595275024523611 \\
12871	0.487724764545261 \\
14041	0.400762404391776 \\
15211	0.327609966253242 \\
16381	0.266250601502127 \\
17551	0.221022214958398 \\
18721	0.185231993232122 \\
19891	0.155142879393177 \\
21061	0.131070318962255 \\
22231	0.111484516824811 \\
23401	0.0950869604478434 \\
24571	0.0826488427230572 \\
25741	0.0727019018579828 \\
26911	0.063548939822789 \\
28081	0.0555982531354983 \\
29251	0.0490006164604731 \\
30421	0.0429289558208536 \\
31591	0.0374290188023848 \\
32761	0.0331448312724392 \\
33931	0.0294841448839341 \\
35101	0.0267967382664894 \\
36271	0.0243267304188189 \\
37441	0.0220766696953493 \\
38611	0.0200062219620973 \\
39781	0.0181074529179187 \\
40951	0.0163410761690875 \\
42121	0.0147048013281011 \\
43291	0.0131839723121143 \\
44461	0.0117718885821002 \\
45631	0.0106409573394344 \\
46801	0.00958347229011713 \\
47971	0.00859315553841627 \\
49141	0.00767284992535949 \\
50311	0.00690796804719462 \\
51481	0.00621006202443919 \\
52651	0.00557843898397342 \\
53821	0.00498878305817907 \\
54991	0.00443890125613677 \\
56161	0.00392377210885148 \\
57331	0.00344045021437256 \\
58501	0.00298819812099516 \\
59671	0.00257231751636772 \\
60841	0.0022588583096656 \\
62011	0.00197192143813113 \\
63181	0.00170334615161044 \\
64351	0.00145126484584063 \\
65521	0.00121461434671655 \\
66691	0.00110081613634372 \\
67861	0.0010212467196779 \\
69031	0.000947519907869121 \\
70201	0.000879174973035302 \\
71371	0.000815795464725211 \\
72541	0.000756999383719537 \\
73711	0.000702435904900545 \\
74881	0.000651782664502998 \\
76051	0.000604743310135603 \\
77221	0.000561045276145089 \\
78391	0.000520437761448778 \\
79561	0.000482689889926557 \\
80731	0.000447589035663332 \\
81901	0.000418732529647026 \\
83071	0.000392369465315345 \\
84241	0.000367770594465744 \\
85411	0.000344799102750148 \\
86581	0.000323338302827103 \\
87751	0.000303281243601294 \\
88921	0.000284529323553995 \\
90091	0.000266991534418504 \\
91261	0.000250583809218607 \\
92431	0.000235228434869228 \\
93601	0.00022085352051382 \\
94771	0.00020739251549623 \\
95941	0.000194783771732376 \\
97111	0.000182970145888528 \\
98281	0.000171898637326695 \\
99451	0.000161520058238551 \\
100621	0.000151788732796865 \\
101791	0.000142662222504752 \\
102961	0.000134101075225324 \\
104131	0.000126068595645512 \\
105301	0.000118530635160707 \\
106471	0.000111455399375626 \\
107641	0.000104813271601223 \\
108811	9.85766508870278e-05 \\
109981	9.27198032796284e-05 \\
111151	8.72187251193535e-05 \\
112321	8.2051017309237e-05 \\
113491	7.71957695896996e-05 \\
114661	7.26334539467022e-05 \\
115831	6.83458263610603e-05 \\
117001	6.43158361856022e-05 \\
118171	6.05275424999108e-05 \\
119341	5.69660368524016e-05 \\
120511	5.36173718569954e-05 \\
121681	5.04684951568857e-05 \\
122851	4.75071883135891e-05 \\
124021	4.47220102188228e-05 \\
125191	4.21022446636665e-05 \\
126361	3.96378517282803e-05 \\
127531	3.73194226908624e-05 \\
128701	3.5138138174462e-05 \\
129871	3.3085729279736e-05 \\
131041	3.11544414701159e-05 \\
132211	2.9337000997165e-05 \\
133381	2.76265836716139e-05 \\
134551	2.60167858012195e-05 \\
135721	2.45015971330753e-05 \\
136891	2.30753756488844e-05 \\
138061	2.1732824077858e-05 \\
139231	2.04689679964565e-05 \\
140401	1.92791354055033e-05 \\
141571	1.81589376693769e-05 \\
142741	1.71042517261299e-05 \\
143911	1.61112034734456e-05 \\
145081	1.5176152247498e-05 \\
146251	1.42956763187208e-05 \\
147421	1.34665593325978e-05 \\
148591	1.2685777629251e-05 \\
149761	1.19504883810961e-05 \\
150931	1.12580184928324e-05 \\
152101	1.06058542116427e-05 \\
153271	9.9916313983095e-06 \\
154441	9.41312641622449e-06 \\
155611	8.86824759549487e-06 \\
156781	8.35502723500747e-06 \\
157951	7.87161410442661e-06 \\
159121	7.41626641670434e-06 \\
160291	6.98734523663092e-06 \\
161461	6.58330829977904e-06 \\
162631	6.20270421275437e-06 \\
163801	5.84416701121526e-06 \\
164971	5.50641105417915e-06 \\
166141	5.18822622935788e-06 \\
167311	4.88847345314625e-06 \\
168481	4.6060804445025e-06 \\
169651	4.34003775656722e-06 \\
170821	4.08939504831229e-06 \\
171991	3.85325758295307e-06 \\
173161	3.63078293758035e-06 \\
174331	3.4211779110227e-06 \\
175501	3.22369561822633e-06 \\
176671	3.03763275888436e-06 \\
177841	2.86232704999057e-06 \\
179011	2.69715481221455e-06 \\
180181	2.54152869993973e-06 \\
181351	2.39489556747019e-06 \\
182521	2.25673446180297e-06 \\
183691	2.12655473469381e-06 \\
184861	2.00389426746606e-06 \\
186031	1.88831780068011e-06 \\
187201	1.77941536255721e-06 \\
188371	1.67680079082855e-06 \\
189541	1.5801103418478e-06 \\
190711	1.48900138075003e-06 \\
191881	1.40315115104706e-06 \\
193051	1.32225561444432e-06 \\
194221	1.24602836043541e-06 \\
195391	1.17419957879061e-06 \\
196561	1.10651509355186e-06 \\
197731	1.04273545303846e-06 \\
198901	9.82635072976912e-07 \\
200071	9.26001430812118e-07 \\
201241	8.72634305759679e-07 \\
202411	8.22345065376595e-07 \\
203581	7.74955991045267e-07 \\
204751	7.30299645368415e-07 \\
205921	6.88218275202157e-07 \\
207091	6.48563249994183e-07 \\
208261	6.11194532595949e-07 \\
209431	5.75980180939073e-07 \\
210601	5.42795879021618e-07 \\
211771	5.11524495094839e-07 \\
212941	4.82055664996484e-07 \\
214111	4.54285400908194e-07 \\
215281	4.28115721429201e-07 \\
216451	4.0345430446509e-07 \\
217621	3.80214159489967e-07 \\
218791	3.58313318737924e-07 \\
219961	3.37674547268296e-07 \\
221131	3.18225068851596e-07 \\
222301	2.9989630773164e-07 \\
223471	2.82623646763458e-07 \\
224641	2.66346197264067e-07 \\
225811	2.51006584850444e-07 \\
226981	2.36550745769115e-07 \\
228151	2.2292773477206e-07 \\
229321	2.10089546426317e-07 \\
230491	1.97990944472703e-07 \\
231661	1.86589301787166e-07 \\
232831	1.75844450112095e-07 \\
234001	1.65718537614712e-07 \\
235171	1.56175895660304e-07 \\
236341	1.47182912524357e-07 \\
237511	1.38707914321134e-07 \\
238681	1.30721053648308e-07 \\
239851	1.23194204337818e-07 \\
241021	1.16100861091706e-07 \\
242191	1.09416047167077e-07 \\
243361	1.03116225447231e-07 \\
244531	9.71792157300477e-08 \\
245701	9.15841160686881e-08 \\
246871	8.63112294968715e-08 \\
248041	8.13419944178939e-08 \\
249211	7.66589186018685e-08 \\
250381	7.22455180124371e-08 \\
251551	6.80862590196618e-08 \\
252721	6.41665026113181e-08 \\
253891	6.04724535446799e-08 \\
255061	5.69911117742627e-08 \\
256231	5.37102257114341e-08 \\
257401	5.06182500359387e-08 \\
258571	4.77043048952019e-08 \\
259741	4.49581367134577e-08 \\
260911	4.23700829976781e-08 \\
262081	3.99310379206597e-08 \\
263251	3.76324201245559e-08 \\
264421	3.5466142578322e-08 \\
265591	3.34245842670278e-08 \\
266761	3.15005631579268e-08 \\
267931	2.96873108318607e-08 \\
269101	2.79784485579526e-08 \\
270271	2.63679655332361e-08 \\
270811	2.4850196900239e-08 \\
270931	2.34198037074584e-08 \\
271051	2.20717552013028e-08 \\
271171	2.08013099523008e-08 \\
271291	1.96039996458452e-08 \\
271411	1.84756134280484e-08 \\
271531	1.74121828067086e-08 \\
271651	1.64099678845453e-08 \\
271771	1.54654444806113e-08 \\
271891	1.45752905855723e-08 \\
272011	1.37363763696996e-08 \\
272131	1.29457515263276e-08 \\
272251	1.22006361680249e-08 \\
272371	1.14984100574311e-08 \\
272491	1.08366043916064e-08 \\
272611	1.02128918100242e-08 \\
272731	9.62507951118852e-09 \\
272851	9.07110075942796e-09 \\
272971	8.5490080015127e-09 \\
273091	8.05696526162691e-09 \\
273211	7.59324303434283e-09 \\
273331	7.1562109571488e-09 \\
273451	6.74433259240104e-09 \\
273571	6.35615987620852e-09 \\
273691	5.9903284554963e-09 \\
273811	5.64555208137918e-09 \\
273931	5.32061888991464e-09 \\
274051	5.0143866281438e-09 \\
274171	4.72577893484427e-09 \\
274291	4.45378123270501e-09 \\
274411	4.19743739765721e-09 \\
274531	3.95584659473869e-09 \\
274651	3.72815944782445e-09 \\
274771	3.51357581918066e-09 \\
274891	3.31134147879553e-09 \\
275011	3.12074560637754e-09 \\
275131	2.94111840437594e-09 \\
275251	2.77182837793433e-09 \\
275371	2.6122807805784e-09 \\
275491	2.46191472763613e-09 \\
275611	2.32020180845893e-09 \\
275731	2.18664414353142e-09 \\
275851	2.06077216402534e-09 \\
275971	1.94214361259881e-09 \\
276091	1.8303416005061e-09 \\
276211	1.7249733308411e-09 \\
276331	1.62566843320278e-09 \\
276451	1.5320779644945e-09 \\
276571	1.44387313216754e-09 \\
276691	1.36074379541995e-09 \\
276811	1.28239807661856e-09 \\
276931	1.20856058494212e-09 \\
277051	1.13897191678092e-09 \\
277171	1.07338760102493e-09 \\
277291	1.0115769888408e-09 \\
277411	9.53322976116056e-10 \\
277531	8.98421170791863e-10 \\
277651	8.46678449573091e-10 \\
277771	7.9791279139485e-10 \\
277891	7.51953332933653e-10 \\
278011	7.08638203672507e-10 \\
278131	6.67815414079342e-10 \\
278251	6.29341689872831e-10 \\
278371	5.93081528332817e-10 \\
278491	5.58907808922982e-10 \\
278611	5.26700294489757e-10 \\
278731	4.96345964329237e-10 \\
278851	4.67738070497603e-10 \\
278971	4.40776304344581e-10 \\
279091	4.15365741801565e-10 \\
279211	3.9141728747083e-10 \\
279331	3.68846619913654e-10 \\
279451	3.47574580228383e-10 \\
279571	3.27526505916609e-10 \\
279691	3.08631953327421e-10 \\
279811	2.90824364590492e-10 \\
279931	2.74041345171838e-10 \\
280051	2.58223997740004e-10 \\
280171	2.43316644610303e-10 \\
280291	2.29266938767125e-10 \\
280411	2.16025530797026e-10 \\
280531	2.03546013377576e-10 \\
280651	1.91784477188151e-10 \\
280771	1.80699566421083e-10 \\
280891	1.7025236775936e-10 \\
281011	1.6040624384317e-10 \\
281131	1.51126611225294e-10 \\
281251	1.42380829348809e-10 \\
281371	1.34138200547085e-10 \\
281491	1.26369747999178e-10 \\
281611	1.19048160218682e-10 \\
281731	1.12147957587183e-10 \\
281851	1.05644659686988e-10 \\
281971	9.9515340412637e-11 \\
282091	9.37389055266635e-11 \\
282211	8.82944828362042e-11 \\
282331	8.31635316167478e-11 \\
282451	7.83275666549343e-11 \\
282571	7.37698790942432e-11 \\
282691	6.94744262119684e-11 \\
282811	6.5426053463824e-11 \\
282931	6.16105499950947e-11 \\
283051	5.80145931294851e-11 \\
283171	5.46254153022119e-11 \\
283291	5.14313036603653e-11 \\
283411	4.8420878417943e-11 \\
283531	4.55836479673621e-11 \\
283651	4.29096203014012e-11 \\
283771	4.03894695466533e-11 \\
283891	3.80143139189215e-11 \\
284011	3.57757157232186e-11 \\
284131	3.36658478872209e-11 \\
284251	3.1677382938966e-11 \\
284371	2.98032709622476e-11 \\
284491	2.80371281746739e-11 \\
284611	2.63724042603997e-11 \\
284731	2.48034925931506e-11 \\
284851	2.33247865466524e-11 \\
284971	2.19312901172941e-11 \\
285091	2.06178962791626e-11 \\
285211	1.93799976067055e-11 \\
285331	1.82133197412782e-11 \\
285451	1.71136993465382e-11 \\
285571	1.60774726865043e-11 \\
285691	1.51008094917415e-11 \\
285811	1.41802680708736e-11 \\
285931	1.33126842882803e-11 \\
286051	1.24951160529463e-11 \\
286171	1.17243437181003e-11 \\
286291	1.09980358153905e-11 \\
286411	1.03136388318603e-11 \\
286531	9.66854374340187e-12 \\
286651	9.06047459281467e-12 \\
286771	8.48732195635193e-12 \\
286891	7.94730947717426e-12 \\
287011	7.43821670923239e-12 \\
287131	6.9583783179894e-12 \\
287251	6.50629550236204e-12 \\
287371	6.08008088320844e-12 \\
287491	5.67851321520152e-12 \\
287611	5.29998267495557e-12 \\
287731	4.94310148368982e-12 \\
287851	4.60698146298455e-12 \\
287971	4.2900683006053e-12 \\
288091	3.99147381813236e-12 \\
288211	3.70986574793619e-12 \\
288331	3.44452244505078e-12 \\
288451	3.19450021990519e-12 \\
288571	2.95891089407974e-12 \\
288691	2.73669975570101e-12 \\
288811	2.52736720440794e-12 \\
288931	2.33008057293205e-12 \\
289051	2.1442292386098e-12 \\
289171	1.96898053417272e-12 \\
289291	1.80377934810849e-12 \\
289411	1.64807056890481e-12 \\
289531	1.50141010735183e-12 \\
289651	1.36313182963477e-12 \\
289771	1.23279164654377e-12 \\
289891	1.10988995771777e-12 \\
290011	9.94260229703059e-13 \\
290131	8.85069795231175e-13 \\
290251	7.82152120848423e-13 \\
290371	6.8539618425234e-13 \\
290491	5.93969318174459e-13 \\
290611	5.07871522614778e-13 \\
290731	4.2676973066591e-13 \\
290851	3.50275364269237e-13 \\
290971	2.78221889971064e-13 \\
};
\addplot [line width=0.01pt, red, forget plot]
table [row sep=\\]{%
1182	2.26755774266133 \\
2362	1.77964193493708 \\
3542	1.40569143249729 \\
4722	1.12519633145723 \\
5902	0.905570228153968 \\
7082	0.724133084350716 \\
8262	0.586075458296991 \\
9442	0.470379150835249 \\
10622	0.382769980975389 \\
11802	0.313413833835201 \\
12982	0.25687107668252 \\
14162	0.212804859529527 \\
15342	0.180297614461483 \\
16522	0.152809695279984 \\
17702	0.132247856639573 \\
18882	0.11572150837542 \\
20062	0.10120635303636 \\
21242	0.0888366064539437 \\
22422	0.0775808824124722 \\
23602	0.0678194853956758 \\
24782	0.0594372967607328 \\
25962	0.0522910662752791 \\
27142	0.0458773206754446 \\
28322	0.0403887844172185 \\
29502	0.0354053652740368 \\
30682	0.0311432683532523 \\
31862	0.0272006669496481 \\
33042	0.0236167487698245 \\
34222	0.0205711510841923 \\
35402	0.0179157454250034 \\
36582	0.0155787755381289 \\
37762	0.0138393060890174 \\
38942	0.0124580038723535 \\
40122	0.0111997968041693 \\
41302	0.0100949760987044 \\
42482	0.00911677784502035 \\
43662	0.00825204184249018 \\
44842	0.00749528713296466 \\
46022	0.00679267691987639 \\
47202	0.00621185878376906 \\
48382	0.00569274332038339 \\
49562	0.0052172696504349 \\
50742	0.0047770496455416 \\
51922	0.00437487500888556 \\
53102	0.00400229916952582 \\
54282	0.00365688897055777 \\
55462	0.00333905434518761 \\
56642	0.00304351616399567 \\
57822	0.00276863158357127 \\
59002	0.00251289015608352 \\
60182	0.00227490015177534 \\
61362	0.00205337796696314 \\
62542	0.00184713876638681 \\
63722	0.00165508810005344 \\
64902	0.00147621435630918 \\
66082	0.00130958194336389 \\
67262	0.00115432511080221 \\
68442	0.00101436115272663 \\
69622	0.000886560809775006 \\
70802	0.000767754680443045 \\
71982	0.000657254487196612 \\
73162	0.000554436250764956 \\
74342	0.000458730763765569 \\
75522	0.000380563901046072 \\
76702	0.000354231743893429 \\
77882	0.000330028154346795 \\
79062	0.000307559983251715 \\
80242	0.000286677887933406 \\
81422	0.000267258833758444 \\
82602	0.000249191673668103 \\
83782	0.000232375075772839 \\
84962	0.000216716382850313 \\
86142	0.000202130682243784 \\
87322	0.000188540012100891 \\
88502	0.000175872675948374 \\
89682	0.000164062645786667 \\
90862	0.000153049038420505 \\
92042	0.00014277565303572 \\
93222	0.00013319056051897 \\
94402	0.00012424573690184 \\
95582	0.000115896734757204 \\
96762	0.00010810238749881 \\
97942	0.000100824542403843 \\
99122	9.40278188703569e-05 \\
100302	8.76793889641458e-05 \\
101482	8.17487777552972e-05 \\
102662	7.62076812973467e-05 \\
103842	7.10298003949172e-05 \\
105022	6.61906885511265e-05 \\
106202	6.16676126855009e-05 \\
107382	5.74394253829991e-05 \\
108562	5.34864475832952e-05 \\
109742	4.97906250842006e-05 \\
110922	4.63348713881095e-05 \\
112102	4.3102789826277e-05 \\
113282	4.00796431501149e-05 \\
114462	3.72516922026445e-05 \\
115122	3.46061250146024e-05 \\
115262	3.21309915328749e-05 \\
115402	2.98151433059113e-05 \\
115542	2.76481776869675e-05 \\
115682	2.56203861683235e-05 \\
115822	2.37227065095214e-05 \\
115962	2.1946678346485e-05 \\
116102	2.02844020076864e-05 \\
116242	1.87285002832271e-05 \\
116382	1.72720829193507e-05 \\
116522	1.59087136289982e-05 \\
116662	1.46323794268377e-05 \\
116802	1.34374621159616e-05 \\
116942	1.23187117660462e-05 \\
117082	1.12712220342037e-05 \\
117222	1.02904054301578e-05 \\
117362	9.37197909778309e-06 \\
117502	8.5119185377347e-06 \\
117642	7.70652354303847e-06 \\
117782	7.14416171915255e-06 \\
117922	6.62437062443066e-06 \\
118062	6.14315014629829e-06 \\
118202	5.69727302968115e-06 \\
118342	5.28394515780795e-06 \\
118482	4.90064621344599e-06 \\
118622	4.54497102053697e-06 \\
118762	4.21483837059755e-06 \\
118902	3.9228681351644e-06 \\
119042	3.65313583683413e-06 \\
119182	3.40279101473939e-06 \\
119322	3.17033945945733e-06 \\
119462	2.95433660041367e-06 \\
119602	2.75366178786873e-06 \\
119742	2.5671374495051e-06 \\
119882	2.3936974375971e-06 \\
120022	2.23242711622218e-06 \\
120162	2.08228851045211e-06 \\
120302	1.94260869823548e-06 \\
120442	1.81258600845391e-06 \\
120582	1.69152413032059e-06 \\
120722	1.57878007883694e-06 \\
120862	1.47375971576436e-06 \\
121002	1.37591370924506e-06 \\
121142	1.28473387472905e-06 \\
121282	1.19977493989731e-06 \\
121422	1.12055232437402e-06 \\
121562	1.04666005540377e-06 \\
121702	9.77775817478932e-07 \\
121842	9.13527701817962e-07 \\
121982	8.53594146865699e-07 \\
122122	7.97677435493416e-07 \\
122262	7.45499366738844e-07 \\
122402	6.96803553745795e-07 \\
122542	6.51351260594346e-07 \\
122682	6.0892071451768e-07 \\
122822	5.69305842357259e-07 \\
122962	5.32315107937276e-07 \\
123102	4.97770440088807e-07 \\
123242	4.65506244773373e-07 \\
123382	4.35368494366983e-07 \\
123522	4.07213884723312e-07 \\
123662	3.80909059849444e-07 \\
123802	3.56329893091978e-07 \\
123942	3.33360824888995e-07 \\
124082	3.11894248372635e-07 \\
124222	2.91829942822286e-07 \\
124362	2.73074548473584e-07 \\
124502	2.55541080240729e-07 \\
124642	2.39148478076157e-07 \\
124782	2.23821190137308e-07 \\
124922	2.09488786262479e-07 \\
125062	1.96085600034923e-07 \\
125202	1.83550396659626e-07 \\
125342	1.71826064765401e-07 \\
125482	1.60859331521657e-07 \\
125622	1.50600496129361e-07 \\
125762	1.41003184683797e-07 \\
125902	1.32024121080043e-07 \\
126042	1.23622914738331e-07 \\
126182	1.15761863261898e-07 \\
126322	1.0840576891713e-07 \\
126462	1.01521768824941e-07 \\
126602	9.50791755327529e-08 \\
126742	8.90493302985185e-08 \\
126882	8.34054659226702e-08 \\
127022	7.8122578905937e-08 \\
127162	7.31773107665035e-08 \\
127302	6.85478377948634e-08 \\
127442	6.42137684137012e-08 \\
127582	6.01560464219553e-08 \\
127722	5.63568627320876e-08 \\
127862	5.27995718258012e-08 \\
128002	4.94686144825174e-08 \\
128142	4.63494453928348e-08 \\
128282	4.34284664896367e-08 \\
128422	4.06929634988451e-08 \\
128562	3.81310478747565e-08 \\
128702	3.57316023991139e-08 \\
128842	3.34842302218696e-08 \\
128982	3.13792076767072e-08 \\
129122	2.9407439483542e-08 \\
129262	2.75604182253808e-08 \\
129402	2.58301852684717e-08 \\
129542	2.42092954572115e-08 \\
129682	2.26907826972322e-08 \\
129822	2.1268129479779e-08 \\
129962	1.99352374052886e-08 \\
130102	1.86863995943476e-08 \\
130242	1.75162754301184e-08 \\
130382	1.64198669660998e-08 \\
130522	1.53924967771779e-08 \\
130662	1.44297860327214e-08 \\
130802	1.35276368995463e-08 \\
130942	1.26822133350579e-08 \\
131082	1.18899240453274e-08 \\
131222	1.11474070529916e-08 \\
131362	1.04515149312867e-08 \\
131502	9.79930120381667e-09 \\
131642	9.18800630023142e-09 \\
131782	8.61504712013073e-09 \\
131922	8.07800437652162e-09 \\
132062	7.57461277034466e-09 \\
132202	7.10275094295554e-09 \\
132342	6.66043215025169e-09 \\
132482	6.24579599151076e-09 \\
132622	5.85709952760638e-09 \\
132762	5.4927106751812e-09 \\
132902	5.15110104570837e-09 \\
133042	4.83083800739692e-09 \\
133182	4.53058057736655e-09 \\
133322	4.24907226070914e-09 \\
133462	3.98513588795169e-09 \\
133602	3.73766861905267e-09 \\
133742	3.50563744699883e-09 \\
133882	3.2880742573127e-09 \\
134022	3.08407255289467e-09 \\
134162	2.89278251353053e-09 \\
134302	2.71340805380049e-09 \\
134442	2.54520288178739e-09 \\
134582	2.38746850067528e-09 \\
134722	2.23954949030158e-09 \\
134862	2.10083234142289e-09 \\
135002	1.97074179197898e-09 \\
135142	1.84873866215796e-09 \\
135282	1.73431768946131e-09 \\
135422	1.62700569683594e-09 \\
135562	1.5263588171166e-09 \\
135702	1.43196110524713e-09 \\
135842	1.34342276192356e-09 \\
135982	1.26037846825966e-09 \\
136122	1.18248572045232e-09 \\
136262	1.10942344200282e-09 \\
136402	1.04089076247149e-09 \\
136542	9.76605629698923e-10 \\
136682	9.16303810605257e-10 \\
136822	8.59737003811034e-10 \\
136962	8.06673117192958e-10 \\
137102	7.56894269482444e-10 \\
137242	7.10196179642963e-10 \\
137382	6.66387334202767e-10 \\
137522	6.25288210098773e-10 \\
137662	5.86730608542751e-10 \\
137802	5.50556544798297e-10 \\
137942	5.1661808164738e-10 \\
138082	4.84776552234223e-10 \\
138222	4.54901893931492e-10 \\
138362	4.26871926695327e-10 \\
138502	4.00572519598796e-10 \\
138642	3.75896036519663e-10 \\
138782	3.52742390852256e-10 \\
138922	3.31016880572577e-10 \\
139062	3.10631131927863e-10 \\
139202	2.9150237779163e-10 \\
139342	2.73552347440642e-10 \\
139482	2.56708709844844e-10 \\
139622	2.40902464643256e-10 \\
139762	2.26069829523112e-10 \\
139902	2.12150463863026e-10 \\
140042	1.99087912822193e-10 \\
140182	1.86829274273492e-10 \\
140322	1.75324810225419e-10 \\
140462	1.64527891310939e-10 \\
140602	1.54395052298639e-10 \\
140742	1.44885103914305e-10 \\
140882	1.35959743463587e-10 \\
141022	1.27582722164732e-10 \\
141162	1.19720344748941e-10 \\
141302	1.1234080332656e-10 \\
141442	1.05414510453983e-10 \\
141582	9.8913321977534e-11 \\
141722	9.28110366338331e-11 \\
141862	8.70832850274894e-11 \\
142002	8.17069190084396e-11 \\
142142	7.66600116719474e-11 \\
142282	7.19228565593255e-11 \\
142422	6.74758027230382e-11 \\
142562	6.33014751727501e-11 \\
142702	5.93826654515794e-11 \\
142842	5.57041079929377e-11 \\
142982	5.22507592748411e-11 \\
143122	4.90087970206332e-11 \\
143262	4.5965287132077e-11 \\
143402	4.31080171559017e-11 \\
143542	4.04255517949537e-11 \\
143682	3.7907121885894e-11 \\
143822	3.5542735421501e-11 \\
143962	3.3322844483763e-11 \\
144102	3.12387338219366e-11 \\
144242	2.92819102298836e-11 \\
144382	2.74446576575826e-11 \\
144522	2.57196486330713e-11 \\
144662	2.40998887512944e-11 \\
144802	2.25790497410117e-11 \\
144942	2.115113639789e-11 \\
145082	1.98102090287477e-11 \\
145222	1.85511606076716e-11 \\
145362	1.73689396198995e-11 \\
145502	1.62587165952743e-11 \\
145642	1.52162171751513e-11 \\
145782	1.42373335343393e-11 \\
145922	1.33180688699497e-11 \\
146062	1.24547594460012e-11 \\
146202	1.16441856157223e-11 \\
146342	1.08827946654344e-11 \\
146482	1.01678110375758e-11 \\
146622	9.49629264113128e-12 \\
146762	8.86568596314419e-12 \\
146902	8.2734374906579e-12 \\
147042	7.71727126647193e-12 \\
147182	7.19491133338579e-12 \\
147322	6.70430377880393e-12 \\
147462	6.24350571243326e-12 \\
147602	5.81068526628314e-12 \\
147742	5.40417710581664e-12 \\
147882	5.02248242995051e-12 \\
148022	4.66382488184536e-12 \\
148162	4.32703872732532e-12 \\
148302	4.01062516530715e-12 \\
148442	3.71336295046376e-12 \\
148582	3.434252882073e-12 \\
148722	3.17207371480777e-12 \\
148862	2.92571522564344e-12 \\
149002	2.69434474731156e-12 \\
149142	2.47701859024119e-12 \\
149282	2.27284857601262e-12 \\
149422	2.08111305965986e-12 \\
149562	1.90081284046073e-12 \\
149702	1.73155934035663e-12 \\
149842	1.57263091438153e-12 \\
149982	1.42313938411576e-12 \\
150122	1.28280719380314e-12 \\
150262	1.15102372078013e-12 \\
150402	1.02706732008073e-12 \\
150542	9.10826969402478e-13 \\
150682	8.01358979174438e-13 \\
150822	6.98718860547842e-13 \\
150962	6.02240479707916e-13 \\
151102	5.11535258596041e-13 \\
151242	4.26381152607291e-13 \\
151382	3.46334072531818e-13 \\
151522	2.71116462613463e-13 \\
151662	2.00339744793609e-13 \\
151802	1.33948407921025e-13 \\
151942	7.16648962395539e-14 \\
152082	1.29896093881143e-14 \\
152222	-4.22439860869872e-14 \\
152362	-9.3869356732057e-14 \\
152502	-1.42497125210639e-13 \\
152642	-1.88238313825195e-13 \\
152782	-2.31148433726958e-13 \\
152922	-2.71671574125776e-13 \\
153062	-3.09530179265494e-13 \\
153202	-3.45168338355961e-13 \\
153342	-3.7864156254841e-13 \\
153482	-4.10171896447764e-13 \\
153622	-4.39703828902793e-13 \\
153762	-4.67514915669653e-13 \\
153902	-4.93549645597113e-13 \\
154042	-5.18252107895023e-13 \\
154182	-5.41289235655995e-13 \\
154322	-5.62938584636186e-13 \\
154462	-5.83311177138057e-13 \\
154602	-6.02518035464072e-13 \\
154742	-6.20559159614231e-13 \\
154882	-6.37379038437302e-13 \\
155022	-6.53310738840673e-13 \\
155162	-6.68354260824344e-13 \\
155302	-6.8228755978339e-13 \\
155442	-6.95499213776429e-13 \\
155582	-7.07989222803462e-13 \\
155722	-7.19591053410795e-13 \\
155862	-7.30693283657047e-13 \\
156002	-7.40907335483598e-13 \\
156142	-7.50621786949068e-13 \\
156282	-7.59725615750995e-13 \\
156422	-7.6832984419184e-13 \\
156562	-7.76378961120372e-13 \\
156702	-7.83983988839054e-13 \\
156842	-7.91033905045424e-13 \\
156982	-7.97695243193175e-13 \\
157122	-8.04023514433538e-13 \\
157262	-8.09907696464052e-13 \\
157402	-8.15458811587177e-13 \\
};
\addplot [line width=0.01pt, red, forget plot]
table [row sep=\\]{%
1181	2.60861604632583 \\
2361	2.16847659072101 \\
3541	1.79911343906749 \\
4721	1.47735854514669 \\
5901	1.20042933594887 \\
7081	0.969684013187687 \\
8261	0.782157746871778 \\
9441	0.651141066097212 \\
10621	0.548432417148831 \\
11801	0.466000993909274 \\
12981	0.399615776568678 \\
14161	0.344811817757295 \\
15341	0.295834632577256 \\
16521	0.251539631225085 \\
17701	0.213560897379088 \\
18881	0.184165651470042 \\
20061	0.160508980932604 \\
21241	0.142151965698788 \\
22421	0.126387105013888 \\
23601	0.112294358729016 \\
24781	0.099589859570514 \\
25961	0.0877628458173014 \\
27141	0.0771599588502414 \\
28321	0.0676100796127562 \\
29501	0.0595261029250156 \\
30681	0.0522042972620791 \\
31861	0.0454247355506744 \\
33041	0.039078817171656 \\
34221	0.0338764935349706 \\
35401	0.0293109018816821 \\
36581	0.0250744150563858 \\
37761	0.0212536751351124 \\
38941	0.0188479110950079 \\
40121	0.0168548534790293 \\
41301	0.0149912680753371 \\
42481	0.0133079725809264 \\
43661	0.0117719147515006 \\
44841	0.0104679771719828 \\
46021	0.00936770004880189 \\
47201	0.00844800419865988 \\
48381	0.00761134872464236 \\
49561	0.00687765852844358 \\
50741	0.00619287136038793 \\
51921	0.00556752473692762 \\
53101	0.00506705500624438 \\
54281	0.00460960061043669 \\
55461	0.00418980887133841 \\
56641	0.00380353988408838 \\
57821	0.0034488615239644 \\
59001	0.00312402479278923 \\
60181	0.00286128078898457 \\
61361	0.0026194392602174 \\
62541	0.00239638644013129 \\
63721	0.0021904648547566 \\
64901	0.00200020408754842 \\
66081	0.00182436324732405 \\
67261	0.001665911324854 \\
68441	0.0015267143869318 \\
69621	0.00140222777633214 \\
70801	0.00128877918840647 \\
71981	0.00118523169823248 \\
73161	0.00109061310558162 \\
74341	0.00100406801324615 \\
75521	0.000924838393168259 \\
76701	0.0008522492126945 \\
77881	0.000785696902366573 \\
79061	0.000724639937446814 \\
80241	0.000668591050813627 \\
81421	0.000617110721534475 \\
82601	0.000569801672444137 \\
83781	0.000526304175053005 \\
84961	0.000486292007978395 \\
86141	0.000449468950534937 \\
87321	0.00041556571951229 \\
88501	0.000384337276934654 \\
89681	0.000355560451504755 \\
90861	0.000329031827757564 \\
92041	0.000304565865616391 \\
93221	0.000281993219744647 \\
94401	0.000261159233303054 \\
95581	0.000241922584840182 \\
96761	0.000224154070311733 \\
97941	0.000207735504863915 \\
99121	0.000192558731149817 \\
100301	0.00017852472271751 \\
101481	0.000165542772461669 \\
102661	0.000153529757366599 \\
103841	0.000142409471800831 \\
105021	0.000132112022514397 \\
106201	0.000122573279253413 \\
107381	0.000113734375568553 \\
108561	0.000105541254970809 \\
109741	9.7944258096172e-05 \\
110921	9.12139589672178e-05 \\
112101	8.51941869663975e-05 \\
113281	7.95912939408061e-05 \\
114461	7.43738595331833e-05 \\
115011	6.9513241173158e-05 \\
115141	6.49832304646103e-05 \\
115271	6.07597916837399e-05 \\
115401	5.68208400593129e-05 \\
115531	5.31460506073e-05 \\
115661	4.97166917988179e-05 \\
115791	4.6515479701803e-05 \\
115921	4.3526449185205e-05 \\
116051	4.07348394891871e-05 \\
116181	3.81269920127769e-05 \\
116311	3.56902585872954e-05 \\
116441	3.34129188263588e-05 \\
116571	3.12841053943802e-05 \\
116701	2.92937362316725e-05 \\
116831	2.74324529292436e-05 \\
116961	2.56915645690525e-05 \\
117091	2.40629964455286e-05 \\
117221	2.25392431665861e-05 \\
117351	2.11133256929341e-05 \\
117481	1.97787519363724e-05 \\
117611	1.8529480579732e-05 \\
117741	1.73598878183134e-05 \\
117871	1.62647367610291e-05 \\
118001	1.52391492515003e-05 \\
118131	1.42785798998291e-05 \\
118261	1.33787921326456e-05 \\
118391	1.25358360896222e-05 \\
118521	1.17475221695362e-05 \\
118651	1.10277875559062e-05 \\
118781	1.03531267527379e-05 \\
118911	9.72057028542084e-06 \\
119041	9.12739528047268e-06 \\
119171	8.571068526253e-06 \\
119301	8.04922958153043e-06 \\
119431	7.5596775917508e-06 \\
119561	7.10035942119713e-06 \\
119691	6.66935882737141e-06 \\
119821	6.2648865552517e-06 \\
119951	5.88527125383598e-06 \\
120081	5.52895113054008e-06 \\
120211	5.19446627061937e-06 \\
120341	4.88045155810912e-06 \\
120471	4.58563014249513e-06 \\
120601	4.30880740159845e-06 \\
120731	4.04886535626536e-06 \\
120861	3.80475750016984e-06 \\
120991	3.57596203548471e-06 \\
121121	3.3601812739481e-06 \\
121251	3.15794218225252e-06 \\
121381	2.96797514803826e-06 \\
121511	2.78952566429291e-06 \\
121641	2.6218866250316e-06 \\
121771	2.4643952359904e-06 \\
121901	2.31643014253491e-06 \\
122031	2.17740875407824e-06 \\
122161	2.04678475024256e-06 \\
122291	1.92404575644067e-06 \\
122421	1.80871117427817e-06 \\
122551	1.70033015806093e-06 \\
122681	1.59847972558413e-06 \\
122811	1.50276299326624e-06 \\
122941	1.412807528689e-06 \\
123071	1.32826381021856e-06 \\
123201	1.24880378710168e-06 \\
123331	1.17411953404201e-06 \\
123461	1.10392199226261e-06 \\
123591	1.03793979194799e-06 \\
123721	9.75918150569743e-07 \\
123851	9.17617841433849e-07 \\
123981	8.62814228730358e-07 \\
124111	8.11296363367742e-07 \\
124241	7.62866137038465e-07 \\
124371	7.1733748890912e-07 \\
124501	6.74535663380826e-07 \\
124631	6.34296513868371e-07 \\
124761	5.96465850710715e-07 \\
124891	5.60898829882195e-07 \\
125021	5.27459380450512e-07 \\
125151	4.96019666396119e-07 \\
125281	4.664595837367e-07 \\
125411	4.38666287794209e-07 \\
125541	4.12533750271393e-07 \\
125671	3.87962343084691e-07 \\
125801	3.64858449564132e-07 \\
125931	3.431340974136e-07 \\
126061	3.22706616595614e-07 \\
126191	3.03498316311934e-07 \\
126321	2.85436183244947e-07 \\
126451	2.68451598006703e-07 \\
126581	2.52480068241301e-07 \\
126711	2.37460979213289e-07 \\
126841	2.23337359051001e-07 \\
126971	2.1005565781218e-07 \\
127101	1.97565541593114e-07 \\
127231	1.85819697517964e-07 \\
127361	1.74773651662186e-07 \\
127491	1.64385597689609e-07 \\
127621	1.54616236147653e-07 \\
127751	1.45428622866373e-07 \\
127881	1.367880277936e-07 \\
128011	1.28661800880003e-07 \\
128141	1.21019247734111e-07 \\
128271	1.13831511161511e-07 \\
128401	1.07071461308283e-07 \\
128531	1.00713591411061e-07 \\
128661	9.47339205414899e-08 \\
128791	8.91099016797625e-08 \\
128921	8.38203361164247e-08 \\
129051	7.8845291906493e-08 \\
129181	7.4166028540823e-08 \\
129311	6.97649248371235e-08 \\
129441	6.56254121600419e-08 \\
129571	6.17319113049852e-08 \\
129701	5.80697733787439e-08 \\
129831	5.46252237887401e-08 \\
129961	5.13853098960126e-08 \\
130091	4.83378520543809e-08 \\
130221	4.54713970921006e-08 \\
130351	4.2775175179699e-08 \\
130481	4.02390585296786e-08 \\
130611	3.78535231493338e-08 \\
130741	3.56096129805472e-08 \\
130871	3.34989057049206e-08 \\
131001	3.15134807693518e-08 \\
131131	2.96458898541019e-08 \\
131261	2.78891280625082e-08 \\
131391	2.62366077752318e-08 \\
131521	2.46821337812619e-08 \\
131651	2.32198796856764e-08 \\
131781	2.18443661492707e-08 \\
131911	2.05504396832978e-08 \\
132041	1.93332538311886e-08 \\
132171	1.81882504057818e-08 \\
132301	1.71111425029125e-08 \\
132431	1.6097898070111e-08 \\
132561	1.5144725251659e-08 \\
132691	1.42480574560899e-08 \\
132821	1.3404540200046e-08 \\
132951	1.26110186182693e-08 \\
133081	1.18645253621708e-08 \\
133211	1.11622694420888e-08 \\
133341	1.05016256801704e-08 \\
133471	9.88012532898708e-09 \\
133601	9.29544563543772e-09 \\
133731	8.74540267981061e-09 \\
133861	8.22794166133178e-09 \\
133991	7.74113001478227e-09 \\
134121	7.28315024955961e-09 \\
134251	6.85229256669473e-09 \\
134381	6.44694880813645e-09 \\
134511	6.06560596194683e-09 \\
134641	5.7068413883421e-09 \\
134771	5.36931665795493e-09 \\
134901	5.05177283338654e-09 \\
135031	4.75302541769196e-09 \\
135161	4.47196063513289e-09 \\
135291	4.20753060170753e-09 \\
135421	3.95874971692578e-09 \\
135551	3.72469066700631e-09 \\
135681	3.50448120522984e-09 \\
135811	3.29730109882576e-09 \\
135941	3.10237840972505e-09 \\
136071	2.91898716309191e-09 \\
136201	2.7464444607439e-09 \\
136331	2.58410787212782e-09 \\
136461	2.43137288080675e-09 \\
136591	2.28767099708094e-09 \\
136721	2.15246759305288e-09 \\
136851	2.0252592936032e-09 \\
136981	1.90557286616766e-09 \\
137111	1.79296322233569e-09 \\
137241	1.68701130842663e-09 \\
137371	1.58732310628906e-09 \\
137501	1.49352813449966e-09 \\
137631	1.40527733893947e-09 \\
137761	1.32224292626049e-09 \\
137891	1.24411614343956e-09 \\
138021	1.17060677817804e-09 \\
138151	1.10144127152267e-09 \\
138281	1.03636310644362e-09 \\
138411	9.75130420854953e-10 \\
138541	9.17515563525484e-10 \\
138671	8.63304594478365e-10 \\
138801	8.12296507834986e-10 \\
138931	7.64301566480441e-10 \\
139061	7.19141690641578e-10 \\
139191	6.76649458686285e-10 \\
139321	6.36666996900459e-10 \\
139451	5.99045646421104e-10 \\
139581	5.63646351814384e-10 \\
139711	5.3033744062958e-10 \\
139841	4.98995456066353e-10 \\
139971	4.69504046751723e-10 \\
140101	4.41754022251217e-10 \\
140231	4.15642242845848e-10 \\
140361	3.91072008110172e-10 \\
140491	3.67952279756167e-10 \\
140621	3.46197237544033e-10 \\
140751	3.25726501326784e-10 \\
140881	3.06463965316084e-10 \\
141011	2.88338186660297e-10 \\
141141	2.71282274422191e-10 \\
141271	2.5523289037821e-10 \\
141401	2.40130582085385e-10 \\
141531	2.25919394303276e-10 \\
141661	2.12546757971666e-10 \\
141791	1.99963212654808e-10 \\
141921	1.88121962452215e-10 \\
142051	1.76979153554413e-10 \\
142181	1.66493929754097e-10 \\
142311	1.56627211200799e-10 \\
142441	1.47342305023557e-10 \\
142571	1.38605238397815e-10 \\
142701	1.30383481788954e-10 \\
142831	1.22646559574946e-10 \\
142961	1.15365828001757e-10 \\
143091	1.08514586205644e-10 \\
143221	1.0206729905704e-10 \\
143351	9.60001522720688e-11 \\
143481	9.02908303679339e-11 \\
143611	8.49180725737142e-11 \\
143741	7.98621724307225e-11 \\
143871	7.51040896140864e-11 \\
144001	7.06267822003781e-11 \\
144131	6.64133192884719e-11 \\
144261	6.24481577560232e-11 \\
144391	5.87167536814093e-11 \\
144521	5.52053402991248e-11 \\
144651	5.1900761466328e-11 \\
144781	4.87909157520505e-11 \\
144911	4.58645343925923e-11 \\
145041	4.31104041354047e-11 \\
145171	4.05187550178709e-11 \\
145301	3.80797060550719e-11 \\
145431	3.57843754628107e-11 \\
145561	3.36242700349487e-11 \\
145691	3.15914516768601e-11 \\
145821	2.96784263831285e-11 \\
145951	2.78780887263963e-11 \\
146081	2.61837218573646e-11 \\
146211	2.45891640382467e-11 \\
146341	2.30886421093146e-11 \\
146471	2.16764939331426e-11 \\
146601	2.03474459503639e-11 \\
146731	1.90967797131236e-11 \\
146861	1.79196657512648e-11 \\
146991	1.68119407284451e-11 \\
147121	1.57693302860196e-11 \\
147251	1.47882261991583e-11 \\
147381	1.38648537095776e-11 \\
147511	1.29959376593547e-11 \\
147641	1.21779808459621e-11 \\
147771	1.1408374245292e-11 \\
147901	1.06840647440265e-11 \\
148031	1.00023322957554e-11 \\
148161	9.36067889867331e-12 \\
148291	8.75693961788215e-12 \\
148421	8.1885609404253e-12 \\
148551	7.65376650946337e-12 \\
148681	7.15044690124955e-12 \\
148811	6.67677024779323e-12 \\
148941	6.23090468110377e-12 \\
149071	5.81129588894669e-12 \\
149201	5.41633404793629e-12 \\
149331	5.04474240159425e-12 \\
149461	4.69502214883732e-12 \\
149591	4.36567448858227e-12 \\
149721	4.05586675356062e-12 \\
149851	3.76432218729406e-12 \\
149981	3.48987505560672e-12 \\
150111	3.23158166892767e-12 \\
150241	2.98838731538353e-12 \\
150371	2.75962586115952e-12 \\
150501	2.54429810553347e-12 \\
150631	2.34162689238815e-12 \\
150761	2.15094608790878e-12 \\
150891	1.97136751367566e-12 \\
151021	1.80228054702525e-12 \\
151151	1.64329660989893e-12 \\
151281	1.49363854617945e-12 \\
151411	1.35275124435452e-12 \\
151541	1.22019061521428e-12 \\
151671	1.09534603609518e-12 \\
151801	9.77884440089838e-13 \\
151931	8.67361737988404e-13 \\
152061	7.63278329429795e-13 \\
152191	6.65412169809088e-13 \\
152321	5.732636587652e-13 \\
152451	4.86499729390744e-13 \\
152581	4.04898337080795e-13 \\
152711	3.2790437032304e-13 \\
152841	2.55739873722405e-13 \\
152971	1.87627691161651e-13 \\
153101	1.23456800338317e-13 \\
153231	6.31716901011714e-14 \\
153361	6.38378239159465e-15 \\
153491	-4.70179450928754e-14 \\
153621	-9.72555369571637e-14 \\
153751	-1.44662060108658e-13 \\
153881	-1.89126492244895e-13 \\
154011	-2.31037411424495e-13 \\
154141	-2.70561351101151e-13 \\
154271	-3.07698311274862e-13 \\
};
\addplot [line width=0.01pt, red, forget plot]
table [row sep=\\]{%
1182	2.53006311764127 \\
2352	2.07936087752551 \\
3522	1.71266547564717 \\
4692	1.40634736380482 \\
5862	1.1501212280517 \\
7032	0.94851522200473 \\
8202	0.783035073728376 \\
9372	0.641039828285147 \\
10542	0.530209881073009 \\
11712	0.438958671339568 \\
12882	0.364137026136876 \\
14052	0.302207075510571 \\
15222	0.250901091783352 \\
16392	0.208553736398417 \\
17562	0.17520228942197 \\
18732	0.147978476843786 \\
19902	0.125699325222043 \\
21072	0.106995528999265 \\
22242	0.0954609643083634 \\
23412	0.0854520663799471 \\
24582	0.0764489125629639 \\
25752	0.0681469950210005 \\
26922	0.0604901524137398 \\
28092	0.0542696168542559 \\
29262	0.0484929349700686 \\
30432	0.0431138070438106 \\
31602	0.0381054784001773 \\
32772	0.0335707504446585 \\
33942	0.0299468455076917 \\
35112	0.0266449118874243 \\
36282	0.0235582994077814 \\
37452	0.0206717527326233 \\
38622	0.0179928147637755 \\
39792	0.0155404821902099 \\
40962	0.0132589464142848 \\
42132	0.0111263175475806 \\
43302	0.00919012496338528 \\
44472	0.00767221069616447 \\
45642	0.00684898088966124 \\
46812	0.00621386390391437 \\
47982	0.00564389486851485 \\
49152	0.00511687876440869 \\
50322	0.00462917367277771 \\
51492	0.0041775510676898 \\
52662	0.00375908743015108 \\
53832	0.00337112897586228 \\
55002	0.00301214880429101 \\
56172	0.00268138193045253 \\
57342	0.00237437522271583 \\
58512	0.00209347560356898 \\
59682	0.00183874769393705 \\
60852	0.00160993660205733 \\
62022	0.00140089302614549 \\
63192	0.00120686786979918 \\
64362	0.00108835787450573 \\
65532	0.00101045118964044 \\
66702	0.000938363340506898 \\
67872	0.000871535212318197 \\
69042	0.000809537912982161 \\
70212	0.000751985649012576 \\
71382	0.000699577532367435 \\
72552	0.000655009732300982 \\
73722	0.000613436225509301 \\
74892	0.000574564192653504 \\
76062	0.000538201793155391 \\
77232	0.000504176689558433 \\
78402	0.000472330016822953 \\
79572	0.00044254903359181 \\
80742	0.000414916573790503 \\
81912	0.000389035656081171 \\
83082	0.000364789130512022 \\
84252	0.000342069017948698 \\
85422	0.000320775167996057 \\
86592	0.000300814565539154 \\
87762	0.000282100740264468 \\
88932	0.000264553237990739 \\
90102	0.000248097144335324 \\
91272	0.000232662654191196 \\
92442	0.000218184681541334 \\
93612	0.000204602504922047 \\
94782	0.000191859444499176 \\
95952	0.000179902567273127 \\
97122	0.000168682417393984 \\
98292	0.00015815276895953 \\
99462	0.000148270399006944 \\
100632	0.000138994878692345 \\
101802	0.000130288380896071 \\
102972	0.000122115502701858 \\
104142	0.00011444310137837 \\
105312	0.00010724014264385 \\
106482	0.000100477560133372 \\
107652	9.4128125102344e-05 \\
108822	8.81663254988507e-05 \\
109992	8.25682536332861e-05 \\
111162	7.7311501743893e-05 \\
112332	7.23750648299326e-05 \\
113502	6.77392501817198e-05 \\
114672	6.3385593092713e-05 \\
115842	5.92967782821474e-05 \\
117012	5.54565666035511e-05 \\
118182	5.18497266461249e-05 \\
119352	4.84619708756573e-05 \\
120522	4.52798959855172e-05 \\
121692	4.2290927160904e-05 \\
122862	3.94832659779132e-05 \\
124032	3.68458416872275e-05 \\
125202	3.43682656464006e-05 \\
126372	3.20407886859608e-05 \\
127542	2.9854261210327e-05 \\
128712	2.78000958498503e-05 \\
129882	2.58702324903992e-05 \\
131052	2.40571055271088e-05 \\
132222	2.23536131898627e-05 \\
133392	2.07530888091667e-05 \\
134562	1.92492738921302e-05 \\
135732	1.78362928916487e-05 \\
136902	1.65086295607075e-05 \\
138072	1.52611036067229e-05 \\
139242	1.40888442632381e-05 \\
140412	1.29872772099415e-05 \\
141582	1.19521092458719e-05 \\
142752	1.09793106646827e-05 \\
143922	1.00650987772677e-05 \\
145092	9.20592250525099e-06 \\
146262	8.39844796901579e-06 \\
147432	7.63954500260233e-06 \\
148602	6.92627452852923e-06 \\
149772	6.25587673785954e-06 \\
150942	5.62576001822412e-06 \\
152112	5.15152613295156e-06 \\
153282	4.77321369091177e-06 \\
154452	4.42326059912901e-06 \\
155622	4.09919941862791e-06 \\
156792	3.81008538280492e-06 \\
157962	3.54469139690616e-06 \\
159132	3.29868633947017e-06 \\
160302	3.07047059316945e-06 \\
161472	2.85861193605186e-06 \\
162642	2.6618191217298e-06 \\
163812	2.47918904161537e-06 \\
164982	2.30978422544048e-06 \\
166152	2.15240859396948e-06 \\
167322	2.00615686724026e-06 \\
168492	1.87019904135388e-06 \\
169662	1.74377285300231e-06 \\
170832	1.62617728316494e-06 \\
172002	1.51676687026869e-06 \\
173172	1.41494669630982e-06 \\
174342	1.32016793785716e-06 \\
175512	1.23192389522853e-06 \\
176682	1.14974643400378e-06 \\
177852	1.07320278075473e-06 \\
179022	1.00189262780548e-06 \\
180192	9.35445509941957e-07 \\
181362	8.73518420096975e-07 \\
182532	8.15793638420104e-07 \\
183702	7.61976751251225e-07 \\
184872	7.11794839125535e-07 \\
186042	6.64994820653853e-07 \\
187212	6.21341931572594e-07 \\
188382	5.80618331802452e-07 \\
189552	5.42621823751421e-07 \\
190722	5.07164676866179e-07 \\
191892	4.74072542888671e-07 \\
193062	4.43183460263619e-07 \\
194232	4.14346935206922e-07 \\
195402	3.87423095216111e-07 \\
196572	3.62281908472539e-07 \\
197742	3.38802463362153e-07 \\
198912	3.16872303840476e-07 \\
200082	2.96386814535587e-07 \\
201252	2.77248654423357e-07 \\
202422	2.59367231414398e-07 \\
203592	2.42658218019276e-07 \\
204762	2.27043102318802e-07 \\
205932	2.12448773406759e-07 \\
207102	1.9880713714171e-07 \\
208272	1.86054759765408e-07 \\
209442	1.74132539443317e-07 \\
210612	1.62985400398163e-07 \\
211782	1.52562010191648e-07 \\
212952	1.42814517323231e-07 \\
214122	1.33698308479868e-07 \\
215292	1.2517178227256e-07 \\
216462	1.1719614051442e-07 \\
217632	1.09735194153693e-07 \\
218802	1.02755182196379e-07 \\
219972	9.62246049507343e-08 \\
221142	9.01140683740032e-08 \\
222312	8.43961395768922e-08 \\
223482	7.90452127641394e-08 \\
224652	7.40373844454467e-08 \\
225822	6.93503371396176e-08 \\
226992	6.4963232404569e-08 \\
228162	6.08566095849916e-08 \\
229332	5.70122933307715e-08 \\
230502	5.34133064999942e-08 \\
231672	5.00437891681749e-08 \\
232842	4.68889240212711e-08 \\
234012	4.39348660785654e-08 \\
235182	4.11686777446185e-08 \\
236352	3.8578267524958e-08 \\
237522	3.61523344594161e-08 \\
238692	3.3880314886936e-08 \\
239862	3.1752333540247e-08 \\
241032	2.97591581377432e-08 \\
242202	2.7892155862741e-08 \\
243372	2.6143254949762e-08 \\
244542	2.45049066038838e-08 \\
245712	2.29700511389375e-08 \\
246882	2.15320856145063e-08 \\
248052	2.01848331937704e-08 \\
249222	1.89225162761097e-08 \\
250392	1.77397296297066e-08 \\
251562	1.66314158001057e-08 \\
252732	1.55928430722874e-08 \\
253902	1.46195835992735e-08 \\
255072	1.37074938066917e-08 \\
256242	1.28526962406283e-08 \\
257412	1.20515614154826e-08 \\
258582	1.13006923818659e-08 \\
259752	1.05969096830805e-08 \\
260922	9.93723670017488e-09 \\
262092	9.31888710642426e-09 \\
263262	8.73925221078764e-09 \\
264432	8.19588957812201e-09 \\
265602	7.68651242655238e-09 \\
266772	7.20897935790887e-09 \\
267942	6.76128525389785e-09 \\
269112	6.34155211676202e-09 \\
270282	5.94802168629727e-09 \\
271347	5.57904661357966e-09 \\
271467	5.23308441024994e-09 \\
271587	4.90869017655271e-09 \\
271707	4.6045106061321e-09 \\
271827	4.31927754673822e-09 \\
271947	4.0518037813797e-09 \\
272067	3.80097620045206e-09 \\
272187	3.56575263760206e-09 \\
272307	3.3451563741238e-09 \\
272427	3.13827247522269e-09 \\
272547	2.94424334912335e-09 \\
272667	2.76226569395632e-09 \\
272787	2.59158672299975e-09 \\
272907	2.43150088952149e-09 \\
273027	2.28134694468807e-09 \\
273147	2.14050538405175e-09 \\
273267	2.00839561648181e-09 \\
273387	1.88447335514041e-09 \\
273507	1.76822850805891e-09 \\
273627	1.65918273564714e-09 \\
273747	1.55688806291465e-09 \\
273867	1.46092421493549e-09 \\
273987	1.37089772866972e-09 \\
274107	1.2864391774059e-09 \\
274227	1.20720261564955e-09 \\
274347	1.13286358072173e-09 \\
274467	1.063117982536e-09 \\
274587	9.97680493775022e-10 \\
274707	9.3628377273447e-10 \\
274827	8.78677242077686e-10 \\
274947	8.24625867590356e-10 \\
275067	7.73909158979791e-10 \\
275187	7.26320559252258e-10 \\
275307	6.81666167956507e-10 \\
275427	6.39764297094558e-10 \\
275547	6.00444527432131e-10 \\
275667	5.63547208898285e-10 \\
275787	5.28922239340091e-10 \\
275907	4.96429286567235e-10 \\
276027	4.65936234039788e-10 \\
276147	4.37319958024318e-10 \\
276267	4.10464051636694e-10 \\
276387	3.85259990576259e-10 \\
276507	3.61605634324746e-10 \\
276627	3.39405448190888e-10 \\
276747	3.1856955962084e-10 \\
276867	2.99013924731639e-10 \\
276987	2.80659495643931e-10 \\
277107	2.63432220481974e-10 \\
277227	2.47262654795577e-10 \\
277347	2.32085461959741e-10 \\
277467	2.17839690730415e-10 \\
277587	2.04467831554922e-10 \\
277707	1.91916260661174e-10 \\
277827	1.80134240856944e-10 \\
277947	1.69074476641384e-10 \\
278067	1.58692670115812e-10 \\
278187	1.48947076894501e-10 \\
278307	1.39798506104682e-10 \\
278427	1.3121015385309e-10 \\
278547	1.23147936292867e-10 \\
278667	1.15579101844787e-10 \\
278787	1.08473563464884e-10 \\
278907	1.01802899443726e-10 \\
279027	9.55402978952691e-11 \\
279147	8.96607232903079e-11 \\
279267	8.41407499230229e-11 \\
279387	7.89582288440727e-11 \\
279507	7.40925654163505e-11 \\
279627	6.95242752257741e-11 \\
279747	6.52350395924373e-11 \\
279867	6.12077610817607e-11 \\
279987	5.74264524821899e-11 \\
280107	5.38762368051948e-11 \\
280227	5.05425146180016e-11 \\
280347	4.74123518223735e-11 \\
280467	4.44732584092833e-11 \\
280587	4.17134660146701e-11 \\
280707	3.91218724082876e-11 \\
280827	3.66885966052166e-11 \\
280947	3.44035910870844e-11 \\
281067	3.22577520250888e-11 \\
281187	3.02428082576967e-11 \\
281307	2.83505996456768e-11 \\
281427	2.65736876947642e-11 \\
281547	2.49050224887526e-11 \\
281667	2.33380537117966e-11 \\
281787	2.18663975815048e-11 \\
281907	2.04843919604514e-11 \\
282027	1.91863747112109e-11 \\
282147	1.79675718747774e-11 \\
282267	1.68227654029351e-11 \\
282387	1.57476254258881e-11 \\
282507	1.47378220738403e-11 \\
282627	1.37895250773568e-11 \\
282747	1.28988486558512e-11 \\
282867	1.20623511179474e-11 \\
282987	1.12765907722689e-11 \\
283107	1.05387365501031e-11 \\
283227	9.84551329352712e-12 \\
283347	9.19453402303816e-12 \\
283467	8.58296766992339e-12 \\
283587	8.0086492992848e-12 \\
283707	7.4691364204682e-12 \\
283827	6.96237512087805e-12 \\
283947	6.48636699907001e-12 \\
284067	6.03928018705346e-12 \\
284187	5.619338327989e-12 \\
284307	5.22476506503722e-12 \\
284427	4.85422813056857e-12 \\
284547	4.50606219004612e-12 \\
284667	4.17904599814278e-12 \\
284787	3.87179177607777e-12 \\
284907	3.58330032312892e-12 \\
285027	3.31218386051546e-12 \\
285147	3.05755420981768e-12 \\
285267	2.81824563685973e-12 \\
285387	2.59359200782683e-12 \\
285507	2.38237207739189e-12 \\
285627	2.18397522289138e-12 \\
285747	1.99767979935928e-12 \\
285867	1.82254211722466e-12 \\
285987	1.65800706497521e-12 \\
286107	1.50351953109862e-12 \\
286227	1.35841338178011e-12 \\
286347	1.22196697205368e-12 \\
286467	1.09373621270947e-12 \\
286587	9.73388036840106e-13 \\
286707	8.60256310630803e-13 \\
286827	7.53952456022944e-13 \\
286947	6.54087894957911e-13 \\
287067	5.60274049377085e-13 \\
287187	4.72066830070617e-13 \\
287307	3.8924419243358e-13 \\
287427	3.11306536104894e-13 \\
287547	2.38253861084559e-13 \\
287667	1.6947554470903e-13 \\
287787	1.04916075827077e-13 \\
287907	4.42423875313125e-14 \\
288027	-1.2712053631958e-14 \\
288147	-6.63358257213531e-14 \\
288267	-1.16573417585641e-13 \\
288387	-1.63979940737136e-13 \\
288507	-2.08444372873373e-13 \\
288627	-2.50299780901742e-13 \\
288747	-2.89435142519778e-13 \\
288867	-3.26350058088565e-13 \\
288987	-3.6098901645687e-13 \\
289107	-3.93629573380849e-13 \\
289227	-4.24271728860504e-13 \\
289347	-4.52915482895833e-13 \\
289467	-4.80004924696686e-13 \\
289587	-5.05373520809371e-13 \\
289707	-5.29298826990043e-13 \\
289827	-5.51725332087472e-13 \\
289947	-5.72764058404118e-13 \\
290067	-5.92637050544909e-13 \\
290187	-6.1123328620738e-13 \\
290307	-6.28830321147689e-13 \\
290427	-6.4520611076091e-13 \\
290547	-6.60693721954431e-13 \\
290667	-6.7523764357702e-13 \\
290787	-6.8889338677991e-13 \\
290907	-7.01660951563099e-13 \\
291027	-7.13817893682744e-13 \\
291147	-7.25142168533921e-13 \\
291267	-7.35855820721554e-13 \\
291387	-7.4584782794318e-13 \\
291507	-7.55229212501263e-13 \\
};
\addplot [line width=0.01pt, red, forget plot]
table [row sep=\\]{%
1190	2.54861904137153 \\
2364	2.10359499075138 \\
3534	1.75653605916943 \\
4704	1.46704650582425 \\
5874	1.22143566750423 \\
7044	1.01860881700016 \\
8214	0.860199229589972 \\
9384	0.721231928962842 \\
10554	0.604761717011328 \\
11724	0.514357388630725 \\
12894	0.437699582821307 \\
14064	0.368420855972855 \\
15234	0.308121530887196 \\
16404	0.258690003226964 \\
17574	0.214182693465841 \\
18744	0.17816655849515 \\
19914	0.153806169569563 \\
21084	0.132439970259573 \\
22254	0.113358837582952 \\
23424	0.0971293369125605 \\
24594	0.0821058419039998 \\
25764	0.0681606670061991 \\
26934	0.0554473756167798 \\
28104	0.0451684818020187 \\
29274	0.0368074091317022 \\
30444	0.0310883543227407 \\
31614	0.0275226174639304 \\
32784	0.0242869693901301 \\
33954	0.0215624753777293 \\
35124	0.0189990616695288 \\
36294	0.0166363617536938 \\
37464	0.0144605888256905 \\
38634	0.0125259251942695 \\
39804	0.0109250806533633 \\
40974	0.00964528951217247 \\
42144	0.00845389517016581 \\
43314	0.00734360142058993 \\
44484	0.00632833297210172 \\
45654	0.00546709999425865 \\
46824	0.00498800606837013 \\
47994	0.00454802147325328 \\
49164	0.00413906509324402 \\
50334	0.00384687981085297 \\
51504	0.00359612248591584 \\
52674	0.00336168268004389 \\
53844	0.00314230770505808 \\
55014	0.00293690209423686 \\
56184	0.00274447248733545 \\
57354	0.00256411264425271 \\
58524	0.00239675858194216 \\
59694	0.00224606899158841 \\
60864	0.00210453519462084 \\
62034	0.00197166508954644 \\
63204	0.00184695722947681 \\
64374	0.00172972200359384 \\
65544	0.00161948287130481 \\
66714	0.00151579815483582 \\
67884	0.0014182575856162 \\
69054	0.0013264794793415 \\
70224	0.00124010827771398 \\
71394	0.00115881238351473 \\
72564	0.00108228223951673 \\
73734	0.00101022861265571 \\
74904	0.000942381052733698 \\
76074	0.000881311246480387 \\
77244	0.000823965527925019 \\
78414	0.000770010379808395 \\
79584	0.000719238190234284 \\
80754	0.000671454953167117 \\
81924	0.000626479233622912 \\
83094	0.000584141253845316 \\
84264	0.000544282061451185 \\
85434	0.000506752768640595 \\
86604	0.00047141385456162 \\
87774	0.000438134524149647 \\
88944	0.000406792117700894 \\
90114	0.000379524032394551 \\
91284	0.000356169599866174 \\
92454	0.000334434235328696 \\
93624	0.000314144079434753 \\
94794	0.000295124858877838 \\
95964	0.000277289800043101 \\
97134	0.000260560088412076 \\
98304	0.000244877610205585 \\
99474	0.000230159788516648 \\
100644	0.000216344195396179 \\
101814	0.000203373042727439 \\
102984	0.000191192689307174 \\
104154	0.000179753249436254 \\
105324	0.000169008259162606 \\
106494	0.00015891438710669 \\
107664	0.000149431181172388 \\
108834	0.000140520844548464 \\
110004	0.000132148035918711 \\
111174	0.000124279689924622 \\
112344	0.000116884854779953 \\
113514	0.000109934544579138 \\
114684	0.000103401604334363 \\
115854	9.72605861513376e-05 \\
117024	9.14876352446914e-05 \\
118194	8.60603847139818e-05 \\
119364	8.09578581814629e-05 \\
120534	7.61603795243482e-05 \\
121704	7.16494890510844e-05 \\
122874	6.74078655509835e-05 \\
124044	6.34192537229405e-05 \\
125214	5.96683965465306e-05 \\
126384	5.61409722061312e-05 \\
127554	5.28235352190154e-05 \\
128724	4.97034614547753e-05 \\
129894	4.67688967611957e-05 \\
131064	4.40087089376151e-05 \\
132234	4.14124428202989e-05 \\
133404	3.89702782621626e-05 \\
134574	3.66729908081176e-05 \\
135744	3.45119148824069e-05 \\
136914	3.2478909319289e-05 \\
138084	3.05663250781962e-05 \\
139254	2.87669750002606e-05 \\
140424	2.70741054710921e-05 \\
141594	2.5481369863023e-05 \\
142764	2.3982803643352e-05 \\
143934	2.25728010379567e-05 \\
145104	2.12460931507952e-05 \\
146274	1.99977274454866e-05 \\
147444	1.88230485009822e-05 \\
148614	1.7717679960505e-05 \\
149784	1.66775075971515e-05 \\
150954	1.56986634253231e-05 \\
152124	1.47775107922077e-05 \\
153294	1.39106303852499e-05 \\
154464	1.30948071027648e-05 \\
155634	1.23270177262436e-05 \\
156804	1.16044193508857e-05 \\
157974	1.09243385227331e-05 \\
159144	1.02842610393283e-05 \\
160314	9.68182237359727e-06 \\
161484	9.11479868032083e-06 \\
162654	8.58109835077947e-06 \\
163824	8.07875408009862e-06 \\
164994	7.60591541665301e-06 \\
166164	7.16084176377585e-06 \\
167334	6.74189580435192e-06 \\
168504	6.34753732431381e-06 \\
169674	5.97631740978377e-06 \\
170844	5.62687299332509e-06 \\
172014	5.29792173092902e-06 \\
173184	4.98825718681095e-06 \\
174354	4.69674430908507e-06 \\
175524	4.42231517605585e-06 \\
176694	4.16396500108052e-06 \\
177864	3.9207483739645e-06 \\
179034	3.69177573172896e-06 \\
180204	3.47621003748966e-06 \\
181374	3.27326366139635e-06 \\
182544	3.08219544609134e-06 \\
183714	2.9023079496926e-06 \\
184884	2.73294485203524e-06 \\
186054	2.57348851712136e-06 \\
187224	2.4233577005095e-06 \\
188394	2.28200539398316e-06 \\
189564	2.14891679833906e-06 \\
190734	2.02360741724528e-06 \\
191904	1.90562126517468e-06 \\
193074	1.7945291808652e-06 \\
194244	1.68992724208783e-06 \\
195414	1.59143527483918e-06 \\
196584	1.49869545029713e-06 \\
197754	1.41137096582034e-06 \\
198924	1.329144804052e-06 \\
200094	1.25171856540929e-06 \\
201264	1.17881137107201e-06 \\
202434	1.11015882880983e-06 \\
203604	1.04551206209225e-06 \\
204774	9.84636796208527e-07 \\
205944	9.2731249762279e-07 \\
207114	8.73331565509616e-07 \\
208284	8.22498570030028e-07 \\
209454	7.7462953690377e-07 \\
210624	7.29551272893314e-07 \\
211794	6.87100732199575e-07 \\
212964	6.47124419550504e-07 \\
214134	6.09477829371929e-07 \\
215304	5.7402491721037e-07 \\
216474	5.40637602797212e-07 \\
217644	5.09195302256238e-07 \\
218814	4.79584488344287e-07 \\
219984	4.51698275838464e-07 \\
221154	4.25436031625814e-07 \\
222324	4.0070300888484e-07 \\
223494	3.77410000529377e-07 \\
224664	3.55473015190011e-07 \\
225834	3.34812971070075e-07 \\
227004	3.15355408619933e-07 \\
228174	2.97030218920913e-07 \\
229344	2.79771389388728e-07 \\
230514	2.63516763687743e-07 \\
231684	2.48207815967127e-07 \\
232854	2.3378943775354e-07 \\
234024	2.20209738610588e-07 \\
235194	2.0741985717887e-07 \\
236364	1.95373784761532e-07 \\
237534	1.84028197069974e-07 \\
238704	1.73342298570578e-07 \\
239874	1.63277673936868e-07 \\
241044	1.537981486055e-07 \\
242214	1.44869658214031e-07 \\
243384	1.36460125366167e-07 \\
244554	1.28539342891809e-07 \\
245724	1.21078864878665e-07 \\
246894	1.14051904254175e-07 \\
248064	1.07433234919352e-07 \\
249234	1.01199101376626e-07 \\
250404	9.53271323544946e-08 \\
251574	8.97962602053326e-08 \\
252744	8.45866448551114e-08 \\
253914	7.96796018609491e-08 \\
255084	7.50575355201732e-08 \\
256254	7.07038743108512e-08 \\
257424	6.66030123830375e-08 \\
258594	6.27402525488208e-08 \\
259764	5.9101752880597e-08 \\
260934	5.56744778612561e-08 \\
262104	5.24461503670359e-08 \\
263274	4.94052080357577e-08 \\
264444	4.65407616889735e-08 \\
265614	4.38425551974042e-08 \\
266784	4.13009296207356e-08 \\
267954	3.89067880690597e-08 \\
269124	3.66515627292507e-08 \\
270294	3.45271840562766e-08 \\
271464	3.25260520184223e-08 \\
272109	3.06410085082476e-08 \\
272229	2.88653119739912e-08 \\
272349	2.71926123285304e-08 \\
272469	2.56169291890096e-08 \\
272589	2.41326296723798e-08 \\
272709	2.27344081338288e-08 \\
272829	2.14172677925895e-08 \\
272949	2.01765018026379e-08 \\
273069	1.90076772099701e-08 \\
273189	1.79066185768129e-08 \\
273309	1.68693931046349e-08 \\
273429	1.58922963122698e-08 \\
273549	1.49718392683518e-08 \\
273669	1.41047353796608e-08 \\
273789	1.32878891778709e-08 \\
273909	1.25183849952748e-08 \\
274029	1.17934763621541e-08 \\
274149	1.11105764588615e-08 \\
274269	1.04672485679025e-08 \\
274389	9.86119758072945e-09 \\
274509	9.29026150453538e-09 \\
274629	8.7524039682485e-09 \\
274749	8.24570667301572e-09 \\
274869	7.76836300842021e-09 \\
274989	7.31867072500947e-09 \\
275109	6.89502682726939e-09 \\
275229	6.49592113433073e-09 \\
275349	6.11993106192088e-09 \\
275469	5.76571695942718e-09 \\
275589	5.43201600367027e-09 \\
275709	5.11763981192459e-09 \\
275829	4.82146811364714e-09 \\
275949	4.54244586389763e-09 \\
276069	4.2795793020467e-09 \\
276189	4.03193223252885e-09 \\
276309	3.79862280519561e-09 \\
276429	3.57882024015765e-09 \\
276549	3.37174199671608e-09 \\
276669	3.17665088678254e-09 \\
276789	2.99285246585512e-09 \\
276909	2.81969264603887e-09 \\
277029	2.65655492048822e-09 \\
277149	2.50285914216164e-09 \\
277269	2.3580583041749e-09 \\
277389	2.22163715202228e-09 \\
277509	2.09311046273086e-09 \\
277629	1.97202115748141e-09 \\
277749	1.85793830320691e-09 \\
277869	1.75045644645877e-09 \\
277989	1.64919328193847e-09 \\
278109	1.55378870880796e-09 \\
278229	1.46390360944437e-09 \\
278349	1.37921846166122e-09 \\
278469	1.29943211746308e-09 \\
278589	1.22426113691176e-09 \\
278709	1.15343823381409e-09 \\
278829	1.08671199816612e-09 \\
278949	1.02384495326291e-09 \\
279069	9.64613888765342e-10 \\
279189	9.08808472921407e-10 \\
279309	8.56230253365453e-10 \\
279429	8.06692657118191e-10 \\
279549	7.60019713830218e-10 \\
279669	7.16045667203957e-10 \\
279789	6.74614197837542e-10 \\
279909	6.35578423224814e-10 \\
280029	5.98799620998847e-10 \\
280149	5.64147117909641e-10 \\
280269	5.31498012268372e-10 \\
280389	5.00736507813571e-10 \\
280509	4.71753303088462e-10 \\
280629	4.44445591440967e-10 \\
280749	4.18716228356431e-10 \\
280869	3.94474231057984e-10 \\
280989	3.71633446238917e-10 \\
281109	3.50112772107281e-10 \\
281229	3.29836047363585e-10 \\
281349	3.10731218533533e-10 \\
281469	2.92730450990319e-10 \\
281589	2.75769906910028e-10 \\
281709	2.59789578738179e-10 \\
281829	2.4473262305591e-10 \\
281949	2.30546026713796e-10 \\
282069	2.17178830475007e-10 \\
282189	2.04584293950205e-10 \\
282309	1.92717286573441e-10 \\
282429	1.81535897425533e-10 \\
282549	1.71000602566806e-10 \\
282669	1.61073876459028e-10 \\
282789	1.51720802588073e-10 \\
282909	1.42907852218599e-10 \\
283029	1.3460416115052e-10 \\
283149	1.2678014194023e-10 \\
283269	1.19408039012114e-10 \\
283389	1.12461762125093e-10 \\
283509	1.05916664328021e-10 \\
283629	9.97497640042866e-11 \\
283749	9.3938967715701e-11 \\
283869	8.84637363363083e-11 \\
283989	8.33048630077826e-11 \\
284109	7.84437514944614e-11 \\
284229	7.38635264063703e-11 \\
284349	6.95477009315937e-11 \\
284469	6.54810650146942e-11 \\
284589	6.16492967786542e-11 \\
284709	5.80388515025732e-11 \\
284829	5.46368505993655e-11 \\
284949	5.14312481492141e-11 \\
285069	4.84108864107213e-11 \\
285189	4.55647741759435e-11 \\
285309	4.28829749488102e-11 \\
285429	4.03561073447634e-11 \\
285549	3.79750120238498e-11 \\
285669	3.57314733356873e-11 \\
285789	3.3617386652196e-11 \\
285909	3.16254245014136e-11 \\
286029	2.97483704336798e-11 \\
286149	2.79796741331495e-11 \\
286269	2.63131183508847e-11 \\
286389	2.47427633937036e-11 \\
286509	2.3263002635332e-11 \\
286629	2.18688400721589e-11 \\
286749	2.05548356113638e-11 \\
286869	1.93168814277556e-11 \\
286989	1.81503145846307e-11 \\
287109	1.70511382791005e-11 \\
287229	1.60153001971253e-11 \\
287349	1.50393031361773e-11 \\
287469	1.4119649893729e-11 \\
287589	1.32530653118579e-11 \\
287709	1.24363852549436e-11 \\
287829	1.16668896765759e-11 \\
287949	1.09418585303445e-11 \\
288069	1.02585162586877e-11 \\
288189	9.61475343785878e-12 \\
288309	9.00818308835483e-12 \\
288429	8.43641823067287e-12 \\
288549	7.89773801912474e-12 \\
288669	7.39014405226612e-12 \\
288789	6.91185997325761e-12 \\
288909	6.46110942525979e-12 \\
289029	6.03644911834067e-12 \\
289149	5.6362137179633e-12 \\
289269	5.25912646764937e-12 \\
289389	4.90374407746685e-12 \\
289509	4.56884530208868e-12 \\
289629	4.25331991849021e-12 \\
289749	3.95589117019313e-12 \\
289869	3.6757819010802e-12 \\
289989	3.41177086582434e-12 \\
290109	3.16291437485461e-12 \\
290229	2.92854629435624e-12 \\
290349	2.7075564013046e-12 \\
290469	2.49950060648985e-12 \\
290589	2.30315766458489e-12 \\
290709	2.11841655328726e-12 \\
290829	1.94405602726988e-12 \\
290949	1.77996506423028e-12 \\
291069	1.6253109969e-12 \\
291189	1.47953871376671e-12 \\
291309	1.34220412562058e-12 \\
291429	1.21280763210052e-12 \\
291549	1.09079412169422e-12 \\
291669	9.75886038645513e-13 \\
291789	8.67472760290866e-13 \\
291909	7.65498775479045e-13 \\
292029	6.69297950395276e-13 \\
292149	5.78537218132169e-13 \\
};
\addplot [line width=0.01pt, red, forget plot]
table [row sep=\\]{%
1186	2.59160755024989 \\
2366	2.13547753272114 \\
3546	1.75498243276314 \\
4726	1.43601517420546 \\
5906	1.1703569091464 \\
7086	0.953745906282165 \\
8266	0.784068275143006 \\
9446	0.651839470216201 \\
10626	0.538466397910526 \\
11806	0.44335665423314 \\
12986	0.366586874505212 \\
14166	0.304458882203183 \\
15346	0.253487493146567 \\
16526	0.21109720818972 \\
17706	0.174517733062425 \\
18886	0.146198600339435 \\
20066	0.121321101653934 \\
21246	0.102414185398836 \\
22426	0.0861880228089404 \\
23606	0.0715789564798998 \\
24786	0.0590865897638622 \\
25966	0.0479787946434843 \\
27146	0.0398091640881483 \\
28326	0.0344823418855301 \\
29506	0.0299045590661188 \\
30686	0.0260560353089417 \\
31866	0.0226480528107314 \\
33046	0.0196206437684238 \\
34226	0.0172064035513328 \\
35406	0.0152454941539963 \\
36586	0.0134194820247918 \\
37766	0.0117196497422286 \\
38946	0.0101342152298403 \\
40126	0.00865536141614559 \\
41306	0.00727383591821251 \\
42486	0.00658612099553796 \\
43666	0.00596133973906304 \\
44846	0.0053820967858042 \\
46026	0.0048453193674059 \\
47206	0.00437542833660365 \\
48386	0.00395321085945743 \\
49566	0.00356094738945839 \\
50746	0.00321053978420854 \\
51926	0.0029194291351009 \\
53106	0.00265074071154497 \\
54286	0.00245481994390467 \\
55466	0.00229143926956266 \\
56646	0.00213960540760522 \\
57826	0.00199864255796917 \\
59006	0.00187131380689293 \\
60186	0.00175255675516461 \\
61366	0.00164164030627201 \\
62546	0.00153799186943648 \\
63726	0.00144115400160127 \\
64906	0.00135061356744909 \\
66086	0.0012659251453776 \\
67266	0.00118668728646593 \\
68446	0.00111252956577451 \\
69626	0.00104310957917625 \\
70806	0.000978110350814021 \\
71986	0.000917238066583781 \\
73166	0.000860220076667917 \\
74346	0.00080680312308834 \\
75526	0.00075675175748674 \\
76706	0.000709846921315704 \\
77886	0.000666069313385032 \\
79066	0.000625309517242434 \\
80246	0.000587106176009045 \\
81426	0.000551273701697941 \\
82606	0.000517660352013671 \\
83786	0.000486124852117964 \\
84966	0.00045653551752256 \\
86146	0.000428769485192715 \\
87326	0.000402712032056052 \\
88506	0.000378255965733543 \\
89686	0.00035530107578069 \\
90866	0.000333753636284961 \\
92046	0.000313525952582716 \\
93226	0.00029453594632034 \\
94406	0.000276706774184454 \\
95586	0.000259966476475415 \\
96766	0.000244247652354923 \\
97946	0.000229487159100372 \\
99126	0.000215625833105637 \\
100306	0.00020260823067636 \\
101486	0.000190382386929533 \\
102666	0.000178899591304238 \\
103846	0.000168114178369705 \\
105026	0.000157983332746692 \\
106206	0.00014846690708592 \\
107386	0.000139527252142779 \\
108566	0.000131129058073165 \\
109746	0.000123239206158809 \\
110926	0.000115826630225913 \\
112106	0.000108862187089331 \\
113286	0.000102318535401658 \\
114466	9.61700223359174e-05 \\
115541	9.0392577572207e-05 \\
115671	8.49636141004306e-05 \\
115801	7.98619353817309e-05 \\
115931	7.50676484483348e-05 \\
116061	7.05620825490194e-05 \\
116191	6.6327712975045e-05 \\
116321	6.23518933762224e-05 \\
116451	5.86440174696889e-05 \\
116581	5.51580185615408e-05 \\
116711	5.18804048016408e-05 \\
116841	4.87985807420266e-05 \\
116971	4.59007330220196e-05 \\
117101	4.3175776087756e-05 \\
117231	4.06133050689372e-05 \\
117361	3.82035522236701e-05 \\
117491	3.59373464514645e-05 \\
117621	3.38060756021608e-05 \\
117751	3.18016513479735e-05 \\
117881	2.99164764174242e-05 \\
118011	2.81434140034786e-05 \\
118141	2.64757591870168e-05 \\
118271	2.49072122219252e-05 \\
118401	2.3431853549305e-05 \\
118531	2.20441204188959e-05 \\
118661	2.07387850019169e-05 \\
118791	1.95109338954613e-05 \\
118921	1.83559489213003e-05 \\
119051	1.72694891336644e-05 \\
119181	1.62474739535123e-05 \\
119311	1.52860673551802e-05 \\
119441	1.43816630385207e-05 \\
119571	1.35308705180859e-05 \\
119701	1.27305020731217e-05 \\
119831	1.19775605036399e-05 \\
119961	1.12692276370008e-05 \\
120091	1.06028535414859e-05 \\
120221	9.97594639995381e-06 \\
120351	9.38616144813276e-06 \\
120481	8.83129833040286e-06 \\
120611	8.30928318668045e-06 \\
120741	7.81816719641615e-06 \\
120871	7.35611756491616e-06 \\
121001	6.9214105720139e-06 \\
121131	6.51242504429161e-06 \\
121261	6.12763622159784e-06 \\
121391	5.76560999399067e-06 \\
121521	5.42499748634739e-06 \\
121651	5.10452996960176e-06 \\
121781	4.80301407718153e-06 \\
121911	4.51932731004856e-06 \\
122041	4.2524138111899e-06 \\
122171	4.00128039462766e-06 \\
122301	3.76499281168341e-06 \\
122431	3.54267224117466e-06 \\
122561	3.33349199044264e-06 \\
122691	3.1366743931116e-06 \\
122821	2.95148789314359e-06 \\
122951	2.77724430264303e-06 \\
123081	2.61329622336381e-06 \\
123211	2.45903462353647e-06 \\
123341	2.31388655708153e-06 \\
123471	2.17731302121216e-06 \\
123601	2.04880694021359e-06 \\
123731	1.92789127007043e-06 \\
123861	1.81411721605906e-06 \\
123991	1.70706255586683e-06 \\
124121	1.60633006424105e-06 \\
124251	1.51154602817671e-06 \\
124381	1.42235885464137e-06 \\
124511	1.33843775740328e-06 \\
124641	1.25947152451733e-06 \\
124771	1.18516735725382e-06 \\
124901	1.11524977913779e-06 \\
125031	1.04945960976988e-06 \\
125161	9.87552998876762e-07 \\
125291	9.29300518703879e-07 \\
125421	8.74486308810596e-07 \\
125551	8.22907273767548e-07 \\
125681	7.74372326539652e-07 \\
125811	7.28701677665811e-07 \\
125941	6.85726166405054e-07 \\
126071	6.45286631739683e-07 \\
126201	6.07233320459866e-07 \\
126331	5.71425330164743e-07 \\
126461	5.3773008595881e-07 \\
126591	5.06022847568399e-07 \\
126721	4.76186245546018e-07 \\
126851	4.48109845618827e-07 \\
126981	4.21689738350217e-07 \\
127111	3.96828153059747e-07 \\
127241	3.73433095224307e-07 \\
127371	3.51418004473913e-07 \\
127501	3.3070143340419e-07 \\
127631	3.11206745151615e-07 \\
127761	2.92861829342961e-07 \\
127891	2.75598834253987e-07 \\
128021	2.59353915177396e-07 \\
128151	2.44066997334702e-07 \\
128281	2.29681553498562e-07 \\
128411	2.16144393716533e-07 \\
128541	2.03405469023643e-07 \\
128671	1.9141768486941e-07 \\
128801	1.80136726979363e-07 \\
128931	1.69520897541631e-07 \\
129061	1.59530959609189e-07 \\
129191	1.50129992770864e-07 \\
129321	1.41283255594704e-07 \\
129451	1.32958056953125e-07 \\
129581	1.25123635397184e-07 \\
129711	1.17751045025649e-07 \\
129841	1.1081304790439e-07 \\
129971	1.04284013702216e-07 \\
130101	9.81398241006737e-08 \\
130231	9.23577842537604e-08 \\
130361	8.69165384664861e-08 \\
130491	8.17959907584154e-08 \\
130621	7.69772311448591e-08 \\
130751	7.24424650266897e-08 \\
130881	6.81749475206495e-08 \\
131011	6.41589217864613e-08 \\
131141	6.03795602960311e-08 \\
131271	5.68229105990525e-08 \\
131401	5.34758425874138e-08 \\
131531	5.0326001033163e-08 \\
131661	4.73617592922082e-08 \\
131791	4.45721766717533e-08 \\
131921	4.19469574630682e-08 \\
132051	3.94764131383951e-08 \\
132181	3.71514268793227e-08 \\
132311	3.49634189378278e-08 \\
132441	3.29043160496312e-08 \\
132571	3.09665207365306e-08 \\
132701	2.91428834953145e-08 \\
132831	2.7426676540987e-08 \\
132961	2.5811568382661e-08 \\
133091	2.42916015635863e-08 \\
133221	2.28611691244218e-08 \\
133351	2.15149949522875e-08 \\
133481	2.02481141298172e-08 \\
133611	1.90558541168784e-08 \\
133741	1.79338180417155e-08 \\
133871	1.68778681586268e-08 \\
134001	1.58841108044427e-08 \\
134131	1.49488815215371e-08 \\
134261	1.40687319571953e-08 \\
134391	1.32404170405387e-08 \\
134521	1.24608831031381e-08 \\
134651	1.17272561661608e-08 \\
134781	1.10368319483634e-08 \\
134911	1.03870651524396e-08 \\
135041	9.7755608607919e-09 \\
135171	9.20006504312454e-09 \\
135301	8.65845678488242e-09 \\
135431	8.14873973853381e-09 \\
135561	7.66903562876564e-09 \\
135691	7.21757659194466e-09 \\
135821	6.79269890335732e-09 \\
135951	6.39283692649428e-09 \\
136081	6.01651739540188e-09 \\
136211	5.6623538080558e-09 \\
136341	5.32904126382405e-09 \\
136471	5.01535168950795e-09 \\
136601	4.72012950947232e-09 \\
136731	4.44228720475337e-09 \\
136861	4.18080114972241e-09 \\
136991	3.93470839243903e-09 \\
137121	3.70310232478133e-09 \\
137251	3.48512985137717e-09 \\
137381	3.27998828097975e-09 \\
137511	3.08692188477622e-09 \\
137641	2.9052200090085e-09 \\
137771	2.73421324470391e-09 \\
137901	2.57327198438517e-09 \\
138031	2.42180347997945e-09 \\
138161	2.27925006646146e-09 \\
138291	2.14508694140747e-09 \\
138421	2.01881994454922e-09 \\
138551	1.89998439203976e-09 \\
138681	1.78814274498507e-09 \\
138811	1.68288322166532e-09 \\
138941	1.58381835424493e-09 \\
139071	1.49058348997144e-09 \\
139201	1.40283534788566e-09 \\
139331	1.32025101962086e-09 \\
139461	1.2425265261129e-09 \\
139591	1.16937587391064e-09 \\
139721	1.10052977841946e-09 \\
139851	1.03573483123398e-09 \\
139981	9.74752556448522e-10 \\
140111	9.17358577989802e-10 \\
140241	8.6334178694969e-10 \\
140371	8.12503286873323e-10 \\
140501	7.64656116203355e-10 \\
140631	7.19624082545778e-10 \\
140761	6.7724165164762e-10 \\
140891	6.3735278166277e-10 \\
141021	5.99810867640826e-10 \\
141151	5.64477631304072e-10 \\
141281	5.31223121047475e-10 \\
141411	4.99925101316023e-10 \\
141541	4.7046827544861e-10 \\
141671	4.42744563233788e-10 \\
141801	4.16651713130989e-10 \\
141931	3.92093857382037e-10 \\
142061	3.68980679343878e-10 \\
142191	3.47227191443977e-10 \\
142321	3.26753402113411e-10 \\
142451	3.07483871697656e-10 \\
142581	2.8934776796774e-10 \\
142711	2.72278644075641e-10 \\
142841	2.5621349486471e-10 \\
142971	2.41093311981189e-10 \\
143101	2.26862528762695e-10 \\
143231	2.13468742682466e-10 \\
143361	2.00862881882813e-10 \\
143491	1.88998316996702e-10 \\
143621	1.7783158279272e-10 \\
143751	1.67321767552409e-10 \\
143881	1.57429957958755e-10 \\
144011	1.48119905230004e-10 \\
144141	1.39357414496999e-10 \\
144271	1.31110289292025e-10 \\
144401	1.23348165015358e-10 \\
144531	1.16042619957568e-10 \\
144661	1.09166675699157e-10 \\
144791	1.02695019155163e-10 \\
144921	9.66041135974649e-11 \\
145051	9.08713104763592e-11 \\
145181	8.54755710655297e-11 \\
145311	8.03970778839869e-11 \\
145441	7.56173457183706e-11 \\
145571	7.1118611000287e-11 \\
145701	6.68844979401229e-11 \\
145831	6.28992968820796e-11 \\
145961	5.9148519415686e-11 \\
146091	5.56181212196805e-11 \\
146221	5.22955012627335e-11 \\
146351	4.91681140246669e-11 \\
146481	4.62246907417807e-11 \\
146611	4.34542402061311e-11 \\
146741	4.08467704104964e-11 \\
146871	3.83926224145625e-11 \\
147001	3.60826368783762e-11 \\
147131	3.3908542640404e-11 \\
147261	3.18623460948686e-11 \\
147391	2.99363311917489e-11 \\
147521	2.81236145482922e-11 \\
147651	2.6417368292897e-11 \\
147781	2.48115972212304e-11 \\
147911	2.33001395955057e-11 \\
148041	2.1877499811751e-11 \\
148171	2.05385153329019e-11 \\
148301	1.92783566888011e-11 \\
148431	1.80921944092916e-11 \\
148561	1.69756986245773e-11 \\
148691	1.59248725317696e-11 \\
148821	1.49358858614335e-11 \\
148951	1.40050193664365e-11 \\
149081	1.31288868665536e-11 \\
149211	1.23041576927108e-11 \\
149341	1.15280562873465e-11 \\
149471	1.07974185148407e-11 \\
149601	1.01098573956904e-11 \\
149731	9.46265288348513e-12 \\
149861	8.85341799872208e-12 \\
149991	8.28020985110811e-12 \\
150121	7.74047492768659e-12 \\
150251	7.23249238276935e-12 \\
150381	6.75448585951699e-12 \\
150511	6.30445695648518e-12 \\
150641	5.88090687259069e-12 \\
150771	5.48222578444779e-12 \\
150901	5.10702591327572e-12 \\
151031	4.75391948029369e-12 \\
151161	4.42151870672092e-12 \\
151291	4.10860234723032e-12 \\
151421	3.81417120109973e-12 \\
151551	3.53694851185082e-12 \\
151681	3.27604610106391e-12 \\
151811	3.03052027916806e-12 \\
151941	2.79942735659233e-12 \\
152071	2.58187915491703e-12 \\
152201	2.37709851802492e-12 \\
152331	2.18441931210123e-12 \\
152461	2.00295335872624e-12 \\
152591	1.83225656869013e-12 \\
152721	1.67155178587564e-12 \\
152851	1.52033940992169e-12 \\
152981	1.37784228471105e-12 \\
153111	1.24394938794126e-12 \\
153241	1.11782805234384e-12 \\
153371	9.99089699860178e-13 \\
153501	8.87234730129194e-13 \\
153631	7.82152120848423e-13 \\
153761	6.8317573820309e-13 \\
153891	5.90028026437039e-13 \\
154021	5.0220938518919e-13 \\
154151	4.19664303308309e-13 \\
154281	3.42059713887011e-13 \\
154411	2.68896016564213e-13 \\
154541	2.00006677886222e-13 \\
154671	1.35280675550575e-13 \\
154801	7.42184091961917e-14 \\
};
\addplot [line width=0.01pt, red, forget plot]
table [row sep=\\]{%
1181	2.54753504718279 \\
2361	2.14497300129736 \\
3541	1.80507025414527 \\
4721	1.51932077234216 \\
5901	1.28204387332263 \\
7081	1.0759709877996 \\
8261	0.907473089495965 \\
9441	0.758564131988636 \\
10621	0.627885677047087 \\
11801	0.521590623918832 \\
12981	0.428692598760273 \\
14161	0.349674827043492 \\
15341	0.283527985918908 \\
16521	0.237535067276266 \\
17701	0.200371144748799 \\
18881	0.173862150869957 \\
20061	0.152582172080139 \\
21241	0.133243978919453 \\
22421	0.116167580260373 \\
23601	0.100684150575871 \\
24781	0.0864870853483878 \\
25961	0.0738809518640939 \\
27141	0.0634816174077592 \\
28321	0.0553318314465414 \\
29501	0.0478449162176822 \\
30681	0.0408425052189499 \\
31861	0.0358195529249681 \\
33041	0.031965377385944 \\
34221	0.0286074850093582 \\
35401	0.0255105105605365 \\
36581	0.0227295071905375 \\
37761	0.0201769540804388 \\
38941	0.0178542791940873 \\
40121	0.016071324481642 \\
41301	0.0147514824265561 \\
42481	0.0135236852501912 \\
43661	0.012447581090295 \\
44841	0.0114546088683846 \\
46021	0.0105395197363191 \\
47201	0.00968535875161908 \\
48381	0.00889337775146193 \\
49561	0.0081705622653529 \\
50741	0.00750454728406796 \\
51921	0.0069070347910079 \\
53101	0.00635093219916449 \\
54281	0.00583305861461886 \\
55461	0.00535627341757278 \\
56641	0.00492691311342619 \\
57821	0.00452872914086683 \\
59001	0.00415985056931273 \\
60181	0.00381814418942222 \\
61361	0.00349985743910242 \\
62541	0.00320532486917757 \\
63721	0.00293186405022605 \\
64901	0.00267688745596933 \\
66081	0.00243898690349564 \\
67261	0.00221688775004958 \\
68441	0.00201063523170525 \\
69621	0.00182953586968709 \\
70801	0.00171357197648236 \\
71981	0.00160588546534451 \\
73161	0.0015053844868142 \\
74341	0.00141151104263709 \\
75521	0.00132378265037569 \\
76701	0.00124175982066616 \\
77881	0.0011650398593453 \\
79061	0.00109325287593393 \\
80241	0.00102605843439607 \\
81421	0.000963142660374616 \\
82601	0.000904215721080348 \\
83781	0.000849009615210838 \\
84961	0.000797276223205745 \\
86141	0.000748785577926181 \\
87321	0.000703324323439503 \\
88501	0.0006606943355012 \\
89681	0.00062071148199061 \\
90861	0.000583204505238744 \\
92041	0.00054801401113086 \\
93221	0.000514991552231836 \\
94401	0.000483998794100604 \\
95581	0.00045490675552684 \\
96761	0.000427595114711299 \\
97941	0.000401951574483361 \\
99121	0.000377871280539954 \\
100301	0.000355256287443428 \\
101481	0.000334015067754878 \\
102661	0.00031406206021739 \\
103841	0.000295317253372951 \\
105021	0.000277705801396966 \\
106201	0.000261157669279755 \\
107381	0.000245607304789341 \\
108561	0.000230993334913443 \\
109741	0.000217258284707311 \\
110921	0.000204348316685088 \\
112101	0.000192212989064966 \\
113281	0.0001808050313441 \\
114461	0.000170080135818873 \\
115011	0.000159996763793135 \\
115141	0.000150515965331721 \\
115271	0.000141601211514131 \\
115401	0.000133218238240029 \\
115531	0.00012533490071398 \\
115661	0.000117921037815993 \\
115791	0.000110948345626249 \\
115921	0.000104390259436704 \\
116051	9.82218436328397e-05 \\
116181	9.24196888830719e-05 \\
116311	8.69618161132846e-05 \\
116441	8.18275867873197e-05 \\
116571	7.69976190532207e-05 \\
116701	7.24537093446664e-05 \\
116831	6.81787590622873e-05 \\
116961	6.41853678489124e-05 \\
117091	6.04294564575669e-05 \\
117221	5.68946258745484e-05 \\
117351	5.35676958980935e-05 \\
117481	5.04362987417961e-05 \\
117611	4.74888209393565e-05 \\
117741	4.47143544908823e-05 \\
117871	4.21026518420686e-05 \\
118001	3.9644084069923e-05 \\
118131	3.7329601981495e-05 \\
118261	3.51506998801376e-05 \\
118391	3.30993817810876e-05 \\
118521	3.11681298855171e-05 \\
118651	2.93498751370858e-05 \\
118781	2.76379697051743e-05 \\
118911	2.60261612519685e-05 \\
119041	2.45085688553859e-05 \\
119171	2.30796604684391e-05 \\
119301	2.17342318082347e-05 \\
119431	2.04673865749627e-05 \\
119561	1.9274517911505e-05 \\
119691	1.81512910176762e-05 \\
119821	1.70936268438782e-05 \\
119951	1.60976867921714e-05 \\
120081	1.51598583577606e-05 \\
120211	1.42767416514977e-05 \\
120341	1.3445136744672e-05 \\
120471	1.26620317851822e-05 \\
120601	1.19245918343003e-05 \\
120731	1.12301483795041e-05 \\
120861	1.05761894807488e-05 \\
120991	9.96035050937483e-06 \\
121121	9.38040544506968e-06 \\
121251	8.83425869352417e-06 \\
121381	8.3199373938081e-06 \\
121511	7.83558418704366e-06 \\
121641	7.37945041401344e-06 \\
121771	6.9498897204423e-06 \\
121901	6.54535204014373e-06 \\
122031	6.16437793760083e-06 \\
122161	5.80559328472452e-06 \\
122291	5.46770425413623e-06 \\
122421	5.14949260699282e-06 \\
122551	4.84981126164241e-06 \\
122681	4.56758012251646e-06 \\
122811	4.30178215532484e-06 \\
122941	4.05145969401e-06 \\
123071	3.81571096508271e-06 \\
123201	3.59368681718264e-06 \\
123331	3.38458764109761e-06 \\
123461	3.18766047385788e-06 \\
123591	3.00219627019649e-06 \\
123721	2.82752733610225e-06 \\
123851	2.66302491175319e-06 \\
123981	2.50809689716913e-06 \\
124111	2.36218571064706e-06 \\
124241	2.22476627331769e-06 \\
124371	2.09534411077517e-06 \\
124501	1.97345356678369e-06 \\
124631	1.85865612145619e-06 \\
124761	1.75053880718812e-06 \\
124891	1.64871271818301e-06 \\
125021	1.55281160668652e-06 \\
125151	1.46249056076631e-06 \\
125281	1.3774247607512e-06 \\
125411	1.29730830694674e-06 \\
125541	1.22185311651757e-06 \\
125671	1.15078788509582e-06 \\
125801	1.08385710834158e-06 \\
125931	1.02082016079086e-06 \\
126061	9.61450429048938e-07 \\
126191	9.05534494832771e-07 \\
126321	8.52871366252561e-07 \\
126451	8.03271753890833e-07 \\
126581	7.56557388292833e-07 \\
126711	7.12560378812732e-07 \\
126841	6.71122608264518e-07 \\
126971	6.32095163877189e-07 \\
127101	5.95337801279072e-07 \\
127231	5.60718439346353e-07 \\
127361	5.28112684861082e-07 \\
127491	4.97403384869255e-07 \\
127621	4.6848020518464e-07 \\
127751	4.41239232484847e-07 \\
127881	4.15582601553943e-07 \\
128011	3.91418142564603e-07 \\
128141	3.68659049343467e-07 \\
128271	3.47223567287447e-07 \\
128401	3.27034699265649e-07 \\
128531	3.0801992861873e-07 \\
128661	2.90110958478529e-07 \\
128791	2.73243465631623e-07 \\
128921	2.57356869926006e-07 \\
129051	2.42394115890221e-07 \\
129181	2.28301467619652e-07 \\
129311	2.15028315653232e-07 \\
129441	2.02526995285446e-07 \\
129571	1.9075261448176e-07 \\
129701	1.79662893562416e-07 \\
129831	1.69218011714101e-07 \\
129961	1.59380465436509e-07 \\
130091	1.50114932595535e-07 \\
130221	1.41388145968868e-07 \\
130351	1.33168773619463e-07 \\
130481	1.25427306152393e-07 \\
130611	1.18135950633036e-07 \\
130741	1.11268531055586e-07 \\
130871	1.04800393974092e-07 \\
131001	9.87083197401262e-08 \\
131131	9.29704395136177e-08 \\
131261	8.75661564370134e-08 \\
131391	8.24760715834039e-08 \\
131521	7.76819142900287e-08 \\
131651	7.31664768216511e-08 \\
131781	6.89135519205131e-08 \\
131911	6.49078751857601e-08 \\
132041	6.11350698953572e-08 \\
132171	5.75815953807179e-08 \\
132301	5.42346987875142e-08 \\
132431	5.10823687238648e-08 \\
132561	4.81132919616378e-08 \\
132691	4.53168139680216e-08 \\
132821	4.26828991040296e-08 \\
132951	4.0202095985542e-08 \\
133081	3.78655021227026e-08 \\
133211	3.56647334998073e-08 \\
133341	3.35918931004819e-08 \\
133471	3.16395434851735e-08 \\
133601	2.98006792576189e-08 \\
133731	2.80687027509607e-08 \\
133861	2.64373996583522e-08 \\
133991	2.49009176056525e-08 \\
134121	2.34537436694104e-08 \\
134251	2.20906861692072e-08 \\
134381	2.08068549611973e-08 \\
134511	1.95976443961854e-08 \\
134641	1.84587164442362e-08 \\
134771	1.73859854291081e-08 \\
134901	1.63756032622864e-08 \\
135031	1.5423945509685e-08 \\
135161	1.45275983465254e-08 \\
135291	1.3683346788973e-08 \\
135421	1.28881623706611e-08 \\
135551	1.21391930396619e-08 \\
135681	1.14337522227892e-08 \\
135811	1.07693099438144e-08 \\
135941	1.01434828869706e-08 \\
136071	9.55402668090244e-09 \\
136201	8.99882696137055e-09 \\
136331	8.47589282093608e-09 \\
136461	7.98334870433237e-09 \\
136591	7.5194284132607e-09 \\
136721	7.08246805647406e-09 \\
136851	6.67090060968434e-09 \\
136981	6.28325041995836e-09 \\
137111	5.91812704398009e-09 \\
137241	5.57422158431464e-09 \\
137371	5.25030058318166e-09 \\
137501	4.94520274729737e-09 \\
137631	4.65783384084872e-09 \\
137761	4.38716357686886e-09 \\
137891	4.1322216759454e-09 \\
138021	3.89209370288413e-09 \\
138151	3.66591901279634e-09 \\
138281	3.45288658776255e-09 \\
138411	3.25223253883067e-09 \\
138541	3.06323771903649e-09 \\
138671	2.88522433722349e-09 \\
138801	2.71755440373056e-09 \\
138931	2.55962662176756e-09 \\
139061	2.41087511065885e-09 \\
139191	2.27076635272994e-09 \\
139321	2.13879824961793e-09 \\
139451	2.0144977908032e-09 \\
139581	1.89741933276366e-09 \\
139711	1.78714304466254e-09 \\
139841	1.68327396465884e-09 \\
139971	1.58543966843894e-09 \\
140101	1.49328927001591e-09 \\
140231	1.40649281110683e-09 \\
140361	1.32473892966445e-09 \\
140491	1.24773469334372e-09 \\
140621	1.17520432274532e-09 \\
140751	1.10688763710343e-09 \\
140881	1.04253977672997e-09 \\
141011	9.81930370347328e-10 \\
141141	9.24841925264985e-10 \\
141271	8.71070104935256e-10 \\
141401	8.20421897085311e-10 \\
141531	7.72716057806377e-10 \\
141661	7.27781557241514e-10 \\
141791	6.85457524074451e-10 \\
141921	6.45591968773118e-10 \\
142051	6.08042394212305e-10 \\
142181	5.72674074827972e-10 \\
142311	5.39360167639558e-10 \\
142441	5.07981601227669e-10 \\
142571	4.7842574346646e-10 \\
142701	4.50586568057076e-10 \\
142831	4.24364710038816e-10 \\
142961	3.99665911476887e-10 \\
143091	3.76401687596228e-10 \\
143221	3.54488938203446e-10 \\
143351	3.33848892974942e-10 \\
143481	3.14407833101882e-10 \\
143611	2.96095981067168e-10 \\
143741	2.78847778201197e-10 \\
143871	2.62601385081496e-10 \\
144001	2.47298626021575e-10 \\
144131	2.32884656004018e-10 \\
144261	2.19307849658179e-10 \\
144391	2.06519690237883e-10 \\
144521	1.94474158998759e-10 \\
144651	1.83128401332056e-10 \\
144781	1.72441561030467e-10 \\
144911	1.62375335399645e-10 \\
145041	1.52893808724741e-10 \\
145171	1.439628971589e-10 \\
145301	1.35550737301315e-10 \\
145431	1.27627075574566e-10 \\
145561	1.20163545780372e-10 \\
145691	1.13133558077294e-10 \\
145821	1.06511799380371e-10 \\
145951	1.00274621939178e-10 \\
146081	9.43996547597692e-11 \\
146211	8.88658591158276e-11 \\
146341	8.36536395709686e-11 \\
146471	7.87437892668663e-11 \\
146601	7.4119099746639e-11 \\
146731	6.97632507318247e-11 \\
146861	6.5660088477415e-11 \\
146991	6.17952911063924e-11 \\
147121	5.81549253197977e-11 \\
147251	5.47259459970917e-11 \\
147381	5.14960851738522e-11 \\
147511	4.84538520417743e-11 \\
147641	4.55881443706119e-11 \\
147771	4.2888914641992e-11 \\
147901	4.03465039156004e-11 \\
148031	3.7951530806879e-11 \\
148161	3.56957241542943e-11 \\
148291	3.35709238186155e-11 \\
148421	3.15694692609725e-11 \\
148551	2.9684310565159e-11 \\
148681	2.7908508837271e-11 \\
148811	2.62358468283708e-11 \\
148941	2.46603293341252e-11 \\
149071	2.31762387059575e-11 \\
149201	2.17784124068032e-11 \\
149331	2.04616323884466e-11 \\
149461	1.92214022476378e-11 \\
149591	1.80530590476735e-11 \\
149721	1.69526614968163e-11 \\
149851	1.59161572810262e-11 \\
149981	1.49398826643221e-11 \\
150111	1.40201739107226e-11 \\
150241	1.31539223957589e-11 \\
150371	1.23379639838106e-11 \\
150501	1.15693010727114e-11 \\
150631	1.08453246383533e-11 \\
150761	1.01633146343261e-11 \\
150891	9.52099510342919e-12 \\
151021	8.91586804385724e-12 \\
151151	8.34593505416592e-12 \\
151281	7.80908671060843e-12 \\
151411	7.3034356340429e-12 \\
151541	6.82703893417624e-12 \\
151671	6.37828678762276e-12 \\
151801	5.95579141560165e-12 \\
151931	5.55760992781984e-12 \\
152061	5.18263210125269e-12 \\
152191	4.82947015711943e-12 \\
152321	4.49673631663927e-12 \\
152451	4.18332035678759e-12 \\
152581	3.88811205453976e-12 \\
152711	3.61000118687116e-12 \\
152841	3.34804406421085e-12 \\
152971	3.10135250813914e-12 \\
153101	2.8689828290851e-12 \\
153231	2.64999133747779e-12 \\
153361	2.44387843295613e-12 \\
153491	2.24964491479795e-12 \\
153621	2.06668016033973e-12 \\
153751	1.89442905806914e-12 \\
153881	1.73194791841524e-12 \\
154011	1.57901469677313e-12 \\
154141	1.43501877047925e-12 \\
154271	1.29934951687005e-12 \\
};
\addplot [line width=0.01pt, red, forget plot]
table [row sep=\\]{%
1188	2.7888866255838 \\
2368	2.26129384536861 \\
3548	1.85642583964018 \\
4728	1.52392451163332 \\
5908	1.24495392264653 \\
7088	1.01173282099969 \\
8268	0.830891537628891 \\
9448	0.681633207157341 \\
10628	0.561793993506216 \\
11808	0.471179635615061 \\
12988	0.405016431189394 \\
14168	0.346942222759005 \\
15348	0.296892121496764 \\
16528	0.255065881466235 \\
17708	0.218996322021556 \\
18888	0.186783234793812 \\
20068	0.159053610372881 \\
21248	0.135945048194655 \\
22428	0.118571435087158 \\
23608	0.104139552631835 \\
24788	0.0909128542194105 \\
25968	0.0788417429269822 \\
27148	0.067926237318154 \\
28328	0.0585000627154866 \\
29508	0.0499731564834456 \\
30688	0.0423553089393127 \\
31868	0.0369528255552933 \\
33048	0.0322715585172124 \\
34228	0.0280957846304477 \\
35408	0.0242769499120309 \\
36588	0.0207142849304812 \\
37768	0.0175056668377846 \\
38948	0.0149787504545199 \\
40128	0.0128768994751576 \\
41308	0.0112675926771071 \\
42488	0.0100740929491022 \\
43668	0.00897175765759201 \\
44848	0.00794571270754335 \\
46028	0.00698943981968947 \\
47208	0.00609722834395532 \\
48388	0.0053077752477963 \\
49568	0.0047269688898009 \\
50748	0.00418530683097712 \\
51928	0.00367926397681734 \\
53108	0.00329338284837633 \\
54288	0.00306856242532894 \\
55468	0.00285880775697117 \\
56648	0.00266290547678699 \\
57828	0.00247980003906095 \\
59008	0.00230854873136138 \\
60188	0.00214830176782366 \\
61368	0.00199828771981053 \\
62548	0.00185780229143689 \\
63728	0.00172619951124636 \\
64908	0.00160288470038183 \\
66088	0.00148730876156061 \\
67268	0.00137896346183014 \\
68448	0.00127769470624545 \\
69628	0.00118390038310667 \\
70808	0.00109606159704184 \\
71988	0.0010137744725694 \\
73168	0.00093724624530761 \\
74348	0.000870278514523593 \\
75528	0.000812543401162824 \\
76708	0.000758625813546909 \\
77888	0.000708229961918161 \\
79068	0.000661099210095606 \\
80248	0.000617001158578656 \\
81428	0.00057572380422044 \\
82608	0.000537072923574189 \\
83788	0.000500914606151925 \\
84968	0.000468443473448821 \\
86148	0.000438141928037983 \\
87328	0.000409852684623513 \\
88508	0.000383432210659951 \\
89688	0.000358748589596558 \\
90868	0.000335680334728639 \\
92048	0.000314115394113912 \\
93228	0.000293950279997013 \\
94408	0.000275089300840525 \\
95588	0.000257443880143426 \\
96768	0.000240931949315348 \\
97948	0.000225477404209862 \\
99128	0.000211009616745139 \\
100308	0.00019746299449519 \\
101488	0.000184776582296653 \\
102668	0.000172893700851551 \\
103848	0.000161761618061196 \\
105028	0.000151331249451492 \\
106208	0.000141581851603123 \\
107388	0.000132519581825141 \\
108568	0.000124037457912785 \\
109748	0.000116096809518573 \\
110928	0.000108661770158225 \\
112108	0.000101698964889207 \\
113288	9.51773218760277e-05 \\
114468	8.90679045694731e-05 \\
115648	8.33437571942097e-05 \\
115892	7.79797621119904e-05 \\
116032	7.2952508065327e-05 \\
116172	6.82401684200529e-05 \\
116312	6.38223885959799e-05 \\
116452	5.96801819500725e-05 \\
116592	5.57958334372888e-05 \\
116732	5.21528104319713e-05 \\
116872	4.87356801449623e-05 \\
117012	4.55300331186348e-05 \\
117152	4.25224123279389e-05 \\
117292	3.9700247448482e-05 \\
117432	3.70517938967918e-05 \\
117572	3.45660762743893e-05 \\
117712	3.22328358796642e-05 \\
117852	3.00424819785206e-05 \\
117992	2.79860465503545e-05 \\
118132	2.60551422453514e-05 \\
118272	2.42419233169588e-05 \\
118412	2.25390493036604e-05 \\
118552	2.09396512580451e-05 \\
118692	1.94984933009068e-05 \\
118832	1.82407751112557e-05 \\
118972	1.70661703102204e-05 \\
119112	1.59688732659369e-05 \\
119252	1.49433763915185e-05 \\
119392	1.39848600011483e-05 \\
119532	1.30887046299044e-05 \\
119672	1.22506746177975e-05 \\
119812	1.14668447963595e-05 \\
119952	1.07335737931646e-05 \\
120092	1.00475060504368e-05 \\
120232	9.40542407801281e-06 \\
120372	8.80448028606207e-06 \\
120512	8.24193224341352e-06 \\
120652	7.71524960879733e-06 \\
120792	7.22207597858571e-06 \\
120932	6.76021608736477e-06 \\
121072	6.3276240852006e-06 \\
121212	5.92239277297235e-06 \\
121352	5.54274369890528e-06 \\
121492	5.18701802970645e-06 \\
121632	4.8536681258593e-06 \\
121772	4.54124975446391e-06 \\
121912	4.24841488699812e-06 \\
122052	3.97390502954176e-06 \\
122192	3.71654504327523e-06 \\
122332	3.47523741345279e-06 \\
122472	3.24895693454286e-06 \\
122612	3.03674577506463e-06 \\
122752	2.83770889542012e-06 \\
122892	2.65100979118804e-06 \\
123032	2.47586653662202e-06 \\
123172	2.31154810681478e-06 \\
123312	2.15737095743407e-06 \\
123452	2.01737455812756e-06 \\
123592	1.89527060839856e-06 \\
123732	1.78083789842409e-06 \\
123872	1.67354971097877e-06 \\
124012	1.57294771735428e-06 \\
124152	1.47854569343409e-06 \\
124292	1.38998470011753e-06 \\
124432	1.30687135568852e-06 \\
124572	1.22885479336698e-06 \\
124712	1.15560862207209e-06 \\
124852	1.08682897220769e-06 \\
124992	1.02223273323876e-06 \\
125132	9.61555958856053e-07 \\
125272	9.04552419800364e-07 \\
125412	8.50992285916785e-07 \\
125552	8.00660923838681e-07 \\
125692	7.53357796090537e-07 \\
125832	7.08895453505054e-07 \\
125972	6.67098608686523e-07 \\
126112	6.27803284969364e-07 \\
126252	5.90856030879827e-07 \\
126392	5.56113197436314e-07 \\
126532	5.23440269351028e-07 \\
126672	4.92711246857791e-07 \\
126812	4.63808074557814e-07 \\
126952	4.36620110455532e-07 \\
127092	4.1104363585065e-07 \\
127232	3.86981398370345e-07 \\
127372	3.64342188918787e-07 \\
127512	3.43040448658183e-07 \\
127652	3.2299590208007e-07 \\
127792	3.04133216888491e-07 \\
127932	2.86381687197856e-07 \\
128072	2.6967493749197e-07 \\
128212	2.53950647566281e-07 \\
128352	2.39150295455737e-07 \\
128492	2.25218918348258e-07 \\
128632	2.12104887764575e-07 \\
128772	1.99759702557145e-07 \\
128912	1.88137791790055e-07 \\
129052	1.77196333550622e-07 \\
129192	1.66895084141583e-07 \\
129332	1.57196218542044e-07 \\
129472	1.48064181582086e-07 \\
129612	1.3946554794364e-07 \\
129752	1.3136889159826e-07 \\
129892	1.23744664015657e-07 \\
130032	1.16565078589481e-07 \\
130172	1.09804004222447e-07 \\
130312	1.03436863962969e-07 \\
130452	9.74405415243851e-08 \\
130592	9.17932921340459e-08 \\
130732	8.64746601547672e-08 \\
130872	8.1465401202685e-08 \\
131012	7.67474091500908e-08 \\
131152	7.230364801325e-08 \\
131292	6.81180870598652e-08 \\
131432	6.41756416341899e-08 \\
131572	6.04621157029861e-08 \\
131712	5.69641488978867e-08 \\
131852	5.36691671659817e-08 \\
131992	5.05653354188063e-08 \\
132132	4.76415140115982e-08 \\
132272	4.48872166103342e-08 \\
132412	4.22925723331247e-08 \\
132552	3.98482890018315e-08 \\
132692	3.7545618225554e-08 \\
132832	3.53763244254068e-08 \\
132972	3.33326532486744e-08 \\
133112	3.14073040352802e-08 \\
133252	2.95934028393674e-08 \\
133392	2.788447694968e-08 \\
133532	2.62744317414132e-08 \\
133672	2.47575280276635e-08 \\
133812	2.33283614647917e-08 \\
133952	2.19818426794305e-08 \\
134092	2.07131787832715e-08 \\
134232	1.9517856109097e-08 \\
134372	1.83916238349902e-08 \\
134512	1.73304784412132e-08 \\
134652	1.63306489997517e-08 \\
134792	1.5388584684306e-08 \\
134932	1.45009398933027e-08 \\
135072	1.36645645354427e-08 \\
135212	1.28764908180479e-08 \\
135352	1.21339237546536e-08 \\
135492	1.14342301182901e-08 \\
135632	1.07749300037874e-08 \\
135772	1.01536870023011e-08 \\
135912	9.56830042975199e-09 \\
136052	9.01669716668607e-09 \\
136192	8.49692416426961e-09 \\
136332	8.00714172743966e-09 \\
136472	7.54561690907707e-09 \\
136612	7.11071657111262e-09 \\
136752	6.7009027215903e-09 \\
136892	6.31472624190721e-09 \\
137032	5.95082194632113e-09 \\
137172	5.60790330839112e-09 \\
137312	5.28475890826385e-09 \\
137452	4.98024693706967e-09 \\
137592	4.69329169971999e-09 \\
137732	4.42287984014911e-09 \\
137872	4.16805645553353e-09 \\
138012	3.92792187664526e-09 \\
138152	3.70162861473844e-09 \\
138292	3.48837814190261e-09 \\
138432	3.28741783794939e-09 \\
138572	3.09803860343294e-09 \\
138712	2.9195725836928e-09 \\
138852	2.75139022676285e-09 \\
138992	2.5928985070145e-09 \\
139132	2.44353859368829e-09 \\
139272	2.3027838524925e-09 \\
139412	2.17013806924626e-09 \\
139552	2.04513389556737e-09 \\
139692	1.92733057291505e-09 \\
139832	1.81631315543385e-09 \\
139972	1.71169040052987e-09 \\
140112	1.61309376967012e-09 \\
140252	1.52017576304786e-09 \\
140392	1.43260941998236e-09 \\
140532	1.35008615398391e-09 \\
140672	1.27231525315352e-09 \\
140812	1.19902299200447e-09 \\
140952	1.12995135470584e-09 \\
141092	1.06485709139292e-09 \\
141232	1.0035111075446e-09 \\
141372	9.45697353760266e-10 \\
141512	8.91212548204123e-10 \\
141652	8.39864677804059e-10 \\
141792	7.91473109273966e-10 \\
141932	7.4586764542417e-10 \\
142072	7.02887803516461e-10 \\
142212	6.62382315663734e-10 \\
142352	6.24208684740779e-10 \\
142492	5.88232629272767e-10 \\
142632	5.54327472812588e-10 \\
142772	5.22374143940851e-10 \\
142912	4.92260121554011e-10 \\
143052	4.6387954588667e-10 \\
143192	4.37132718911215e-10 \\
143332	4.11925216159403e-10 \\
143472	3.88168719389625e-10 \\
143612	3.65779573296976e-10 \\
143752	3.4467911858016e-10 \\
143892	3.24793136829982e-10 \\
144032	3.06051572973587e-10 \\
144172	2.88388757319069e-10 \\
144312	2.71742406354747e-10 \\
144452	2.56054177860676e-10 \\
144592	2.41268727219079e-10 \\
144732	2.27334262525858e-10 \\
144872	2.14201545389869e-10 \\
145012	2.01824723600197e-10 \\
145152	1.90160165391973e-10 \\
145292	1.79166737002134e-10 \\
145432	1.6880602471403e-10 \\
145572	1.59041613212452e-10 \\
145712	1.49838974561334e-10 \\
145852	1.41165912292962e-10 \\
145992	1.32992006296462e-10 \\
146132	1.25288390773193e-10 \\
146272	1.18027976281354e-10 \\
146412	1.11185449735984e-10 \\
146552	1.04736552763995e-10 \\
146692	9.86588033491387e-11 \\
146832	9.29307741870389e-11 \\
146972	8.75322592186478e-11 \\
147112	8.24443846525469e-11 \\
147252	7.76493314091908e-11 \\
147392	7.31300020540004e-11 \\
147532	6.88709644869334e-11 \\
147672	6.48567866079475e-11 \\
147812	6.10734796069323e-11 \\
147952	5.75080538744999e-11 \\
148092	5.41476308235644e-11 \\
148232	5.09807751569724e-11 \\
148372	4.79958850441164e-11 \\
148512	4.518274643317e-11 \\
148652	4.25314783392139e-11 \\
148792	4.00327548888413e-11 \\
148932	3.76778608313089e-11 \\
149072	3.54584139827807e-11 \\
149212	3.33666427820845e-11 \\
149352	3.13951642461063e-11 \\
149492	2.95371505032449e-11 \\
149632	2.77860512376549e-11 \\
149772	2.61357047115496e-11 \\
149912	2.45802822540497e-11 \\
150052	2.31142882611834e-11 \\
150192	2.17326712181887e-11 \\
150332	2.04305461437571e-11 \\
150472	1.92034166346389e-11 \\
150612	1.80467862875844e-11 \\
150752	1.69567138108562e-11 \\
150892	1.59293134238681e-11 \\
151032	1.496097690179e-11 \\
151172	1.40484845978506e-11 \\
151312	1.31883948206735e-11 \\
151452	1.23778209903946e-11 \\
151592	1.16138210159988e-11 \\
151732	1.08938968956807e-11 \\
151872	1.02153285830298e-11 \\
152012	9.57561807624074e-12 \\
152152	8.97293350732298e-12 \\
152292	8.40483238562229e-12 \\
152432	7.8694273319968e-12 \\
152572	7.36477545615344e-12 \\
152712	6.88926693470648e-12 \\
152852	6.44106989966531e-12 \\
152992	6.0185745276442e-12 \\
153132	5.62039303986239e-12 \\
153272	5.24519316869032e-12 \\
153412	4.89158713534721e-12 \\
153552	4.55829818335474e-12 \\
153692	4.24410506738582e-12 \\
153832	3.94800858671829e-12 \\
153972	3.66906505178122e-12 \\
154112	3.40605321724752e-12 \\
154252	3.15814041584872e-12 \\
154392	2.92460500261882e-12 \\
154532	2.70439226568442e-12 \\
154672	2.49694709353321e-12 \\
154812	2.30132579659426e-12 \\
154952	2.11713979680894e-12 \\
155092	1.94333438230387e-12 \\
155232	1.77963199732289e-12 \\
155372	1.62536650805123e-12 \\
155512	1.4798717806741e-12 \\
155652	1.3427592371329e-12 \\
155792	1.21369581052022e-12 \\
155932	1.09195985587007e-12 \\
156072	9.77162795123832e-13 \\
156212	8.69082583676573e-13 \\
156352	7.67164110015983e-13 \\
156492	6.71074307234676e-13 \\
156632	5.80535619576494e-13 \\
156772	4.9515946898282e-13 \\
156912	4.1483483315119e-13 \\
157052	3.3895108941806e-13 \\
157192	2.67619260085894e-13 \\
157332	2.00284233642378e-13 \\
157472	1.36723965482588e-13 \\
157612	7.69939667577546e-14 \\
157752	2.06501482580279e-14 \\
157892	-3.24740234702858e-14 \\
158032	-8.26561041833429e-14 \\
};
\addplot [line width=0.01pt, red, forget plot]
table [row sep=\\]{%
1186	2.41493865268754 \\
2366	1.95997079244078 \\
3546	1.58573518741245 \\
4726	1.28564184866164 \\
5906	1.0428426201766 \\
7086	0.850178095089241 \\
8266	0.69970232896394 \\
9446	0.577855968661201 \\
10626	0.476056758424252 \\
11806	0.396185555402267 \\
12986	0.32629253338264 \\
14166	0.27156929172238 \\
15346	0.226651412001302 \\
16526	0.187638564141574 \\
17706	0.153727131562971 \\
18886	0.126072674374842 \\
20066	0.103870096659369 \\
21246	0.0850542194208977 \\
22426	0.0746430767266489 \\
23606	0.0654011172084689 \\
24786	0.0568904567396384 \\
25966	0.0496541688757619 \\
27146	0.0432835619823533 \\
28326	0.0374997149309022 \\
29506	0.0321468588344461 \\
30686	0.0274388034419225 \\
31866	0.0232847922369012 \\
33046	0.0204807857775163 \\
34226	0.0178676806146199 \\
35406	0.0154267372659466 \\
36586	0.0131550048499178 \\
37766	0.0113220595652412 \\
38946	0.00978126711277549 \\
40126	0.0084116177290578 \\
41306	0.00760369656324877 \\
42486	0.00700189520457983 \\
43666	0.00644711318554003 \\
44846	0.00592908405631604 \\
46026	0.00548952081138149 \\
47206	0.00510531594427804 \\
48386	0.0047473342578499 \\
49566	0.00441333677704397 \\
50746	0.00410272785054161 \\
51926	0.00381524797621485 \\
53106	0.00355064598542587 \\
54286	0.00330104793485075 \\
55466	0.00306860496874944 \\
56646	0.00285170752280101 \\
57826	0.00264788308488145 \\
59006	0.00245728199501294 \\
60186	0.00227868135958648 \\
61366	0.00211108894197043 \\
62546	0.00195485352285885 \\
63726	0.00180817149459467 \\
64906	0.00167005461150421 \\
66086	0.00154105157352202 \\
67266	0.00142004876002272 \\
68446	0.00130912032839142 \\
69626	0.00122100663632613 \\
70806	0.00113871777455099 \\
71986	0.00106152180566826 \\
73166	0.000989399226672016 \\
74346	0.000922110005867283 \\
75526	0.000858955022167818 \\
76706	0.000799875537273953 \\
77886	0.000744751704596036 \\
79066	0.000692605771242139 \\
80246	0.000644103500106852 \\
81426	0.000598426156744991 \\
82606	0.000555459237327416 \\
83786	0.000515140314714946 \\
84966	0.000477244288321632 \\
86146	0.000441691270975708 \\
87326	0.000408350721955331 \\
88506	0.000377036945008902 \\
89686	0.000347693325826681 \\
90866	0.00032009493173385 \\
92046	0.00029409029571148 \\
93226	0.000269691715412412 \\
94406	0.000246685349640563 \\
95586	0.000225179073938486 \\
96766	0.000204930584521923 \\
97946	0.000185918843002009 \\
99126	0.000167994758711487 \\
100306	0.000153690184210131 \\
101486	0.000143182979109924 \\
102666	0.00013341751758239 \\
103846	0.000124328936783991 \\
105026	0.000115910125551566 \\
106206	0.000108066094922254 \\
107386	0.000100759504168157 \\
108566	9.40149229787246e-05 \\
109746	8.77157985738619e-05 \\
110926	8.18302620970157e-05 \\
112106	7.63293076028737e-05 \\
113286	7.11863815435199e-05 \\
114466	6.63770771392458e-05 \\
115542	6.18788877915777e-05 \\
115682	5.76710042519135e-05 \\
115822	5.37341453406559e-05 \\
115962	5.00504146982439e-05 \\
116102	4.66031779567944e-05 \\
116242	4.33769561247188e-05 \\
116382	4.03573320094108e-05 \\
116522	3.75308672663777e-05 \\
116662	3.48850282308466e-05 \\
116802	3.24081191053205e-05 \\
116942	3.00892213949133e-05 \\
117082	2.79181387125016e-05 \\
117222	2.58853462571929e-05 \\
117362	2.39819444005085e-05 \\
117502	2.21996159174287e-05 \\
117642	2.05305864817151e-05 \\
117782	1.89675880987172e-05 \\
117922	1.75038252047122e-05 \\
118062	1.61329431938584e-05 \\
118202	1.48489991655953e-05 \\
118342	1.36464347150755e-05 \\
118482	1.25200505983747e-05 \\
118622	1.1464983133147e-05 \\
118762	1.04766822035529e-05 \\
118902	9.550890749499e-06 \\
119042	8.68362563344327e-06 \\
119182	7.87115978861941e-06 \\
119322	7.11000555447816e-06 \\
119462	6.39689912007579e-06 \\
119602	5.93003959376004e-06 \\
119742	5.50723496972605e-06 \\
119882	5.11583332352128e-06 \\
120022	4.75315622278094e-06 \\
120162	4.41681804569383e-06 \\
120302	4.10467883821264e-06 \\
120442	3.8148097571411e-06 \\
120582	3.54568661398158e-06 \\
120722	3.30867768011389e-06 \\
120862	3.08863641834689e-06 \\
121002	2.88418933863666e-06 \\
121142	2.6941202393993e-06 \\
121282	2.51732482026146e-06 \\
121422	2.35279621080275e-06 \\
121562	2.19961419439807e-06 \\
121702	2.05693609572899e-06 \\
121842	1.92398893411605e-06 \\
121982	1.80006260958043e-06 \\
122122	1.68450394055775e-06 \\
122262	1.57671141404192e-06 \\
122402	1.47613053430584e-06 \\
122542	1.38224968193601e-06 \\
122682	1.29459640918483e-06 \\
122822	1.21273411424205e-06 \\
122962	1.13625904363257e-06 \\
123102	1.06479758299471e-06 \\
123242	9.98003802488157e-07 \\
123382	9.35557227521588e-07 \\
123522	8.77160809820143e-07 \\
123662	8.22539078015883e-07 \\
123802	7.71436449997775e-07 \\
123942	7.23615688646984e-07 \\
124082	6.78856490188284e-07 \\
124222	6.36954189336958e-07 \\
124362	5.97718572581396e-07 \\
124502	5.60972788610226e-07 \\
124642	5.26552347668297e-07 \\
124782	4.94304201292817e-07 \\
124922	4.64085895213184e-07 \\
125062	4.35764790973625e-07 \\
125202	4.0921734689725e-07 \\
125342	3.84328457447847e-07 \\
125482	3.60990842218634e-07 \\
125622	3.39104483160213e-07 \\
125762	3.18576104496593e-07 \\
125902	2.99318692942308e-07 \\
126042	2.81251053502185e-07 \\
126182	2.64297399077407e-07 \\
126322	2.48386971601899e-07 \\
126462	2.33453690046126e-07 \\
126602	2.19435826065428e-07 \\
126742	2.06275701852832e-07 \\
126882	1.93919411639598e-07 \\
127022	1.82316562902241e-07 \\
127162	1.71420036498837e-07 \\
127302	1.61185763902782e-07 \\
127442	1.51572520534771e-07 \\
127582	1.42541733916257e-07 \\
127722	1.34057306200308e-07 \\
127862	1.26085447527124e-07 \\
128002	1.18594522757753e-07 \\
128142	1.11554908532874e-07 \\
128282	1.04938859823989e-07 \\
128422	9.87203862545805e-08 \\
128562	9.28751366924274e-08 \\
128702	8.73802920575706e-08 \\
128842	8.22144658463309e-08 \\
128982	7.73576098733741e-08 \\
129122	7.27909287290274e-08 \\
129262	6.84967978448192e-08 \\
129402	6.44586888309817e-08 \\
129542	6.06610984221767e-08 \\
129682	5.70894836959823e-08 \\
129822	5.3730199678359e-08 \\
129962	5.05704432218756e-08 \\
130102	4.75981989933594e-08 \\
130242	4.48021895138595e-08 \\
130382	4.21718290288808e-08 \\
130522	3.96971794325296e-08 \\
130662	3.7368909744373e-08 \\
130802	3.51782584728788e-08 \\
130942	3.31169973666334e-08 \\
131082	3.11773983852071e-08 \\
131222	2.93522028349535e-08 \\
131362	2.76345915595222e-08 \\
131502	2.60181584610386e-08 \\
131642	2.44968846319082e-08 \\
131782	2.3065113985421e-08 \\
131922	2.17175319949803e-08 \\
132062	2.04491434896426e-08 \\
132202	1.92552536137924e-08 \\
132342	1.81314497860185e-08 \\
132482	1.7073583880034e-08 \\
132622	1.60777561819536e-08 \\
132762	1.51403002357497e-08 \\
132902	1.42577691319978e-08 \\
133042	1.34269211859994e-08 \\
133182	1.26447086135073e-08 \\
133322	1.19082649296942e-08 \\
133462	1.1214894402034e-08 \\
133602	1.05620616697166e-08 \\
133742	9.94738197368505e-09 \\
133882	9.36861199729577e-09 \\
134022	8.82364159515703e-09 \\
134162	8.3104860215677e-09 \\
134302	7.82727754833346e-09 \\
134442	7.37225930302898e-09 \\
134582	6.94377805254831e-09 \\
134722	6.54027854096739e-09 \\
134862	6.16029716127287e-09 \\
135002	5.80245668180268e-09 \\
135142	5.4654612502425e-09 \\
135282	5.14809100904401e-09 \\
135422	4.84919843168896e-09 \\
135562	4.56770299361864e-09 \\
135702	4.30258823014285e-09 \\
135842	4.05289724003666e-09 \\
135982	3.81772913282674e-09 \\
136122	3.59623619772265e-09 \\
136262	3.38762029539197e-09 \\
136402	3.1911304154697e-09 \\
136542	3.00605923486685e-09 \\
136682	2.83174111936901e-09 \\
136822	2.66754912603417e-09 \\
136962	2.5128933378582e-09 \\
137102	2.36721847679533e-09 \\
137242	2.23000118371175e-09 \\
137382	2.10074929674064e-09 \\
137522	1.97899907572463e-09 \\
137662	1.86431409199272e-09 \\
137802	1.75628317444776e-09 \\
137942	1.6545193548545e-09 \\
138082	1.55865836903857e-09 \\
138222	1.4683567695073e-09 \\
138362	1.38329164789397e-09 \\
138502	1.30315841451178e-09 \\
138642	1.22767079835384e-09 \\
138782	1.1565587376694e-09 \\
138922	1.08956821343043e-09 \\
139062	1.02645986155281e-09 \\
139202	9.670082512514e-10 \\
139342	9.11001163395042e-10 \\
139482	8.58238646817e-10 \\
139622	8.08532407692297e-10 \\
139762	7.6170508789275e-10 \\
139902	7.17589820897757e-10 \\
140042	6.76029232593578e-10 \\
140182	6.36874941672971e-10 \\
140322	5.99987726168649e-10 \\
140462	5.65235969141042e-10 \\
140602	5.32495880722905e-10 \\
140742	5.01650998518954e-10 \\
140882	4.72591132893996e-10 \\
141022	4.45213310662496e-10 \\
141162	4.19419721175984e-10 \\
141302	3.95118604501477e-10 \\
141442	3.72223696309959e-10 \\
141582	3.50653395209122e-10 \\
141722	3.30330873765661e-10 \\
141862	3.11183967482975e-10 \\
142002	2.93144841734261e-10 \\
142142	2.76149048072938e-10 \\
142282	2.60136190366467e-10 \\
142422	2.45049536218289e-10 \\
142562	2.30835295322862e-10 \\
142702	2.17443008043716e-10 \\
142842	2.048252678577e-10 \\
142982	1.92936999710014e-10 \\
143122	1.81736126148024e-10 \\
143262	1.71182790165147e-10 \\
143402	1.61239632756605e-10 \\
143542	1.51871293319061e-10 \\
143682	1.4304452067293e-10 \\
143822	1.34728006528917e-10 \\
143962	1.26892274465717e-10 \\
144102	1.19509346863111e-10 \\
144242	1.12553188991171e-10 \\
144382	1.05999098387599e-10 \\
144522	9.98237603688779e-11 \\
144662	9.40052480302711e-11 \\
144802	8.85230777569745e-11 \\
144942	8.33577096237548e-11 \\
145082	7.84907694395542e-11 \\
145222	7.39050487474913e-11 \\
145362	6.95842827802551e-11 \\
145502	6.55132059712571e-11 \\
145642	6.16773854211772e-11 \\
145782	5.80631653868124e-11 \\
145922	5.46577227922285e-11 \\
146062	5.14490672287593e-11 \\
146202	4.84257078881001e-11 \\
146342	4.55769311180632e-11 \\
146482	4.28928004225781e-11 \\
146622	4.03637678836333e-11 \\
146762	3.79808962058803e-11 \\
146902	3.57355256497272e-11 \\
147042	3.36198846540015e-11 \\
147182	3.1626479213287e-11 \\
147322	2.97481483890749e-11 \\
147462	2.797834186552e-11 \\
147602	2.6310786882533e-11 \\
147742	2.4739488235781e-11 \\
147882	2.32590058324433e-11 \\
148022	2.18639550908506e-11 \\
148162	2.05495065408456e-11 \\
148302	1.9310997245725e-11 \\
148442	1.81439863133903e-11 \\
148582	1.70443104074991e-11 \\
148722	1.60081392586164e-11 \\
148862	1.50318646419123e-11 \\
149002	1.41119893548591e-11 \\
149142	1.32451272172318e-11 \\
149282	1.24284471603175e-11 \\
149422	1.16588405596474e-11 \\
149562	1.09336428799622e-11 \\
149702	1.02503561194567e-11 \\
149842	9.60659329862779e-12 \\
149982	8.9999674379726e-12 \\
150122	8.42825809144188e-12 \\
150262	7.88963339104498e-12 \\
150402	7.38209493533759e-12 \\
150542	6.9038663674803e-12 \\
150682	6.45328235293618e-12 \\
150822	6.02867755716829e-12 \\
150962	5.62855317909339e-12 \\
151102	5.25157695108192e-12 \\
151242	4.89630558320187e-12 \\
151382	4.56157334127738e-12 \\
151522	4.24627000228384e-12 \\
151662	3.94900778744045e-12 \\
151802	3.66895402947875e-12 \\
151942	3.40505401652536e-12 \\
152082	3.15641957016055e-12 \\
152222	2.92216251196464e-12 \\
152362	2.70139466351793e-12 \\
152502	2.49333886870318e-12 \\
152642	2.29744001600807e-12 \\
152782	2.11269890471044e-12 \\
152922	1.93861593444922e-12 \\
153062	1.77469150486331e-12 \\
153202	1.62014845983549e-12 \\
153342	1.4745427101559e-12 \\
153482	1.33737465546346e-12 \\
153622	1.2079781619434e-12 \\
153762	1.08624220729325e-12 \\
153902	9.71389635395781e-13 \\
154042	8.63198401646059e-13 \\
154182	7.6127992798547e-13 \\
154322	6.65301147506625e-13 \\
154462	5.74817970999675e-13 \\
154602	4.89497331557232e-13 \\
154742	4.0911718457437e-13 \\
154882	3.33399974294935e-13 \\
155022	2.62123656113999e-13 \\
155162	1.94844140821715e-13 \\
155302	1.31616939569312e-13 \\
155442	7.18869408444789e-14 \\
155582	1.55986334959834e-14 \\
155722	-3.73590047786365e-14 \\
155862	-8.73190408867686e-14 \\
156002	-1.34392497130875e-13 \\
156142	-1.78690395813419e-13 \\
156282	-2.20601314993019e-13 \\
156422	-2.5995872121598e-13 \\
156562	-2.96984659087229e-13 \\
156702	-3.32012195514153e-13 \\
156842	-3.64930308194289e-13 \\
156982	-3.95905530581331e-13 \\
157122	-4.25270929582666e-13 \\
157262	-4.5280446059337e-13 \\
157402	-4.78728168218368e-13 \\
157542	-5.03264097062583e-13 \\
157682	-5.26245713672324e-13 \\
157822	-5.47950573803746e-13 \\
};
\addplot [line width=0.01pt, red, forget plot]
table [row sep=\\]{%
1181	2.29211321943523 \\
2361	1.84641105864483 \\
3541	1.53319176828603 \\
4721	1.27240848206626 \\
5901	1.04735034735252 \\
7081	0.862902972255876 \\
8261	0.709934589387834 \\
9441	0.587712311783133 \\
10621	0.486599689222084 \\
11801	0.40171212619455 \\
12981	0.330440398716718 \\
14161	0.272396495674267 \\
15341	0.221424870460077 \\
16521	0.179552641426813 \\
17701	0.142950937308796 \\
18881	0.112773837568117 \\
20061	0.0886181179041607 \\
21241	0.0682308529896365 \\
22421	0.0529366548080245 \\
23601	0.0417598425722309 \\
24781	0.0330900234445943 \\
25961	0.025796175465953 \\
27141	0.0206380200637976 \\
28321	0.017201104291569 \\
29501	0.0144422270905017 \\
30681	0.0122018338646535 \\
31861	0.0104084071379218 \\
33041	0.00904579450527376 \\
34221	0.00802672749643984 \\
35401	0.00711228385741225 \\
36581	0.0062828153643984 \\
37761	0.00553068827437669 \\
38941	0.00485168110003747 \\
40121	0.00423437067382754 \\
41301	0.00366576359524351 \\
42481	0.00314204349791486 \\
43661	0.0026583124568586 \\
44841	0.00221106582118452 \\
46021	0.00179715698942373 \\
47201	0.00149651267532225 \\
48381	0.00132036702724969 \\
49561	0.00121532287173204 \\
50741	0.00112020947900876 \\
51921	0.00103322645268505 \\
53101	0.000953527448351366 \\
54281	0.000880405242013216 \\
55461	0.00081323339806294 \\
56641	0.000751454000797624 \\
57821	0.000694569204365614 \\
59001	0.000642134010444717 \\
60181	0.000593749988764436 \\
61361	0.000549059805515228 \\
62541	0.000507742449707083 \\
63721	0.000469509063317386 \\
64901	0.000434099294304435 \\
66081	0.000401278102868718 \\
67261	0.000372061740203378 \\
68441	0.000345693413902282 \\
69621	0.000321326616033901 \\
70801	0.000298788066702915 \\
71981	0.000277933177666057 \\
73161	0.000258625529369949 \\
74341	0.000240736382297546 \\
75521	0.000224151004751094 \\
76701	0.000208765071887329 \\
77881	0.000194483603764117 \\
79061	0.000181220024796369 \\
80241	0.000168895328162444 \\
81421	0.000157437332394317 \\
82601	0.000146780019116111 \\
83781	0.00013686294230103 \\
84961	0.000127630700636816 \\
86141	0.00011903246563677 \\
87321	0.000111021559053848 \\
88501	0.000103555073949324 \\
89681	9.66192642817276e-05 \\
90861	9.0284377638028e-05 \\
92041	8.43791300869645e-05 \\
93221	7.88726550502994e-05 \\
94401	7.3736548681258e-05 \\
95581	6.89446144525041e-05 \\
96761	6.44726632547665e-05 \\
97941	6.02983376091215e-05 \\
99121	5.64009549948175e-05 \\
100301	5.27613674754512e-05 \\
101481	4.93618354047154e-05 \\
102661	4.61859133910614e-05 \\
103841	4.32183470086556e-05 \\
105021	4.04449789920269e-05 \\
106201	3.78526638387111e-05 \\
107381	3.54291899078429e-05 \\
108561	3.31632082281041e-05 \\
109741	3.1044167330796e-05 \\
110921	2.90622535145113e-05 \\
112101	2.72083360185471e-05 \\
113281	2.5473916644092e-05 \\
114461	2.38510834129069e-05 \\
115017	2.23324679012316e-05 \\
115157	2.09112059209593e-05 \\
115297	1.95809012595305e-05 \\
115437	1.83355922118156e-05 \\
115577	1.71697206712285e-05 \\
115717	1.60781035632995e-05 \\
115857	1.50559064286404e-05 \\
115997	1.40986189809955e-05 \\
116137	1.32020324784543e-05 \\
116277	1.23622187639949e-05 \\
116417	1.15755108414106e-05 \\
116557	1.08384848682697e-05 \\
116697	1.01479434513885e-05 \\
116837	9.50090014800731e-06 \\
116977	8.89456507807695e-06 \\
117117	8.32633156133689e-06 \\
117257	7.79376370518836e-06 \\
117397	7.29458486864454e-06 \\
117537	6.82666693818668e-06 \\
117677	6.38802035518582e-06 \\
117817	5.97678483948982e-06 \\
117957	5.59122075627361e-06 \\
118097	5.24624098446891e-06 \\
118237	4.92317165301515e-06 \\
118377	4.62007143936516e-06 \\
118517	4.33566616514502e-06 \\
118657	4.06877079217605e-06 \\
118797	3.81841827729046e-06 \\
118937	3.58465687921861e-06 \\
119077	3.3653804143996e-06 \\
119217	3.15967066044642e-06 \\
119357	2.96667097804404e-06 \\
119497	2.7855811925992e-06 \\
119637	2.61565346248993e-06 \\
119777	2.45618854738394e-06 \\
119917	2.30653240007594e-06 \\
120057	2.16607304132088e-06 \\
120197	2.03423768446731e-06 \\
120337	1.91049008696487e-06 \\
120477	1.79432810076818e-06 \\
120617	1.68528140681579e-06 \\
120757	1.58290941471018e-06 \\
120897	1.48679931299966e-06 \\
121037	1.39656425768298e-06 \\
121177	1.31184168689069e-06 \\
121317	1.23229175280626e-06 \\
121457	1.15759585983533e-06 \\
121597	1.08745530169596e-06 \\
121737	1.02158999198965e-06 \\
121877	9.59737276706729e-07 \\
122017	9.01650827778155e-07 \\
122157	8.47099608347612e-07 \\
122297	7.95866906211362e-07 \\
122437	7.47749430318745e-07 \\
122577	7.02556466281035e-07 \\
122717	6.60109086336735e-07 \\
122857	6.20239409609979e-07 \\
122997	5.82789911440784e-07 \\
123137	5.47612775181339e-07 \\
123277	5.14569286735878e-07 \\
123417	4.83529267514271e-07 \\
123557	4.54370542357641e-07 \\
123697	4.26978442658044e-07 \\
123837	4.01245339676226e-07 \\
123977	3.77070208612551e-07 \\
124117	3.54358218712658e-07 \\
124257	3.33020349962965e-07 \\
124397	3.12973034377606e-07 \\
124537	2.94137818490636e-07 \\
124677	2.76441047664111e-07 \\
124817	2.59813570713252e-07 \\
124957	2.44190461462512e-07 \\
125097	2.29510759564011e-07 \\
125237	2.15717225693357e-07 \\
125377	2.02756112788194e-07 \\
125517	1.90576952052712e-07 \\
125657	1.7913235089706e-07 \\
125797	1.68377804643516e-07 \\
125937	1.58271518835296e-07 \\
126077	1.48774244312921e-07 \\
126217	1.39849119507041e-07 \\
126357	1.31461525665344e-07 \\
126497	1.23578949018377e-07 \\
126637	1.16170852160202e-07 \\
126777	1.09208552923068e-07 \\
126917	1.02665111467726e-07 \\
127057	9.65152235354871e-08 \\
127197	9.07351208057072e-08 \\
127337	8.53024769709165e-08 \\
127477	8.01963199181799e-08 \\
127617	7.53969490174811e-08 \\
127757	7.08858576836668e-08 \\
127897	6.66456610454169e-08 \\
128037	6.26600266673272e-08 \\
128177	5.89136117667977e-08 \\
128317	5.53920015411435e-08 \\
128457	5.20816541560443e-08 \\
128597	4.89698465666599e-08 \\
128737	4.60446255012847e-08 \\
128877	4.32947601103351e-08 \\
129017	4.07096985011179e-08 \\
129157	3.82795261599789e-08 \\
129297	3.59949278161409e-08 \\
129437	3.38471502492332e-08 \\
129577	3.18279692046453e-08 \\
129717	2.99296563643914e-08 \\
129857	2.81449498706898e-08 \\
129997	2.64670259042532e-08 \\
130137	2.48894721499582e-08 \\
130277	2.34062628168275e-08 \\
130417	2.20117351568128e-08 \\
130557	2.07005679819794e-08 \\
130697	1.94677604592464e-08 \\
130837	1.83086126814835e-08 \\
130977	1.72187079594543e-08 \\
131117	1.61938953913143e-08 \\
131257	1.52302738198884e-08 \\
131397	1.43241766781266e-08 \\
131537	1.34721581668273e-08 \\
131677	1.26709795988944e-08 \\
131817	1.19175966872831e-08 \\
131957	1.12091483317478e-08 \\
132097	1.05429454611006e-08 \\
132237	9.91646020853665e-09 \\
132377	9.32731680780563e-09 \\
132517	8.77328171222658e-09 \\
132657	8.25225593414913e-09 \\
132797	7.76226621868048e-09 \\
132937	7.30145754967992e-09 \\
133077	6.86808637739844e-09 \\
133217	6.46051362407363e-09 \\
133357	6.07719835565845e-09 \\
133497	5.7166923972396e-09 \\
133637	5.37763372721045e-09 \\
133777	5.05874248046823e-09 \\
133917	4.75881511974308e-09 \\
134057	4.47672021675061e-09 \\
134197	4.21139439987783e-09 \\
134337	3.96183724715726e-09 \\
134477	3.72710895479855e-09 \\
134617	3.50632545220719e-09 \\
134757	3.29865573744925e-09 \\
134897	3.10331838004885e-09 \\
135037	2.91957885645289e-09 \\
135177	2.7467463303843e-09 \\
135317	2.58417154341828e-09 \\
135457	2.43124348431323e-09 \\
135597	2.28738822327657e-09 \\
135737	2.15206602538487e-09 \\
135877	2.02476874155977e-09 \\
136017	1.90501936447873e-09 \\
136157	1.7923688089283e-09 \\
136297	1.68639507913682e-09 \\
136437	1.58670143690642e-09 \\
136577	1.49291440321164e-09 \\
136717	1.40468292553209e-09 \\
136857	1.32167704558483e-09 \\
136997	1.24358645603451e-09 \\
137137	1.17011911271447e-09 \\
137277	1.1010006240042e-09 \\
137417	1.03597269651701e-09 \\
137557	9.74792468966257e-10 \\
137697	9.17231901542692e-10 \\
137837	8.63076166091048e-10 \\
137977	8.12122924465086e-10 \\
138117	7.64182661594504e-10 \\
138257	7.19076687083486e-10 \\
138397	6.7663663561035e-10 \\
138537	6.3670491101675e-10 \\
138677	5.99132687906234e-10 \\
138817	5.637800781777e-10 \\
138957	5.3051552040273e-10 \\
139097	4.99215613292137e-10 \\
139237	4.69763894450637e-10 \\
139377	4.42051006910305e-10 \\
139517	4.15973755441001e-10 \\
139657	3.91435661661887e-10 \\
139797	3.68345354218036e-10 \\
139937	3.46617456958853e-10 \\
140077	3.26171145648146e-10 \\
140217	3.06930758586788e-10 \\
140357	2.88824908434293e-10 \\
140497	2.71786593231127e-10 \\
140637	2.55752641287188e-10 \\
140777	2.40663766692961e-10 \\
140917	2.26464069719157e-10 \\
141057	2.13100981305558e-10 \\
141197	2.00525152038722e-10 \\
141337	1.88690008062764e-10 \\
141477	1.77551973123968e-10 \\
141617	1.67069746925819e-10 \\
141757	1.57204582684756e-10 \\
141897	1.47920231619025e-10 \\
142037	1.39182165792562e-10 \\
142177	1.30958521804558e-10 \\
142317	1.23218601988384e-10 \\
142457	1.15933929123457e-10 \\
142597	1.09077802346036e-10 \\
142737	1.02625019593461e-10 \\
142877	9.65515445372489e-11 \\
143017	9.08352837392101e-11 \\
143157	8.54549209172717e-11 \\
143297	8.03907496127465e-11 \\
143437	7.56243401234258e-11 \\
143577	7.11378733697643e-11 \\
143717	6.69150290733e-11 \\
143857	6.29402641116883e-11 \\
143997	5.9198923541004e-11 \\
144137	5.5677296106893e-11 \\
144277	5.23624477111184e-11 \\
144417	4.92423324338631e-11 \\
144557	4.63051263999148e-11 \\
144697	4.35405600462957e-11 \\
144837	4.09380307431206e-11 \\
144977	3.8488268128134e-11 \\
145117	3.61821683725339e-11 \\
145257	3.40114048036355e-11 \\
145397	3.19680393268129e-11 \\
145537	3.00443558920449e-11 \\
145677	2.82336931611837e-11 \\
145817	2.65289457068718e-11 \\
145957	2.49242293470786e-11 \\
146097	2.34134933663199e-11 \\
146237	2.19914086940776e-11 \\
146377	2.06525907486821e-11 \\
146517	1.93921545488251e-11 \\
146657	1.82057147135595e-11 \\
146797	1.70886638173329e-11 \\
146937	1.60370605684079e-11 \\
147077	1.50470191861984e-11 \\
147217	1.41149314458744e-11 \\
147357	1.32375221895131e-11 \\
147497	1.24113497257383e-11 \\
147637	1.16335829858372e-11 \\
147777	1.09013353899456e-11 \\
147917	1.02118868916534e-11 \\
148057	9.56285051145755e-12 \\
148197	8.95178375870387e-12 \\
148337	8.37635516504065e-12 \\
148477	7.83467735132604e-12 \\
148617	7.32469640496447e-12 \\
148757	6.84446943566286e-12 \\
148897	6.3922200865818e-12 \\
149037	5.96650506778929e-12 \\
149177	5.56582557820207e-12 \\
149317	5.18840526098074e-12 \\
149457	4.83307838194946e-12 \\
149597	4.49845716232744e-12 \\
149737	4.18343137909005e-12 \\
149877	3.88689080921267e-12 \\
150017	3.60750318506575e-12 \\
150157	3.34454686168328e-12 \\
150297	3.09696712719187e-12 \\
150437	2.86376478086936e-12 \\
150577	2.64427368890097e-12 \\
150717	2.43749465056453e-12 \\
150857	2.24281704319651e-12 \\
150997	2.05957473298213e-12 \\
151137	1.88687954150168e-12 \\
151277	1.72428737954533e-12 \\
151417	1.57135415790322e-12 \\
151557	1.42724720930687e-12 \\
151697	1.29146693339521e-12 \\
151837	1.16368026326086e-12 \\
151977	1.04338759854272e-12 \\
152117	9.29978316577262e-13 \\
152257	8.23396906213247e-13 \\
152397	7.22866211333439e-13 \\
152537	6.28219698484145e-13 \\
152677	5.39179811909207e-13 \\
152817	4.55191440096314e-13 \\
152957	3.76088049591772e-13 \\
153097	3.01647595790655e-13 \\
153237	2.31648034088039e-13 \\
153377	1.65645275274073e-13 \\
153517	1.03472785895065e-13 \\
153657	4.49640324973188e-14 \\
153797	-1.02695629777827e-14 \\
153937	-6.21724893790088e-14 \\
154077	-1.11077813613747e-13 \\
154217	-1.5709655798446e-13 \\
154357	-2.00450767096072e-13 \\
154497	-2.41251463251047e-13 \\
154637	-2.79831713356771e-13 \\
154777	-3.15913961657088e-13 \\
154917	-3.50164341966774e-13 \\
155057	-3.82305298529673e-13 \\
155197	-4.12447853648246e-13 \\
155337	-4.40925074229881e-13 \\
155477	-4.6779247142581e-13 \\
155617	-4.93050045236032e-13 \\
155757	-5.17030862567935e-13 \\
155897	-5.39401856514132e-13 \\
156037	-5.60440582830779e-13 \\
156177	-5.80369086122801e-13 \\
156317	-5.99131855238966e-13 \\
156457	-6.16673379028043e-13 \\
156597	-6.33326724397421e-13 \\
156737	-6.49036380195867e-13 \\
156877	-6.6374683527215e-13 \\
157017	-6.77624623079964e-13 \\
157157	-6.90780765921772e-13 \\
157297	-7.02993219192649e-13 \\
};
\addplot [line width=0.01pt, red, forget plot]
table [row sep=\\]{%
1171	2.52430460413437 \\
2341	2.01167707145962 \\
3511	1.62363353814232 \\
4681	1.33291681993213 \\
5851	1.09828910976205 \\
7021	0.908679129099792 \\
8191	0.751059710387066 \\
9361	0.616326344471472 \\
10531	0.501902077661716 \\
11701	0.412101593649623 \\
12871	0.342658551297501 \\
14041	0.283968118707058 \\
15211	0.235488467559051 \\
16381	0.198982673249253 \\
17551	0.167526179924386 \\
18721	0.142056853268688 \\
19891	0.123599012972727 \\
21061	0.107628452499175 \\
22231	0.0929078542067469 \\
23401	0.07942688888628 \\
24571	0.0688040619473749 \\
25741	0.0602102290940356 \\
26911	0.0523875165012482 \\
28081	0.0452174582105339 \\
29251	0.0388949953365847 \\
30421	0.033733899825814 \\
31591	0.0290614863149674 \\
32761	0.0249509607479785 \\
33931	0.0211855328867768 \\
35101	0.0177867391826022 \\
36271	0.0149116279667021 \\
37441	0.0124047462359967 \\
38611	0.0108326318461808 \\
39781	0.00956307349968682 \\
40951	0.00840067814344431 \\
42121	0.00736191555259674 \\
43291	0.00646230832720773 \\
44461	0.00570739743108489 \\
45631	0.00513749303108341 \\
46801	0.00463143118278625 \\
47971	0.00416770934334826 \\
49141	0.00375984778105726 \\
50311	0.00336408886161749 \\
51481	0.00303697433394778 \\
52651	0.00272950732994931 \\
53821	0.00245564607203508 \\
54991	0.00228042028537678 \\
56161	0.00210722128575053 \\
57331	0.00194856499863588 \\
58501	0.00180833162687327 \\
59671	0.00168148336523172 \\
60841	0.00156660959281035 \\
62011	0.00145555558375654 \\
63181	0.00135276352580538 \\
64351	0.00125752849395938 \\
65521	0.00116741025806383 \\
66691	0.00108896131769015 \\
67861	0.00101432716021121 \\
69031	0.00094800208840945 \\
70201	0.000884812084411934 \\
71371	0.000829819897109896 \\
72541	0.000775810297247992 \\
73711	0.000725995709432548 \\
74881	0.000678968274588465 \\
76051	0.00063656717557975 \\
77221	0.000596238894121082 \\
78391	0.000558436719481969 \\
79561	0.000523571844556148 \\
80731	0.000490915611932818 \\
81901	0.000460168945205675 \\
83071	0.000431480353903779 \\
84241	0.000404380122483616 \\
85411	0.000378958530207751 \\
86581	0.000355365305354349 \\
87751	0.000332571926192793 \\
88921	0.000311875008329021 \\
90091	0.000292345145187167 \\
91261	0.00027390168864716 \\
92431	0.000257349904246806 \\
93601	0.000241700169218706 \\
94771	0.000226779402968169 \\
95941	0.000212731674794076 \\
97111	0.000199713514512756 \\
98281	0.000187391311224472 \\
99451	0.000176008871104028 \\
100621	0.000165383026072052 \\
101791	0.000155298905658852 \\
102961	0.000145655891164498 \\
104131	0.000137010454337227 \\
105301	0.00012852276345604 \\
106471	0.000120692783792176 \\
107641	0.000113264278706227 \\
108811	0.000106299313363201 \\
109981	9.98327178367009e-05 \\
111151	9.37122848761462e-05 \\
112321	8.79986905908958e-05 \\
113491	8.26572320387631e-05 \\
114661	7.77221608994627e-05 \\
115831	7.29567322175217e-05 \\
117001	6.84161116117354e-05 \\
118171	6.41870022419222e-05 \\
119341	6.02843882355764e-05 \\
120511	5.66706283790519e-05 \\
121681	5.3184826606878e-05 \\
122851	4.99995232537964e-05 \\
124021	4.70042156959782e-05 \\
125191	4.41412810925912e-05 \\
126361	4.15177406616896e-05 \\
127531	3.89559492299796e-05 \\
128701	3.65815137559622e-05 \\
129871	3.44204156212946e-05 \\
131041	3.23484687821951e-05 \\
132211	3.03857059678059e-05 \\
133381	2.85390482326897e-05 \\
134551	2.68764380818021e-05 \\
135721	2.52696224927185e-05 \\
136891	2.37449614662877e-05 \\
138061	2.2301160315541e-05 \\
139231	2.09699156294985e-05 \\
140401	1.97307751713671e-05 \\
141571	1.85462901569067e-05 \\
142741	1.74247814480055e-05 \\
143911	1.63645148574809e-05 \\
145081	1.53619279713024e-05 \\
146251	1.44341110707158e-05 \\
147421	1.35754610682826e-05 \\
148591	1.27688993378783e-05 \\
149761	1.19843674197262e-05 \\
150931	1.12639474251774e-05 \\
152101	1.05950987197567e-05 \\
153271	9.95149193788736e-06 \\
154441	9.34511483868805e-06 \\
155611	8.78397153242227e-06 \\
156781	8.2468373948652e-06 \\
157951	7.74550840859645e-06 \\
159121	7.27903719288658e-06 \\
160291	6.83179566762782e-06 \\
161461	6.42289966318321e-06 \\
162631	6.04614072508003e-06 \\
163801	5.67927271871715e-06 \\
164971	5.33772916305741e-06 \\
166141	5.01414905090813e-06 \\
167311	4.71545675745366e-06 \\
168481	4.43167775238118e-06 \\
169651	4.16545904230459e-06 \\
170821	3.9171146324346e-06 \\
171991	3.6800344320187e-06 \\
173161	3.45870164081274e-06 \\
174331	3.25365232795027e-06 \\
175501	3.05580415310702e-06 \\
176671	2.87417656724687e-06 \\
177841	2.70182696676668e-06 \\
179011	2.54281963130154e-06 \\
180181	2.39205635638173e-06 \\
181351	2.25054144864512e-06 \\
182521	2.11653955339841e-06 \\
183691	1.98945752777346e-06 \\
184861	1.87054335970549e-06 \\
186031	1.75728276496923e-06 \\
187201	1.65138205393545e-06 \\
188371	1.55314270428031e-06 \\
189541	1.46175552667716e-06 \\
190711	1.37433610714055e-06 \\
191881	1.29449997932074e-06 \\
193051	1.21731597424457e-06 \\
194221	1.14317798716179e-06 \\
195391	1.07464581960981e-06 \\
196561	1.01024130783056e-06 \\
197731	9.50773993313447e-07 \\
198901	8.95276147550028e-07 \\
200071	8.41710471877199e-07 \\
201241	7.91589704507434e-07 \\
202411	7.45112777755796e-07 \\
203581	7.00610297110771e-07 \\
204751	6.59762845522938e-07 \\
205921	6.2087786362186e-07 \\
207091	5.839850142908e-07 \\
208261	5.4965882928526e-07 \\
209431	5.16558905483411e-07 \\
210601	4.8572131833291e-07 \\
211771	4.56926703862326e-07 \\
212941	4.29865292306264e-07 \\
214111	4.04541239007461e-07 \\
215281	3.80450346837424e-07 \\
216451	3.57867000533307e-07 \\
217621	3.36911367748716e-07 \\
218791	3.16752726770009e-07 \\
219961	2.97557088635436e-07 \\
221131	2.8005177588053e-07 \\
222301	2.63417444135161e-07 \\
223471	2.48103170952252e-07 \\
224641	2.33306234165287e-07 \\
225811	2.19374084731161e-07 \\
226981	2.06442266226237e-07 \\
228151	1.94328967639823e-07 \\
229321	1.82567316542315e-07 \\
230491	1.71649302993515e-07 \\
231661	1.61609447657529e-07 \\
232831	1.52100733619154e-07 \\
234001	1.43163327082085e-07 \\
235171	1.34613569235942e-07 \\
236341	1.26613433737433e-07 \\
237511	1.19155172784158e-07 \\
238681	1.12095197457585e-07 \\
239851	1.05377957027031e-07 \\
241021	9.91508850578526e-08 \\
242191	9.32603299585999e-08 \\
243361	8.78473951915204e-08 \\
244531	8.27243554768486e-08 \\
245701	7.77960626474616e-08 \\
246871	7.32021030458441e-08 \\
248041	6.89303564227473e-08 \\
249211	6.47828076183465e-08 \\
250381	6.09087877911918e-08 \\
251551	5.72488415007122e-08 \\
252721	5.38567768160192e-08 \\
253891	5.06759155194381e-08 \\
255061	4.77002941590143e-08 \\
256231	4.48517901152634e-08 \\
257401	4.21726935595501e-08 \\
258571	3.96803603308626e-08 \\
259741	3.73143533294318e-08 \\
260911	3.51123642383655e-08 \\
262081	3.30375363732927e-08 \\
263251	3.11197650426642e-08 \\
264421	2.9264197454193e-08 \\
265591	2.75046982411986e-08 \\
266761	2.58808871556049e-08 \\
267931	2.43457637383315e-08 \\
269101	2.29310158084139e-08 \\
270271	2.15618093934999e-08 \\
270811	2.02693934969744e-08 \\
270931	1.90453006698732e-08 \\
271051	1.78970284703261e-08 \\
271171	1.68196720995439e-08 \\
271291	1.58086674306723e-08 \\
271411	1.4859764141395e-08 \\
271531	1.39690018996497e-08 \\
271651	1.31326889918348e-08 \\
271771	1.23473821167508e-08 \\
271891	1.16098686220312e-08 \\
272011	1.09171486850634e-08 \\
272131	1.02664214352011e-08 \\
272251	9.6550691885966e-09 \\
272371	9.08064595739333e-09 \\
272491	8.54086412704902e-09 \\
272611	8.03358424228406e-09 \\
272731	7.55680451547391e-09 \\
272851	7.10865105668645e-09 \\
272971	6.68736971354278e-09 \\
273091	6.29131746698874e-09 \\
273211	5.91895527035646e-09 \\
273331	5.5688412214927e-09 \\
273451	5.23962340182038e-09 \\
273571	4.9300350468684e-09 \\
273691	4.63888782942234e-09 \\
273811	4.36506758516586e-09 \\
273931	4.10752903912126e-09 \\
274051	3.86529136475744e-09 \\
274171	3.63743452025389e-09 \\
274291	3.42309453005285e-09 \\
274411	3.22146070930174e-09 \\
274531	3.03177172256142e-09 \\
274651	2.85331303029324e-09 \\
274771	2.68541328063421e-09 \\
274891	2.52744186690634e-09 \\
275011	2.37880681819291e-09 \\
275131	2.23895163520282e-09 \\
275251	2.10735356942493e-09 \\
275371	1.98352162472659e-09 \\
275491	1.86699383730726e-09 \\
275611	1.75733683160928e-09 \\
275731	1.65414260067109e-09 \\
275851	1.55702783999345e-09 \\
275971	1.46563228220487e-09 \\
276091	1.37961730928282e-09 \\
276211	1.29866428721925e-09 \\
276331	1.22247378886442e-09 \\
276451	1.15076426165928e-09 \\
276571	1.08327080639015e-09 \\
276691	1.01974445554376e-09 \\
276811	9.5995111859537e-10 \\
276931	9.03670305252291e-10 \\
277051	8.50695014431579e-10 \\
277171	8.00829957903204e-10 \\
277291	7.53892226423858e-10 \\
277411	7.09709291335514e-10 \\
277531	6.68118782520821e-10 \\
277651	6.28968099825045e-10 \\
277771	5.92113746922251e-10 \\
277891	5.57420443136891e-10 \\
278011	5.24761234466098e-10 \\
278131	4.94016216823212e-10 \\
278251	4.65073313193898e-10 \\
278371	4.37826164212396e-10 \\
278491	4.12175515940305e-10 \\
278611	3.88027443509742e-10 \\
278731	3.65293517656795e-10 \\
278851	3.4389102676613e-10 \\
278971	3.23741589092208e-10 \\
279091	3.0477176338195e-10 \\
279211	2.86912327229771e-10 \\
279331	2.70097999521823e-10 \\
279451	2.54267717991752e-10 \\
279571	2.39363640019974e-10 \\
279691	2.25331420189434e-10 \\
279811	2.12120099263302e-10 \\
279931	1.99681438051158e-10 \\
280051	1.87970139453597e-10 \\
280171	1.76943792951079e-10 \\
280291	1.66562041936658e-10 \\
280411	1.56787194338648e-10 \\
280531	1.4758366750911e-10 \\
280651	1.38918210268457e-10 \\
280771	1.30759125749336e-10 \\
280891	1.23076659974686e-10 \\
281011	1.15843223902345e-10 \\
281131	1.09032283202026e-10 \\
281251	1.02619412967186e-10 \\
281371	9.65809654474015e-11 \\
281491	9.08952912936911e-11 \\
281611	8.55414628020412e-11 \\
281731	8.05004951587307e-11 \\
281851	7.57537921280971e-11 \\
281971	7.12840897421074e-11 \\
282091	6.70754563003584e-11 \\
282211	6.31125707251101e-11 \\
282331	5.938088909474e-11 \\
282451	5.5866922199499e-11 \\
282571	5.25581245192086e-11 \\
282691	4.9442505645203e-11 \\
282811	4.65084082357237e-11 \\
282931	4.37457847723977e-11 \\
283051	4.11441436476423e-11 \\
283171	3.86943810326557e-11 \\
283291	3.63873375874846e-11 \\
283411	3.42150197063518e-11 \\
283531	3.21693782723287e-11 \\
283651	3.02428637688479e-11 \\
283771	2.84288703689128e-11 \\
283891	2.67206257120733e-11 \\
284011	2.51119680605427e-11 \\
284131	2.35970132322905e-11 \\
284251	2.21703766456471e-11 \\
284371	2.08269512746995e-11 \\
284491	1.95617966269879e-11 \\
284611	1.83704718104138e-11 \\
284731	1.72484249105764e-11 \\
284851	1.61918811691919e-11 \\
284971	1.51968992945228e-11 \\
285091	1.42598710617392e-11 \\
285211	1.33773547794647e-11 \\
285331	1.25462418232303e-11 \\
285451	1.17635901020208e-11 \\
285571	1.10265685471234e-11 \\
285691	1.03325126232789e-11 \\
285811	9.67875779522842e-12 \\
285931	9.063194639225e-12 \\
286051	8.48332515346328e-12 \\
286171	7.93731746995263e-12 \\
286291	7.42317318724872e-12 \\
286411	6.93889390390723e-12 \\
286531	6.48275877423998e-12 \\
286651	6.05321348601251e-12 \\
286771	5.64875923814157e-12 \\
286891	5.26773069609021e-12 \\
287011	4.90885110338013e-12 \\
287131	4.57089921468423e-12 \\
287251	4.25265378467543e-12 \\
287371	3.9530045903291e-12 \\
287491	3.67061936401569e-12 \\
287611	3.4047764607692e-12 \\
287731	3.15431014641376e-12 \\
287851	2.91855428713461e-12 \\
287971	2.69639865990712e-12 \\
288091	2.48723264206774e-12 \\
288211	2.290223566348e-12 \\
288331	2.10459427663068e-12 \\
288451	1.92984517255468e-12 \\
288571	1.765254609154e-12 \\
288691	1.61015645261386e-12 \\
288811	1.46427314717812e-12 \\
288931	1.32677202557829e-12 \\
289051	1.19709797630208e-12 \\
289171	1.07508446589577e-12 \\
289291	9.60176382847067e-13 \\
289411	8.51929637946114e-13 \\
289531	7.49955653134293e-13 \\
289651	6.5403238380668e-13 \\
289771	5.63660229602192e-13 \\
289891	4.78450612462211e-13 \\
290011	3.98236998933044e-13 \\
290131	3.2263081095607e-13 \\
290251	2.51465515077598e-13 \\
290371	1.8446355554147e-13 \\
290491	1.21236354289067e-13 \\
290611	6.18394224716212e-14 \\
290731	5.82867087928207e-15 \\
290851	-4.69624339416441e-14 \\
290971	-9.66449142936199e-14 \\
};
\addplot [line width=0.01pt, red, forget plot]
table [row sep=\\]{%
1175	2.46712204238466 \\
2335	1.97924581537419 \\
3495	1.61028450653786 \\
4655	1.30965752634909 \\
5815	1.07124717268159 \\
6975	0.893137179992121 \\
8135	0.748250813904106 \\
9295	0.628748783068013 \\
10455	0.522099712806489 \\
11615	0.432016009591254 \\
12775	0.363437462397334 \\
13935	0.306396841686845 \\
15095	0.259980788523038 \\
16255	0.219168479764186 \\
17415	0.184196408995993 \\
18575	0.153426629060855 \\
19735	0.127437284869699 \\
20895	0.107205346213411 \\
22055	0.0917270606319053 \\
23215	0.0784079696306166 \\
24375	0.0675127116475849 \\
25535	0.0577814893115525 \\
26695	0.0488413203436681 \\
27855	0.0405996253429795 \\
29015	0.0352140110701172 \\
30175	0.0308806290903333 \\
31335	0.026824689687831 \\
32495	0.0230392819887061 \\
33655	0.0196593921207799 \\
34815	0.0165997217918566 \\
35975	0.0137378826821703 \\
37135	0.0110588571918924 \\
38295	0.00951268344509265 \\
39455	0.00835032606491959 \\
40615	0.00732549141056088 \\
41775	0.00641833540117587 \\
42935	0.00559268469713103 \\
44095	0.00483046411040283 \\
45255	0.00412482605036085 \\
46415	0.00346916693142274 \\
47575	0.00309883548044598 \\
48735	0.00285344460211756 \\
49895	0.00262883855309887 \\
51055	0.00242105024138356 \\
52215	0.00222853009237006 \\
53375	0.00204993490897504 \\
54535	0.00188408485343794 \\
55695	0.0017326329416294 \\
56855	0.00159298112491207 \\
58015	0.00146305769613492 \\
59175	0.00134211685834956 \\
60335	0.00122948144039786 \\
61495	0.00112453436733945 \\
62655	0.0010267116516382 \\
63815	0.000935496527103286 \\
64975	0.000850414480487283 \\
66135	0.000771193363120815 \\
67295	0.00069734443779923 \\
68455	0.000632407373548027 \\
69615	0.000579274535615626 \\
70775	0.00053366834584484 \\
71935	0.000491661649946795 \\
73095	0.000452917342241121 \\
74255	0.000417156182782652 \\
75415	0.000384127448974925 \\
76575	0.000353604443222466 \\
77735	0.000325381626158638 \\
78895	0.000299272234059234 \\
80055	0.00027510624755428 \\
81215	0.000252728642667899 \\
82375	0.000231997872506717 \\
83535	0.000212784539533717 \\
84695	0.000194970227045232 \\
85855	0.000181570512639284 \\
87015	0.000170131163209852 \\
88175	0.000159443566519812 \\
89335	0.000149441596784594 \\
90495	0.000140077057727117 \\
91655	0.000131306311989932 \\
92815	0.000123089136516319 \\
93975	0.000115388356800417 \\
95135	0.000108169565712746 \\
96295	0.000101400878702362 \\
97455	9.50527173084748e-05 \\
98615	8.90976164016899e-05 \\
99775	8.3510051617719e-05 \\
100935	7.82662841384285e-05 \\
102095	7.33442205086976e-05 \\
103255	6.87232855858255e-05 \\
104415	6.43843070424754e-05 \\
105575	6.03094100924961e-05 \\
106735	5.64819213185741e-05 \\
107895	5.28862806336017e-05 \\
109055	4.95079605493665e-05 \\
110215	4.63333920275866e-05 \\
111375	4.33498962820744e-05 \\
112535	4.05456219759182e-05 \\
113695	3.79094873195207e-05 \\
114855	3.54311266341778e-05 \\
116015	3.3100840985345e-05 \\
117175	3.09095525397596e-05 \\
118335	2.88487623274492e-05 \\
119495	2.6910511126188e-05 \\
120655	2.50873432099952e-05 \\
121815	2.33722727307484e-05 \\
122975	2.17587525187524e-05 \\
124135	2.02406451132453e-05 \\
125295	1.88121958457077e-05 \\
126455	1.74680078166567e-05 \\
127615	1.62030186219297e-05 \\
128775	1.50124786917893e-05 \\
129935	1.38919311222785e-05 \\
131095	1.28496550627633e-05 \\
132255	1.2012009960094e-05 \\
133415	1.12303979991624e-05 \\
134575	1.05007190375206e-05 \\
135735	9.81925508597437e-06 \\
136895	9.18260183857544e-06 \\
138055	8.58763059796708e-06 \\
139215	8.03145897670943e-06 \\
140375	7.51142619886958e-06 \\
141535	7.025071958211e-06 \\
142695	6.57011812782438e-06 \\
143855	6.14445277991038e-06 \\
145015	5.74611610332232e-06 \\
146175	5.37328789551594e-06 \\
147335	5.0242763780517e-06 \\
148495	4.69750813597569e-06 \\
149655	4.39151902376089e-06 \\
150815	4.10494590674659e-06 \\
151975	3.83651913715699e-06 \\
153135	3.58505567593648e-06 \\
154295	3.34945278945842e-06 \\
155455	3.12868226104435e-06 \\
156615	2.93069926865508e-06 \\
157775	2.74853017739884e-06 \\
158935	2.57803238884469e-06 \\
160095	2.41841760828532e-06 \\
161255	2.26895867366794e-06 \\
162415	2.12898104456904e-06 \\
163575	1.99785826526799e-06 \\
164735	1.87500808274121e-06 \\
165895	1.75988896095047e-06 \\
167055	1.65199692575557e-06 \\
168215	1.5508627018157e-06 \\
169375	1.4560491038984e-06 \\
170535	1.36714865539567e-06 \\
171695	1.28378140829e-06 \\
172855	1.20559294547462e-06 \\
174015	1.13225254327887e-06 \\
175175	1.06345148354059e-06 \\
176335	9.98901496074289e-07 \\
177495	9.38333324151941e-07 \\
178655	8.81495398119547e-07 \\
179815	8.28152610377053e-07 \\
180975	7.78085181896149e-07 \\
182135	7.31087612337866e-07 \\
183295	6.86967708163344e-07 \\
184455	6.45545680577619e-07 \\
185615	6.06653309143113e-07 \\
186775	5.70133165678222e-07 \\
187935	5.35837892390312e-07 \\
189095	5.03629532522254e-07 \\
190255	4.73378906629129e-07 \\
191415	4.44965033929989e-07 \\
192575	4.18274591906798e-07 \\
193735	3.93201415649358e-07 \\
194895	3.69646029618753e-07 \\
196055	3.47515212550942e-07 \\
197215	3.2672159155922e-07 \\
198375	3.07183263825728e-07 \\
199535	2.88823444272168e-07 \\
200695	2.71570135823573e-07 \\
201855	2.55355823042258e-07 \\
203015	2.40117185079658e-07 \\
204175	2.25794828667691e-07 \\
205335	2.12333037818979e-07 \\
206495	1.99679540624498e-07 \\
207655	1.87785291705378e-07 \\
208815	1.76604267210223e-07 \\
209975	1.6609327585515e-07 \\
211135	1.5621178001135e-07 \\
212295	1.46921729171634e-07 \\
213455	1.38187404463697e-07 \\
214615	1.29975272877836e-07 \\
215775	1.22253850820542e-07 \\
216935	1.1499357649436e-07 \\
218095	1.08166690493405e-07 \\
219255	1.01747124059415e-07 \\
220415	9.57103940546489e-08 \\
221575	9.0033505706355e-08 \\
222735	8.46948592925223e-08 \\
223895	7.96741657649314e-08 \\
225055	7.49523648146955e-08 \\
226215	7.05115497101616e-08 \\
227375	6.63348969087707e-08 \\
228535	6.24065991106093e-08 \\
229695	5.87118034744982e-08 \\
230855	5.52365528316834e-08 \\
232015	5.19677309518407e-08 \\
233175	4.88930118058839e-08 \\
234335	4.60008111602406e-08 \\
235495	4.32802410021971e-08 \\
236655	4.07210677400016e-08 \\
237815	3.831367245688e-08 \\
238975	3.60490131079416e-08 \\
240135	3.39185901587769e-08 \\
241295	3.19144133342775e-08 \\
242455	3.00289708654589e-08 \\
243615	2.82552007901948e-08 \\
244775	2.65864635307089e-08 \\
245935	2.50165165804894e-08 \\
247095	2.3539490523472e-08 \\
248255	2.21498662744679e-08 \\
249415	2.08424545955488e-08 \\
250575	1.96123757234545e-08 \\
251735	1.84550407733575e-08 \\
252895	1.73661347524501e-08 \\
254055	1.63415997400662e-08 \\
255215	1.53776192890476e-08 \\
256375	1.44706043814224e-08 \\
257535	1.36171794951068e-08 \\
258695	1.28141700028728e-08 \\
259855	1.20585901819403e-08 \\
261015	1.13476315011241e-08 \\
262175	1.06786526843372e-08 \\
263335	1.00491691079618e-08 \\
264495	9.45684375253109e-09 \\
265655	8.89947843196737e-09 \\
266815	8.37500546690961e-09 \\
267975	7.88147969110753e-09 \\
269135	7.41707123497193e-09 \\
270295	6.98005925281464e-09 \\
271455	6.56882415128734e-09 \\
272615	6.18184309297831e-09 \\
273775	5.81768294649621e-09 \\
274935	5.47499545699992e-09 \\
276095	5.15251269428418e-09 \\
277255	4.84904133513098e-09 \\
278415	4.5634588330401e-09 \\
279575	4.29470925489284e-09 \\
280735	4.04179878454869e-09 \\
281895	3.80379266973208e-09 \\
283055	3.57981139176289e-09 \\
284215	3.36902739039857e-09 \\
285375	3.17066206623196e-09 \\
286535	2.98398283860024e-09 \\
287695	2.80830048104974e-09 \\
288855	2.64296629026717e-09 \\
290015	2.48737008767819e-09 \\
291175	2.34093777695676e-09 \\
292335	2.20312895704566e-09 \\
293495	2.07343520131076e-09 \\
294655	1.95137839220649e-09 \\
295815	1.83650816776293e-09 \\
296975	1.72840114442963e-09 \\
298135	1.62665853009614e-09 \\
299295	1.5309055689805e-09 \\
300455	1.44078915464974e-09 \\
301615	1.35597721939718e-09 \\
302775	1.27615740197484e-09 \\
303935	1.20103565981466e-09 \\
305095	1.13033504778315e-09 \\
306255	1.06379516306987e-09 \\
307415	1.00117125700905e-09 \\
308575	9.42232403211563e-10 \\
309735	8.86761830631855e-10 \\
310895	8.34555147211091e-10 \\
312055	7.85420395388314e-10 \\
313215	7.39176442277056e-10 \\
314375	6.95653201709945e-10 \\
315535	6.5469063503798e-10 \\
316695	6.16137696418662e-10 \\
317855	5.79852887927501e-10 \\
319015	5.45702649734636e-10 \\
320175	5.13561193571377e-10 \\
321335	4.83310391707903e-10 \\
322495	4.54838888774844e-10 \\
323655	4.28042101763282e-10 \\
324815	4.02821387357477e-10 \\
325975	3.79084041934874e-10 \\
327135	3.56742746454586e-10 \\
328295	3.35715455435093e-10 \\
329455	3.15924730820427e-10 \\
330615	2.97297853002476e-10 \\
331775	2.79766376731772e-10 \\
332935	2.63265964584036e-10 \\
334095	2.47735831848672e-10 \\
335255	2.33118913062214e-10 \\
336415	2.19361417919117e-10 \\
337575	2.06412831271763e-10 \\
338735	1.9422580210815e-10 \\
339895	1.8275531088463e-10 \\
341055	1.7195928014857e-10 \\
342215	1.61797963915689e-10 \\
343375	1.52233947670055e-10 \\
344535	1.43232370408697e-10 \\
345695	1.34760036463177e-10 \\
346855	1.26785804077656e-10 \\
348015	1.19280363364282e-10 \\
349175	1.12216125280895e-10 \\
350335	1.0556722163102e-10 \\
351495	9.93092275081153e-11 \\
352655	9.34189947621178e-11 \\
353815	8.78751515998033e-11 \\
354975	8.26572699175188e-11 \\
356135	7.77460873457869e-11 \\
357295	7.31234517381552e-11 \\
358455	6.87726542381029e-11 \\
359615	6.4677596611773e-11 \\
360775	6.0823290848333e-11 \\
361935	5.71955260930679e-11 \\
363095	5.3780979669682e-11 \\
364255	5.05671615691483e-11 \\
365415	4.75421924051034e-11 \\
366575	4.46949699473009e-11 \\
367735	4.20153356550657e-11 \\
368895	3.94930199654198e-11 \\
370055	3.71190300718638e-11 \\
371215	3.48844841902007e-11 \\
372375	3.27812221811996e-11 \\
373535	3.08019165728979e-11 \\
374695	2.89385737595182e-11 \\
375855	2.71849764921228e-11 \\
377015	2.55342968991101e-11 \\
378175	2.39807063096009e-11 \\
379335	2.25183760527159e-11 \\
380495	2.11420325690881e-11 \\
381655	1.98465688328042e-11 \\
382815	1.86272663960096e-11 \\
383975	1.74796288554546e-11 \\
385135	1.63993263413431e-11 \\
386295	1.53825840953914e-11 \\
387455	1.4425682870467e-11 \\
388615	1.35249034194374e-11 \\
389775	1.26771371178336e-11 \\
390935	1.18790532965818e-11 \\
392095	1.11280429315741e-11 \\
393255	1.04210529094928e-11 \\
394415	9.75564073968371e-12 \\
395575	9.12930842034143e-12 \\
396735	8.53989101656794e-12 \\
397895	7.98494603770905e-12 \\
399055	7.46269712692538e-12 \\
400215	6.97120139392382e-12 \\
401375	6.50840492610882e-12 \\
402535	6.07303096700207e-12 \\
403695	5.66313662631046e-12 \\
404855	5.27722310295076e-12 \\
406015	4.9141246627471e-12 \\
407175	4.57239801576748e-12 \\
408335	4.25065538323111e-12 \\
409495	3.94784205326459e-12 \\
410655	3.66284780284332e-12 \\
411815	3.39456240894265e-12 \\
412975	3.14198667084042e-12 \\
414135	2.90434343241941e-12 \\
415295	2.68068900410867e-12 \\
416455	2.47013520748851e-12 \\
417615	2.27196039759292e-12 \\
418775	2.08544292945589e-12 \\
419935	1.90986115811143e-12 \\
421095	1.74449343859351e-12 \\
422255	1.58900670399476e-12 \\
423415	1.44262379819793e-12 \\
424575	1.30478960969072e-12 \\
425735	1.1751155604145e-12 \\
426895	1.0529355165545e-12 \\
428055	9.38027433505795e-13 \\
429215	8.29891710907305e-13 \\
430375	7.28084259549178e-13 \\
431535	6.32216501372795e-13 \\
432695	5.4195536947077e-13 \\
433855	4.57078819238177e-13 \\
435015	3.77198272616397e-13 \\
436175	3.01980662698043e-13 \\
437335	2.3125945602942e-13 \\
438495	1.64590563400679e-13 \\
439655	1.01973984811821e-13 \\
440815	4.2854608750531e-14 \\
441975	-1.26565424807268e-14 \\
443135	-6.50035580918029e-14 \\
444295	-1.14241949233929e-13 \\
445455	-1.60482738209566e-13 \\
446615	-2.04225525379798e-13 \\
447775	-2.45192754988466e-13 \\
448935	-2.83939538547884e-13 \\
450095	-3.20243831453126e-13 \\
451255	-3.54438700611581e-13 \\
451999	-3.86690679476942e-13 \\
452119	-4.16999768049209e-13 \\
452239	-4.45588010933307e-13 \\
452359	-4.72399896978004e-13 \\
452479	-4.97712981939458e-13 \\
452599	-5.21471754666436e-13 \\
452719	-5.43842748612633e-13 \\
452839	-5.64992497231742e-13 \\
452959	-5.84865489372532e-13 \\
453079	-6.03572747337466e-13 \\
453199	-6.21169782277775e-13 \\
};
\addplot [line width=2.0pt, red!50!black]
table [row sep=\\]{%
1181	2.54807704427716 \\
2361	2.09147793413845 \\
3541	1.72731876211738 \\
4721	1.42351149441392 \\
5901	1.16331151906464 \\
7081	0.954243147767496 \\
8261	0.783551674435691 \\
9441	0.651490268156706 \\
10621	0.543449407529678 \\
11801	0.451877359042069 \\
12981	0.375027008779758 \\
14161	0.312539617875403 \\
15341	0.26002412313769 \\
16521	0.216724965177324 \\
17701	0.180412275574293 \\
18881	0.151290607580613 \\
20061	0.126568305045871 \\
21241	0.107416899356293 \\
22421	0.0941844092575552 \\
23601	0.0814966876407969 \\
24781	0.0700494142260932 \\
25961	0.0606729190355166 \\
27141	0.0525560598675913 \\
28321	0.0455989396827407 \\
29501	0.0403374353943588 \\
30681	0.0358291818026825 \\
31861	0.0316815052865917 \\
33041	0.0278844799237931 \\
34221	0.0243623098883076 \\
35401	0.0212399802707025 \\
36581	0.0184518540087793 \\
37761	0.0161326280198119 \\
38941	0.0142412777378045 \\
40121	0.0125905024279196 \\
41301	0.0112292857738691 \\
42481	0.0101138576811793 \\
43661	0.00907891216938855 \\
44841	0.00780896170185391 \\
46021	0.00691921035467535 \\
47201	0.00621286134384175 \\
48381	0.00566831909444915 \\
49561	0.0051670742074218 \\
50741	0.00470311165915965 \\
51921	0.00427621303828768 \\
53101	0.00388069329983848 \\
54281	0.00351400897321003 \\
55461	0.00320382965696853 \\
56641	0.00294761184339837 \\
57821	0.00268895850725981 \\
59001	0.00241659742312367 \\
60181	0.0021999714413114 \\
61361	0.00201176133991837 \\
62541	0.00182057630670218 \\
63721	0.00161674258276279 \\
64901	0.00142938979470408 \\
66081	0.00127870516429102 \\
67261	0.00117955639387196 \\
68441	0.00109613728513069 \\
69621	0.00101846461517069 \\
70801	0.000946630439272622 \\
71981	0.000880139308390759 \\
73161	0.000818550899387038 \\
74341	0.000761466029739921 \\
75521	0.00071102533460804 \\
76701	0.000665657710173739 \\
77881	0.000624243287545789 \\
79061	0.000586171488413612 \\
80241	0.000550533853200863 \\
81421	0.000517152568440371 \\
82601	0.000485871826957085 \\
83781	0.000456548716167737 \\
84961	0.000429051512404299 \\
86141	0.000403258573076493 \\
87321	0.000379057391746651 \\
88501	0.000356343769167211 \\
89681	0.000334484366287302 \\
90861	0.000311065237672659 \\
92041	0.00028902453297025 \\
93221	0.000268332640044389 \\
94401	0.000248849152692687 \\
95581	0.000231396805874529 \\
96761	0.000217325106419752 \\
97941	0.000203724509688363 \\
99121	0.000189975021187172 \\
100301	0.000177266796910769 \\
101481	0.000165462899266833 \\
102661	0.000154414331512698 \\
103841	0.000144032681482664 \\
105021	0.000134561238425812 \\
106201	0.000125548021354727 \\
107381	0.000117213579680364 \\
108561	0.000109402766838518 \\
109741	0.000102437982125114 \\
110921	9.62762605581369e-05 \\
112101	9.04655049977499e-05 \\
113281	8.50248539500664e-05 \\
114461	7.99265008142314e-05 \\
115008	7.51778074231102e-05 \\
115676.5	7.0651279289291e-05 \\
115918.5	6.63659738986966e-05 \\
116067	6.23572723709165e-05 \\
116207	5.86252125440168e-05 \\
116347	5.51440001180237e-05 \\
116487	5.18266608818818e-05 \\
116627	4.87533557836928e-05 \\
116767	4.58631129574005e-05 \\
116907	4.31217628781289e-05 \\
117047	4.05777961949849e-05 \\
117187	3.81376859604488e-05 \\
117327	3.58598259652121e-05 \\
117467	3.37530724505153e-05 \\
117607	3.16936386748901e-05 \\
117747	2.9689741487815e-05 \\
117887	2.78063151109564e-05 \\
118027	2.60359640232855e-05 \\
118167	2.43717602250171e-05 \\
118307	2.28072124762724e-05 \\
118447	2.13362376679238e-05 \\
118587	1.99837306486539e-05 \\
118727	1.87599552584072e-05 \\
118867	1.76125539897987e-05 \\
119007	1.65365624960612e-05 \\
119147	1.55272899324821e-05 \\
119287	1.45805061242954e-05 \\
119427	1.36921904743126e-05 \\
119567	1.28586169751976e-05 \\
119707	1.20763112128053e-05 \\
119847	1.13420310871581e-05 \\
119987	1.06307051016552e-05 \\
120127	9.96034903305576e-06 \\
120267	9.3339208185883e-06 \\
120407	8.74777487763767e-06 \\
120547	8.19913940114381e-06 \\
120687	7.68547570090261e-06 \\
120827	7.20442034457536e-06 \\
120967	6.75381021281174e-06 \\
121107	6.33162967927925e-06 \\
121247	5.93601100135066e-06 \\
121387	5.56518164523956e-06 \\
121527	5.21757798860323e-06 \\
121667	4.89169979650539e-06 \\
121807	4.58613432330335e-06 \\
121947	4.29957730124775e-06 \\
122087	4.03083057948228e-06 \\
122227	3.77872458812156e-06 \\
122367	3.54221603310334e-06 \\
122507	3.32206107617194e-06 \\
122647	3.11329541879957e-06 \\
122787	2.91843552219229e-06 \\
122927	2.73503176356771e-06 \\
123067	2.56365180406615e-06 \\
123207	2.40965059822296e-06 \\
123347	2.26499030736571e-06 \\
123487	2.12977421631955e-06 \\
123627	2.00165626307314e-06 \\
123767	1.88152076502357e-06 \\
123907	1.76882339858464e-06 \\
124047	1.66293842190912e-06 \\
124187	1.56338362550246e-06 \\
124327	1.46987477039895e-06 \\
124467	1.38199878813028e-06 \\
124607	1.29941274301659e-06 \\
124747	1.22179461775618e-06 \\
124887	1.14884263208159e-06 \\
125027	1.0802735250226e-06 \\
125167	1.01582149197865e-06 \\
125307	9.55237083821725e-07 \\
125447	8.98286177164476e-07 \\
125587	8.44749012796076e-07 \\
125727	7.94419295624404e-07 \\
125867	7.4710335268291e-07 \\
126007	7.02619346759636e-07 \\
126147	6.60796539098119e-07 \\
126287	6.21474599837857e-07 \\
126427	5.84502962919231e-07 \\
126567	5.49740221067463e-07 \\
126707	5.17053560245007e-07 \\
126847	4.86318229186988e-07 \\
126987	4.58063132646735e-07 \\
127127	4.30242665661051e-07 \\
127267	4.07917152001058e-07 \\
127407	3.81283094041951e-07 \\
127547	3.59554950313257e-07 \\
127687	3.38313488057462e-07 \\
127827	3.1840802566796e-07 \\
127967	2.98632040707236e-07 \\
128107	2.82952031416261e-07 \\
128247	2.64560230611721e-07 \\
128387	2.4901623557616e-07 \\
128527	2.34390633568893e-07 \\
128667	2.20628267266765e-07 \\
128807	2.07677771257853e-07 \\
128947	1.95490899856932e-07 \\
129087	1.84022309057674e-07 \\
129227	1.73229377009587e-07 \\
129367	1.6307203809518e-07 \\
129507	1.53512626555052e-07 \\
129647	1.445157306601e-07 \\
129787	1.36048191878846e-07 \\
129927	1.2807841326401e-07 \\
130067	1.20577122009191e-07 \\
130207	1.13516625988019e-07 \\
130347	1.06870891192212e-07 \\
130487	1.00615433373807e-07 \\
130627	9.47272245088726e-08 \\
130767	9.03081431302688e-08 \\
130907	8.50038695543454e-08 \\
131047	7.90581285992076e-08 \\
131187	7.44344234027139e-08 \\
131327	7.00820296239968e-08 \\
131467	6.59847727568774e-08 \\
131607	6.29472526880726e-08 \\
131747	5.88009253510435e-08 \\
131887	5.55993839301827e-08 \\
132027	5.24451777561552e-08 \\
132167	4.89677078330253e-08 \\
132307	4.62551201230177e-08 \\
132447	4.37742915804229e-08 \\
132587	4.07655755818581e-08 \\
132727	3.83850455309975e-08 \\
132867	3.6143784076792e-08 \\
133007	3.40336148707543e-08 \\
133147	3.23528835388665e-08 \\
133287	3.04610066614863e-08 \\
133427	2.85664466526114e-08 \\
133567	2.70067816443564e-08 \\
133707	2.51960569630505e-08 \\
133847	2.37258873103663e-08 \\
133987	2.23416227185247e-08 \\
134127	2.10382307685997e-08 \\
134267	1.9810977025525e-08 \\
134407	1.86554049430576e-08 \\
134547	1.75673198210546e-08 \\
134687	1.65427730958179e-08 \\
134827	1.55780476851497e-08 \\
134967	1.4669644721188e-08 \\
135107	1.38142704497746e-08 \\
135247	1.3008824184535e-08 \\
135387	1.22503870936264e-08 \\
135527	1.15611560880247e-08 \\
135667	1.08817778121661e-08 \\
135807	1.02313389982456e-08 \\
135947	9.6348825029402e-09 \\
136087	9.07599667643311e-09 \\
136227	8.54389387017207e-09 \\
136367	8.04324051806304e-09 \\
136507	7.57215040669834e-09 \\
136647	7.1288536185321e-09 \\
136787	6.71168864929683e-09 \\
136927	6.31909585768753e-09 \\
137067	5.94961041544551e-09 \\
137207	5.60185708931016e-09 \\
137347	5.27454407928118e-09 \\
137487	4.96645752301461e-09 \\
137627	4.67645744350875e-09 \\
137767	4.40347214247794e-09 \\
137907	4.14649475866113e-09 \\
138047	3.90457882692985e-09 \\
138187	3.6768347810856e-09 \\
138327	3.46242617910164e-09 \\
138467	3.26056653898732e-09 \\
138607	3.07051667425284e-09 \\
138747	2.89158097466213e-09 \\
138887	2.72310579640944e-09 \\
139027	2.56447590940567e-09 \\
139167	2.41511283194384e-09 \\
139307	2.27447244371959e-09 \\
139447	2.1420427653851e-09 \\
139587	2.01734207116999e-09 \\
139727	1.89991739008022e-09 \\
139867	1.78934211891857e-09 \\
140007	1.6852153006397e-09 \\
140147	1.58715918185948e-09 \\
140287	1.49481865774348e-09 \\
140427	1.40785921809439e-09 \\
140567	1.32596628121817e-09 \\
140707	1.24884363961186e-09 \\
140847	1.17621257178513e-09 \\
140987	1.1078106765261e-09 \\
141127	1.04339098472295e-09 \\
141267	9.82720904652012e-10 \\
141407	9.25581833399747e-10 \\
141547	8.71767491528175e-10 \\
141687	8.21084034097197e-10 \\
141827	7.73348773908111e-10 \\
141967	7.28390014970159e-10 \\
142107	6.860458312552e-10 \\
142247	6.461637336308e-10 \\
142387	6.0860033679333e-10 \\
142527	5.73220637622995e-10 \\
142667	5.39897293538871e-10 \\
142807	5.08510678010055e-10 \\
142947	4.78947881354941e-10 \\
143087	4.51102988296981e-10 \\
143227	4.248756901859e-10 \\
143367	4.00172117664965e-10 \\
143507	3.76903508403359e-10 \\
143647	3.54986318118478e-10 \\
143787	3.34337946217289e-10 \\
143927	3.14833104031464e-10 \\
144067	2.96464019999831e-10 \\
144207	2.79327394547835e-10 \\
144347	2.63076060935674e-10 \\
144487	2.475319393902e-10 \\
144627	2.33078945033327e-10 \\
144767	2.1946749972912e-10 \\
144907	2.06972383676174e-10 \\
145047	1.94679772302919e-10 \\
145187	1.83230153272262e-10 \\
145327	1.72874881076979e-10 \\
145467	1.62427127303744e-10 \\
145607	1.52926560303968e-10 \\
145747	1.43978828859304e-10 \\
145887	1.35551736502038e-10 \\
146027	1.27614974143597e-10 \\
146167	1.20139898029947e-10 \\
146307	1.13514364574741e-10 \\
146447	1.06473774241778e-10 \\
146587	1.00639108158163e-10 \\
146727	9.43480293891241e-11 \\
146867	8.92138030117451e-11 \\
147007	8.39929237272941e-11 \\
147147	7.87440668226225e-11 \\
147287	7.40753014483175e-11 \\
147427	6.97122914949944e-11 \\
147567	6.56051324376961e-11 \\
147707	6.17370043975995e-11 \\
147847	5.80943071426532e-11 \\
147987	5.46632183962004e-11 \\
148127	5.14829845421616e-11 \\
148267	4.83998952027775e-11 \\
148407	4.55296911283654e-11 \\
148547	4.28287960652085e-11 \\
148687	4.02856636938509e-11 \\
148827	3.81861209319823e-11 \\
148967	3.56903950837761e-11 \\
149107	3.37908034886425e-11 \\
149247	3.15381609716781e-11 \\
149387	2.96433433355503e-11 \\
149527	2.78662093400328e-11 \\
149667	2.61929922196202e-11 \\
149807	2.4617308191921e-11 \\
149947	2.31332175637533e-11 \\
150087	2.17354467757502e-11 \\
150227	2.04190553354522e-11 \\
150367	1.93826066308134e-11 \\
150507	1.80176984443392e-11 \\
150647	1.71025971162919e-11 \\
150787	1.58789092985501e-11 \\
150927	1.49028012152996e-11 \\
151067	1.39838696178174e-11 \\
151207	1.31182287255172e-11 \\
151347	1.24620314068125e-11 \\
151487	1.15357723373677e-11 \\
151627	1.08124065256732e-11 \\
151767	1.01312847000656e-11 \\
151907	9.48968681413476e-12 \\
152047	8.89827100891694e-12 \\
152187	8.31840152315522e-12 \\
152327	7.78171971305142e-12 \\
152467	7.39397432170108e-12 \\
152607	6.91408041930686e-12 \\
152747	6.46188658137703e-12 \\
152887	5.93158855366482e-12 \\
153027	5.53429524430271e-12 \\
153167	5.1600945738528e-12 \\
153307	4.80765427468555e-12 \\
153447	4.47569759032262e-12 \\
153587	4.16305878658818e-12 \\
153727	3.86857212930636e-12 \\
153867	3.59123841775499e-12 \\
154007	3.32939231739715e-12 \\
154147	3.08203462751067e-12 \\
154287	2.84977597075908e-12 \\
154427	2.630118345337e-12 \\
154567	2.49206211222486e-12 \\
154707	2.23104867913548e-12 \\
154847	2.05091499339005e-12 \\
154987	1.87505566628943e-12 \\
155127	1.71257452663554e-12 \\
155267	1.55969681614465e-12 \\
155407	1.41592293445569e-12 \\
155547	1.28047572545142e-12 \\
155687	1.15280007761953e-12 \\
155827	1.03278496865755e-12 \\
155967	9.19653242448248e-13 \\
156107	8.60478355235728e-13 \\
156247	7.13151759867969e-13 \\
156387	6.187272916236e-13 \\
156527	5.29853938502356e-13 \\
156667	4.46143122445619e-13 \\
156807	4.06508160466501e-13 \\
156947	2.93209900803504e-13 \\
157087	2.2332136140335e-13 \\
157227	1.57596158345541e-13 \\
157367	9.56457135714572e-14 \\
157507	3.73590047786365e-14 \\
157647	-1.75970349403087e-14 \\
157787	-6.9333427887841e-14 \\
157927	-1.17961196366423e-13 \\
};
\end{axis}

\end{tikzpicture}
}
\scalebox{.9}{% This file was created by matplotlib2tikz v0.6.18.
\begin{tikzpicture}

\begin{axis}[
legend cell align={left},
legend entries={{20\% \adaalgo }},
legend style={at={(0.5,1.05)}, anchor=north},
tick align=outside,
tick pos=left,
xlabel={Number of Subspaces explored},
xmajorgrids,
xmin=0, xmax=400000,
ylabel={Suboptimality},
ymajorgrids,
ymin=1e-12, ymax=18.3325629648828,
ymode=log
]
\addlegendimage{no markers, blue}
\addplot [line width=0.01pt, blue, forget plot]
table [row sep=\\]{%
1190	2.86017367456298 \\
2380	2.3178498772078 \\
3570	1.87875922300314 \\
4760	1.52459855658915 \\
5950	1.23682197802005 \\
7136	1.00105570677978 \\
8316	0.817669053660461 \\
9496	0.666995409909048 \\
10676	0.547366989168348 \\
11856	0.44950206109311 \\
13036	0.368441442568725 \\
14216	0.302736933572566 \\
15396	0.24599045795425 \\
16576	0.196683496568551 \\
17756	0.156056190882283 \\
18936	0.125851443079824 \\
20116	0.10320514792655 \\
21296	0.0860731481373342 \\
22476	0.0726891326548061 \\
23656	0.0616596636193447 \\
24836	0.0521175725396099 \\
26016	0.0450324372629828 \\
27196	0.0399387259741098 \\
28376	0.0352854519813786 \\
29556	0.0310488999745205 \\
30736	0.0272270943039664 \\
31916	0.0239402650958494 \\
33096	0.0209170953678463 \\
34276	0.0182369517116656 \\
35456	0.0159086171408831 \\
36636	0.0137599861327999 \\
37816	0.0118100532234705 \\
38996	0.0103395351896191 \\
40176	0.00913328093829591 \\
41356	0.00839890592930376 \\
42536	0.00772939876393103 \\
43716	0.00711245102132979 \\
44896	0.00654188716496723 \\
46076	0.00602542529727296 \\
47256	0.00555125891280395 \\
48436	0.0051134696779096 \\
49616	0.00472041313291638 \\
50796	0.0043827543703267 \\
51976	0.00406978102726208 \\
53156	0.00378366262820234 \\
54336	0.00352627095417524 \\
55516	0.00328716202576107 \\
56696	0.00306485405818441 \\
57876	0.00285801450161477 \\
59056	0.00266551986454983 \\
60236	0.00248670798535444 \\
61416	0.00231986868786765 \\
62596	0.00216412815184835 \\
63776	0.00201868559113982 \\
64956	0.00188280567667026 \\
66136	0.00175581219223608 \\
67316	0.00163708245197003 \\
68496	0.00152604235447584 \\
69676	0.00142350083847398 \\
70856	0.00132783854974866 \\
72036	0.00123835550455303 \\
73216	0.00115461991609794 \\
74396	0.00107623527617123 \\
75576	0.00100283632122733 \\
76756	0.00093408593429245 \\
77936	0.000869672478316152 \\
79116	0.000809307466563602 \\
80296	0.000752723512373654 \\
81476	0.000699672513532201 \\
82656	0.00065002457190011 \\
83836	0.00060536641929676 \\
85016	0.000564037101380255 \\
86196	0.000526409701547659 \\
87376	0.000491464850650436 \\
88556	0.000458660140785383 \\
89736	0.00042786001029993 \\
90916	0.0003989383960607 \\
92096	0.000371777575168819 \\
93276	0.000346267539042266 \\
94456	0.00032230544520373 \\
95636	0.000299795115362322 \\
96816	0.000278646573649233 \\
97996	0.000258775620933294 \\
99176	0.00024010344175468 \\
100356	0.000222556240856764 \\
101536	0.00020606490666053 \\
102716	0.000190564699334872 \\
103896	0.000175994961375492 \\
105076	0.000162298848827847 \\
106256	0.000149423081484967 \\
107436	0.000137640086838364 \\
108616	0.000127541963893363 \\
109796	0.000118169121281819 \\
110976	0.00010946572141407 \\
112156	0.000101381065343409 \\
113336	9.38687383644221e-05 \\
114516	8.68861669270915e-05 \\
115696	8.04532745292685e-05 \\
116876	7.45129247506915e-05 \\
118056	6.89912903438783e-05 \\
119236	6.38577311314337e-05 \\
120416	5.90840385088498e-05 \\
121517	5.46442020142313e-05 \\
121907	5.05142128605085e-05 \\
122297	4.66718896928731e-05 \\
122687	4.30967229096413e-05 \\
123077	3.97697350084103e-05 \\
123467	3.66733548667342e-05 \\
123857	3.37913042033899e-05 \\
124247	3.11084947423046e-05 \\
124637	2.86406438361753e-05 \\
125027	2.64522522652189e-05 \\
125417	2.44318969835899e-05 \\
125807	2.25663666277853e-05 \\
126197	2.08436438888238e-05 \\
126587	1.92526656454528e-05 \\
126977	1.77832378685516e-05 \\
127367	1.64259656674215e-05 \\
127757	1.51721895622625e-05 \\
128147	1.4013926989298e-05 \\
128537	1.29438185659803e-05 \\
128927	1.19550787168188e-05 \\
129317	1.10414502907386e-05 \\
129707	1.01971628396202e-05 \\
130097	9.41689425271175e-06 \\
130487	8.69573547002123e-06 \\
130877	8.02915802100568e-06 \\
131267	7.41298415535274e-06 \\
131657	6.84335935646763e-06 \\
132047	6.31672704054509e-06 \\
132437	5.82980526558918e-06 \\
132827	5.3795652862898e-06 \\
133217	4.9632118063192e-06 \\
133607	4.578164791047e-06 \\
133997	4.22204271399718e-06 \\
134387	3.89264712530313e-06 \\
134777	3.58794843130505e-06 \\
135167	3.30607279369621e-06 \\
135557	3.04529005490384e-06 \\
135947	2.80400261126745e-06 \\
136337	2.58073515546631e-06 \\
136727	2.37412522247071e-06 \\
137117	2.18291447279251e-06 \\
137507	2.00594065685733e-06 \\
137897	1.84213020448798e-06 \\
138287	1.69049139253641e-06 \\
138677	1.55010804331335e-06 \\
139067	1.42013371312588e-06 \\
139457	1.29978633239825e-06 \\
139847	1.18834326073847e-06 \\
140237	1.0851367261977e-06 \\
140627	9.89549617580465e-07 \\
141017	9.0101160205025e-07 \\
141407	8.18995543605539e-07 \\
141797	7.43014197834846e-07 \\
142187	6.72617161912026e-07 \\
142577	6.07388060125391e-07 \\
142967	5.46941946455437e-07 \\
143357	4.90922906881686e-07 \\
143747	4.39001846541665e-07 \\
144137	3.90874447253609e-07 \\
144527	3.46259281969186e-07 \\
144917	3.04896073499705e-07 \\
145307	2.66544088078913e-07 \\
145697	2.30980650994805e-07 \\
146087	1.97999776518909e-07 \\
146477	1.71873611398699e-07 \\
146867	1.58362868540873e-07 \\
147257	1.46089922492543e-07 \\
147647	1.34908224924679e-07 \\
148037	1.24704103321438e-07 \\
148427	1.15378424037083e-07 \\
148817	1.06843735026008e-07 \\
149207	9.90226377561854e-08 \\
149597	9.18464596044721e-08 \\
149987	8.52541285345509e-08 \\
150377	7.91912137532158e-08 \\
150767	7.36091044784004e-08 \\
151157	6.8464302560578e-08 \\
151547	6.37178131923655e-08 \\
151937	5.93346174415643e-08 \\
152327	5.5283212785362e-08 \\
152717	5.15352146512882e-08 \\
153107	4.80650079182077e-08 \\
153497	4.48494408833433e-08 \\
153887	4.18675584201722e-08 \\
154277	3.91003667221668e-08 \\
154667	3.65306249694441e-08 \\
155057	3.41426635852571e-08 \\
155447	3.19222218658766e-08 \\
155837	2.98563048173328e-08 \\
156227	2.79330558683455e-08 \\
156617	2.61416449598428e-08 \\
157007	2.44721669040437e-08 \\
157397	2.29155527331493e-08 \\
157787	2.14634897077737e-08 \\
158177	2.0108349929604e-08 \\
158567	1.88431264480649e-08 \\
158957	1.76613765279221e-08 \\
159347	1.65571698573785e-08 \\
159737	1.55250429734188e-08 \\
160127	1.45599577394684e-08 \\
160517	1.3657263764344e-08 \\
160907	1.28126654286298e-08 \\
161297	1.20221908539442e-08 \\
161687	1.12821649245198e-08 \\
162077	1.05891848067863e-08 \\
162467	9.94009646815286e-09 \\
162857	9.33197558117271e-09 \\
163247	8.76210753952833e-09 \\
163637	8.22797235899841e-09 \\
164027	7.72722791309022e-09 \\
164417	7.25769677689669e-09 \\
164807	6.81735329299826e-09 \\
165197	6.40431180309875e-09 \\
165587	6.01681621192895e-09 \\
165977	5.65323043932864e-09 \\
166367	5.31202898335081e-09 \\
166757	4.99178920421173e-09 \\
167147	4.69118366375199e-09 \\
167537	4.40897346409841e-09 \\
167927	4.14400114223668e-09 \\
168317	3.89518617360807e-09 \\
168707	3.66151803321557e-09 \\
169097	3.44205269842135e-09 \\
169487	3.23590687578701e-09 \\
169877	3.04225472591568e-09 \\
170267	2.86032320051532e-09 \\
170657	2.68938876724079e-09 \\
171047	2.5287742455582e-09 \\
171437	2.37784542056474e-09 \\
171827	2.23600798987533e-09 \\
172217	2.10270567624349e-09 \\
172607	1.97741673035878e-09 \\
172997	1.85965226551232e-09 \\
173387	1.74895398163954e-09 \\
173777	1.64489238896337e-09 \\
174167	1.54706414345895e-09 \\
174557	1.45509149174217e-09 \\
174947	1.36861977306779e-09 \\
175337	1.2873164201288e-09 \\
175727	1.21086901616607e-09 \\
176117	1.13898485087915e-09 \\
176507	1.07138869998025e-09 \\
176897	1.00782238110497e-09 \\
177287	9.48043588078207e-10 \\
177677	8.91824725179902e-10 \\
178067	8.38952185500119e-10 \\
178457	7.89225018671402e-10 \\
178847	7.42454986379926e-10 \\
179237	6.98464786008657e-10 \\
179627	6.57088106148507e-10 \\
180017	6.18168738419911e-10 \\
180407	5.8155974480556e-10 \\
180797	5.47123235605795e-10 \\
181187	5.14729592282492e-10 \\
181577	4.84256801325245e-10 \\
181967	4.55590620784818e-10 \\
182357	4.28623025960917e-10 \\
182747	4.03253264114056e-10 \\
183137	3.79386133619874e-10 \\
183527	3.56932039480284e-10 \\
183917	3.35807159856927e-10 \\
184307	3.159325023816e-10 \\
184697	2.97233682111653e-10 \\
185087	2.7964097704114e-10 \\
185477	2.63088606455852e-10 \\
185867	2.47514952977923e-10 \\
186257	2.32861840920862e-10 \\
186647	2.1907481384531e-10 \\
187037	2.06102357402926e-10 \\
187427	1.93896287914441e-10 \\
187817	1.82411252769299e-10 \\
188207	1.716044528699e-10 \\
188597	1.61435864676207e-10 \\
188987	1.51867463049626e-10 \\
189377	1.42864053920277e-10 \\
189767	1.34391886508212e-10 \\
190157	1.26419597012983e-10 \\
190547	1.18917931057894e-10 \\
190937	1.11858688978117e-10 \\
191327	1.05215947066029e-10 \\
191717	9.89650583704815e-11 \\
192107	9.30825971856564e-11 \\
192497	8.7547191718329e-11 \\
192887	8.2338080797939e-11 \\
193277	7.74361130773116e-11 \\
193667	7.28231919211453e-11 \\
194057	6.84821643837097e-11 \\
194447	6.43968767199965e-11 \\
194837	6.05523409191733e-11 \\
195227	5.69345126599785e-11 \\
195617	5.35296251769068e-11 \\
196007	5.03255215278386e-11 \\
196397	4.7310211304108e-11 \\
196787	4.44723702308636e-11 \\
197177	4.18017287451278e-11 \\
197567	3.92885168842838e-11 \\
197957	3.69231312191687e-11 \\
198347	3.46971895659465e-11 \\
198737	3.260225422963e-11 \\
199127	3.06307756936519e-11 \\
199517	2.87753154637471e-11 \\
199907	2.70290456683142e-11 \\
200297	2.53856935472641e-11 \\
200687	2.38390418516587e-11 \\
201077	2.23833729329215e-11 \\
201467	2.1013357720534e-11 \\
201857	1.97241667443393e-11 \\
202247	1.85108040007265e-11 \\
202637	1.73687175752946e-11 \\
203027	1.6293910665155e-11 \\
203417	1.52822754451165e-11 \\
203807	1.43303147126517e-11 \\
204197	1.34343647317792e-11 \\
204587	1.25910948334251e-11 \\
204977	1.17974519042718e-11 \\
205367	1.1050493853304e-11 \\
205757	1.03474451229602e-11 \\
206147	9.68586322258602e-12 \\
206537	9.06313912807377e-12 \\
206927	8.47699688222292e-12 \\
207317	7.92549359474037e-12 \\
207707	7.40635330842565e-12 \\
208097	6.91768864413689e-12 \\
208487	6.45783426733715e-12 \\
208877	6.02501382118703e-12 \\
209267	5.61761748230083e-12 \\
209657	5.23431298304899e-12 \\
210047	4.87337947774336e-12 \\
210437	4.53376225451052e-12 \\
210827	4.21407353456971e-12 \\
211217	3.91325860604752e-12 \\
211607	3.63004071246564e-12 \\
211997	3.36358718655561e-12 \\
212387	3.11278780529278e-12 \\
212777	2.87669887910624e-12 \\
213167	2.65448774072752e-12 \\
213557	2.44532172288814e-12 \\
213947	2.2484791806221e-12 \\
214337	2.06323846896339e-12 \\
214727	1.88887794294601e-12 \\
215117	1.72467595760395e-12 \\
215507	1.57035495718105e-12 \\
215897	1.42491574095516e-12 \\
216287	1.28802524201888e-12 \\
216677	1.15923937116236e-12 \\
217067	1.03811403917575e-12 \\
217457	9.23927601093055e-13 \\
217847	8.16624545763034e-13 \\
218237	7.15538739370913e-13 \\
218627	6.20337115009306e-13 \\
219017	5.30853139224519e-13 \\
219407	4.46587211655469e-13 \\
219797	3.67261776546002e-13 \\
220187	2.92654789291191e-13 \\
220577	2.22211138378725e-13 \\
220967	1.5620837956476e-13 \\
221357	9.3869356732057e-14 \\
221747	3.53606033343112e-14 \\
222137	-1.9817480989559e-14 \\
222527	-7.17204073907851e-14 \\
222917	-1.20570220474292e-13 \\
223307	-1.66477942542542e-13 \\
223697	-2.09887662805386e-13 \\
224087	-2.50521825506667e-13 \\
224477	-2.88880031007466e-13 \\
224867	-3.24906768156552e-13 \\
225257	-3.58935103861313e-13 \\
225647	-3.90965038121749e-13 \\
226037	-4.20996570937859e-13 \\
226427	-4.49307258065801e-13 \\
226817	-4.76008121808036e-13 \\
227207	-5.01043651013333e-13 \\
227597	-5.24635890286618e-13 \\
227987	-5.46951373081583e-13 \\
228377	-5.67879077095768e-13 \\
228767	-5.87474513480402e-13 \\
229157	-6.06015237991642e-13 \\
229547	-6.23501250629488e-13 \\
229937	-6.3993255139394e-13 \\
230327	-6.55420162587461e-13 \\
230717	-6.69964084210051e-13 \\
231107	-6.8361982741294e-13 \\
231497	-6.96498414498592e-13 \\
231887	-7.08544334315775e-13 \\
232277	-7.20035142620645e-13 \\
232667	-7.30804305959509e-13 \\
233057	-7.40907335483598e-13 \\
233447	-7.50455253495375e-13 \\
233837	-7.59392548843607e-13 \\
234227	-7.67830243830758e-13 \\
234617	-7.75768338456828e-13 \\
235007	-7.8337336617551e-13 \\
235397	-7.90256748928186e-13 \\
235787	-7.96918087075937e-13 \\
236177	-8.03024313711376e-13 \\
236567	-8.09019518044352e-13 \\
236957	-8.14459610865015e-13 \\
237347	-8.19622147929522e-13 \\
};
\addplot [line width=0.01pt, blue, forget plot]
table [row sep=\\]{%
1190	2.58465000992032 \\
2380	2.0986731141083 \\
3570	1.71892756514415 \\
4760	1.42169972654321 \\
5950	1.1822342501746 \\
7138	0.977004405322384 \\
8318	0.805273582774765 \\
9498	0.672521029889478 \\
10678	0.564201522203172 \\
11858	0.471523288047803 \\
13038	0.39644875511099 \\
14218	0.335871101597056 \\
15398	0.281936546443094 \\
16578	0.235240075671051 \\
17758	0.196349455321759 \\
18938	0.166821869133859 \\
20118	0.141980684934023 \\
21298	0.121197539092819 \\
22478	0.102311996876003 \\
23658	0.0860575517553028 \\
24838	0.0724084031158851 \\
26018	0.0612252074276944 \\
27198	0.0526105310851888 \\
28378	0.0461098769798602 \\
29558	0.0402259976698257 \\
30738	0.0349236629935289 \\
31918	0.0300284226797732 \\
33098	0.0261643242660087 \\
34278	0.022742607731855 \\
35458	0.0198156995905381 \\
36638	0.0172969436194487 \\
37818	0.0150100033560754 \\
38998	0.0129234717646813 \\
40178	0.0114088607455843 \\
41358	0.0100078352651009 \\
42538	0.00872098722805265 \\
43718	0.00761726654377382 \\
44898	0.00676402245697871 \\
46078	0.00610840201766349 \\
47258	0.0055629328152299 \\
48438	0.00505577222183906 \\
49618	0.00458373985373622 \\
50798	0.00414726690044376 \\
51978	0.00374832313352474 \\
53158	0.00338889108702284 \\
54338	0.00306100421714428 \\
55518	0.00275514700193141 \\
56698	0.00250634431203833 \\
57878	0.00228026027050837 \\
59058	0.00207184793484355 \\
60238	0.00187945851537535 \\
61418	0.00170163153962977 \\
62598	0.00153706866894909 \\
63778	0.00138461226597525 \\
64958	0.00124322719812947 \\
66138	0.0011119853262197 \\
67318	0.000991892366476954 \\
68498	0.000881492658306671 \\
69678	0.000779046326274957 \\
70858	0.000683900621143207 \\
72038	0.000595549018843733 \\
73218	0.000514088285625824 \\
74398	0.000474932020549823 \\
75578	0.00044204620976529 \\
76758	0.000411626509514873 \\
77938	0.000383455670386446 \\
79118	0.00035734832265738 \\
80298	0.00033313737751367 \\
81478	0.000310671339712076 \\
82658	0.000289812536820144 \\
83838	0.000270435621665233 \\
85018	0.000252426273405992 \\
86198	0.00023568006382263 \\
87378	0.000220101463346001 \\
88558	0.000205602966072915 \\
89738	0.000192104316676267 \\
90918	0.000179531825024615 \\
92098	0.000167817756664634 \\
93278	0.000156899789213238 \\
94458	0.000146720526249255 \\
95638	0.000137227061555389 \\
96818	0.000128370587609949 \\
97998	0.000120108198371494 \\
99178	0.000112397311872958 \\
100358	0.000105197699805382 \\
101538	9.84738368075e-05 \\
102718	9.21928313681875e-05 \\
103898	8.63242117551954e-05 \\
105078	8.08397320470333e-05 \\
106258	7.57131956605761e-05 \\
107438	7.09202945131104e-05 \\
108618	6.64384622178216e-05 \\
109798	6.22467399074011e-05 \\
110978	5.83256534496535e-05 \\
112158	5.46571009534258e-05 \\
113338	5.12242495971371e-05 \\
114518	4.80114409099341e-05 \\
115698	4.50041037317606e-05 \\
116878	4.2188674161947e-05 \\
118058	3.95525218702675e-05 \\
119238	3.7083882217237e-05 \\
120418	3.47717936763914e-05 \\
121598	3.26060401037997e-05 \\
122058	3.05770974490804e-05 \\
122438	2.86760845289447e-05 \\
122818	2.68947175323664e-05 \\
123198	2.52252679467935e-05 \\
123578	2.36605236303977e-05 \\
123958	2.21937527734029e-05 \\
124338	2.08186705207281e-05 \\
124718	1.95294080427288e-05 \\
125098	1.83204838630768e-05 \\
125478	1.71867772667533e-05 \\
125858	1.61235036273966e-05 \\
126238	1.51261915063983e-05 \\
126618	1.41906613888021e-05 \\
126998	1.33130059308817e-05 \\
127378	1.24895716064333e-05 \\
127758	1.17169416460894e-05 \\
128138	1.0991920174841e-05 \\
128518	1.03115174568402e-05 \\
128898	9.67293616771503e-06 \\
129278	9.07355861845538e-06 \\
129658	8.51093486192722e-06 \\
130038	7.98277161684346e-06 \\
130418	7.49267116212371e-06 \\
130798	7.04506887672096e-06 \\
131178	6.62472527551694e-06 \\
131558	6.22989913723337e-06 \\
131938	5.85898629884563e-06 \\
132318	5.51049168395767e-06 \\
132698	5.18301985913405e-06 \\
133078	4.87526747322553e-06 \\
133458	4.58601645564327e-06 \\
133838	4.31412782797569e-06 \\
134218	4.05853605262019e-06 \\
134598	3.81824385925489e-06 \\
134978	3.5923174966368e-06 \\
135358	3.37988236714937e-06 \\
135738	3.18011900324322e-06 \\
136118	2.99225935607161e-06 \\
136498	2.81558336423515e-06 \\
136878	2.6494157783219e-06 \\
137258	2.49312321870532e-06 \\
137638	2.34611144528385e-06 \\
138018	2.20782282361887e-06 \\
138398	2.07773396787569e-06 \\
138778	1.95535354896581e-06 \\
139158	1.84022025462305e-06 \\
139538	1.73190088670339e-06 \\
139918	1.62998858938002e-06 \\
140298	1.5341011951886e-06 \\
140678	1.44387968092907e-06 \\
141058	1.35898672642965e-06 \\
141438	1.27910536668052e-06 \\
141818	1.20393773322958e-06 \\
142198	1.1332038756251e-06 \\
142578	1.0666406602966e-06 \\
142958	1.00400073804741e-06 \\
143338	9.45051579714917e-07 \\
143718	8.89574570228646e-07 \\
144098	8.37364163064347e-07 \\
144478	7.882270846582e-07 \\
144858	7.41981590612895e-07 \\
145238	6.98456768144506e-07 \\
145618	6.57491881328465e-07 \\
145998	6.18935756813155e-07 \\
146378	5.82646208779902e-07 \\
146758	5.4848949732067e-07 \\
147138	5.16339821843292e-07 \\
147518	4.86078844452731e-07 \\
147898	4.57595243086306e-07 \\
148278	4.3078429157184e-07 \\
148658	4.0554746505439e-07 \\
149038	3.81792070569542e-07 \\
149418	3.59430897711732e-07 \\
149798	3.38381892894812e-07 \\
150178	3.18567852042317e-07 \\
150558	2.999161315409e-07 \\
150938	2.82358377678982e-07 \\
151318	2.6583027112892e-07 \\
151698	2.50271288082526e-07 \\
152078	2.35624474376195e-07 \\
152458	2.21836234159944e-07 \\
152838	2.088561306679e-07 \\
153218	1.96636699090202e-07 \\
153598	1.85133270991233e-07 \\
153978	1.74303808109233e-07 \\
154358	1.64108747646718e-07 \\
154738	1.54510855054912e-07 \\
155118	1.45475087143243e-07 \\
155498	1.3696846212774e-07 \\
155878	1.28959938283657e-07 \\
156258	1.214202990929e-07 \\
156638	1.14322046107507e-07 \\
157018	1.07639296365036e-07 \\
157398	1.01347688963305e-07 \\
157778	9.54242938555616e-08 \\
158158	8.98475279176303e-08 \\
158538	8.45970759000281e-08 \\
158918	7.96538154879123e-08 \\
159298	7.49997470239627e-08 \\
159678	7.06179277831787e-08 \\
160058	6.64924090787444e-08 \\
160438	6.26081784194099e-08 \\
160818	5.89511044979396e-08 \\
161198	5.55078845110302e-08 \\
161578	5.22659962531868e-08 \\
161958	4.92136519869568e-08 \\
162338	4.63397551997424e-08 \\
162718	4.36338602471942e-08 \\
163098	4.10861333843826e-08 \\
163478	3.86873180713287e-08 \\
163858	3.64286997234231e-08 \\
164238	3.43020746806921e-08 \\
164618	3.22997201762654e-08 \\
164998	3.04143656371103e-08 \\
165378	2.8639166871347e-08 \\
165758	2.69676803110741e-08 \\
166138	2.53938398087072e-08 \\
166518	2.39119347100747e-08 \\
166898	2.25165887046685e-08 \\
167278	2.12027400081638e-08 \\
167658	1.99656237098722e-08 \\
168038	1.88007535095736e-08 \\
168418	1.77039058413264e-08 \\
168798	1.66711043858569e-08 \\
169178	1.56986054711261e-08 \\
169558	1.47828846941422e-08 \\
169938	1.39206234872624e-08 \\
170318	1.31086979604511e-08 \\
170698	1.23441666333157e-08 \\
171078	1.16242603320771e-08 \\
171458	1.0946371753473e-08 \\
171838	1.03080460278626e-08 \\
172218	9.70697211499782e-09 \\
172598	9.14097403326153e-09 \\
172978	8.60800269952833e-09 \\
173358	8.10612937884869e-09 \\
173738	7.63353774635434e-09 \\
174118	7.18851794756503e-09 \\
174498	6.76945971500587e-09 \\
174878	6.37484726118132e-09 \\
175258	6.00325328337092e-09 \\
175638	5.65333369006993e-09 \\
176018	5.32382343765292e-09 \\
176398	5.01353086823642e-09 \\
176778	4.72133426798749e-09 \\
177158	4.44617731520935e-09 \\
177538	4.18706547211656e-09 \\
177918	3.94306248763243e-09 \\
178298	3.71328645609736e-09 \\
178678	3.49690726375584e-09 \\
179058	3.29314303604278e-09 \\
179438	3.10125780611514e-09 \\
179818	2.92055851724982e-09 \\
180198	2.75039246933062e-09 \\
180578	2.59014526493573e-09 \\
180958	2.43923836684701e-09 \\
181338	2.29712654453706e-09 \\
181718	2.16329720803543e-09 \\
182098	2.03726696623718e-09 \\
182478	1.91858123832489e-09 \\
182858	1.80681142269989e-09 \\
183238	1.70155445289311e-09 \\
183618	1.6024304105855e-09 \\
183998	1.50908163742969e-09 \\
184378	1.42117151380461e-09 \\
184758	1.33838290450328e-09 \\
185138	1.26041715953207e-09 \\
185518	1.18699311491e-09 \\
185898	1.11784609346799e-09 \\
186278	1.05272673911472e-09 \\
186658	9.91400295191625e-10 \\
187038	9.33645660783355e-10 \\
187418	8.79254558050491e-10 \\
187798	8.28031143651486e-10 \\
188178	7.79790676475045e-10 \\
188558	7.34359351106662e-10 \\
188938	6.91573465161355e-10 \\
189318	6.51278919683307e-10 \\
189698	6.13330553012048e-10 \\
190078	5.77591752204398e-10 \\
190458	5.43933453833745e-10 \\
190838	5.12234810123857e-10 \\
191218	4.82381468103199e-10 \\
191598	4.54266013694138e-10 \\
191978	4.27787139045677e-10 \\
192358	4.02849531511151e-10 \\
192738	3.7936337404787e-10 \\
193118	3.57244345217111e-10 \\
193498	3.36412619983406e-10 \\
193878	3.16793258292591e-10 \\
194258	2.98315705471452e-10 \\
194638	2.80913348138512e-10 \\
195018	2.64523847270937e-10 \\
195398	2.49087805936909e-10 \\
195778	2.34550046052107e-10 \\
196158	2.20858331623219e-10 \\
196538	2.07963091192198e-10 \\
196918	1.95818250503521e-10 \\
197298	1.84379789214262e-10 \\
197678	1.73606906628265e-10 \\
198058	1.6346074493967e-10 \\
198438	1.53904722299814e-10 \\
198818	1.44904643839538e-10 \\
199198	1.36428035535374e-10 \\
199578	1.28444532787597e-10 \\
199958	1.20925436331021e-10 \\
200338	1.13843545701542e-10 \\
200718	1.07173547814199e-10 \\
201098	1.00891517362811e-10 \\
201478	9.49748613088275e-11 \\
201858	8.9402207859024e-11 \\
202238	8.41535174878061e-11 \\
202618	7.92102494706626e-11 \\
202998	7.45543071722921e-11 \\
203378	7.01690372473252e-11 \\
203758	6.60388965734171e-11 \\
204138	6.21488416285843e-11 \\
204518	5.84849391138675e-11 \\
204898	5.50339218641227e-11 \\
205278	5.17836884483813e-11 \\
205658	4.87223594802799e-11 \\
206038	4.58389992630259e-11 \\
206418	4.31232272113391e-11 \\
206798	4.05653843849052e-11 \\
207178	3.81562004214686e-11 \\
207558	3.58870155814373e-11 \\
207938	3.37496697255801e-11 \\
208318	3.17365578261786e-11 \\
208698	2.9840574455875e-11 \\
209078	2.80547807207654e-11 \\
209458	2.63726263050046e-11 \\
209838	2.47882825377133e-11 \\
210218	2.32960317703146e-11 \\
210598	2.18904894211391e-11 \\
210978	2.05666594865761e-11 \\
211358	1.93196569853171e-11 \\
211738	1.81452075587174e-11 \\
212118	1.70390368481321e-11 \\
212498	1.59970370283702e-11 \\
212878	1.50155998746015e-11 \\
213258	1.40911171619962e-11 \\
213638	1.32203692437827e-11 \\
214018	1.24004140289458e-11 \\
214398	1.16278098261091e-11 \\
214778	1.09001696557698e-11 \\
215158	1.02148289826687e-11 \\
215538	9.56934531615161e-12 \\
215918	8.96133167671564e-12 \\
216298	8.38862312946276e-12 \\
216678	7.84916576179739e-12 \\
217058	7.34107219457769e-12 \\
217438	6.86239953751056e-12 \\
217818	6.41159347836151e-12 \\
218198	5.98698868259362e-12 \\
218578	5.58697532682118e-12 \\
218958	5.21033216571709e-12 \\
219338	4.85544937589566e-12 \\
219718	4.52116122318102e-12 \\
220098	4.20635748454856e-12 \\
220478	3.90970589236872e-12 \\
220858	3.63042929052426e-12 \\
221238	3.3673064336881e-12 \\
221618	3.11944914344053e-12 \\
221998	2.88602475251309e-12 \\
222378	2.66603406018362e-12 \\
222758	2.45897746609103e-12 \\
223138	2.26396679181562e-12 \\
223518	2.08011385893769e-12 \\
223898	1.90703008939863e-12 \\
224278	1.74399383823243e-12 \\
224658	1.59039448277554e-12 \\
225038	1.44573242266688e-12 \\
225418	1.30945254639414e-12 \\
225798	1.18099974244501e-12 \\
226178	1.06009645506333e-12 \\
226558	9.46243083888021e-13 \\
226938	8.38940028558e-13 \\
227318	7.3790973331711e-13 \\
227698	6.42652597804272e-13 \\
228078	5.5300208856579e-13 \\
228458	4.68569627543047e-13 \\
228838	3.88966636677424e-13 \\
229218	3.13915560212763e-13 \\
229598	2.43360886997834e-13 \\
229978	1.76914038974019e-13 \\
230358	1.1429746038516e-13 \\
230738	5.52335954751015e-14 \\
231118	-2.77555756156289e-16 \\
231498	-5.26245713672324e-14 \\
231878	-1.01918473660589e-13 \\
232258	-1.48381307241152e-13 \\
232638	-1.92235116713846e-13 \\
233018	-2.33424390927439e-13 \\
233398	-2.72226685638088e-13 \\
233778	-3.08808534299487e-13 \\
234158	-3.43225448062867e-13 \\
234538	-3.75810493835615e-13 \\
};
\addplot [line width=0.01pt, blue, forget plot]
table [row sep=\\]{%
1190	2.90684846601852 \\
2380	2.36002442962834 \\
3570	1.92110832868437 \\
4760	1.54912365075835 \\
5950	1.24341214977799 \\
7140	0.986109995172377 \\
8322	0.777550995174252 \\
9492	0.619375151712152 \\
10662	0.500847875716659 \\
11832	0.411314935591174 \\
13002	0.337104305539529 \\
14172	0.272961917732563 \\
15342	0.226525424617561 \\
16512	0.189706357278162 \\
17682	0.158600802642308 \\
18852	0.133420118227457 \\
20022	0.112818108577116 \\
21192	0.0942381627576962 \\
22362	0.080874261371441 \\
23532	0.0692010798088727 \\
24702	0.059121318877898 \\
25872	0.0510422521308072 \\
27042	0.0439362144281956 \\
28212	0.0374511592450323 \\
29382	0.0315682319052099 \\
30552	0.0263206101661421 \\
31722	0.022097313719087 \\
32892	0.0181965684795256 \\
34062	0.0145904700422533 \\
35232	0.0119331148411401 \\
36402	0.00999760401141492 \\
37572	0.00863717289191479 \\
38742	0.00746319284512531 \\
39912	0.00640054541662138 \\
41082	0.00545702101579976 \\
42252	0.0047041292424978 \\
43422	0.0041271659194812 \\
44592	0.00361043730049665 \\
45762	0.00315390080901828 \\
46932	0.00280394493776781 \\
48102	0.00253111691705765 \\
49272	0.00229188754522741 \\
50442	0.00207366505574169 \\
51612	0.00188131897106453 \\
52782	0.00171334297607889 \\
53952	0.00157623321237682 \\
55122	0.00145031529033146 \\
56292	0.00133432503602637 \\
57462	0.00122737268403877 \\
58632	0.00113298571519643 \\
59802	0.00104586863210399 \\
60972	0.000965184189198387 \\
62142	0.000890405140731909 \\
63312	0.000821059040549799 \\
64482	0.000759214669436181 \\
65652	0.000708085127358871 \\
66822	0.000660741046687174 \\
67992	0.000616829153034792 \\
69162	0.000576050178369814 \\
70332	0.000538366120235501 \\
71502	0.000503440927173304 \\
72672	0.000470921364845223 \\
73842	0.000440675362069654 \\
75012	0.000412481771077999 \\
76182	0.000386178020022332 \\
77352	0.000361625953513856 \\
78522	0.000338699424566713 \\
79692	0.000317282876711678 \\
80862	0.000297270168816532 \\
82032	0.000278563586472436 \\
83202	0.000261073000736134 \\
84372	0.000244715144641805 \\
85542	0.000229412984894151 \\
86712	0.000215095171364144 \\
87882	0.000201695550912095 \\
89052	0.000189152735008247 \\
90222	0.000177409712838805 \\
91392	0.000166413503275231 \\
92562	0.000156114840375809 \\
93732	0.00014646788807976 \\
94902	0.00013742998052213 \\
96072	0.00012896138499735 \\
97242	0.000121025085074111 \\
98412	0.000113586581737546 \\
99582	0.000106613710735226 \\
100752	0.000100076474552513 \\
101922	9.39468876395289e-05 \\
103092	8.81988336793871e-05 \\
104262	8.28079338272003e-05 \\
105432	7.77514249661326e-05 \\
106602	7.30080471296746e-05 \\
107772	6.85579393223112e-05 \\
108942	6.43825430478029e-05 \\
110112	6.04645129231862e-05 \\
111282	5.67876338080087e-05 \\
112452	5.3336743936705e-05 \\
113622	5.00976635813832e-05 \\
114792	4.70571288194721e-05 \\
115962	4.42027300116554e-05 \\
117132	4.1522854632936e-05 \\
118302	3.90066341267592e-05 \\
119472	3.66438944807146e-05 \\
120642	3.44251102468163e-05 \\
121812	3.23413617506207e-05 \\
122982	3.03842952550926e-05 \\
124152	2.8572531242943e-05 \\
125322	2.68704276566445e-05 \\
126492	2.52707564219468e-05 \\
127662	2.37672287174973e-05 \\
128832	2.23539610081924e-05 \\
130002	2.10254430715318e-05 \\
131172	1.97765122995075e-05 \\
132342	1.86023302543026e-05 \\
133512	1.74983610425894e-05 \\
134682	1.64603513183037e-05 \\
135852	1.54843117587955e-05 \\
137022	1.45664998841277e-05 \\
138192	1.37034041025053e-05 \\
139362	1.28917288759189e-05 \\
140532	1.21283809174644e-05 \\
141702	1.14104563340711e-05 \\
142872	1.07352286442541e-05 \\
144042	1.01001376008325e-05 \\
145212	9.50277875993999e-06 \\
146382	8.9408937397617e-06 \\
147552	8.41236111953636e-06 \\
148722	7.91518793369361e-06 \\
149892	7.44750171549624e-06 \\
151062	7.00754305626994e-06 \\
152232	6.59365864080774e-06 \\
153402	6.20429472819595e-06 \\
154572	5.83799104630911e-06 \\
155742	5.49337507366188e-06 \\
156912	5.16915668175022e-06 \\
158082	4.86412311306861e-06 \\
159252	4.57713427476358e-06 \\
160422	4.30711832477559e-06 \\
161592	4.05306753420431e-06 \\
162762	3.81403440380401e-06 \\
163932	3.58912802411737e-06 \\
165102	3.3775106566547e-06 \\
166272	3.17839452862456e-06 \\
167442	2.99103882211993e-06 \\
168612	2.81474684965533e-06 \\
169782	2.64886340151094e-06 \\
170952	2.49277225489175e-06 \\
172122	2.3458938354648e-06 \\
173292	2.20768302111596e-06 \\
174462	2.07762707871151e-06 \\
175632	1.95524372587075e-06 \\
176802	1.84007931114394e-06 \\
177972	1.73170710249249e-06 \\
179142	1.62972568146236e-06 \\
180312	1.53375743161543e-06 \\
181482	1.44344712027511e-06 \\
182652	1.3584605644823e-06 \\
183822	1.27848337722059e-06 \\
184992	1.20321978885896e-06 \\
186162	1.13239154070355e-06 \\
187332	1.06573684216515e-06 \\
188502	1.00300939337439e-06 \\
189672	9.4397746391861e-07 \\
190842	8.88423028477803e-07 \\
192012	8.36140952420461e-07 \\
193182	7.86938227970158e-07 \\
194352	7.40633253781819e-07 \\
195522	6.97055159648574e-07 \\
196692	6.56043169677822e-07 \\
197862	6.17446004547162e-07 \\
199032	5.81121318954381e-07 \\
200202	5.4693517231863e-07 \\
201372	5.14761531511532e-07 \\
202542	4.84481802676129e-07 \\
203712	4.55984391023456e-07 \\
204882	4.29164286830375e-07 \\
206052	4.03922676139867e-07 \\
207222	3.80166574054375e-07 \\
208392	3.57808480566657e-07 \\
209562	3.36766055986093e-07 \\
210732	3.16961816571038e-07 \\
211902	2.98322846703503e-07 \\
213072	2.80780529993141e-07 \\
214242	2.64270295147195e-07 \\
215412	2.48731376550904e-07 \\
216582	2.34106590168981e-07 \\
217752	2.20342122270178e-07 \\
218922	2.0738732964265e-07 \\
220092	1.95194553298528e-07 \\
221262	1.83718942226019e-07 \\
222432	1.72918287411061e-07 \\
223602	1.62752866683658e-07 \\
224772	1.53185296669633e-07 \\
225942	1.44180397010363e-07 \\
227112	1.35705057968671e-07 \\
228282	1.27728120580262e-07 \\
229452	1.20220259913761e-07 \\
230622	1.13153878156247e-07 \\
231792	1.06503002195169e-07 \\
232962	1.00243188250193e-07 \\
234132	9.43514310014493e-08 \\
235302	8.88060798787116e-08 \\
236472	8.35867584592087e-08 \\
237642	7.86742898051251e-08 \\
238812	7.40506253538165e-08 \\
239982	6.96987791370951e-08 \\
241152	6.56027644430068e-08 \\
242322	6.17475356401442e-08 \\
243492	5.81189322779174e-08 \\
244662	5.47036269060719e-08 \\
245832	5.14890768354981e-08 \\
247002	4.84634765651748e-08 \\
248172	4.56157157047166e-08 \\
249342	4.29353373410102e-08 \\
250512	4.04124996800093e-08 \\
251682	3.80379399089747e-08 \\
252852	3.58029401126281e-08 \\
254022	3.36992951877058e-08 \\
255192	3.1719282256315e-08 \\
256362	2.98556334099587e-08 \\
257532	2.81015075653812e-08 \\
258702	2.64504662617071e-08 \\
259872	2.48964497351345e-08 \\
261042	2.3433754436919e-08 \\
262212	2.20570122166919e-08 \\
263382	2.07611707270239e-08 \\
264552	1.95414747716782e-08 \\
265722	1.83934483199977e-08 \\
266892	1.73128790748045e-08 \\
268062	1.62958023741666e-08 \\
269232	1.53384859813421e-08 \\
270402	1.44374180943707e-08 \\
271572	1.3589292469085e-08 \\
272742	1.2790997871992e-08 \\
273912	1.20396058678196e-08 \\
275082	1.13323597727977e-08 \\
276252	1.06666653842957e-08 \\
277422	1.00400809888157e-08 \\
278512	9.45030814714087e-09 \\
278882	8.89518370073006e-09 \\
279252	8.37267144504494e-09 \\
279622	7.88085535718963e-09 \\
279992	7.41793160230486e-09 \\
280362	6.98220231631907e-09 \\
280732	6.57206972176638e-09 \\
281102	6.18602979951532e-09 \\
281472	5.82266684867605e-09 \\
281842	5.48064865713016e-09 \\
282212	5.15872095041559e-09 \\
282582	4.85570333941254e-09 \\
282952	4.57048493496259e-09 \\
283322	4.30201974044309e-09 \\
283692	4.0493238762096e-09 \\
284062	3.81147052808117e-09 \\
284432	3.58758794893888e-09 \\
284802	3.3768552398783e-09 \\
285172	3.1784999077189e-09 \\
285542	2.99179486740186e-09 \\
285912	2.8160556664325e-09 \\
286282	2.65063798687848e-09 \\
286652	2.4949358135018e-09 \\
287022	2.34837821411205e-09 \\
287392	2.21042817383221e-09 \\
287762	2.08057976402998e-09 \\
288132	1.95835725413929e-09 \\
288502	1.84331289121431e-09 \\
288872	1.73502462397224e-09 \\
289242	1.63309604728212e-09 \\
289612	1.53715329354043e-09 \\
289982	1.4468448106264e-09 \\
290352	1.36183986310101e-09 \\
290722	1.28182670033894e-09 \\
291092	1.20651227897284e-09 \\
291462	1.13562059755878e-09 \\
291832	1.06889186390902e-09 \\
292202	1.00608155140236e-09 \\
292572	9.46959621828114e-10 \\
292942	8.91309359651871e-10 \\
293312	8.38926816904007e-10 \\
293682	7.89620258068169e-10 \\
294052	7.43208938835949e-10 \\
294422	6.99522828551125e-10 \\
294792	6.58401722031243e-10 \\
295162	6.1969529507877e-10 \\
295532	5.83261494657705e-10 \\
295902	5.48967038493942e-10 \\
296272	5.16686193829941e-10 \\
296642	4.86300666402428e-10 \\
297012	4.57699211864337e-10 \\
297382	4.30777080673295e-10 \\
297752	4.05435574002411e-10 \\
298122	3.81582043740281e-10 \\
298492	3.59128893290261e-10 \\
298862	3.37994021659682e-10 \\
299232	3.18100046303726e-10 \\
299602	2.99373970058525e-10 \\
299972	2.81747403185761e-10 \\
300342	2.65155675194251e-10 \\
300712	2.49537945862244e-10 \\
301082	2.34837149726275e-10 \\
301452	2.20999440969649e-10 \\
301822	2.07974082400142e-10 \\
302192	1.95713389938845e-10 \\
302562	1.84172455064413e-10 \\
302932	1.73309144813061e-10 \\
303302	1.63083435644751e-10 \\
303672	1.53458079577007e-10 \\
304042	1.44397716006495e-10 \\
304412	1.3586926028708e-10 \\
304782	1.27841459640621e-10 \\
305152	1.20284893156963e-10 \\
305522	1.13171860771644e-10 \\
305892	1.06476438777037e-10 \\
306262	1.00174202266601e-10 \\
306632	9.4241670023365e-11 \\
307002	8.86574702541054e-11 \\
307372	8.34009528105639e-11 \\
307742	7.84530773678682e-11 \\
308112	7.37956917795657e-11 \\
308482	6.94116430999259e-11 \\
308852	6.52849996285454e-11 \\
309222	6.14003847765332e-11 \\
309592	5.77439207560815e-11 \\
309962	5.43020628462898e-11 \\
310332	5.10622100158287e-11 \\
310702	4.80125939006371e-11 \\
311072	4.51418347147126e-11 \\
311442	4.24396073839262e-11 \\
311812	3.98960309233587e-11 \\
312182	3.75017794596033e-11 \\
312552	3.52480267196142e-11 \\
312922	3.31264460307068e-11 \\
313292	3.11294878763135e-11 \\
313662	2.92496582510182e-11 \\
314032	2.74802403055219e-11 \\
314402	2.58146837239792e-11 \\
314772	2.42467712574523e-11 \\
315142	2.27709517908181e-11 \\
315512	2.13817852312559e-11 \\
315882	2.00739980193987e-11 \\
316252	1.88430937519968e-11 \\
316622	1.76843539811955e-11 \\
316992	1.6593726392955e-11 \\
317362	1.55671031620841e-11 \\
317732	1.46005429968454e-11 \\
318102	1.369093727277e-11 \\
318472	1.28345112315742e-11 \\
318842	1.20284893156963e-11 \\
319212	1.1269651878365e-11 \\
319582	1.05554454066237e-11 \\
319952	9.88314985406191e-12 \\
320322	9.25026721887434e-12 \\
320692	8.65446603270925e-12 \\
321062	8.09369238297108e-12 \\
321432	7.56578133476182e-12 \\
321802	7.0690120423933e-12 \\
322172	6.60127508211872e-12 \\
322542	6.16090511940115e-12 \\
322912	5.74645886430858e-12 \\
323282	5.35643751575776e-12 \\
323652	4.98912022806053e-12 \\
324022	4.64334126704102e-12 \\
324392	4.31799040967462e-12 \\
324762	4.01162436602931e-12 \\
325132	3.72329944653416e-12 \\
325502	3.45184991701331e-12 \\
325872	3.19644311019829e-12 \\
326242	2.95591329191325e-12 \\
326612	2.72959432834341e-12 \\
326982	2.51643150761538e-12 \\
327352	2.31581420706561e-12 \\
327722	2.12702078172811e-12 \\
328092	1.94916305318316e-12 \\
328462	1.78185244337214e-12 \\
328832	1.62436730732907e-12 \\
329202	1.47615253354161e-12 \\
329572	1.33643096589253e-12 \\
329942	1.20520260438184e-12 \\
330312	1.08146824828736e-12 \\
330682	9.65061364155417e-13 \\
331052	8.55537862776146e-13 \\
331422	7.52342632637237e-13 \\
331792	6.55198117982536e-13 \\
332162	5.63826763055886e-13 \\
332532	4.77784478647436e-13 \\
332902	3.96793709001031e-13 \\
333272	3.20521387209283e-13 \\
333642	2.48856490969729e-13 \\
334012	1.81243908770057e-13 \\
334382	1.17628129459035e-13 \\
334752	5.77871084317394e-14 \\
335122	1.4432899320127e-15 \\
335492	-5.16253706450698e-14 \\
335862	-1.01529895601971e-13 \\
336232	-1.48603351846077e-13 \\
336602	-1.92790228226158e-13 \\
336972	-2.34423591649602e-13 \\
337342	-2.73725486721332e-13 \\
337712	-3.10584891138888e-13 \\
338082	-3.45334871809655e-13 \\
338452	-3.78030939884866e-13 \\
};
\addplot [line width=0.01pt, blue, forget plot]
table [row sep=\\]{%
1190	2.97182255057902 \\
2380	2.49552960757698 \\
3570	2.08710263529157 \\
4760	1.73711946882184 \\
5950	1.44177125850708 \\
7140	1.19265559476237 \\
8323	0.986214811671495 \\
9503	0.811022145026387 \\
10683	0.659346538317636 \\
11863	0.536368649099615 \\
13043	0.440888013697335 \\
14223	0.362785964403803 \\
15403	0.299777830261389 \\
16583	0.246246410040694 \\
17763	0.202571032028758 \\
18943	0.166225385445486 \\
20123	0.136766766569343 \\
21303	0.113205814516875 \\
22483	0.0967606772152382 \\
23663	0.0826882084209871 \\
24843	0.069849387108706 \\
26023	0.0582935439239045 \\
27203	0.0503382910590552 \\
28383	0.0435569076789707 \\
29563	0.0377452261891676 \\
30743	0.0335976286363168 \\
31923	0.0299981576987723 \\
33103	0.0267956922345664 \\
34283	0.0241675893585939 \\
35463	0.0217801549570842 \\
36643	0.0196040025591744 \\
37823	0.0178449088130673 \\
39003	0.0163062173098458 \\
40183	0.0150268053210553 \\
41363	0.0138575224372722 \\
42543	0.012769282456503 \\
43723	0.0117555767872108 \\
44903	0.0108106054067679 \\
46083	0.00994299913327512 \\
47263	0.00915026322048657 \\
48443	0.00841076547475433 \\
49623	0.00772054141700002 \\
50803	0.0070759818494715 \\
51983	0.00649180455525544 \\
53163	0.00597692284315887 \\
54343	0.00549829336259511 \\
55523	0.00505253468342531 \\
56703	0.00463681361247226 \\
57883	0.00424868097752673 \\
59063	0.00388598783704847 \\
60243	0.00354685210001371 \\
61423	0.00322973692074974 \\
62603	0.00293348287038026 \\
63783	0.00265682956081442 \\
64963	0.00239766030824612 \\
66143	0.00215479761291304 \\
67323	0.00192715755626466 \\
68503	0.00172427562836691 \\
69683	0.00153723022986968 \\
70863	0.0013641933564561 \\
72043	0.00120201038488715 \\
73223	0.0010510884251474 \\
74403	0.000916945103328903 \\
75583	0.000791476802259605 \\
76763	0.00067405287650224 \\
77943	0.000564104316184477 \\
79123	0.000461112505451333 \\
80303	0.000397687464625207 \\
81483	0.000366439569009114 \\
82663	0.000337569732976606 \\
83843	0.000310839164725518 \\
85023	0.000286068571724196 \\
86203	0.000263097952064861 \\
87383	0.000241782453927419 \\
88563	0.00022199057058353 \\
89743	0.000203744240269699 \\
90923	0.000186858488211206 \\
92103	0.000171143107988103 \\
93283	0.000156509105803004 \\
94463	0.000143843536876964 \\
95643	0.000132550475493209 \\
96823	0.000122161030225332 \\
98003	0.000112596512565177 \\
99183	0.000103786013612772 \\
100363	9.56654340968366e-05 \\
101543	8.81767302682301e-05 \\
102723	8.12672609069032e-05 \\
103903	7.48892158715653e-05 \\
105083	6.89991142742152e-05 \\
106263	6.35573625370478e-05 \\
107443	5.85952565196335e-05 \\
108623	5.40388969945682e-05 \\
109803	4.98444969999623e-05 \\
110983	4.59825148427795e-05 \\
112163	4.24259037809516e-05 \\
113343	3.91498818790303e-05 \\
114523	3.61317353918533e-05 \\
115703	3.33506406093864e-05 \\
116883	3.07875016845593e-05 \\
118063	2.8424802894389e-05 \\
119243	2.62464739837487e-05 \\
120423	2.42377673799754e-05 \\
121603	2.2385146186743e-05 \\
122463	2.06761819716017e-05 \\
122843	1.90994614575057e-05 \\
123223	1.76445013124193e-05 \\
123603	1.63016703113028e-05 \\
123983	1.50621182062327e-05 \\
124363	1.39177107104649e-05 \\
124743	1.28609700510984e-05 \\
125123	1.18850205990673e-05 \\
125503	1.09835391292057e-05 \\
125883	1.01507093018816e-05 \\
126263	9.38117999993437e-06 \\
126643	8.67002718202281e-06 \\
127023	8.0127189484025e-06 \\
127403	7.40508353930247e-06 \\
127783	6.84328001437917e-06 \\
128163	6.32377138065632e-06 \\
128543	5.84329995984012e-06 \\
128923	5.39886480216412e-06 \\
129303	4.98770097318024e-06 \\
129683	4.6072605539571e-06 \\
130063	4.25519520758089e-06 \\
130443	3.92934018117463e-06 \\
130823	3.62769962186649e-06 \\
131203	3.34843309462984e-06 \\
131583	3.08984320340766e-06 \\
131963	2.8503642200417e-06 \\
132343	2.62855164023801e-06 \\
132723	2.42307258552232e-06 \\
133103	2.23269698440554e-06 \\
133483	2.05628946520209e-06 \\
133863	1.89280190343588e-06 \\
134243	1.74126656887763e-06 \\
134623	1.60078982286427e-06 \\
135003	1.47054632121391e-06 \\
135383	1.34977368038136e-06 \\
135763	1.2377675698283e-06 \\
136143	1.13387719530289e-06 \\
136523	1.03750114083256e-06 \\
136903	9.48083541230105e-07 \\
137283	8.65110557801696e-07 \\
137663	7.88107131777238e-07 \\
138043	7.1663399531241e-07 \\
138423	6.50284916026145e-07 \\
138803	5.88684160418573e-07 \\
139183	5.31484153798445e-07 \\
139563	4.84348756435438e-07 \\
139943	4.51489941100114e-07 \\
140323	4.2112306902764e-07 \\
140703	3.93013643851248e-07 \\
141083	3.66966528331236e-07 \\
141463	3.4280760458083e-07 \\
141843	3.20380328155956e-07 \\
142223	2.99543546689041e-07 \\
142603	2.80169679622855e-07 \\
142983	2.62143158069605e-07 \\
143363	2.45359077111207e-07 \\
143743	2.29722027289547e-07 \\
144123	2.1514507725362e-07 \\
144503	2.0154888585866e-07 \\
144883	1.8886092228998e-07 \\
145263	1.77014780999851e-07 \\
145643	1.65949577357605e-07 \\
146023	1.55609412799684e-07 \\
146403	1.45942900042773e-07 \\
146783	1.36902741698641e-07 \\
147163	1.28445353908546e-07 \\
147543	1.20530530944851e-07 \\
147923	1.13121144285078e-07 \\
148303	1.06182873049754e-07 \\
148683	9.96839614741951e-08 \\
149063	9.35950015823472e-08 \\
149443	8.78887352895319e-08 \\
149823	8.25398775994302e-08 \\
150203	7.75249553996993e-08 \\
150583	7.28221615786673e-08 \\
150963	6.84112231308376e-08 \\
151343	6.42732804201351e-08 \\
151723	6.03907780449831e-08 \\
152103	5.67473646406746e-08 \\
152483	5.33278021186412e-08 \\
152863	5.01178823442139e-08 \\
153243	4.71043507532798e-08 \\
153623	4.42748359641421e-08 \\
154003	4.1617787049919e-08 \\
154383	3.9122413253434e-08 \\
154763	3.6778630640999e-08 \\
155143	3.45770119203337e-08 \\
155523	3.25087408659108e-08 \\
155903	3.0565569464347e-08 \\
156283	2.87397792786415e-08 \\
156663	2.70241448663278e-08 \\
157043	2.54119008058495e-08 \\
157423	2.38967104992938e-08 \\
157803	2.24726371955697e-08 \\
158183	2.11341179001678e-08 \\
158563	1.98759386171865e-08 \\
158943	1.86932110901594e-08 \\
159323	1.75813518188406e-08 \\
159703	1.65360621307009e-08 \\
160083	1.55533098067373e-08 \\
160463	1.46293121505714e-08 \\
160843	1.37605192795931e-08 \\
161223	1.29436001916616e-08 \\
161603	1.2175428276695e-08 \\
161983	1.14530685491054e-08 \\
162363	1.07737654353457e-08 \\
162743	1.01349317271904e-08 \\
163123	9.53413786808355e-09 \\
163503	8.96910196113154e-09 \\
163883	8.43768094282993e-09 \\
164263	7.9378617012793e-09 \\
164643	7.46775313809067e-09 \\
165023	7.02557839682427e-09 \\
165403	6.60966803511798e-09 \\
165783	6.21845358539375e-09 \\
166163	5.85046083800833e-09 \\
166543	5.50430451218276e-09 \\
166923	5.17868270488719e-09 \\
167303	4.87237178381505e-09 \\
167683	4.58422139137937e-09 \\
168063	4.31315050342107e-09 \\
168443	4.0581425997388e-09 \\
168823	3.81824222239757e-09 \\
169203	3.59255114545931e-09 \\
169583	3.38022482226918e-09 \\
169963	3.18046905478653e-09 \\
170343	2.99253732904958e-09 \\
170723	2.81572765103988e-09 \\
171103	2.64937977112467e-09 \\
171483	2.4928730746332e-09 \\
171863	2.34562347323219e-09 \\
172243	2.20708235021405e-09 \\
172623	2.07673306329426e-09 \\
173003	1.9540903894999e-09 \\
173383	1.83869741654519e-09 \\
173763	1.73012476567536e-09 \\
174143	1.62796848224289e-09 \\
174523	1.53184842588416e-09 \\
174903	1.441407382341e-09 \\
175283	1.35630906505924e-09 \\
175663	1.27623750456607e-09 \\
176043	1.2008950500686e-09 \\
176423	1.13000186985346e-09 \\
176803	1.06329506310843e-09 \\
177183	1.00052677254325e-09 \\
177563	9.41464239900824e-10 \\
177943	8.8588830715608e-10 \\
178323	8.33592750382195e-10 \\
178703	7.84383835661373e-10 \\
179083	7.38079264372971e-10 \\
179463	6.94507118481624e-10 \\
179843	6.53506138093007e-10 \\
180223	6.149243891862e-10 \\
180603	5.78618930546781e-10 \\
180983	5.44455258655319e-10 \\
181363	5.12307030131609e-10 \\
181743	4.8205511804511e-10 \\
182123	4.53587611914941e-10 \\
182503	4.26799151576063e-10 \\
182883	4.01590649623529e-10 \\
183263	3.77868736300968e-10 \\
183643	3.55545759500586e-10 \\
184023	3.34539007607049e-10 \\
184403	3.14770931542085e-10 \\
184783	2.96168423119525e-10 \\
185163	2.78662759534143e-10 \\
185543	2.62189103761301e-10 \\
185923	2.46686726601553e-10 \\
186303	2.32098118502222e-10 \\
186683	2.18369489157766e-10 \\
187063	2.05450212398262e-10 \\
187443	1.93292271077894e-10 \\
187823	1.81851089742224e-10 \\
188203	1.71084257871712e-10 \\
188583	1.60951973970924e-10 \\
188963	1.51416712501629e-10 \\
189343	1.42443501438549e-10 \\
189723	1.33999034090948e-10 \\
190103	1.26052113191832e-10 \\
190483	1.18573706409109e-10 \\
190863	1.11535780611405e-10 \\
191243	1.04912634135701e-10 \\
191623	9.86797865643041e-11 \\
192003	9.28140897471508e-11 \\
192383	8.72940053575633e-11 \\
192763	8.20991608030397e-11 \\
193143	7.72105157587077e-11 \\
193523	7.26097515446611e-11 \\
193903	6.82800482820767e-11 \\
194283	6.42054187593999e-11 \\
194663	6.03707084323446e-11 \\
195043	5.67620950242542e-11 \\
195423	5.33660338142283e-11 \\
195803	5.01699237709374e-11 \\
196183	4.71620520414717e-11 \\
196563	4.43314829290387e-11 \\
196943	4.16675027814506e-11 \\
197323	3.91607302141495e-11 \\
197703	3.68013397533673e-11 \\
198083	3.45810047264195e-11 \\
198463	3.24914539717724e-11 \\
198843	3.05248604171027e-11 \\
199223	2.86741741462038e-11 \\
199603	2.69325117763231e-11 \\
199983	2.52933785027665e-11 \\
200363	2.37508901435035e-11 \\
200743	2.22991625165037e-11 \\
201123	2.09329220623999e-11 \\
201503	1.96471727775815e-11 \\
201883	1.84371407030426e-11 \\
202263	1.72983294355333e-11 \\
202643	1.62266311498627e-11 \\
203023	1.52179935319907e-11 \\
203403	1.42688083570874e-11 \\
203783	1.33754674003228e-11 \\
204163	1.25347510149254e-11 \\
204543	1.17435505764263e-11 \\
204923	1.09989795049614e-11 \\
205303	1.0298206731818e-11 \\
205683	9.63867874403945e-12 \\
206063	9.01795305097153e-12 \\
206443	8.4339757400187e-12 \\
206823	7.88424880937555e-12 \\
207203	7.36694039105146e-12 \\
207583	6.8800520836021e-12 \\
207963	6.42175201903683e-12 \\
208343	5.99054139627242e-12 \\
208723	5.58464385846946e-12 \\
209103	5.20278264914964e-12 \\
209483	4.84329243377601e-12 \\
209863	4.50500747817273e-12 \\
210243	4.18665102586147e-12 \\
210623	3.88689080921267e-12 \\
211003	3.60494967210911e-12 \\
211383	3.33955085807247e-12 \\
211763	3.0899172109855e-12 \\
212143	2.8547719743699e-12 \\
212523	2.6335045255621e-12 \\
212903	2.42533770844489e-12 \\
213283	2.22938334459855e-12 \\
213663	2.04497530020831e-12 \\
214043	1.87150295261063e-12 \\
214423	1.70818914568827e-12 \\
214803	1.55447876792891e-12 \\
215183	1.40981670782026e-12 \\
215563	1.27364785384998e-12 \\
215943	1.14541709450577e-12 \\
216323	1.02490238518271e-12 \\
216703	9.11382080914791e-13 \\
217083	8.04578625945851e-13 \\
217463	7.04047931066043e-13 \\
217843	6.0945692936798e-13 \\
218223	5.20417042793042e-13 \\
218603	4.36650715585074e-13 \\
218983	3.57769369685457e-13 \\
219363	2.83606471640496e-13 \\
219743	2.13717932240343e-13 \\
220123	1.48103751484996e-13 \\
220503	8.60977955596809e-14 \\
220883	2.79221090693227e-14 \\
221263	-2.68118860446975e-14 \\
221643	-7.84372566897673e-14 \\
222023	-1.27009514017118e-13 \\
222403	-1.72695191480443e-13 \\
222783	-2.15716333684668e-13 \\
223163	-2.56072940629792e-13 \\
223543	-2.94264612676898e-13 \\
223923	-3.29958282918597e-13 \\
224303	-3.6365355171597e-13 \\
224683	-3.95516952522712e-13 \\
225063	-4.25270929582666e-13 \\
225443	-4.53470594408145e-13 \\
225823	-4.79893902394224e-13 \\
226203	-5.04818409297059e-13 \\
226583	-5.28188603965418e-13 \\
226963	-5.50282042155459e-13 \\
227343	-5.71098723867181e-13 \\
227723	-5.90583137949352e-13 \\
228103	-6.08901817855667e-13 \\
228483	-6.2622129703982e-13 \\
228863	-6.42541575501809e-13 \\
229243	-6.57807142090405e-13 \\
229623	-6.72295552561764e-13 \\
230003	-6.85895784613422e-13 \\
230383	-6.9860783824538e-13 \\
230763	-7.10709269213794e-13 \\
231143	-7.21978032913739e-13 \\
231523	-7.32691685101372e-13 \\
231903	-7.42683692322998e-13 \\
232283	-7.52065076881081e-13 \\
232663	-7.60946861078082e-13 \\
233043	-7.69329044914002e-13 \\
233423	-7.7715611723761e-13 \\
233803	-7.84650122653829e-13 \\
234183	-7.91533505406505e-13 \\
234563	-7.98083821251794e-13 \\
234943	-8.04301070189695e-13 \\
};
\addplot [line width=0.01pt, blue, forget plot]
table [row sep=\\]{%
1190	2.55682190438143 \\
2380	2.0254324892414 \\
3570	1.63125062912031 \\
4760	1.32524026654354 \\
5946	1.07509317986305 \\
7126	0.892506362262448 \\
8306	0.752490994217243 \\
9486	0.637145121196819 \\
10666	0.536849764412615 \\
11846	0.457000033972018 \\
13026	0.395650497542633 \\
14206	0.342079407632106 \\
15386	0.295247949055942 \\
16566	0.252554598952609 \\
17746	0.215314650964115 \\
18926	0.187632421138864 \\
20106	0.163378129488487 \\
21286	0.142369816390002 \\
22466	0.124368133740804 \\
23646	0.107551841360565 \\
24826	0.0932701958283143 \\
26006	0.0808116572177218 \\
27186	0.0691595956590271 \\
28366	0.0584782877421064 \\
29546	0.0497403006943276 \\
30726	0.0428798258129514 \\
31906	0.0369069489687631 \\
33086	0.0314497828592404 \\
34266	0.0263579337957193 \\
35446	0.0219433259921791 \\
36626	0.0178683038025572 \\
37806	0.0154123443174307 \\
38986	0.0136571850494879 \\
40166	0.0121638420237713 \\
41346	0.0108742520708055 \\
42526	0.00971745931974061 \\
43706	0.00867410194050533 \\
44886	0.00770617028948145 \\
46066	0.00691461686688016 \\
47246	0.006224470241739 \\
48426	0.00558333746824002 \\
49606	0.00498812914978669 \\
50786	0.00443756139572055 \\
51966	0.00393041751519768 \\
53146	0.00349070700689147 \\
54326	0.0030944034285178 \\
55506	0.00275139734969326 \\
56686	0.00244562219400613 \\
57866	0.00216166203594309 \\
59046	0.00190068514817054 \\
60226	0.00174929724145101 \\
61406	0.00161572886702355 \\
62586	0.00149339457987208 \\
63766	0.00138042211223594 \\
64946	0.00127918570403734 \\
66126	0.00119098030258635 \\
67306	0.00110890879917053 \\
68486	0.00103250840312447 \\
69666	0.000961579657870137 \\
70846	0.000895587691594035 \\
72026	0.000833940478254191 \\
73206	0.00077638002619651 \\
74386	0.000722653797893047 \\
75566	0.000672674318742184 \\
76746	0.000626027701702281 \\
77926	0.000582445094097717 \\
79106	0.000541626431594611 \\
80286	0.000503509494954923 \\
81466	0.000467897924320393 \\
82646	0.000434618040121848 \\
83826	0.000403506918652818 \\
85006	0.000374424169971299 \\
86186	0.000347215456829642 \\
87366	0.000321772651015295 \\
88546	0.000297969636726958 \\
89726	0.000275694924445402 \\
90906	0.000254860120757427 \\
92086	0.000235353868071164 \\
93266	0.000217113634485955 \\
94446	0.000201214141264616 \\
95626	0.000186623472762326 \\
96806	0.000173058005150384 \\
97986	0.00016047077905762 \\
99166	0.000148780830462714 \\
100346	0.000137919262298503 \\
101526	0.000127834264111726 \\
102706	0.000118460952136745 \\
103886	0.000109741558541265 \\
105066	0.000101647850874853 \\
106246	9.41246904751503e-05 \\
107426	8.7127794930919e-05 \\
108606	8.0608352894862e-05 \\
109786	7.45491544877908e-05 \\
110966	6.89114449897721e-05 \\
112146	6.36644834480227e-05 \\
113326	5.88175336955765e-05 \\
114506	5.43511060164814e-05 \\
115686	5.02015755816099e-05 \\
116866	4.63494964935141e-05 \\
118046	4.27782902664986e-05 \\
119226	3.94588880539004e-05 \\
120327	3.63772966969211e-05 \\
120717	3.35125569781991e-05 \\
121107	3.09859512250821e-05 \\
121497	2.87253681519606e-05 \\
121887	2.66330034800477e-05 \\
122277	2.46958455027424e-05 \\
122667	2.2901974272127e-05 \\
123057	2.12404450777859e-05 \\
123447	1.97011966099336e-05 \\
123837	1.82749707807184e-05 \\
124227	1.69532417432228e-05 \\
124617	1.57281526302899e-05 \\
125007	1.45924588688118e-05 \\
125397	1.35394771493824e-05 \\
125787	1.2563039304303e-05 \\
126177	1.16574504810396e-05 \\
126567	1.0817451104983e-05 \\
126957	1.00381822028539e-05 \\
127347	9.31515372731839e-06 \\
127737	8.64421557444928e-06 \\
128127	8.02153102591463e-06 \\
128517	7.44355238574412e-06 \\
128907	6.90699860633748e-06 \\
129297	6.40883472469156e-06 \\
129687	5.94625294775275e-06 \\
130077	5.51665524500811e-06 \\
130467	5.11763731925186e-06 \\
130857	4.74697383923139e-06 \\
131247	4.4026048297563e-06 \\
131637	4.08262312528995e-06 \\
132027	3.78526279759539e-06 \\
132417	3.50888848210662e-06 \\
132807	3.25198552808548e-06 \\
133197	3.01315090922571e-06 \\
133587	2.7910848326429e-06 \\
133977	2.58458299123854e-06 \\
134367	2.39252941025558e-06 \\
134757	2.21388983950854e-06 \\
135147	2.04770564976586e-06 \\
135537	1.89308819426026e-06 \\
135927	1.74921359802349e-06 \\
136317	1.61531794262704e-06 \\
136707	1.49069281613068e-06 \\
137097	1.37468119915107e-06 \\
137487	1.26667366101563e-06 \\
137877	1.16610484396373e-06 \\
138267	1.07245021119251e-06 \\
138657	9.85223038818539e-07 \\
139047	9.03971634880207e-07 \\
139437	8.28276765507674e-07 \\
139827	7.57749274771236e-07 \\
140217	6.92954647552035e-07 \\
140607	6.48010374582064e-07 \\
140997	6.06291481353161e-07 \\
141387	5.67499562176277e-07 \\
141777	5.31397951919477e-07 \\
142167	4.9777470856327e-07 \\
142557	4.66437863555491e-07 \\
142947	4.37213019066629e-07 \\
143337	4.09941387891077e-07 \\
143727	3.84478102855024e-07 \\
144117	3.60690750378367e-07 \\
144507	3.38458090887173e-07 \\
144897	3.17668937765969e-07 \\
145287	2.98221170591528e-07 \\
145677	2.80020865106589e-07 \\
146067	2.62981522280992e-07 \\
146457	2.47023382804468e-07 \\
146847	2.32072818184825e-07 \\
147237	2.18061784018619e-07 \\
147627	2.04927332769778e-07 \\
148017	1.92611175842128e-07 \\
148407	1.81059288395513e-07 \\
148797	1.70221554629535e-07 \\
149187	1.60051447151144e-07 \\
149577	1.50505735874162e-07 \\
149967	1.41544224729895e-07 \\
150357	1.33129513468777e-07 \\
150747	1.25226779335019e-07 \\
151137	1.17803579835485e-07 \\
151527	1.10829671495782e-07 \\
151917	1.04276845547258e-07 \\
152307	9.8118776548084e-08 \\
152697	9.23308845490745e-08 \\
153087	8.68902080841671e-08 \\
153477	8.17752884296752e-08 \\
153867	7.69660618571422e-08 \\
154257	7.2443762322294e-08 \\
154647	6.81908295940836e-08 \\
155037	6.41908268761426e-08 \\
155427	6.0428362924636e-08 \\
155817	5.68890213270556e-08 \\
156207	5.35592941663943e-08 \\
156597	5.04265211809241e-08 \\
156987	4.7478833808956e-08 \\
157377	4.47051018981348e-08 \\
157767	4.20948857993153e-08 \\
158157	3.96383911249742e-08 \\
158547	3.73264266162465e-08 \\
158937	3.51503653406304e-08 \\
159327	3.3102108831784e-08 \\
159717	3.11740523395443e-08 \\
160107	2.93590546873723e-08 \\
160497	2.76504074636641e-08 \\
160887	2.60418092645764e-08 \\
161277	2.45273391041856e-08 \\
161667	2.31014329887813e-08 \\
162057	2.17588617124065e-08 \\
162447	2.04947101511976e-08 \\
162837	1.93043575014151e-08 \\
163227	1.81834596824082e-08 \\
163617	1.71279319616247e-08 \\
164007	1.61339328008658e-08 \\
164397	1.51978498674765e-08 \\
164787	1.4316285379401e-08 \\
165177	1.34860434486406e-08 \\
165567	1.27041171471554e-08 \\
165957	1.1967678126279e-08 \\
166347	1.12740651259102e-08 \\
166737	1.06207740935282e-08 \\
167127	1.00054488028078e-08 \\
167517	9.42587191632427e-09 \\
167907	8.87995649234696e-09 \\
168297	8.36573887941228e-09 \\
168687	7.88137038965075e-09 \\
169077	7.42511130358281e-09 \\
169467	6.99532376469136e-09 \\
169857	6.59046595075097e-09 \\
170247	6.20908674475729e-09 \\
170637	5.84981973972276e-09 \\
171027	5.51137857573991e-09 \\
171417	5.19255211051117e-09 \\
171807	4.89220014499026e-09 \\
172197	4.60924881595659e-09 \\
172587	4.34268743187971e-09 \\
172977	4.0915638654937e-09 \\
173367	3.85498172272847e-09 \\
173757	3.6320969010184e-09 \\
174147	3.42211448067786e-09 \\
174537	3.22428561627675e-09 \\
174927	3.03790570477247e-09 \\
175317	2.86231033319595e-09 \\
175707	2.69687461251777e-09 \\
176097	2.54100912533417e-09 \\
176487	2.39415942626664e-09 \\
176877	2.2558031553821e-09 \\
177267	2.12544798428027e-09 \\
177657	2.00263056138184e-09 \\
178047	1.8869142359712e-09 \\
178437	1.77788789246236e-09 \\
178827	1.67516411853086e-09 \\
179217	1.57837792835736e-09 \\
179607	1.48718576342688e-09 \\
179997	1.40126377168315e-09 \\
180387	1.3203070858836e-09 \\
180777	1.24402838030946e-09 \\
181167	1.17215698258732e-09 \\
181557	1.10443792999959e-09 \\
181947	1.04063102579488e-09 \\
182337	9.80510173054228e-10 \\
182727	9.23861931401149e-10 \\
183117	8.70485739046245e-10 \\
183507	8.20192302963818e-10 \\
183897	7.72803487869567e-10 \\
184287	7.28151317019865e-10 \\
184677	6.86077805678309e-10 \\
185067	6.46433517825784e-10 \\
185457	6.09078454338885e-10 \\
185847	5.7387999907732e-10 \\
186237	5.40713751551181e-10 \\
186627	5.09462361186763e-10 \\
187017	4.80014916703908e-10 \\
187407	4.52267556738661e-10 \\
187797	4.26121693486436e-10 \\
188187	4.01485067413887e-10 \\
188577	3.78270470502429e-10 \\
188967	3.5639580175939e-10 \\
189357	3.35783567617653e-10 \\
189747	3.16360992957954e-10 \\
190137	2.98059132930462e-10 \\
190527	2.80813594599749e-10 \\
190917	2.6456320467716e-10 \\
191307	2.49250509121168e-10 \\
191697	2.34821384559325e-10 \\
192087	2.21224982777102e-10 \\
192477	2.08412842539474e-10 \\
192867	1.96340166347397e-10 \\
193257	1.84963877547517e-10 \\
193647	1.74244008110946e-10 \\
194037	1.64142588410243e-10 \\
194427	1.54624091308619e-10 \\
194817	1.45654432959219e-10 \\
195207	1.37202305072748e-10 \\
195597	1.29237842649843e-10 \\
195987	1.21732846025679e-10 \\
196377	1.14660614336515e-10 \\
196767	1.07996445120051e-10 \\
197157	1.01716690625864e-10 \\
197547	9.57990908823092e-11 \\
197937	9.02228292076757e-11 \\
198327	8.49682546544273e-11 \\
198717	8.00167154757503e-11 \\
199107	7.53506701478557e-11 \\
199497	7.09537983922814e-11 \\
199887	6.68105015755316e-11 \\
200277	6.2906346798286e-11 \\
200667	5.92271232058295e-11 \\
201057	5.576011874453e-11 \\
201447	5.24930654499656e-11 \\
201837	4.94144725138312e-11 \\
202227	4.65134042393345e-11 \\
202617	4.37795355523463e-11 \\
203007	4.12035405794597e-11 \\
203397	3.87758714026631e-11 \\
203787	3.6488256860423e-11 \\
204177	3.43325368135083e-11 \\
204567	3.23011617453517e-11 \\
204957	3.03868596951418e-11 \\
205347	2.8583024835882e-11 \\
205737	2.68832178740297e-11 \\
206127	2.52813880941005e-11 \\
206517	2.37718178475177e-11 \\
206907	2.23494001083679e-11 \\
207297	2.10088613172843e-11 \\
207687	1.97457605821683e-11 \\
208077	1.85554349663164e-11 \\
208467	1.74337211333864e-11 \\
208857	1.6376733302792e-11 \\
209247	1.53806412050983e-11 \\
209637	1.4442003148929e-11 \\
210027	1.35574329540589e-11 \\
210417	1.27238775071703e-11 \\
210807	1.1938394717248e-11 \\
211197	1.11982090267304e-11 \\
211587	1.05007114115097e-11 \\
211977	9.8433483586291e-12 \\
212367	9.22384391088826e-12 \\
212757	8.64014415569159e-12 \\
213147	8.09013966929228e-12 \\
213537	7.5717765390948e-12 \\
213927	7.08327840825973e-12 \\
214317	6.62303545340137e-12 \\
214707	6.18921580652909e-12 \\
215097	5.78043168886211e-12 \\
215487	5.39518429931718e-12 \\
215877	5.03225239256722e-12 \\
216267	4.69013716752897e-12 \\
216657	4.36772840117783e-12 \\
217047	4.06397138164039e-12 \\
217437	3.77775588589202e-12 \\
217827	3.50797169090811e-12 \\
218217	3.25367510711772e-12 \\
218607	3.01408897840361e-12 \\
218997	2.78838063749731e-12 \\
219387	2.57571741713036e-12 \\
219777	2.37515562773183e-12 \\
220167	2.18619566894063e-12 \\
220557	2.00822691809321e-12 \\
220947	1.84052773022358e-12 \\
221337	1.68237646036573e-12 \\
221727	1.53338453046103e-12 \\
222117	1.39294131784595e-12 \\
222507	1.26065824446187e-12 \\
222897	1.13592468764523e-12 \\
223287	1.01835206933742e-12 \\
223677	9.07773856084759e-13 \\
224067	8.03301869467532e-13 \\
224457	7.05158154090668e-13 \\
224847	6.12399020383236e-13 \\
225237	5.25024468345237e-13 \\
225627	4.42812453371744e-13 \\
226017	3.65207863950445e-13 \\
226407	2.92210700081341e-13 \\
226797	2.23210339100888e-13 \\
227187	1.5842882561401e-13 \\
227577	9.72000258059325e-14 \\
227967	3.95794508278868e-14 \\
228357	-1.47104550762833e-14 \\
228747	-6.59472476627343e-14 \\
229137	-1.14186438082697e-13 \\
229527	-1.59539048638635e-13 \\
229917	-2.02449168540397e-13 \\
230307	-2.42861286636753e-13 \\
230697	-2.80886425230165e-13 \\
231087	-3.16746628925557e-13 \\
231477	-3.50497408874162e-13 \\
231867	-3.82416320832135e-13 \\
232257	-4.12281320194552e-13 \\
232647	-4.40592007322493e-13 \\
233037	-4.67237359913497e-13 \\
233427	-4.92328400270026e-13 \\
233817	-5.1592063954331e-13 \\
234207	-5.38291633489507e-13 \\
234597	-5.59274848654923e-13 \\
234987	-5.7903681849325e-13 \\
235377	-5.97744076458184e-13 \\
235767	-6.15285600247262e-13 \\
236157	-6.31827923314177e-13 \\
236547	-6.47482067961391e-13 \\
};
\addplot [line width=0.01pt, blue, forget plot]
table [row sep=\\]{%
1190	2.76289739875642 \\
2380	2.26349144826947 \\
3570	1.90750037251748 \\
4760	1.59921302768253 \\
5950	1.33658113464023 \\
7139	1.10514624781468 \\
8319	0.91303135396776 \\
9499	0.755401052504913 \\
10679	0.626370980734505 \\
11859	0.527311527670798 \\
13039	0.441583978938529 \\
14219	0.368501670234918 \\
15399	0.307799389665201 \\
16579	0.257740424966333 \\
17759	0.218454618892223 \\
18939	0.185387596134288 \\
20119	0.155263920142774 \\
21299	0.131935987176408 \\
22479	0.113411002501316 \\
23659	0.0988516809739453 \\
24839	0.0868702405674149 \\
26019	0.0771220853391376 \\
27199	0.0681969477505585 \\
28379	0.060037674696219 \\
29559	0.0528413851438469 \\
30739	0.0465488127916502 \\
31919	0.0408577534462171 \\
33099	0.0356521581064006 \\
34279	0.0308356179992573 \\
35459	0.0263819090701009 \\
36639	0.0222914067675753 \\
37819	0.0188447610664262 \\
38999	0.0164913890786553 \\
40179	0.014669901311766 \\
41359	0.0131199398377984 \\
42539	0.0118631514471092 \\
43719	0.0107000288212262 \\
44899	0.00963095018079013 \\
46079	0.00869785088131642 \\
47259	0.0078475521147125 \\
48439	0.00705956492094412 \\
49619	0.00643120650560175 \\
50799	0.00587135499530606 \\
51979	0.00542371510706724 \\
53159	0.00501040480497522 \\
54339	0.00462790858518386 \\
55519	0.00427362015249261 \\
56699	0.00394520699798279 \\
57879	0.00364056940466784 \\
59059	0.00335781044104116 \\
60239	0.00309521107672067 \\
61419	0.00285213533120537 \\
62599	0.00263123616883465 \\
63779	0.00242622197384712 \\
64959	0.00223818375080048 \\
66139	0.00206408500463123 \\
67319	0.00190274902821613 \\
68499	0.00175314909636259 \\
69679	0.00161437590727342 \\
70859	0.00148559785293872 \\
72039	0.00136605338257562 \\
73219	0.00125504427336093 \\
74399	0.00115192966550443 \\
75579	0.00105612075605011 \\
76759	0.000967076064028183 \\
77939	0.000884297193654959 \\
79119	0.000807325033614525 \\
80299	0.000735736339678772 \\
81479	0.00066914065551249 \\
82659	0.000610495425402147 \\
83839	0.000562276577295073 \\
85019	0.000517800760891696 \\
86199	0.000476705782569586 \\
87379	0.00043875413781852 \\
88559	0.000403670804337009 \\
89739	0.000371209673410311 \\
90919	0.000341165752960393 \\
92099	0.000313351058642442 \\
93279	0.000287593072033077 \\
94459	0.000263739537727015 \\
95639	0.000241682140287247 \\
96819	0.000221245808260606 \\
97999	0.000202307075399644 \\
99179	0.000185537859609286 \\
100359	0.00017305014637603 \\
101539	0.000161486319586235 \\
102719	0.00015071885044593 \\
103899	0.000140686997449457 \\
105079	0.000131338230841604 \\
106259	0.000122624261134963 \\
107439	0.000114500462914335 \\
108619	0.000106925545695025 \\
109799	9.98612724425474e-05 \\
110979	9.32722085085214e-05 \\
112159	8.71254959226575e-05 \\
113339	8.13906495015781e-05 \\
114519	7.60393719412833e-05 \\
115699	7.1045385561852e-05 \\
116879	6.63842787747559e-05 \\
118059	6.20333656461969e-05 \\
119239	5.7971557180958e-05 \\
120419	5.41792431448251e-05 \\
121599	5.06381834033398e-05 \\
122139	4.73314078827669e-05 \\
122519	4.42431243725117e-05 \\
122899	4.1358633469879e-05 \\
123279	3.86642500525558e-05 \\
123659	3.61472307204846e-05 \\
124039	3.37957067155781e-05 \\
124419	3.15986218626385e-05 \\
124799	2.95456751306933e-05 \\
125179	2.76272674423206e-05 \\
125559	2.58344523956788e-05 \\
125939	2.41588905927625e-05 \\
126319	2.25928072930537e-05 \\
126699	2.11289531382164e-05 \\
127079	1.97605677099699e-05 \\
127459	1.84813457098087e-05 \\
127839	1.7285405557621e-05 \\
128219	1.61672602334573e-05 \\
128599	1.51217901899758e-05 \\
128979	1.41442181842422e-05 \\
129359	1.32300858874945e-05 \\
129739	1.23752321396497e-05 \\
130119	1.15757727311983e-05 \\
130499	1.08280815994122e-05 \\
130879	1.01287733343947e-05 \\
131259	9.47468690509767e-06 \\
131639	8.86287051271584e-06 \\
132019	8.29056749218626e-06 \\
132399	7.75520318768708e-06 \\
132779	7.25437273141338e-06 \\
133159	6.78582966229246e-06 \\
133539	6.347475323909e-06 \\
133919	5.93734898907128e-06 \\
134299	5.55361865628434e-06 \\
134679	5.19457247488697e-06 \\
135059	4.85861075127936e-06 \\
135439	4.54423849871599e-06 \\
135819	4.25005849058468e-06 \\
136199	3.97476478292136e-06 \\
136579	3.71713667463025e-06 \\
136959	3.47603307387923e-06 \\
137339	3.25038724269255e-06 \\
137719	3.04019154823143e-06 \\
138099	2.84438933145781e-06 \\
138479	2.66126907522146e-06 \\
138859	2.49000252083853e-06 \\
139239	2.3298164997243e-06 \\
139619	2.17998893364824e-06 \\
139999	2.03984535135371e-06 \\
140379	1.90875567235294e-06 \\
140759	1.7861312227585e-06 \\
141139	1.67142196294545e-06 \\
141519	1.56411390977995e-06 \\
141899	1.46372673609507e-06 \\
142279	1.36981153670002e-06 \\
142659	1.28194874676746e-06 \\
143039	1.19974620127472e-06 \\
143419	1.12283732572882e-06 \\
143799	1.05087944918258e-06 \\
144179	9.83552230715556e-07 \\
144559	9.20556189554222e-07 \\
144939	8.61611335667423e-07 \\
145319	8.06455889845736e-07 \\
145699	7.54845090267242e-07 \\
146079	7.065500763348e-07 \\
146459	6.61356848785655e-07 \\
146839	6.19065297025045e-07 \\
147219	5.79488291296837e-07 \\
147599	5.42450834695174e-07 \\
147979	5.07789271464443e-07 \\
148359	4.75350546425002e-07 \\
148739	4.44991514692017e-07 \\
149119	4.16578295248105e-07 \\
149499	3.89985668147741e-07 \\
149879	3.65096510634988e-07 \\
150259	3.41801270287156e-07 \\
150639	3.19997472908451e-07 \\
151019	2.99589263230704e-07 \\
151399	2.80486973702754e-07 \\
151779	2.6260672408851e-07 \\
152159	2.45870045489927e-07 \\
152539	2.30203528794881e-07 \\
152919	2.15538498105072e-07 \\
153299	2.01810702371574e-07 \\
153679	1.88960030234053e-07 \\
154059	1.76930241235773e-07 \\
154439	1.65668715523815e-07 \\
154819	1.55126219536506e-07 \\
155199	1.45256687400508e-07 \\
155579	1.36017016039158e-07 \\
155959	1.27366873270418e-07 \\
156339	1.19268519505056e-07 \\
156719	1.11686640325015e-07 \\
157099	1.04588189664412e-07 \\
157479	9.79422433711186e-08 \\
157859	9.17198628713756e-08 \\
158239	8.58939666614766e-08 \\
158619	8.04392112918606e-08 \\
158999	7.53318782908963e-08 \\
159379	7.05497709141412e-08 \\
159759	6.60721148904031e-08 \\
160139	6.18794677720302e-08 \\
160519	5.79536325595598e-08 \\
160899	5.42775783207716e-08 \\
161279	5.0835364751034e-08 \\
161659	4.76120718406747e-08 \\
162039	4.45937349269343e-08 \\
162419	4.17672829655658e-08 \\
162799	3.91204811878154e-08 \\
163179	3.66418771435839e-08 \\
163559	3.43207512409904e-08 \\
163939	3.21470687292269e-08 \\
164319	3.01114365663935e-08 \\
164699	2.8205061730624e-08 \\
165079	2.64197133614807e-08 \\
165459	2.47476858450391e-08 \\
165839	2.31817659512856e-08 \\
166219	2.17152008596955e-08 \\
166599	2.03416688493441e-08 \\
166979	1.90552516543541e-08 \\
167359	1.78504083736541e-08 \\
167739	1.67219519897621e-08 \\
168119	1.56650259430791e-08 \\
168499	1.46750832596965e-08 \\
168879	1.37478675110714e-08 \\
169259	1.2879392496945e-08 \\
169639	1.2065926868754e-08 \\
170019	1.13039765881062e-08 \\
170399	1.05902703828598e-08 \\
170779	9.92174514768962e-09 \\
171159	9.2955328434563e-09 \\
171539	8.70894817373014e-09 \\
171919	8.1594764278492e-09 \\
172299	7.64476310033402e-09 \\
172679	7.16260351030229e-09 \\
173059	6.71093314252857e-09 \\
173439	6.28781871014894e-09 \\
173819	5.89144999452174e-09 \\
174199	5.52013168508836e-09 \\
174579	5.1722763849682e-09 \\
174959	4.84639756104244e-09 \\
175339	4.54110293812704e-09 \\
175719	4.25508878132419e-09 \\
176099	3.98713423388486e-09 \\
176479	3.73609593262714e-09 \\
176859	3.50090290091032e-09 \\
177239	3.28055238529856e-09 \\
177619	3.07410519262419e-09 \\
177999	2.88068174869593e-09 \\
178379	2.69945815700723e-09 \\
178759	2.52966292357826e-09 \\
179139	2.37057329321999e-09 \\
179519	2.22151269602122e-09 \\
179899	2.08184686156798e-09 \\
180279	1.95098237565361e-09 \\
180659	1.82836285000931e-09 \\
181039	1.71346764554769e-09 \\
181419	1.60580887476058e-09 \\
181799	1.50492912576183e-09 \\
182179	1.41040096268696e-09 \\
182559	1.32182287337912e-09 \\
182939	1.23881988001173e-09 \\
183319	1.16103959779679e-09 \\
183699	1.08815279009633e-09 \\
184079	1.01985064837606e-09 \\
184459	9.55844070560374e-10 \\
184839	8.95862217742405e-10 \\
185219	8.39651181916423e-10 \\
185599	7.86973486377462e-10 \\
185979	7.37606198342178e-10 \\
186359	6.91340762415393e-10 \\
186739	6.47981945878229e-10 \\
187119	6.07346395398167e-10 \\
187499	5.69262692540207e-10 \\
187879	5.33570188032684e-10 \\
188259	5.00118391144611e-10 \\
188639	4.68766025996104e-10 \\
189019	4.39381142580686e-10 \\
189399	4.11839951031112e-10 \\
189779	3.86026322019006e-10 \\
190159	3.61831620221409e-10 \\
190539	3.39154260231567e-10 \\
190919	3.17898485313606e-10 \\
191299	2.97975255580951e-10 \\
191679	2.7930058266179e-10 \\
192059	2.61795973788281e-10 \\
192439	2.45388043218497e-10 \\
192819	2.30007846102609e-10 \\
193199	2.15590822971734e-10 \\
193579	2.02076522182182e-10 \\
193959	1.89408322359697e-10 \\
194339	1.77532877287945e-10 \\
194719	1.66400671020028e-10 \\
195099	1.55965018677762e-10 \\
195479	1.46182066451672e-10 \\
195859	1.37011013645605e-10 \\
196239	1.28413446542908e-10 \\
196619	1.20353393917583e-10 \\
196999	1.1279704947853e-10 \\
197379	1.05713104936456e-10 \\
197759	9.90716952919968e-11 \\
198139	9.28452315029915e-11 \\
198519	8.70077343506637e-11 \\
198899	8.15347234173203e-11 \\
199279	7.64034946421077e-11 \\
199659	7.15925652094995e-11 \\
200039	6.70818955939012e-11 \\
200419	6.28526120038941e-11 \\
200799	5.88872839379917e-11 \\
201179	5.5169147028522e-11 \\
201559	5.16830467311991e-11 \\
201939	4.84144391244001e-11 \\
202319	4.53493909091662e-11 \\
202699	4.24754675876216e-11 \\
203079	3.97806787511001e-11 \\
203459	3.72538111470533e-11 \\
203839	3.48842621455958e-11 \\
204219	3.26623172952623e-11 \\
204599	3.0578706233797e-11 \\
204979	2.86248247327592e-11 \\
205359	2.67926236752203e-11 \\
205739	2.50744980334616e-11 \\
206119	2.34631203355207e-11 \\
206499	2.19521067990058e-11 \\
206879	2.05349626192231e-11 \\
207259	1.92061921922004e-11 \\
207639	1.7959800313605e-11 \\
208019	1.67910130244309e-11 \\
208399	1.5694778809916e-11 \\
208779	1.46666567779619e-11 \\
209159	1.37024835922261e-11 \\
209539	1.27981514275177e-11 \\
209919	1.19500520590066e-11 \\
210299	1.11545217507114e-11 \\
210679	1.04083408558608e-11 \\
211059	9.70862279459084e-12 \\
211439	9.05225894243245e-12 \\
211819	8.43652925297533e-12 \\
212199	7.85915776901902e-12 \\
212579	7.31736893300194e-12 \\
212959	6.80933087693347e-12 \\
213339	6.33276764361312e-12 \\
213719	5.8857363427478e-12 \\
214099	5.46646061749811e-12 \\
214479	5.07310859987342e-12 \\
214859	4.70407046648802e-12 \\
215239	4.35801394971236e-12 \\
215619	4.0332737150095e-12 \\
215999	3.72851749474989e-12 \\
216379	3.44285711051384e-12 \\
216759	3.17473825006687e-12 \\
217139	2.9233282461405e-12 \\
217519	2.68746136455889e-12 \\
217899	2.46608289344863e-12 \\
218279	2.25841567669249e-12 \\
218659	2.06357153587078e-12 \\
219039	1.88088433716871e-12 \\
219419	1.70935487986412e-12 \\
219799	1.54848356359594e-12 \\
220179	1.39743772109568e-12 \\
220559	1.25593979660721e-12 \\
220939	1.12304610055958e-12 \\
221319	9.98479077196635e-13 \\
221699	8.81406059249912e-13 \\
222079	7.71605002114484e-13 \\
222459	6.68631816580501e-13 \\
222839	5.72042413438112e-13 \\
223219	4.81392703477468e-13 \\
223599	3.9629410863995e-13 \\
223979	3.16469073169401e-13 \\
224359	2.41529019007203e-13 \\
224739	1.71196390397199e-13 \\
225119	1.05193631583234e-13 \\
225499	4.33542091116124e-14 \\
225879	-1.47659662275146e-14 \\
226259	-6.9333427887841e-14 \\
226639	-1.20403687020598e-13 \\
227019	-1.68476343986868e-13 \\
227399	-2.13384865332955e-13 \\
227779	-2.55739873722405e-13 \\
228159	-2.95485858003985e-13 \\
228539	-3.32622818177697e-13 \\
228919	-3.67650354604621e-13 \\
229299	-4.00346422679831e-13 \\
229679	-4.31266133915642e-13 \\
230059	-4.60131932555896e-13 \\
230439	-4.87276885507981e-13 \\
230819	-5.12700992771897e-13 \\
231199	-5.36626298952569e-13 \\
231579	-5.58941781747535e-13 \\
231959	-5.80036019215413e-13 \\
232339	-5.99853500204972e-13 \\
232719	-6.18338713564981e-13 \\
233099	-6.35713703900365e-13 \\
233479	-6.52033982362354e-13 \\
233859	-6.67355060102182e-13 \\
234239	-6.8184347057354e-13 \\
234619	-6.95277169171504e-13 \\
};
\addplot [line width=0.01pt, blue, forget plot]
table [row sep=\\]{%
1190	2.16404992606559 \\
2380	1.75199549625971 \\
3570	1.43680138198886 \\
4760	1.1895189957336 \\
5942	0.987161706675005 \\
7092	0.81496135519179 \\
8242	0.663523432657523 \\
9392	0.54165028333571 \\
10542	0.441264830128194 \\
11692	0.360111219710996 \\
12842	0.290869184285328 \\
13992	0.230897808406309 \\
15142	0.185581329225646 \\
16292	0.151933823323499 \\
17442	0.124941779181465 \\
18592	0.104123433883767 \\
19742	0.0879973012284718 \\
20892	0.0740286839488013 \\
22042	0.0619504787194664 \\
23192	0.0529047816073856 \\
24342	0.0446497065828306 \\
25492	0.0370052059993233 \\
26642	0.0300761756190294 \\
27792	0.0240146011452091 \\
28942	0.0187942449210455 \\
30092	0.0149654903791764 \\
31242	0.0124061319444858 \\
32392	0.010470367967186 \\
33542	0.0089559524137377 \\
34692	0.00766409287370473 \\
35842	0.00663590460314434 \\
36992	0.00570662636196156 \\
38142	0.00490655723591865 \\
39292	0.00418342561357188 \\
40442	0.0035266737699754 \\
41592	0.00294323088959259 \\
42742	0.00240496999149142 \\
43892	0.00192576869464495 \\
45042	0.00156181966752811 \\
46192	0.0013681435287814 \\
47342	0.00121467100880396 \\
48492	0.00108740966815268 \\
49642	0.000977200798458866 \\
50792	0.000875269086887454 \\
51942	0.000780972422055326 \\
53092	0.000693967872643586 \\
54242	0.000616576223780585 \\
55392	0.00054537871553817 \\
56542	0.00047805009828894 \\
57692	0.000423579346462366 \\
58842	0.000375598807271948 \\
59992	0.000332099614633485 \\
61142	0.000295960723324828 \\
62292	0.000269452823965344 \\
63442	0.000246032712539312 \\
64592	0.0002245742816116 \\
65742	0.000204959650313807 \\
66892	0.000186931216088315 \\
68042	0.000170346663537935 \\
69192	0.000155085054399318 \\
70342	0.000141038539858029 \\
71492	0.000128101556485238 \\
72642	0.000116270832537513 \\
73792	0.000108532492655911 \\
74942	0.000101460715275281 \\
76092	9.48897964993534e-05 \\
77242	8.87868476778242e-05 \\
78392	8.309626600983e-05 \\
79542	7.78130855627612e-05 \\
80692	7.28816289903844e-05 \\
81842	6.82821784797616e-05 \\
82992	6.39993092076407e-05 \\
84142	5.99987818112324e-05 \\
85292	5.62646748148432e-05 \\
86442	5.27725794799982e-05 \\
87592	4.95071578183182e-05 \\
88742	4.64488454516854e-05 \\
89892	4.35885705979966e-05 \\
91042	4.09116061281245e-05 \\
92192	3.84060155109167e-05 \\
93342	3.60587041212113e-05 \\
94492	3.3857935130388e-05 \\
95642	3.17988455457163e-05 \\
96792	2.98666730325636e-05 \\
97942	2.80561116117184e-05 \\
99092	2.63562438618803e-05 \\
100242	2.4761838978149e-05 \\
101392	2.32653064381472e-05 \\
102542	2.18606730664761e-05 \\
103692	2.05417438435274e-05 \\
104842	1.93038534898049e-05 \\
105992	1.81421420442085e-05 \\
107142	1.70502396133476e-05 \\
108292	1.6025353350213e-05 \\
109442	1.50612033749109e-05 \\
110592	1.41554745709538e-05 \\
111742	1.33042852566434e-05 \\
112892	1.25035873185797e-05 \\
114042	1.17597384663992e-05 \\
115192	1.10658727497936e-05 \\
116342	1.04134422992397e-05 \\
117492	9.80096886976645e-06 \\
118642	9.22649601381442e-06 \\
119792	8.68619477828458e-06 \\
120942	8.17797090202355e-06 \\
122092	7.69986733800687e-06 \\
123242	7.2500534620823e-06 \\
124392	6.82681582619971e-06 \\
125542	6.42854979376883e-06 \\
126692	6.05375191548019e-06 \\
127842	5.70101295244152e-06 \\
128992	5.3690114737992e-06 \\
130142	5.05650796428503e-06 \\
131292	4.76233938917492e-06 \\
132442	4.48541417152892e-06 \\
133592	4.22470754174453e-06 \\
134742	3.97925722450676e-06 \\
135892	3.74815943532392e-06 \\
137042	3.53056515783878e-06 \\
138192	3.3256766805434e-06 \\
139342	3.13274436963829e-06 \\
140492	2.95106366149378e-06 \\
141642	2.77997225650584e-06 \\
142792	2.61884749952479e-06 \\
143942	2.46710393186911e-06 \\
145092	2.32419100459902e-06 \\
146242	2.18959093911675e-06 \\
147392	2.06281672587849e-06 \\
148542	1.9434102510596e-06 \\
149692	1.83094054218014e-06 \\
150842	1.72500212569648e-06 \\
151992	1.62521348634481e-06 \\
153142	1.53121562446179e-06 \\
154292	1.44267070240067e-06 \\
155442	1.35926077582393e-06 \\
156592	1.28068660221192e-06 \\
157742	1.20666652420054e-06 \\
158892	1.13693542075355e-06 \\
160042	1.0712437221172e-06 \\
161192	1.00935648633671e-06 \\
162342	9.51052530340224e-07 \\
163492	8.96123614702038e-07 \\
164642	8.44373677422183e-07 \\
165792	7.9561811422435e-07 \\
166942	7.49683102485577e-07 \\
168092	7.06404964023744e-07 \\
169242	6.65629569296389e-07 \\
170392	6.27211773740477e-07 \\
171542	5.91014890749531e-07 \\
172692	5.56910194293714e-07 \\
173842	5.24776451682474e-07 \\
174992	4.94499484193778e-07 \\
176142	4.65971754237682e-07 \\
177292	4.39091975779071e-07 \\
178442	4.13764749074286e-07 \\
179592	3.89900216224426e-07 \\
180742	3.67413737711875e-07 \\
181892	3.46225586977944e-07 \\
183042	3.26260664818001e-07 \\
184192	3.07448228265006e-07 \\
185342	2.89721637625195e-07 \\
186492	2.73018116614399e-07 \\
187642	2.57278528148497e-07 \\
188792	2.42447161458159e-07 \\
189942	2.28471533914032e-07 \\
191092	2.15302202233314e-07 \\
192242	2.0289258667594e-07 \\
193392	1.91198803678461e-07 \\
194542	1.80179509978728e-07 \\
195692	1.69795755233793e-07 \\
196842	1.60010841965263e-07 \\
197992	1.50790195829753e-07 \\
199142	1.4210124243963e-07 \\
200292	1.33913290678578e-07 \\
201442	1.26197424343832e-07 \\
202592	1.18926398340324e-07 \\
203742	1.1207454264639e-07 \\
204892	1.05617670664859e-07 \\
206042	9.95329930697508e-08 \\
207192	9.37990378702125e-08 \\
208342	8.83955727504215e-08 \\
209492	8.33035341263333e-08 \\
210642	7.85049596996323e-08 \\
211792	7.39829242868417e-08 \\
212942	6.9721479645235e-08 \\
214092	6.57055988506627e-08 \\
215242	6.19211218411131e-08 \\
216392	5.83547061783207e-08 \\
217542	5.49937795857325e-08 \\
218692	5.18264952065195e-08 \\
219842	4.88416896371469e-08 \\
220992	4.6028843792012e-08 \\
222142	4.33780457109734e-08 \\
223292	4.08799546991467e-08 \\
224442	3.85257691859486e-08 \\
225592	3.63071955833405e-08 \\
226742	3.42164181987847e-08 \\
227892	3.22460719792694e-08 \\
229042	3.03892173647569e-08 \\
230192	2.86393140869201e-08 \\
231342	2.69901995753052e-08 \\
232492	2.54360664198039e-08 \\
233642	2.39714420535719e-08 \\
234792	2.25911691575931e-08 \\
235942	2.12903880636439e-08 \\
237092	2.0064519157259e-08 \\
238242	1.89092467794971e-08 \\
239392	1.78205039058632e-08 \\
240542	1.67944581019874e-08 \\
241692	1.58274977568595e-08 \\
242842	1.49162192597529e-08 \\
243992	1.40574154539053e-08 \\
245142	1.32480636461096e-08 \\
246292	1.24853156147076e-08 \\
247442	1.17664867849143e-08 \\
248592	1.10890480686798e-08 \\
249742	1.0450615373081e-08 \\
250892	9.84894249489443e-09 \\
252042	9.28191190574523e-09 \\
253192	8.74752903445852e-09 \\
254342	8.24391349629749e-09 \\
255492	7.76929348633715e-09 \\
256642	7.32199911812614e-09 \\
257792	6.90045703910513e-09 \\
258942	6.50318426886898e-09 \\
260092	6.12878325867428e-09 \\
261242	5.77593700645807e-09 \\
262392	5.44340433838997e-09 \\
263542	5.13001557900239e-09 \\
264692	4.83466849887648e-09 \\
265842	4.55632398477235e-09 \\
266992	4.29400354162723e-09 \\
268142	4.04678401899616e-09 \\
269292	3.81379638980661e-09 \\
270442	3.5942210874218e-09 \\
271592	3.38728578519465e-09 \\
272742	3.19226295397712e-09 \\
273892	3.00846653145115e-09 \\
275042	2.8352503678164e-09 \\
276192	2.67200522818811e-09 \\
277342	2.5181570717514e-09 \\
278492	2.37316510887098e-09 \\
279642	2.23651913655587e-09 \\
280792	2.10773881681448e-09 \\
281942	1.9863714562085e-09 \\
283092	1.87199017398498e-09 \\
284242	1.76419290287555e-09 \\
285392	1.66260044620614e-09 \\
286542	1.56685575625204e-09 \\
287692	1.47662210236987e-09 \\
288842	1.39158240486381e-09 \\
289992	1.31143745862872e-09 \\
291142	1.23590560008324e-09 \\
292292	1.16472131939105e-09 \\
293442	1.09763426126008e-09 \\
294592	1.03440866983107e-09 \\
295742	9.74822222943317e-10 \\
296892	9.18665310489786e-10 \\
298042	8.65740812372451e-10 \\
299192	8.1586243316778e-10 \\
300342	7.68854979682487e-10 \\
301492	7.24552917663601e-10 \\
302642	6.82800815887674e-10 \\
303792	6.43451902870851e-10 \\
304942	6.06367567268506e-10 \\
306092	5.71417857475609e-10 \\
307242	5.38479483225274e-10 \\
308392	5.07436925811788e-10 \\
309542	4.78181161334135e-10 \\
310692	4.50609050073325e-10 \\
311842	4.24623891603915e-10 \\
312992	4.00134259059826e-10 \\
314142	3.77054221178952e-10 \\
315292	3.55302398613588e-10 \\
316442	3.34802519041943e-10 \\
317592	3.15482417967416e-10 \\
318742	2.97274260763203e-10 \\
319892	2.80114098583084e-10 \\
321042	2.63941479783369e-10 \\
322192	2.48699782989803e-10 \\
323342	2.3433516238569e-10 \\
324492	2.20797324868016e-10 \\
325642	2.0803847533557e-10 \\
326792	1.96014038333914e-10 \\
327942	1.84681658854657e-10 \\
329092	1.74001479891217e-10 \\
330242	1.63935864883058e-10 \\
331392	1.54449564249148e-10 \\
332542	1.45509160276447e-10 \\
333692	1.3708340018681e-10 \\
334842	1.29142418980877e-10 \\
335992	1.21658572105332e-10 \\
337142	1.14605491763342e-10 \\
338292	1.07958142425701e-10 \\
339442	1.01693375942347e-10 \\
340592	9.57891543862388e-11 \\
341742	9.02247720979688e-11 \\
342892	8.49805781300006e-11 \\
344042	8.0038253802428e-11 \\
345192	7.53802575914619e-11 \\
346342	7.09904357520941e-11 \\
347492	6.68531341396772e-11 \\
348642	6.29539753660424e-11 \\
349792	5.92793036879868e-11 \\
350942	5.58158519403662e-11 \\
352092	5.25519072702707e-11 \\
353242	4.94757568247906e-11 \\
354392	4.65767424628893e-11 \\
355542	4.38443725769844e-11 \\
356692	4.12692657825176e-11 \\
357842	3.88425402952919e-11 \\
358992	3.65552588199591e-11 \\
360142	3.43997608176494e-11 \\
361292	3.23683302383415e-11 \\
362442	3.0453639610073e-11 \\
363592	2.86491941281497e-11 \\
364742	2.69486655213313e-11 \\
365892	2.53459475629825e-11 \\
367042	2.38354336268287e-11 \\
368192	2.24117946423519e-11 \\
369342	2.10702011393948e-11 \\
370492	1.98058236478005e-11 \\
371642	1.86139992308654e-11 \\
372792	1.74910641526083e-11 \\
373942	1.64324664986282e-11 \\
375092	1.54348755998512e-11 \\
376242	1.44947942537499e-11 \\
377392	1.36087807689478e-11 \\
378542	1.27736154986735e-11 \\
379692	1.1986633907668e-11 \\
380842	1.12448939049159e-11 \\
381992	1.05458419774607e-11 \\
383142	9.8870356346481e-12 \\
384292	9.26608789697525e-12 \\
385442	8.68100036299779e-12 \\
386592	8.12955258666648e-12 \\
387742	7.60980167768821e-12 \\
388892	7.11991576807236e-12 \\
390042	6.65828503443322e-12 \\
391192	6.22329965338508e-12 \\
392342	5.81312775693732e-12 \\
393492	5.4266591220653e-12 \\
394642	5.06239494768579e-12 \\
395792	4.71911398847169e-12 \\
396942	4.39559499909592e-12 \\
398092	4.09072775653385e-12 \\
399242	3.80340203776086e-12 \\
400392	3.53256313090355e-12 \\
401542	3.27737836869346e-12 \\
402692	3.03679303925719e-12 \\
403842	2.81014100877996e-12 \\
404992	2.59636756538839e-12 \\
406142	2.39508413102385e-12 \\
407292	2.20540252726664e-12 \\
408442	2.02643457569707e-12 \\
409592	1.8579582317102e-12 \\
410742	1.69897429458388e-12 \\
411892	1.54937174201564e-12 \\
413042	1.40820688443455e-12 \\
414192	1.27520216608445e-12 \\
415342	1.14996900890674e-12 \\
416492	1.03173025678416e-12 \\
417642	9.20485909716717e-13 \\
418792	8.15514322738409e-13 \\
419942	7.16648962395539e-13 \\
421092	6.23445739478257e-13 \\
422242	5.35627098230407e-13 \\
423392	4.52859971744601e-13 \\
424542	3.74866804264684e-13 \\
425692	3.01370040034499e-13 \\
426842	2.32036612146658e-13 \\
427992	1.66588964845005e-13 \\
429142	1.05193631583234e-13 \\
430292	4.71289673953379e-14 \\
431442	-7.43849426498855e-15 \\
432592	-5.90083537588271e-14 \\
433742	-1.07580611086178e-13 \\
434892	-1.53266288549503e-13 \\
436042	-1.96453964207421e-13 \\
437192	-2.37143638059933e-13 \\
438342	-2.75501843560733e-13 \\
439492	-3.11528580709819e-13 \\
440642	-3.4555691641458e-13 \\
441792	-3.77697872977478e-13 \\
442942	-4.07840428096051e-13 \\
444092	-4.36317648677687e-13 \\
445242	-4.63185045873615e-13 \\
446392	-4.88442619683838e-13 \\
447542	-5.12312414713278e-13 \\
448692	-5.34794430961938e-13 \\
449842	-5.55944179581047e-13 \\
450992	-5.76039216326762e-13 \\
452142	-5.94690963140465e-13 \\
453292	-6.12510042685699e-13 \\
454442	-6.29163388055076e-13 \\
455592	-6.44928555004753e-13 \\
456742	-6.59805543534731e-13 \\
457892	-6.73794353645008e-13 \\
459042	-6.86950496486816e-13 \\
460192	-6.99440505513849e-13 \\
};
\addplot [line width=0.01pt, blue, forget plot]
table [row sep=\\]{%
1190	2.72877374678762 \\
2380	2.04987958508941 \\
3570	1.60467997352245 \\
4760	1.26930349681101 \\
5950	1.02445723536507 \\
7132	0.835149966758151 \\
8312	0.689585485933972 \\
9492	0.577185950427373 \\
10672	0.480597109028634 \\
11852	0.400929159034974 \\
13032	0.333643259026759 \\
14212	0.281111626187858 \\
15392	0.236700424941436 \\
16572	0.197076221448015 \\
17752	0.16092544990623 \\
18932	0.131016529735089 \\
20112	0.108144629296046 \\
21292	0.0893362805685541 \\
22472	0.074300501841401 \\
23652	0.0612336276054199 \\
24832	0.0509199502424182 \\
26012	0.0424512901997248 \\
27192	0.0345773821872128 \\
28372	0.0272883485439432 \\
29552	0.0215760752585816 \\
30732	0.0173743083591525 \\
31912	0.0136247409592585 \\
33092	0.0106931297089234 \\
34272	0.00830831779727487 \\
35452	0.00616520451004604 \\
36632	0.00495715298627541 \\
37812	0.0042627786888797 \\
38992	0.00362730674110667 \\
40172	0.00304413223333105 \\
41352	0.002581055161961 \\
42532	0.00218024795337418 \\
43712	0.00184068701186063 \\
44892	0.00157189964570681 \\
46072	0.00132518228550788 \\
47252	0.00110329504804857 \\
48432	0.000933886376549842 \\
49612	0.000780886016348459 \\
50792	0.000685314050855257 \\
51972	0.00060058358948295 \\
53152	0.000534672688532967 \\
54332	0.000480664344761572 \\
55512	0.000433441673413137 \\
56692	0.000391437564624897 \\
57872	0.000353742587326622 \\
59052	0.000319822268616488 \\
60232	0.000292770536127474 \\
61412	0.000268387127104752 \\
62592	0.000246291152974476 \\
63772	0.000226233936490239 \\
64952	0.000208001839197003 \\
66132	0.000191406812887207 \\
67312	0.000176282722279597 \\
68492	0.000162482471147307 \\
69672	0.000149875560957524 \\
70852	0.000138345997939215 \\
72032	0.000127790492612967 \\
73212	0.000118116906466292 \\
74392	0.000109242908319429 \\
75572	0.000101094809232427 \\
76752	9.36065499310135e-05 \\
77932	8.6718818902376e-05 \\
79112	8.03782827513144e-05 \\
80292	7.45369132433837e-05 \\
81472	6.91513978112734e-05 \\
82652	6.41826222679276e-05 \\
83832	5.95952161132063e-05 \\
85012	5.53571522016183e-05 \\
86192	5.14393937068314e-05 \\
87372	4.78155823085413e-05 \\
88552	4.44617623630017e-05 \\
89732	4.13561365362192e-05 \\
90912	3.84788489867227e-05 \\
92092	3.58117927111157e-05 \\
93272	3.33384381072666e-05 \\
94452	3.10436802016256e-05 \\
95632	2.89137023137265e-05 \\
96812	2.6935854218757e-05 \\
97992	2.50985431137773e-05 \\
99172	2.33911359108774e-05 \\
100352	2.18038715599222e-05 \\
101532	2.03277822656278e-05 \\
102712	1.89546475962699e-05 \\
103892	1.76781135481363e-05 \\
105072	1.64898743086339e-05 \\
106252	1.53835023445348e-05 \\
107432	1.43530711933515e-05 \\
108612	1.33931122182629e-05 \\
109792	1.24985761270069e-05 \\
110972	1.16647981080464e-05 \\
112152	1.08874661698954e-05 \\
113332	1.01625923502602e-05 \\
114512	9.48648650545003e-06 \\
115692	8.85573242226378e-06 \\
116872	8.26716602309041e-06 \\
118052	7.71785546282988e-06 \\
119232	7.20508293305944e-06 \\
120412	6.72632801490547e-06 \\
121192	6.27925243262828e-06 \\
121572	5.86168608063264e-06 \\
121952	5.47161420588749e-06 \\
122332	5.10716564139369e-06 \\
122712	4.76660199760781e-06 \\
123092	4.44830772433624e-06 \\
123472	4.15078096993593e-06 \\
123852	3.87262516576836e-06 \\
124232	3.61254127528854e-06 \\
124612	3.36932064914919e-06 \\
124992	3.14183843813653e-06 \\
125372	2.92904751258982e-06 \\
125752	2.72997285216681e-06 \\
126132	2.54370635999202e-06 \\
126512	2.36940207420933e-06 \\
126892	2.20627173735943e-06 \\
127272	2.05358070137773e-06 \\
127652	1.91064413507247e-06 \\
128032	1.77682351581998e-06 \\
128412	1.65430859683235e-06 \\
128792	1.55234693305362e-06 \\
129172	1.45700025822038e-06 \\
129552	1.36777695486368e-06 \\
129932	1.28424811862704e-06 \\
130312	1.20602078884646e-06 \\
130692	1.13273311724837e-06 \\
131072	1.06405129246534e-06 \\
131452	9.99666910361796e-07 \\
131832	9.39294658219847e-07 \\
132212	8.82670262047913e-07 \\
132592	8.29548661540613e-07 \\
132972	7.79702377995406e-07 \\
133352	7.32920054202779e-07 \\
133732	6.89005138887477e-07 \\
134112	6.47774703488313e-07 \\
134492	6.09058369405169e-07 \\
134872	5.72697338274697e-07 \\
135252	5.38543509343015e-07 \\
135632	5.06458676163835e-07 \\
136012	4.7631379535007e-07 \\
136392	4.47988314777881e-07 \\
136772	4.21369562908502e-07 \\
137152	3.96352184739435e-07 \\
137532	3.7283762666096e-07 \\
137912	3.50733661114155e-07 \\
138292	3.29953948885464e-07 \\
138672	3.10417635984717e-07 \\
139052	2.92048981387349e-07 \\
139432	2.74777012199134e-07 \\
139812	2.58535205355237e-07 \\
140192	2.43261192633959e-07 \\
140572	2.28896486764718e-07 \\
140952	2.15386227631065e-07 \\
141332	2.02678946625845e-07 \\
141712	1.90726347049086e-07 \\
142092	1.79483100715139e-07 \\
142472	1.68906657882495e-07 \\
142852	1.58957070228727e-07 \\
143232	1.49596826815035e-07 \\
143612	1.40790699487692e-07 \\
143992	1.32505600380917e-07 \\
144372	1.24710447246823e-07 \\
144752	1.17376039332484e-07 \\
145132	1.10474939307714e-07 \\
145512	1.03981365573436e-07 \\
145892	9.78710887888923e-08 \\
146272	9.21213366145146e-08 \\
146652	8.67107050050997e-08 \\
147032	8.16190736108169e-08 \\
147412	7.68275271734176e-08 \\
147792	7.23182829731606e-08 \\
148172	6.80746214398731e-08 \\
148552	6.40808213714372e-08 \\
148932	6.03220995376574e-08 \\
149312	5.67845547805312e-08 \\
149692	5.34551129471872e-08 \\
150072	5.03214788727391e-08 \\
150452	4.73720885296736e-08 \\
150832	4.45960647854626e-08 \\
151212	4.19831773235124e-08 \\
151592	3.9523802175534e-08 \\
151972	3.72088870270737e-08 \\
152352	3.50299158569101e-08 \\
152732	3.2978877517742e-08 \\
153112	3.10482348719887e-08 \\
153492	2.9230897646837e-08 \\
153872	2.75201945121317e-08 \\
154252	2.59098488775145e-08 \\
154632	2.43939552446726e-08 \\
155012	2.2966957002879e-08 \\
155392	2.1623625834355e-08 \\
155772	2.03590420078115e-08 \\
156152	1.91685763928362e-08 \\
156532	1.80478728073474e-08 \\
156912	1.69928321969159e-08 \\
157292	1.59995969806204e-08 \\
157672	1.50645371177482e-08 \\
158052	1.41842365630751e-08 \\
158432	1.33554801662328e-08 \\
158812	1.25752425694792e-08 \\
159192	1.18406763283119e-08 \\
159572	1.11491016974163e-08 \\
159952	1.04979964166141e-08 \\
160332	9.88498699561191e-09 \\
160712	9.30783927710621e-09 \\
161092	8.7644505541995e-09 \\
161472	8.25284174332808e-09 \\
161852	7.77115011230123e-09 \\
162232	7.31762239691847e-09 \\
162612	6.89060841718714e-09 \\
162992	6.48855474905119e-09 \\
163372	6.10999945083179e-09 \\
163752	5.75356629006762e-09 \\
164132	5.41795952546664e-09 \\
164512	5.10196029868126e-09 \\
164892	4.80442047257057e-09 \\
165272	4.52425952257585e-09 \\
165652	4.26046048440654e-09 \\
166032	4.01206601274851e-09 \\
166412	3.77817471752806e-09 \\
166792	3.55793822182093e-09 \\
167172	3.35055783118321e-09 \\
167552	3.15528159156031e-09 \\
167932	2.97140184679634e-09 \\
168312	2.79825229654307e-09 \\
168692	2.63520544274698e-09 \\
169072	2.48167097982588e-09 \\
169452	2.33709290808903e-09 \\
169832	2.20094759084688e-09 \\
170212	2.072742644188e-09 \\
170592	1.95201393937694e-09 \\
170972	1.83832493672043e-09 \\
171352	1.73126429858783e-09 \\
171732	1.63044500123277e-09 \\
172112	1.53550255843626e-09 \\
172492	1.44609368923909e-09 \\
172872	1.36189548527454e-09 \\
173252	1.28260380094503e-09 \\
173632	1.20793225422133e-09 \\
174012	1.13761122744194e-09 \\
174392	1.0713867015788e-09 \\
174772	1.00901964561473e-09 \\
175152	9.50285183876076e-10 \\
175532	8.94971374787445e-10 \\
175912	8.42878489226706e-10 \\
176292	7.93818955013847e-10 \\
176672	7.4761558055414e-10 \\
177052	7.04102276483098e-10 \\
177432	6.63121668686983e-10 \\
177812	6.24526430570427e-10 \\
178192	5.88177451188443e-10 \\
178572	5.5394389075758e-10 \\
178952	5.21702181455197e-10 \\
179332	4.91336527019826e-10 \\
179712	4.62737514972389e-10 \\
180092	4.35802283149656e-10 \\
180472	4.10433742548122e-10 \\
180852	3.86540688346315e-10 \\
181232	3.6403724479328e-10 \\
181612	3.42842587652825e-10 \\
181992	3.22880222558553e-10 \\
182372	3.04078651147677e-10 \\
182752	2.86370205326847e-10 \\
183132	2.69691158294449e-10 \\
183512	2.53981724540608e-10 \\
183892	2.39185227179917e-10 \\
184272	2.25248819596402e-10 \\
184652	2.12122319709351e-10 \\
185032	1.99758598551369e-10 \\
185412	1.88113302712622e-10 \\
185792	1.77144576785082e-10 \\
186172	1.66813118873677e-10 \\
186552	1.57081847529383e-10 \\
186932	1.47915790726927e-10 \\
187312	1.39282196887081e-10 \\
187692	1.31149924254004e-10 \\
188072	1.23490051517905e-10 \\
188452	1.16274823103169e-10 \\
188832	1.09478648369077e-10 \\
189212	1.03077046897937e-10 \\
189592	9.70472036065928e-11 \\
189972	9.13673026126105e-11 \\
190352	8.60171933680931e-11 \\
190732	8.09776135035634e-11 \\
191112	7.62305218948711e-11 \\
191492	7.17587655962859e-11 \\
191872	6.75466349520093e-11 \\
192252	6.35790864400576e-11 \\
192632	5.9841520627657e-11 \\
193012	5.63208923942682e-11 \\
193392	5.30045451974104e-11 \\
193772	4.98804886284177e-11 \\
194152	4.69377314793462e-11 \\
194532	4.41656711203109e-11 \\
194912	4.15543710552413e-11 \\
195292	3.90945054107306e-11 \\
195672	3.67773589360354e-11 \\
196052	3.45944939361686e-11 \\
196432	3.25381388499579e-11 \\
196812	3.06011882500457e-11 \\
197192	2.8776481197923e-11 \\
197572	2.70574673777446e-11 \\
197952	2.54382626074801e-11 \\
198332	2.39127051493426e-11 \\
198712	2.24757989997215e-11 \\
199092	2.11219930434936e-11 \\
199472	1.98467353662579e-11 \\
199852	1.8645363031311e-11 \\
200232	1.75136571911594e-11 \\
200612	1.64475100206118e-11 \\
200992	1.5442980227931e-11 \\
201372	1.44969036774967e-11 \\
201752	1.36055056110251e-11 \\
202132	1.27657329151987e-11 \\
202512	1.1974643499002e-11 \\
202892	1.12293507825711e-11 \\
203272	1.05272457417982e-11 \\
203652	9.86588588602899e-12 \\
204032	9.24260668000443e-12 \\
204412	8.65563176688511e-12 \\
204792	8.10257416716809e-12 \\
205172	7.5815465017115e-12 \\
205552	7.09071690252472e-12 \\
205932	6.62830901276834e-12 \\
206312	6.1926019867542e-12 \\
206692	5.78220804570151e-12 \\
207072	5.39551736622457e-12 \\
207452	5.03114216954259e-12 \\
207832	4.68791672147972e-12 \\
208212	4.36461977670888e-12 \\
208592	4.05997457875173e-12 \\
208972	3.77298192688613e-12 \\
209352	3.50264262038991e-12 \\
209732	3.24784643623843e-12 \\
210112	3.00787172946571e-12 \\
210492	2.78171929934956e-12 \\
210872	2.56872301207522e-12 \\
211252	2.367994689223e-12 \\
211632	2.1788681969781e-12 \\
212012	2.00078842382823e-12 \\
212392	1.8329227025049e-12 \\
212772	1.67471592149582e-12 \\
213152	1.52572399159112e-12 \\
213532	1.3853918012785e-12 \\
213912	1.25310872789441e-12 \\
214292	1.12843068222901e-12 \\
214672	1.01113561967736e-12 \\
215052	9.00446384122233e-13 \\
215432	7.96362975563625e-13 \\
215812	6.98052726733067e-13 \\
216192	6.05571148781792e-13 \\
216572	5.18418641348717e-13 \\
216952	4.36262137526455e-13 \\
217332	3.58990615012544e-13 \\
217712	2.8599345114344e-13 \\
218092	2.17437179372837e-13 \\
218472	1.52600154734728e-13 \\
218852	9.16489106828067e-14 \\
219232	3.41393580072236e-14 \\
219612	-1.99840144432528e-14 \\
219992	-7.08877401223162e-14 \\
220372	-1.18960397088586e-13 \\
220752	-1.64257496493292e-13 \\
221132	-2.06890060638898e-13 \\
221512	-2.4713564528156e-13 \\
221892	-2.84938739270046e-13 \\
222272	-3.20687920662976e-13 \\
222652	-3.54272167157887e-13 \\
223032	-3.86024545662167e-13 \\
223412	-4.15834033873352e-13 \\
223792	-4.43922676396369e-13 \\
224172	-4.70512517836141e-13 \\
224552	-4.95437024738976e-13 \\
224932	-5.18918241709798e-13 \\
225312	-5.41178213353533e-13 \\
225692	-5.62050406216485e-13 \\
226072	-5.81701353752351e-13 \\
226452	-6.00297589414822e-13 \\
226832	-6.17728090901437e-13 \\
227212	-6.34159391665889e-13 \\
227592	-6.49702514010642e-13 \\
227972	-6.64246435633231e-13 \\
228352	-6.78124223441046e-13 \\
228732	-6.91058321677929e-13 \\
229112	-7.03270774948805e-13 \\
229492	-7.14761583253676e-13 \\
229872	-7.25586257743771e-13 \\
230252	-7.35800309570322e-13 \\
230632	-7.45570272187024e-13 \\
231012	-7.54618589837719e-13 \\
231392	-7.63167307127333e-13 \\
231772	-7.71216424055865e-13 \\
232152	-7.78765940623316e-13 \\
232532	-7.85926879132148e-13 \\
232912	-7.92699239582362e-13 \\
233292	-7.99083021973956e-13 \\
233672	-8.05078226306932e-13 \\
234052	-8.10684852581289e-13 \\
};
\addplot [line width=0.01pt, blue, forget plot]
table [row sep=\\]{%
1190	2.65407726924642 \\
2380	2.19216188805613 \\
3570	1.81254608638987 \\
4760	1.49169509840465 \\
5950	1.22875883817758 \\
7133	1.02505661501562 \\
8313	0.856627044006287 \\
9493	0.715269845542025 \\
10673	0.590479190828263 \\
11853	0.488757847344876 \\
13033	0.403294544606053 \\
14213	0.339835109887786 \\
15393	0.286805441030919 \\
16573	0.23828021924751 \\
17753	0.195959277096327 \\
18933	0.162470537836306 \\
20113	0.135461338929478 \\
21293	0.113390334406427 \\
22473	0.0944040256955014 \\
23653	0.0776385265852766 \\
24833	0.0626872646127258 \\
26013	0.0512490497685081 \\
27193	0.0434777103175562 \\
28373	0.0377274928813796 \\
29553	0.0326787420852958 \\
30733	0.0289516212040887 \\
31913	0.0259289272295943 \\
33093	0.0234264504467878 \\
34273	0.0212357675930823 \\
35453	0.0193251941553406 \\
36633	0.0176865696176716 \\
37813	0.01625775843898 \\
38993	0.0149675592932533 \\
40173	0.0137644641035312 \\
41353	0.0126422115047874 \\
42533	0.0116035177964685 \\
43713	0.010643592252902 \\
44893	0.00975731622857978 \\
46073	0.00894442491699232 \\
47253	0.00819159780093215 \\
48433	0.00757259703325536 \\
49613	0.00706874416550524 \\
50793	0.00659944724648798 \\
51973	0.00616444076886852 \\
53153	0.00576713603209134 \\
54333	0.00539778163010124 \\
55513	0.00505462278659891 \\
56693	0.00474093101945366 \\
57873	0.00444710159031497 \\
59053	0.00417477755517343 \\
60233	0.00391929042245071 \\
61413	0.00367952077699146 \\
62593	0.00345438229212663 \\
63773	0.00324347236390649 \\
64953	0.00304538711386687 \\
66133	0.00285917119436069 \\
67313	0.00268405914580222 \\
68493	0.00251938329927753 \\
69673	0.00236448633194752 \\
70853	0.00221874312290049 \\
72033	0.00208159233871852 \\
73213	0.00195254401870237 \\
74393	0.00183103954968383 \\
75573	0.00171665228763396 \\
76753	0.00160895808020228 \\
77933	0.00150761640099856 \\
79113	0.00141384343298434 \\
80293	0.00132561979986384 \\
81473	0.0012426474805976 \\
82653	0.00116456645710128 \\
83833	0.00109107894427862 \\
85013	0.00102189155361565 \\
86193	0.000956766813944909 \\
87373	0.000895462599875996 \\
88553	0.000837751938296494 \\
89733	0.000783438413769877 \\
90913	0.000732287422058109 \\
92093	0.00068410873627428 \\
93273	0.000638720341856913 \\
94453	0.000595982481158497 \\
95633	0.000555742679995141 \\
96813	0.000517832049686084 \\
97993	0.000483886223760777 \\
99173	0.000454281706846693 \\
100353	0.000426538120505027 \\
101533	0.000400528366442221 \\
102713	0.000376138088519329 \\
103893	0.000353257530466722 \\
105073	0.000331799094649865 \\
106253	0.000311663406625307 \\
107433	0.000292775269020107 \\
108613	0.000275037171758641 \\
109793	0.00025839843703146 \\
110973	0.000242770598844755 \\
112153	0.000228099747689947 \\
113333	0.000214322610167905 \\
114513	0.000201385379399399 \\
115693	0.000189238191760865 \\
116873	0.000177828020395399 \\
118053	0.000167115894294023 \\
119233	0.000157058621600814 \\
120413	0.000147604239709942 \\
121273	0.000138722123920465 \\
121653	0.000130375435709928 \\
122033	0.000122534772237282 \\
122413	0.000115169007896598 \\
122793	0.000108248983256787 \\
123173	0.000101747373199845 \\
123553	9.563856560596e-05 \\
123933	8.98985490279114e-05 \\
124313	8.45048084971989e-05 \\
124693	7.94362287394601e-05 \\
125073	7.46730041650268e-05 \\
125453	7.01965550762851e-05 \\
125833	6.5989449590298e-05 \\
126213	6.20353308308785e-05 \\
126593	5.83188489871045e-05 \\
126973	5.48255978767287e-05 \\
127353	5.15420556810309e-05 \\
127733	4.8455529553959e-05 \\
128113	4.55541038291707e-05 \\
128493	4.28265915741743e-05 \\
128873	4.02624892593129e-05 \\
129253	3.78519343300909e-05 \\
129633	3.55856654849496e-05 \\
130013	3.34549854785204e-05 \\
130393	3.14517262801628e-05 \\
130773	2.95682164346278e-05 \\
131153	2.77972504776902e-05 \\
131533	2.61320602739645e-05 \\
131913	2.45662881516728e-05 \\
132293	2.31093162147977e-05 \\
132673	2.17405265676152e-05 \\
133053	2.04533520480887e-05 \\
133433	1.92428672192713e-05 \\
133813	1.81044536538e-05 \\
134193	1.70337771359663e-05 \\
134573	1.60267691693416e-05 \\
134953	1.5079609978097e-05 \\
135333	1.41887127173002e-05 \\
135713	1.3350708776283e-05 \\
136093	1.25624340848174e-05 \\
136473	1.18209163388339e-05 \\
136853	1.11233630762375e-05 \\
137233	1.04671505344878e-05 \\
137613	9.84981323648437e-06 \\
137993	9.26903424725012e-06 \\
138373	8.72263605661283e-06 \\
138753	8.20857204231196e-06 \\
139133	7.72491847478296e-06 \\
139513	7.26986702531685e-06 \\
139893	6.84171774462117e-06 \\
140273	6.43887248025221e-06 \\
140653	6.05982870466315e-06 \\
141033	5.70317372411422e-06 \\
141413	5.3675792485719e-06 \\
141793	5.05179629406438e-06 \\
142173	4.75465040095102e-06 \\
142553	4.47503714329223e-06 \\
142933	4.21191791549758e-06 \\
143313	3.96431597510238e-06 \\
143693	3.73131272707328e-06 \\
144073	3.51204423487683e-06 \\
144453	3.30569794376734e-06 \\
144833	3.11150960174977e-06 \\
145213	2.92876036878109e-06 \\
145593	2.75677409972142e-06 \\
145973	2.59491479248641e-06 \\
146353	2.44258418957699e-06 \\
146733	2.2992195246041e-06 \\
147113	2.16429140464935e-06 \\
147493	2.03730181969064e-06 \\
147873	1.91778227287553e-06 \\
148253	1.80529202098434e-06 \\
148633	1.69941642319538e-06 \\
149013	1.5997653869948e-06 \\
149393	1.50597190734514e-06 \\
149773	1.41769069461617e-06 \\
150153	1.33459688350657e-06 \\
150533	1.25638481968116e-06 \\
150913	1.18276691840613e-06 \\
151293	1.11347259240668e-06 \\
151673	1.04824724272978e-06 \\
152053	9.86851309781045e-07 \\
152433	9.29059382648312e-07 \\
152813	8.746593590514e-07 \\
153193	8.2345165702824e-07 \\
153573	7.75248473583456e-07 \\
153953	7.29873085969501e-07 \\
154333	6.87159196655074e-07 \\
154713	6.46950315541517e-07 \\
155093	6.09099179871286e-07 \\
155473	5.73467207998224e-07 \\
155853	5.39923986297985e-07 \\
156233	5.08346785998981e-07 \\
156613	4.7862010871258e-07 \\
156993	4.50635259441334e-07 \\
157373	4.24289944400691e-07 \\
157753	3.9948789304356e-07 \\
158133	3.76138501623213e-07 \\
158513	3.54156499793312e-07 \\
158893	3.3346163502701e-07 \\
159273	3.13978376187407e-07 \\
159653	2.95635635638725e-07 \\
160033	2.78366506900607e-07 \\
160413	2.62108017567986e-07 \\
160793	2.46800898051625e-07 \\
161173	2.32389362586627e-07 \\
161553	2.18820904562822e-07 \\
161933	2.06046102624313e-07 \\
162313	1.94018439259036e-07 \\
162693	1.82694129768901e-07 \\
163073	1.72031961176433e-07 \\
163453	1.61993140512795e-07 \\
163833	1.52541153153329e-07 \\
164213	1.43641627647906e-07 \\
164593	1.35262210487763e-07 \\
164973	1.27372446701024e-07 \\
165353	1.19943668830391e-07 \\
165733	1.12948891073383e-07 \\
166113	1.06362710472485e-07 \\
166493	1.00161214877659e-07 \\
166873	9.43218936844126e-08 \\
167253	8.8823556565476e-08 \\
167633	8.36462556441653e-08 \\
168013	7.87712122196638e-08 \\
168393	7.41807484883061e-08 \\
168773	6.98582222624644e-08 \\
169153	6.57879666854377e-08 \\
169533	6.19552330549666e-08 \\
169913	5.83461363112825e-08 \\
170293	5.49476046329822e-08 \\
170673	5.17473319194828e-08 \\
171053	4.87337328269888e-08 \\
171433	4.58958998028614e-08 \\
171813	4.322356411679e-08 \\
172193	4.07070582242319e-08 \\
172573	3.83372799617199e-08 \\
172953	3.6105660350394e-08 \\
173333	3.40041313995343e-08 \\
173713	3.20250972407621e-08 \\
174093	3.01614060393973e-08 \\
174473	2.84063248479072e-08 \\
174853	2.67535139597541e-08 \\
175233	2.51970043718686e-08 \\
175613	2.37311766349002e-08 \\
175993	2.23507390928468e-08 \\
176373	2.10507097309076e-08 \\
176753	1.98263978568036e-08 \\
177133	1.8673386670276e-08 \\
177513	1.75875172203632e-08 \\
177893	1.65648741390356e-08 \\
178273	1.56017702646061e-08 \\
178653	1.46947342072323e-08 \\
179033	1.38404968597072e-08 \\
179413	1.30359805727842e-08 \\
179793	1.22782868872129e-08 \\
180173	1.15646868192876e-08 \\
180553	1.08926107578178e-08 \\
180933	1.0259638860699e-08 \\
181313	9.66349278375134e-09 \\
181693	9.10202696546847e-09 \\
182073	8.57322113301251e-09 \\
182453	8.07517264167501e-09 \\
182833	7.60609042416149e-09 \\
183213	7.16428716351913e-09 \\
183593	6.74817440815545e-09 \\
183973	6.35625629907821e-09 \\
184353	5.98712424082493e-09 \\
184733	5.63945157239232e-09 \\
185113	5.31198912634423e-09 \\
185493	5.00356051036377e-09 \\
185873	4.71305811045042e-09 \\
186253	4.43943848349448e-09 \\
186633	4.18171930416378e-09 \\
187013	3.93897525707843e-09 \\
187393	3.71033526125331e-09 \\
187773	3.49497863982862e-09 \\
188153	3.29213284411267e-09 \\
188533	3.10107012291283e-09 \\
188913	2.92110519106714e-09 \\
189293	2.75159256490909e-09 \\
189673	2.59192445284384e-09 \\
190053	2.4415285349022e-09 \\
190433	2.29986568678342e-09 \\
190813	2.16642798145372e-09 \\
191193	2.04073752341216e-09 \\
191573	1.9223440617111e-09 \\
191953	1.81082310257707e-09 \\
192333	1.70577574287734e-09 \\
192713	1.60682533945078e-09 \\
193093	1.51361784217485e-09 \\
193473	1.42581940698605e-09 \\
193853	1.34311600730186e-09 \\
194233	1.26521149113046e-09 \\
194613	1.19182708147036e-09 \\
194993	1.12270021057626e-09 \\
195373	1.0575836872917e-09 \\
195753	9.96244586826123e-10 \\
196133	9.38463473598716e-10 \\
196513	8.84033790615746e-10 \\
196893	8.32760860269843e-10 \\
197273	7.84461551273097e-10 \\
197653	7.38963168434026e-10 \\
198033	6.96103119590674e-10 \\
198413	6.55728193965643e-10 \\
198793	6.17694118076884e-10 \\
199173	5.81865111648483e-10 \\
199553	5.48113332499156e-10 \\
199933	5.16317988363824e-10 \\
200313	4.86365725471671e-10 \\
200693	4.58149462811974e-10 \\
201073	4.31568669689852e-10 \\
201453	4.06528199992096e-10 \\
201833	3.82938958320977e-10 \\
202213	3.60716678748929e-10 \\
202593	3.39781869307387e-10 \\
202973	3.20060145053702e-10 \\
203353	3.01481173359264e-10 \\
203733	2.83978507376048e-10 \\
204113	2.67489641547769e-10 \\
204493	2.51956011609877e-10 \\
204873	2.37322106411142e-10 \\
205253	2.23535912002859e-10 \\
205633	2.10548134482735e-10 \\
206013	1.98312477550644e-10 \\
206393	1.86785475975171e-10 \\
206773	1.75925995993254e-10 \\
207153	1.6569517979903e-10 \\
207533	1.56056889633049e-10 \\
207913	1.46976653070396e-10 \\
208293	1.38421996087601e-10 \\
208673	1.30362776129544e-10 \\
209053	1.22769905352982e-10 \\
209433	1.15616627383019e-10 \\
209813	1.08877462601242e-10 \\
210193	1.02528263656865e-10 \\
210573	9.65466040447893e-11 \\
210953	9.0911111971792e-11 \\
211333	8.56016924011271e-11 \\
211713	8.05996380748297e-11 \\
212093	7.58870188910521e-11 \\
212473	7.14471259932736e-11 \\
212853	6.72640831922422e-11 \\
213233	6.33230134994278e-11 \\
213613	5.96099281047202e-11 \\
213993	5.61117263764288e-11 \\
214373	5.28159183055266e-11 \\
214753	4.97107355279525e-11 \\
215133	4.67851868357627e-11 \\
215513	4.40287806213746e-11 \\
215893	4.14317469221714e-11 \\
216273	3.89849819093513e-11 \\
216653	3.66798813544733e-11 \\
217033	3.45078965402479e-11 \\
217413	3.24615334612588e-11 \\
217793	3.05335201566947e-11 \\
218173	2.87170287549543e-11 \\
218553	2.70055644513434e-11 \\
218933	2.53930210192266e-11 \\
219313	2.38737363211783e-11 \\
219693	2.24422702643778e-11 \\
220073	2.10934603117607e-11 \\
220453	1.98228100600772e-11 \\
220833	1.8625434528019e-11 \\
221213	1.74974479349999e-11 \\
221593	1.64344093889213e-11 \\
221973	1.54330437318606e-11 \\
222353	1.44895206943829e-11 \\
222733	1.36005096074143e-11 \\
223113	1.27627908241834e-11 \\
223493	1.19735887871286e-11 \\
223873	1.12299058940835e-11 \\
224253	1.05292996543938e-11 \\
224633	9.86905002164917e-12 \\
225013	9.24710308325416e-12 \\
225393	8.66090532625208e-12 \\
225773	8.10873590495476e-12 \\
226153	7.58831886216171e-12 \\
226533	7.09809988563848e-12 \\
226913	6.63613608509195e-12 \\
227293	6.20087314828766e-12 \\
227673	5.79070125183989e-12 \\
228053	5.40434363927034e-12 \\
228433	5.04019048719329e-12 \\
228813	4.69707606143288e-12 \\
229193	4.37377911666204e-12 \\
229573	4.06930045215859e-12 \\
229953	3.78230780029298e-12 \\
230333	3.51191298264553e-12 \\
230713	3.25706128734282e-12 \\
231093	3.0169200471164e-12 \\
231473	2.79065659469779e-12 \\
231853	2.57743826281853e-12 \\
232233	2.37659891766384e-12 \\
232613	2.18741691426771e-12 \\
232993	2.00900407421045e-12 \\
233373	1.8410828417359e-12 \\
233753	1.68270952727312e-12 \\
234133	1.5334955527635e-12 \\
};
\addplot [line width=0.01pt, blue, forget plot]
table [row sep=\\]{%
1190	2.44207331041438 \\
2380	2.04601665592347 \\
3570	1.70213723957762 \\
4760	1.39946416992214 \\
5950	1.15029238572866 \\
7130	0.956421697256779 \\
8310	0.798588116856739 \\
9490	0.669297573716351 \\
10670	0.563473604538579 \\
11850	0.475288787495049 \\
13030	0.399408991350697 \\
14210	0.333099145607922 \\
15390	0.276649956993659 \\
16570	0.234170128112151 \\
17750	0.198189426896229 \\
18930	0.166773103475135 \\
20110	0.13907071032374 \\
21290	0.117444892595636 \\
22470	0.100720037145856 \\
23650	0.0855604442591086 \\
24830	0.0726296300566147 \\
26010	0.0632448449127107 \\
27190	0.0549206125057536 \\
28370	0.0475768326986557 \\
29550	0.0411567594378672 \\
30730	0.0353963098882126 \\
31910	0.0301454252061531 \\
33090	0.0253019940796324 \\
34270	0.0211166840408937 \\
35450	0.017771912373683 \\
36630	0.0147178322754385 \\
37810	0.0120347489087562 \\
38990	0.0101042564579638 \\
40170	0.00914185957458324 \\
41350	0.00825231716457597 \\
42530	0.00742957392937699 \\
43710	0.00666815096200002 \\
44890	0.00599783482645005 \\
46070	0.00540755400036735 \\
47250	0.00486190037879997 \\
48430	0.00435717871358599 \\
49610	0.00389004183322439 \\
50790	0.0034578965446822 \\
51970	0.00307282878090315 \\
53150	0.00273104818744879 \\
54330	0.0024245858585793 \\
55510	0.00214381641256961 \\
56690	0.00191497243123184 \\
57870	0.00171681737064511 \\
59050	0.0015329456580454 \\
60230	0.00136338583710505 \\
61410	0.00120598280092954 \\
62590	0.0010597355663175 \\
63770	0.000924394414449026 \\
64950	0.000800727712315208 \\
66130	0.000725426524554262 \\
67310	0.000657440639947326 \\
68490	0.000594876219623786 \\
69670	0.0005371605289336 \\
70850	0.000483891779326007 \\
72030	0.000434704205472147 \\
73210	0.000389264184536608 \\
74390	0.000355639591932855 \\
75570	0.000325879936927453 \\
76750	0.000298604665187829 \\
77930	0.000273556415516796 \\
79110	0.000250550819905104 \\
80290	0.000229393879937645 \\
81470	0.00020987454053234 \\
82650	0.000191842368288064 \\
83830	0.000175164475421929 \\
85010	0.000159940419013627 \\
86190	0.000147120303152559 \\
87370	0.000135374917563214 \\
88550	0.0001246028903511 \\
89730	0.000114714430074381 \\
90910	0.000105629315443501 \\
92090	9.72756654529228e-05 \\
93270	8.95889200014932e-05 \\
94450	8.25109692593196e-05 \\
95630	7.59988581819915e-05 \\
96810	7.00421312162525e-05 \\
97990	6.4565860783572e-05 \\
99170	5.95296244549903e-05 \\
100350	5.48966399340212e-05 \\
101530	5.06333686829397e-05 \\
102710	4.67092054163842e-05 \\
103890	4.30962029489002e-05 \\
105070	3.97688256739137e-05 \\
106250	3.67037281184812e-05 \\
107430	3.38795556670823e-05 \\
108610	3.12767649499612e-05 \\
109790	2.88774617251963e-05 \\
110970	2.66652543620682e-05 \\
112150	2.46251212695503e-05 \\
113330	2.27432908126435e-05 \\
114510	2.1007132433748e-05 \\
115690	1.94050578429805e-05 \\
116870	1.7926431269133e-05 \\
118050	1.65614878772091e-05 \\
119230	1.53012595541235e-05 \\
120410	1.41375073481331e-05 \\
121037	1.30626599277872e-05 \\
121427	1.20697574868545e-05 \\
121817	1.11524005846353e-05 \\
122207	1.03047034600845e-05 \\
122597	9.52125140712967e-06 \\
122987	8.79706183637419e-06 \\
123377	8.12754868928511e-06 \\
123767	7.50848989855557e-06 \\
124157	6.93599762302544e-06 \\
124547	6.40649100808188e-06 \\
124937	5.91667124621997e-06 \\
125327	5.46349873592478e-06 \\
125717	5.04417215402286e-06 \\
126107	4.65610927363747e-06 \\
126497	4.2969293779227e-06 \\
126887	3.96443712769035e-06 \\
127277	3.65660776319165e-06 \\
127667	3.3715735208717e-06 \\
128057	3.1076111636219e-06 \\
128447	2.86313053049492e-06 \\
128837	2.63666401856266e-06 \\
129227	2.42685691820288e-06 \\
129617	2.23421330530771e-06 \\
130007	2.05655694979212e-06 \\
130397	1.89195515132301e-06 \\
130787	1.7394277156968e-06 \\
131177	1.59807031968606e-06 \\
131567	1.46704814174559e-06 \\
131957	1.34559028197589e-06 \\
132347	1.23298468163169e-06 \\
132737	1.12857347772666e-06 \\
133127	1.03174875021317e-06 \\
133517	9.41948620603039e-07 \\
133907	8.58653672219933e-07 \\
134297	7.8138366027547e-07 \\
134687	7.09694487788237e-07 \\
135077	6.43175423253872e-07 \\
135467	5.81446539138497e-07 \\
135857	5.24156353376437e-07 \\
136247	4.70979656552739e-07 \\
136637	4.23380777436755e-07 \\
137027	3.92742877775021e-07 \\
137417	3.64572707467481e-07 \\
137807	3.38619749962188e-07 \\
138197	3.14685252689184e-07 \\
138587	2.92592542472381e-07 \\
138977	2.721828875063e-07 \\
139367	2.53313273412825e-07 \\
139757	2.35854576757788e-07 \\
140147	2.19689992864147e-07 \\
140537	2.04713675011625e-07 \\
140927	1.90829549717719e-07 \\
141317	1.77950281066153e-07 \\
141707	1.65996362599952e-07 \\
142097	1.54895318627002e-07 \\
142487	1.44580998562294e-07 \\
142877	1.34992952927071e-07 \\
143267	1.26075879069987e-07 \\
143657	1.17779128228079e-07 \\
144047	1.10056266933167e-07 \\
144437	1.02864683160231e-07 \\
144827	9.61652362185816e-08 \\
145217	8.99219405048335e-08 \\
145607	8.41016838948327e-08 \\
145997	7.86739717262286e-08 \\
146387	7.36106986476415e-08 \\
146777	6.88859407849129e-08 \\
147167	6.44757695567044e-08 \\
147557	6.03580818658855e-08 \\
147947	5.65124473883571e-08 \\
148337	5.2919969351084e-08 \\
148727	4.95631588548484e-08 \\
149117	4.64258198551448e-08 \\
149507	4.34929446346821e-08 \\
149897	4.07506194899376e-08 \\
150287	3.81859376341609e-08 \\
150677	3.57869202605166e-08 \\
151067	3.3542444433099e-08 \\
151457	3.14421765290618e-08 \\
151847	2.94765126196417e-08 \\
152237	2.7636522237362e-08 \\
152627	2.59138984159968e-08 \\
153017	2.43009103395586e-08 \\
153407	2.27903609317792e-08 \\
153797	2.13755477207478e-08 \\
154187	2.00502262015512e-08 \\
154577	1.88085767516277e-08 \\
154967	1.76451739886119e-08 \\
155357	1.65549582931135e-08 \\
155747	1.55332101070549e-08 \\
156137	1.45755250091639e-08 \\
156527	1.367779278727e-08 \\
156917	1.28361756224216e-08 \\
157307	1.2047089659184e-08 \\
157697	1.13071872420711e-08 \\
158087	1.06133407062892e-08 \\
158477	9.96262727870345e-09 \\
158867	9.35231475596154e-09 \\
159257	8.77984901448414e-09 \\
159647	8.24284185352298e-09 \\
160037	7.73905933781904e-09 \\
160427	7.26641208315115e-09 \\
160817	6.82294493126179e-09 \\
161207	6.40682862318442e-09 \\
161597	6.01635108399279e-09 \\
161987	5.64990959572853e-09 \\
162377	5.30600391401848e-09 \\
162767	4.98322916264726e-09 \\
163157	4.68026978284186e-09 \\
163547	4.395893815623e-09 \\
163937	4.12894712864542e-09 \\
164327	3.87834836468315e-09 \\
164717	3.64308494482657e-09 \\
165107	3.42220757287848e-09 \\
165497	3.21482668264039e-09 \\
165887	3.02010894071003e-09 \\
166277	2.8372725835446e-09 \\
166667	2.66558564110397e-09 \\
167057	2.5043611073805e-09 \\
167447	2.35295538608682e-09 \\
167837	2.21076479345328e-09 \\
168227	2.07722333778193e-09 \\
168617	1.9518001104224e-09 \\
169007	1.83399712083698e-09 \\
169397	1.72334774228844e-09 \\
169787	1.61941371423779e-09 \\
170177	1.52178419865479e-09 \\
170567	1.43007355957181e-09 \\
170957	1.34392030837205e-09 \\
171347	1.26298499436572e-09 \\
171737	1.18694931661167e-09 \\
172127	1.11551445858282e-09 \\
172517	1.048399922432e-09 \\
172907	9.85342751835816e-10 \\
173297	9.26095700126695e-10 \\
173687	8.70427285803999e-10 \\
174077	8.18119683110297e-10 \\
174467	7.68968777542511e-10 \\
174857	7.22782722561988e-10 \\
175247	6.79381773061039e-10 \\
175637	6.38596620028409e-10 \\
176027	6.00269001171938e-10 \\
176417	5.64249813539419e-10 \\
176807	5.30398946985144e-10 \\
177197	4.98585561725662e-10 \\
177587	4.68686034427179e-10 \\
177977	4.40584568828228e-10 \\
178367	4.14172640628152e-10 \\
178757	3.89347942775231e-10 \\
179147	3.66014496488987e-10 \\
179537	3.44082484726727e-10 \\
179927	3.23467197471672e-10 \\
180317	3.04089198266411e-10 \\
180707	2.85873769101386e-10 \\
181097	2.68750854903743e-10 \\
181487	2.52654674959274e-10 \\
181877	2.37523223312053e-10 \\
182267	2.23298601831345e-10 \\
182657	2.09926076522038e-10 \\
183047	1.97354355080392e-10 \\
183437	1.85535198315989e-10 \\
183827	1.74423420151726e-10 \\
184217	1.63976610068062e-10 \\
184607	1.54154689013808e-10 \\
184997	1.44919964917278e-10 \\
185387	1.36237465753197e-10 \\
185777	1.28073884830826e-10 \\
186167	1.20398024883173e-10 \\
186557	1.13180742555841e-10 \\
186947	1.06394448806668e-10 \\
187337	1.0001321992803e-10 \\
187727	9.40129640802922e-11 \\
188117	8.8370755157996e-11 \\
188507	8.30651103456148e-11 \\
188897	7.80758235841006e-11 \\
189287	7.33841876154884e-11 \\
189677	6.8972161315628e-11 \\
190067	6.48230358279989e-11 \\
190457	6.09209904745001e-11 \\
190847	5.72516478669627e-11 \\
191237	5.38007416395203e-11 \\
191627	5.05552266716336e-11 \\
192017	4.75029460211829e-11 \\
192407	4.46322978575608e-11 \\
192797	4.19324575062774e-11 \\
193187	3.93933219378084e-11 \\
193577	3.7005065678386e-11 \\
193967	3.47587514326619e-11 \\
194357	3.26461080391027e-11 \\
194747	3.06589198473262e-11 \\
195137	2.87899148965209e-11 \\
195527	2.70317657147245e-11 \\
195917	2.53781440306966e-11 \\
196307	2.38227215731968e-11 \\
196697	2.23595586490433e-11 \\
197087	2.09833816988692e-11 \\
197477	1.96886396075513e-11 \\
197867	1.84709469941424e-11 \\
198257	1.7325363366183e-11 \\
198647	1.62477253873305e-11 \\
199037	1.52339807435453e-11 \\
199427	1.42802991653923e-11 \\
199817	1.33833499837976e-11 \\
200207	1.25393584404776e-11 \\
200597	1.17454379555681e-11 \\
200987	1.09983688822979e-11 \\
201377	1.02957087300126e-11 \\
201767	9.63462642999957e-12 \\
202157	9.01273500275579e-12 \\
202547	8.4275919576271e-12 \\
202937	7.87703235971549e-12 \\
203327	7.35916882987908e-12 \\
203717	6.87183643321987e-12 \\
204107	6.41336983520091e-12 \\
204497	5.98193716783157e-12 \\
204887	5.57609514117985e-12 \\
205277	5.19412290955756e-12 \\
205667	4.83479922763763e-12 \\
206057	4.4967918277905e-12 \\
206447	4.1787129312354e-12 \\
206837	3.87934129264522e-12 \\
207227	3.59767771129782e-12 \\
207617	3.33266747531979e-12 \\
208007	3.08331138398898e-12 \\
208397	2.84866574773446e-12 \\
208787	2.62778687698528e-12 \\
209177	2.42006414907792e-12 \\
209567	2.22449836329019e-12 \\
209957	2.04058991926104e-12 \\
210347	1.86745063857074e-12 \\
210737	1.70446989855577e-12 \\
211127	1.55120361000627e-12 \\
211517	1.40693012795623e-12 \\
211907	1.27109434089334e-12 \\
212297	1.14336318191022e-12 \\
212687	1.02312602834331e-12 \\
213077	9.09883279831547e-13 \\
213467	8.03412891769995e-13 \\
213857	7.03270774948805e-13 \\
214247	6.08901817855667e-13 \\
214637	5.20083975885655e-13 \\
215027	4.36595204433843e-13 \\
215417	3.57880391987919e-13 \\
215807	2.83828516245421e-13 \\
216197	2.14273043752655e-13 \\
216587	1.4871437414854e-13 \\
216977	8.6985973979381e-14 \\
217367	2.89213097914853e-14 \\
217757	-2.57016630200724e-14 \\
218147	-7.72160113626796e-14 \\
218537	-1.25566224085105e-13 \\
218927	-1.71196390397199e-13 \\
219317	-2.14106510298961e-13 \\
219707	-2.54463117244086e-13 \\
220097	-2.92432744686266e-13 \\
220487	-3.28181926079196e-13 \\
220877	-3.6187719487657e-13 \\
221267	-3.93629573380849e-13 \\
221657	-4.23439061592035e-13 \\
222047	-4.5147219296382e-13 \\
222437	-4.7795101210113e-13 \\
222827	-5.02820007852733e-13 \\
223217	-5.26245713672324e-13 \\
223607	-5.48394663013596e-13 \\
223997	-5.69100322422855e-13 \\
224387	-5.88640247656258e-13 \\
224777	-6.07014438713804e-13 \\
225167	-6.24333917897957e-13 \\
225557	-6.40598685208715e-13 \\
225947	-6.55975274099774e-13 \\
226337	-6.70463684571132e-13 \\
226727	-6.84119427774021e-13 \\
227117	-6.9683148140598e-13 \\
227507	-7.08821890071931e-13 \\
227897	-7.20312698376802e-13 \\
228287	-7.30804305959509e-13 \\
228677	-7.40962846634829e-13 \\
229067	-7.50455253495375e-13 \\
229457	-7.59337037692376e-13 \\
229847	-7.67719221528296e-13 \\
230237	-7.75601805003134e-13 \\
230627	-7.83095810419354e-13 \\
231017	-7.90034704323261e-13 \\
231407	-7.96640531319781e-13 \\
231797	-8.02802269106451e-13 \\
232187	-8.08686451136964e-13 \\
232577	-8.14182055108859e-13 \\
232967	-8.19344592173366e-13 \\
233357	-8.24285084632947e-13 \\
233747	-8.28892510185142e-13 \\
234137	-8.33055846527486e-13 \\
234527	-8.37219182869831e-13 \\
234917	-8.41049452304787e-13 \\
235307	-8.4471318828605e-13 \\
235697	-8.48154879662388e-13 \\
236087	-8.51207992980108e-13 \\
236477	-8.54205595146595e-13 \\
236867	-8.5703666385939e-13 \\
};
\addplot [line width=0.01pt, blue, forget plot]
table [row sep=\\]{%
1190	2.72471009819269 \\
2380	2.21382820459037 \\
3570	1.79158926013173 \\
4760	1.45297461546301 \\
5950	1.18461720212599 \\
7140	0.964876712108072 \\
8309	0.798266670843059 \\
9469	0.664332970919718 \\
10629	0.554608674488876 \\
11789	0.466291166043895 \\
12949	0.395373921087407 \\
14109	0.332199114681165 \\
15269	0.277110059206315 \\
16429	0.23409371164834 \\
17589	0.199437014873896 \\
18749	0.169390416967308 \\
19909	0.142573934349464 \\
21069	0.124272978994612 \\
22229	0.109038260346101 \\
23389	0.0960581351202176 \\
24549	0.0839115911614433 \\
25709	0.073271704157895 \\
26869	0.0642561697731538 \\
28029	0.0567703964711723 \\
29189	0.0502674623943992 \\
30349	0.0443862901363095 \\
31509	0.0388859802579415 \\
32669	0.033914230012435 \\
33829	0.0297413228712088 \\
34989	0.0265628377405959 \\
36149	0.0236169512586593 \\
37309	0.0211584434091872 \\
38469	0.019262819807238 \\
39629	0.0175078921175373 \\
40789	0.01584561859352 \\
41949	0.0144337023396481 \\
43109	0.0133056376483541 \\
44269	0.0122630619268041 \\
45429	0.0113133603687326 \\
46589	0.010425670162349 \\
47749	0.00961909839625402 \\
48909	0.00888118271570604 \\
50069	0.0081914852787815 \\
51229	0.00754917032455804 \\
52389	0.00695231553620324 \\
53549	0.00639989884020858 \\
54709	0.00588713524660422 \\
55869	0.00540957833694644 \\
57029	0.00496300965951396 \\
58189	0.00454783109395684 \\
59349	0.00415967154113528 \\
60509	0.00379506901336019 \\
61669	0.00345579660630918 \\
62829	0.00314936882506911 \\
63989	0.00287360647931334 \\
65149	0.0026183052706763 \\
66309	0.002378508202803 \\
67469	0.00215428029679077 \\
68629	0.00194531602216386 \\
69789	0.00176555427782049 \\
70949	0.00162809287615195 \\
72109	0.0014997545421479 \\
73269	0.00137964618675113 \\
74429	0.00126708493348615 \\
75589	0.0011628079935207 \\
76749	0.00106528077049078 \\
77909	0.000974042980238288 \\
79069	0.000890069810937144 \\
80229	0.00081885688122979 \\
81389	0.000751796314729491 \\
82549	0.000690978595040859 \\
83709	0.000634842075718345 \\
84869	0.000582464982357145 \\
86029	0.000534328935412109 \\
87189	0.000489109263915044 \\
88349	0.00044661046705996 \\
89509	0.000407528011037717 \\
90669	0.000370019515237241 \\
91829	0.000336122789572124 \\
92989	0.000303999217873496 \\
94149	0.000273937221570986 \\
95309	0.000246099791348431 \\
96469	0.000219912425278967 \\
97629	0.000198925919854875 \\
98789	0.00017972834164115 \\
99949	0.00016370175730307 \\
101109	0.000151847178987419 \\
102269	0.000140770807407653 \\
103429	0.000130759305495165 \\
104589	0.000121141009794323 \\
105749	0.000112450351736015 \\
106909	0.000104281088418201 \\
108069	9.66953835550011e-05 \\
109229	8.96304631192169e-05 \\
110389	8.31427637270621e-05 \\
111549	7.69907967072547e-05 \\
112709	7.15766321042199e-05 \\
113869	6.63715614949711e-05 \\
115029	6.15092010571305e-05 \\
116189	5.6942604839294e-05 \\
117349	5.26728756530814e-05 \\
118509	4.87683443042841e-05 \\
119669	4.51998113316643e-05 \\
120829	4.1824476862018e-05 \\
121989	3.8732271144315e-05 \\
123149	3.57443856431328e-05 \\
124309	3.30011319966683e-05 \\
125469	3.04552198706065e-05 \\
126629	2.81328378033407e-05 \\
127789	2.58794871356716e-05 \\
128949	2.38654375428471e-05 \\
130109	2.19434145286579e-05 \\
131269	2.01265975723897e-05 \\
132429	1.84703455933799e-05 \\
133589	1.68831019091709e-05 \\
134749	1.54520340757136e-05 \\
135909	1.409154828097e-05 \\
137069	1.28353812047233e-05 \\
138229	1.16484030553976e-05 \\
139389	1.0579957678003e-05 \\
140549	9.57647370819492e-06 \\
141709	8.71213549996641e-06 \\
142869	8.03850148944907e-06 \\
144029	7.42504434964042e-06 \\
145189	6.85650806570015e-06 \\
146349	6.33391534959893e-06 \\
147509	5.82090589179574e-06 \\
148669	5.36620164176549e-06 \\
149829	4.94861178534922e-06 \\
150989	4.5696480048818e-06 \\
152149	4.22513011788528e-06 \\
153309	3.88210113627663e-06 \\
154469	3.57778184589819e-06 \\
155629	3.31106693357563e-06 \\
156789	3.04117640970158e-06 \\
157949	2.80323042262109e-06 \\
159109	2.58222293431087e-06 \\
160269	2.3678420071338e-06 \\
161429	2.17382853118231e-06 \\
162589	1.99671300143978e-06 \\
163749	1.83956189830914e-06 \\
164909	1.68708205694479e-06 \\
166069	1.54362483922865e-06 \\
167229	1.41376097334556e-06 \\
168389	1.29574660789888e-06 \\
169549	1.18320622527479e-06 \\
170709	1.08231732021125e-06 \\
171869	9.87250439010712e-07 \\
173029	9.00615276855365e-07 \\
174189	8.18157773696537e-07 \\
175349	7.49434267754978e-07 \\
176509	6.92684618219808e-07 \\
177669	6.43454676041433e-07 \\
178829	5.99184203031733e-07 \\
179989	5.51149218708247e-07 \\
181149	5.09235841494871e-07 \\
182309	4.71746620156477e-07 \\
183469	4.37345059667305e-07 \\
184629	4.05440491790809e-07 \\
185789	3.73786049723268e-07 \\
186949	3.46940306150678e-07 \\
188109	3.1965717200988e-07 \\
189269	2.97917359448618e-07 \\
190429	2.77055658026715e-07 \\
191589	2.57777819367977e-07 \\
192749	2.38247747641473e-07 \\
193909	2.21377505882359e-07 \\
195069	2.0591772192935e-07 \\
196229	1.9104727611019e-07 \\
197389	1.77554418223647e-07 \\
198549	1.64890307907672e-07 \\
199709	1.53951101367067e-07 \\
200869	1.4316338564635e-07 \\
202029	1.33245356304101e-07 \\
203189	1.23892555936589e-07 \\
204349	1.14785032201414e-07 \\
205509	1.07574978147618e-07 \\
206669	1.00253110424386e-07 \\
207829	9.34798249363489e-08 \\
208989	8.72103821381032e-08 \\
210149	8.08341519831224e-08 \\
211309	7.51827148870454e-08 \\
212469	7.02374758354551e-08 \\
213629	6.51949011998276e-08 \\
214789	6.07609614289295e-08 \\
215949	5.6755922073215e-08 \\
217109	5.29180526620543e-08 \\
218269	4.93937513112996e-08 \\
219429	4.64070422090401e-08 \\
220589	4.32439625774883e-08 \\
221749	4.03414245897515e-08 \\
222909	3.75655909157224e-08 \\
224069	3.51903040596291e-08 \\
225229	3.28617609235415e-08 \\
226389	3.05343179074491e-08 \\
227549	2.8663631135295e-08 \\
228709	2.67916670515511e-08 \\
229869	2.51137434181814e-08 \\
231029	2.33297610674477e-08 \\
232189	2.19127234135641e-08 \\
233349	2.04677662041242e-08 \\
234509	1.9103054027525e-08 \\
235669	1.78875425027591e-08 \\
236829	1.67024423736173e-08 \\
237989	1.55490692543836e-08 \\
239149	1.44798671941615e-08 \\
240309	1.34882200963915e-08 \\
241469	1.25862859579051e-08 \\
242629	1.18372497914798e-08 \\
243789	1.10624653881786e-08 \\
244949	1.03782842875155e-08 \\
246109	9.71486013856548e-09 \\
247269	9.05912667104047e-09 \\
248429	8.44992920079335e-09 \\
249589	7.8980581519339e-09 \\
250749	7.38156419322067e-09 \\
251909	6.89570206491297e-09 \\
253069	6.48052062013349e-09 \\
254229	6.0591802131249e-09 \\
255389	5.64316193774062e-09 \\
256549	5.28322502413303e-09 \\
257709	4.94141166873519e-09 \\
258869	4.63558647023632e-09 \\
260029	4.34085167810849e-09 \\
261189	4.06506495131964e-09 \\
262349	3.79572245856608e-09 \\
263509	3.53489060245238e-09 \\
264669	3.31191574165501e-09 \\
265829	3.09876618809213e-09 \\
266989	2.90811713599126e-09 \\
268149	2.7222290532869e-09 \\
269309	2.55444060348609e-09 \\
270469	2.40102993043223e-09 \\
271629	2.25465524028579e-09 \\
272789	2.11461076427e-09 \\
273949	1.98210625690365e-09 \\
275109	1.85120185847154e-09 \\
276269	1.73567937800101e-09 \\
277429	1.62643853940381e-09 \\
278589	1.52447693357871e-09 \\
279749	1.43191736245996e-09 \\
280909	1.34337813095797e-09 \\
282069	1.25993138144764e-09 \\
283229	1.17757759099391e-09 \\
284389	1.09990794250336e-09 \\
285549	1.03026365216863e-09 \\
286709	9.65799551444491e-10 \\
287869	9.0763591087395e-10 \\
289029	8.4869372640739e-10 \\
290189	7.95556120980478e-10 \\
291349	7.47351680541186e-10 \\
292509	6.99542090920602e-10 \\
293669	6.54447773751343e-10 \\
294829	6.14718886904342e-10 \\
295989	5.76309500122107e-10 \\
297149	5.41429179268249e-10 \\
298309	5.07750286260489e-10 \\
299469	4.76181205577575e-10 \\
300629	4.47379133738934e-10 \\
301789	4.19932200124151e-10 \\
302949	3.92656795966673e-10 \\
304109	3.69299146818491e-10 \\
305269	3.46425998998257e-10 \\
306429	3.25031668246822e-10 \\
307589	3.04306579934632e-10 \\
308749	2.85208578976182e-10 \\
309909	2.67155630950811e-10 \\
311069	2.50960974224057e-10 \\
312229	2.36240194073645e-10 \\
313389	2.20901463787726e-10 \\
314549	2.07152905939978e-10 \\
315709	1.94023519473063e-10 \\
316869	1.81891501860321e-10 \\
318029	1.70524427911545e-10 \\
319189	1.59811497368878e-10 \\
320349	1.49611711908193e-10 \\
321509	1.40722877794985e-10 \\
322669	1.31954336346496e-10 \\
323829	1.23767052162549e-10 \\
324989	1.15950138379617e-10 \\
326149	1.08722475467005e-10 \\
327309	1.01884500836036e-10 \\
328469	9.55628909338202e-11 \\
329629	8.97359964113775e-11 \\
330789	8.41456349043312e-11 \\
331949	7.89435183889964e-11 \\
333109	7.38955008294795e-11 \\
334269	6.93791135653044e-11 \\
335429	6.5027816464891e-11 \\
336589	6.09665096185097e-11 \\
337749	5.71850344854852e-11 \\
338909	5.3617998929667e-11 \\
340069	5.02961006176861e-11 \\
341229	4.72180072819128e-11 \\
342389	4.43672321104316e-11 \\
343549	4.15587009250373e-11 \\
344709	3.90239507375156e-11 \\
345869	3.65739660779241e-11 \\
347029	3.43122752433089e-11 \\
348189	3.22000759389596e-11 \\
349349	3.01564329063808e-11 \\
350509	2.8291369247313e-11 \\
351669	2.64737676225479e-11 \\
352829	2.47821207999266e-11 \\
353989	2.3203716725817e-11 \\
355149	2.17170170735415e-11 \\
356309	2.03422279021481e-11 \\
357469	1.90556459500613e-11 \\
358629	1.78650427784532e-11 \\
359789	1.67141300799756e-11 \\
360949	1.56379909022064e-11 \\
362109	1.46363476893896e-11 \\
363269	1.36901601166528e-11 \\
364429	1.28008714739281e-11 \\
365589	1.19718679414405e-11 \\
366749	1.11931575119684e-11 \\
367909	1.04536379552655e-11 \\
369069	9.76257963358762e-12 \\
370229	9.12803166386311e-12 \\
371389	8.52717896293598e-12 \\
372549	7.96385180024117e-12 \\
373709	7.42833572431323e-12 \\
374869	6.92529367185557e-12 \\
376029	6.45689057776622e-12 \\
377189	6.01318994597477e-12 \\
378349	5.59335910921277e-12 \\
379509	5.20705700779445e-12 \\
380669	4.83346696000808e-12 \\
381829	4.48119319429452e-12 \\
382989	4.15301126821532e-12 \\
384149	3.84797749219956e-12 \\
385309	3.56564777703738e-12 \\
386469	3.29763993889287e-12 \\
387629	3.04117842020446e-12 \\
388789	2.80503398286669e-12 \\
389949	2.57988075347271e-12 \\
391109	2.37043717987717e-12 \\
392269	2.17342810415744e-12 \\
393429	1.98646654681056e-12 \\
394589	1.81166193158333e-12 \\
395749	1.64707136818265e-12 \\
396909	1.49402712423807e-12 \\
398069	1.35008670909542e-12 \\
399229	1.21586074541824e-12 \\
400389	1.0886291867962e-12 \\
401549	9.6739283250713e-13 \\
402709	8.55093773566296e-13 \\
403869	7.50122186587987e-13 \\
405029	6.50923759337729e-13 \\
406189	5.57498491815522e-13 \\
407349	4.6956882826521e-13 \\
408509	3.86857212930636e-13 \\
409669	3.09141601206875e-13 \\
410829	2.35922392732846e-13 \\
411989	1.67310609811011e-13 \\
413149	1.02862163231521e-13 \\
414309	4.25770529943748e-14 \\
415469	-1.39332989590457e-14 \\
416629	-6.70574706873595e-14 \\
417789	-1.1740608485411e-13 \\
418949	-1.64590563400679e-13 \\
420109	-2.08888462083223e-13 \\
421269	-2.5068835896036e-13 \\
422429	-2.89601675973472e-13 \\
423589	-3.26405569239796e-13 \\
424749	-3.60766971851945e-13 \\
425909	-3.93296506473462e-13 \\
427069	-4.23772128499422e-13 \\
428229	-4.5236037138352e-13 \\
429389	-4.79227768579449e-13 \\
430549	-5.04540853540902e-13 \\
431709	-5.28355137419112e-13 \\
432869	-5.50726131365309e-13 \\
434029	-5.71487301925799e-13 \\
435189	-5.9136029406659e-13 \\
436349	-6.10012040880292e-13 \\
437509	-6.27498053518138e-13 \\
438669	-6.43984865433822e-13 \\
439829	-6.59472476627343e-13 \\
440989	-6.73960887098701e-13 \\
442149	-6.87561119150359e-13 \\
443309	-7.00495217387243e-13 \\
444469	-7.12596648355657e-13 \\
445629	-7.23976434358065e-13 \\
446789	-7.34690086545697e-13 \\
447949	-7.44682093767324e-13 \\
449109	-7.54007967174175e-13 \\
450269	-7.63000773673639e-13 \\
451429	-7.71327446358328e-13 \\
452589	-7.79099007530704e-13 \\
453749	-7.86537501795692e-13 \\
454909	-7.93531906850831e-13 \\
456069	-8.00026711544888e-13 \\
457229	-8.06188449331557e-13 \\
458389	-8.11906097908377e-13 \\
459075	-8.17290679577809e-13 \\
459445	-8.22397705491085e-13 \\
459815	-8.27282686799435e-13 \\
460185	-8.31723578897936e-13 \\
460555	-8.35997937542743e-13 \\
};
\addplot [line width=0.01pt, blue, forget plot]
table [row sep=\\]{%
1190	2.7361417616515 \\
2380	2.22113566116038 \\
3570	1.81928612832563 \\
4760	1.48911668555392 \\
5950	1.22446517955533 \\
7140	1.01777113586391 \\
8326	0.856079103738896 \\
9506	0.725984919965438 \\
10686	0.616352405818881 \\
11866	0.520820996845611 \\
13046	0.435120329333045 \\
14226	0.358724612595396 \\
15406	0.297044825786193 \\
16586	0.254111777075834 \\
17766	0.217718667870862 \\
18946	0.186173104307672 \\
20126	0.159694447613047 \\
21306	0.138778082860684 \\
22486	0.1218681496038 \\
23666	0.10668415917513 \\
24846	0.0925819290924576 \\
26026	0.0796405454314048 \\
27206	0.0691880143670311 \\
28386	0.060677223035069 \\
29566	0.053545186631443 \\
30746	0.0470665217651362 \\
31926	0.0409999437713253 \\
33106	0.0353691181033437 \\
34286	0.0304777984091411 \\
35466	0.0265709900515295 \\
36646	0.0229564351035198 \\
37826	0.019594261414905 \\
39006	0.016597715374591 \\
40186	0.0145137677518952 \\
41366	0.0127528607418139 \\
42546	0.0111209765015173 \\
43726	0.00960826092859729 \\
44906	0.00833791123431449 \\
46086	0.0074087314055909 \\
47266	0.00654973418530941 \\
48446	0.00575454992150126 \\
49626	0.00505327713275044 \\
50806	0.00445234643604409 \\
51986	0.00389499454488296 \\
53166	0.00338025561165883 \\
54346	0.00290428971861573 \\
55526	0.00246535854284047 \\
56706	0.00205562910078738 \\
57886	0.00167779775074828 \\
59066	0.00133202168394581 \\
60246	0.00102251002529363 \\
61426	0.000785973886555325 \\
62606	0.00061418663215268 \\
63786	0.000551030866014879 \\
64966	0.000511664484912966 \\
66146	0.000475115886151012 \\
67326	0.000441364945343192 \\
68506	0.000410602325448317 \\
69686	0.000382338212859823 \\
70866	0.00035586167556273 \\
72046	0.000331420845575803 \\
73226	0.000308993829453652 \\
74406	0.000288056772283385 \\
75586	0.000268679213705492 \\
76766	0.000250431604761459 \\
77946	0.000233556855960659 \\
79126	0.000217958718403077 \\
80306	0.000203297342637188 \\
81486	0.000189797158376748 \\
82666	0.000177208685771812 \\
83846	0.000165410793482257 \\
85026	0.000154474743353172 \\
86206	0.000144211022104768 \\
87386	0.000134585750613037 \\
88566	0.000125585826456054 \\
89746	0.000117262481120872 \\
90926	0.000109483949395195 \\
92106	0.000102202974040233 \\
93286	9.53619394410965e-05 \\
94466	8.90153738729582e-05 \\
95646	8.30699844129312e-05 \\
96826	7.75125262758025e-05 \\
98006	7.23444073071899e-05 \\
99186	6.74823318287943e-05 \\
100366	6.29452397111208e-05 \\
101546	5.86948757261285e-05 \\
102726	5.47179084974259e-05 \\
103906	5.10090646861361e-05 \\
105086	4.75267031615489e-05 \\
106266	4.42715578402031e-05 \\
107446	4.12203371766284e-05 \\
108626	3.83623351814522e-05 \\
109806	3.56922337957921e-05 \\
110986	3.33605976957085e-05 \\
112166	3.12224898992852e-05 \\
113346	2.92326578947288e-05 \\
114526	2.73680727072012e-05 \\
115706	2.56262191014045e-05 \\
116886	2.39997155976335e-05 \\
118066	2.24780526947854e-05 \\
119246	2.1054725545866e-05 \\
120426	1.97246960036979e-05 \\
121606	1.8479117117276e-05 \\
122706	1.73140374680747e-05 \\
123086	1.62231915232236e-05 \\
123466	1.52022814124408e-05 \\
123846	1.4254582154094e-05 \\
124226	1.33742506580603e-05 \\
124606	1.25503194089438e-05 \\
124986	1.17789444816774e-05 \\
125366	1.1056577691515e-05 \\
125746	1.03799338962984e-05 \\
126126	9.74596807828965e-06 \\
126506	9.15185521732642e-06 \\
126886	8.59497223942851e-06 \\
127266	8.0728817447584e-06 \\
127646	7.58331729899808e-06 \\
128026	7.12417009723909e-06 \\
128406	6.69347684306709e-06 \\
128786	6.28940870794903e-06 \\
129166	5.91026125168481e-06 \\
129546	5.55444520594639e-06 \\
129926	5.2204780298104e-06 \\
130306	4.9069761640097e-06 \\
130686	4.61264791190574e-06 \\
131066	4.33628689194832e-06 \\
131446	4.07676600666651e-06 \\
131826	3.83303188106199e-06 \\
132206	3.60409972965936e-06 \\
132586	3.38904861468814e-06 \\
132966	3.18701706070179e-06 \\
133346	2.99719899699014e-06 \\
133726	2.81883999886379e-06 \\
134106	2.65123380516208e-06 \\
134486	2.49371908739304e-06 \\
134866	2.34567645202022e-06 \\
135246	2.20652565724455e-06 \\
135626	2.07572302635128e-06 \\
136006	1.95275904313341e-06 \\
136386	1.83715611640212e-06 \\
136766	1.7284664988737e-06 \\
137146	1.62627034916429e-06 \\
137526	1.53017392884314e-06 \\
137906	1.43980792005616e-06 \\
138286	1.35482585883473e-06 \\
138666	1.27490267337604e-06 \\
139046	1.19973332057821e-06 \\
139426	1.12903151344712e-06 \\
139806	1.06252853421251e-06 \\
140186	9.99972123660875e-07 \\
140566	9.4112544618552e-07 \\
140946	8.85766120728437e-07 \\
141326	8.33685316115051e-07 \\
141706	7.84686904398146e-07 \\
142086	7.38586670212538e-07 \\
142466	6.9521156997876e-07 \\
142846	6.54399039123899e-07 \\
143226	6.1599634443299e-07 \\
143606	5.79859976201913e-07 \\
143986	5.45855080635871e-07 \\
144366	5.13854927775004e-07 \\
144746	4.8374041239363e-07 \\
145126	4.55399587373506e-07 \\
145506	4.28727226109427e-07 \\
145886	4.03624411782211e-07 \\
146266	3.79998153332561e-07 \\
146646	3.57761024694092e-07 \\
147026	3.36830826952461e-07 \\
147406	3.1713027054403e-07 \\
147786	2.98586679214896e-07 \\
148166	2.81131709412019e-07 \\
148546	2.64701089602859e-07 \\
148926	2.49234374138929e-07 \\
149306	2.34674713051053e-07 \\
149686	2.20968635000762e-07 \\
150066	2.08065844831129e-07 \\
150446	1.95919032497383e-07 \\
150826	1.8448369371038e-07 \\
151206	1.73717962570485e-07 \\
151586	1.63582452861188e-07 \\
151966	1.54040109945353e-07 \\
152346	1.4505607165427e-07 \\
152726	1.36597537225835e-07 \\
153106	1.28633644014275e-07 \\
153486	1.2113535263758e-07 \\
153866	1.1407533789809e-07 \\
154246	1.07427887086065e-07 \\
154626	1.01168803556817e-07 \\
155006	9.52753173022458e-08 \\
155386	8.9726000018775e-08 \\
155766	8.45006850602736e-08 \\
156146	7.95803930531136e-08 \\
156526	7.49472619521185e-08 \\
156906	7.05844798720712e-08 \\
157286	6.6476223359313e-08 \\
157666	6.26075993270803e-08 \\
158046	5.89645893223079e-08 \\
158426	5.55339987329262e-08 \\
158806	5.23034071053807e-08 \\
159186	4.92611234026441e-08 \\
159566	4.63961422059178e-08 \\
159946	4.36981035245587e-08 \\
160326	4.11572552705408e-08 \\
160706	3.87644166766066e-08 \\
161086	3.65109446565093e-08 \\
161466	3.43887033849022e-08 \\
161846	3.23900324894488e-08 \\
162226	3.05077207940485e-08 \\
162606	2.8734979062861e-08 \\
162986	2.70654152423333e-08 \\
163366	2.5493011257538e-08 \\
163746	2.40121013628247e-08 \\
164126	2.26173504924709e-08 \\
164506	2.13037361085355e-08 \\
164886	2.00665289940005e-08 \\
165266	1.89012762663587e-08 \\
165646	1.780378527938e-08 \\
166026	1.67701080799887e-08 \\
166406	1.57965276970096e-08 \\
166786	1.48795437637794e-08 \\
167166	1.40158603612051e-08 \\
167546	1.32023742493992e-08 \\
167926	1.24361628772718e-08 \\
168306	1.17144742239894e-08 \\
168686	1.10347172510572e-08 \\
169066	1.03944515217336e-08 \\
169446	9.79137926293561e-09 \\
169826	9.22333648345486e-09 \\
170206	8.68828531341848e-09 \\
170586	8.184306954373e-09 \\
170966	7.70959396323434e-09 \\
171346	7.2624446456615e-09 \\
171726	6.84125628369614e-09 \\
172106	6.44451930709167e-09 \\
172486	6.07081240833196e-09 \\
172866	5.71879704702738e-09 \\
173246	5.3872121763554e-09 \\
173626	5.07487057932465e-09 \\
174006	4.78065420583818e-09 \\
174386	4.50350967629021e-09 \\
174766	4.24244539498631e-09 \\
175146	3.99652683169549e-09 \\
175526	3.76487468978226e-09 \\
175906	3.54666007673643e-09 \\
176286	3.34110245026054e-09 \\
176666	3.14746723129034e-09 \\
177046	2.96506158514731e-09 \\
177426	2.79323397744946e-09 \\
177806	2.63137017730841e-09 \\
178186	2.47889170301718e-09 \\
178566	2.33525365711529e-09 \\
178946	2.1999432830988e-09 \\
179326	2.0724768567959e-09 \\
179706	1.95239935329994e-09 \\
180086	1.83928205998996e-09 \\
180466	1.73272113324074e-09 \\
180846	1.63233604411062e-09 \\
181226	1.53776869016298e-09 \\
181606	1.44868139706489e-09 \\
181986	1.36475636347555e-09 \\
182366	1.28569432877867e-09 \\
182746	1.21121290774795e-09 \\
183126	1.14104670156934e-09 \\
183506	1.07494535495078e-09 \\
183886	1.01267327856647e-09 \\
184266	9.5400848332261e-10 \\
184646	8.98741636667921e-10 \\
185026	8.46675896060134e-10 \\
185406	7.97625632209531e-10 \\
185786	7.51415984989734e-10 \\
186166	7.07882308326191e-10 \\
186546	6.66869670595815e-10 \\
186926	6.28231799915113e-10 \\
187306	5.91831084140182e-10 \\
187686	5.5753807126635e-10 \\
188066	5.25230359205153e-10 \\
188446	4.94793039873542e-10 \\
188826	4.66117811015465e-10 \\
189206	4.39102476601505e-10 \\
189586	4.13650946828881e-10 \\
189966	3.89672683009934e-10 \\
190346	3.67082086949466e-10 \\
190726	3.45799000545099e-10 \\
191106	3.25747762097706e-10 \\
191486	3.06856762222196e-10 \\
191866	2.89058998959035e-10 \\
192246	2.72291134084668e-10 \\
192626	2.56493604133823e-10 \\
193006	2.41610065288e-10 \\
193386	2.2758772644238e-10 \\
193766	2.1437651653855e-10 \\
194146	2.01929695187175e-10 \\
194526	1.90202964489572e-10 \\
194906	1.79154524548863e-10 \\
195286	1.68745351025734e-10 \\
195666	1.58938251448859e-10 \\
196046	1.49698198281811e-10 \\
196426	1.40992939545725e-10 \\
196806	1.32791000417853e-10 \\
197186	1.25063237543799e-10 \\
197566	1.17782561481761e-10 \\
197946	1.10922937501812e-10 \\
198326	1.04460051719713e-10 \\
198706	9.83708114965509e-11 \\
199086	9.26336785056492e-11 \\
199466	8.72282801545055e-11 \\
199846	8.21352985624912e-11 \\
200226	7.73368591389101e-11 \\
200606	7.28160309826364e-11 \\
200986	6.85563827929059e-11 \\
201366	6.45429820700372e-11 \\
201746	6.07616734704663e-11 \\
202126	5.71988567621418e-11 \\
202506	5.38420419360364e-11 \\
202886	5.0679183072333e-11 \\
203266	4.76993999853903e-11 \\
203646	4.4891590444962e-11 \\
204026	4.22461510218852e-11 \\
204406	3.97534782869968e-11 \\
204786	3.74050790341585e-11 \\
205166	3.51922935237781e-11 \\
205546	3.31074057058345e-11 \\
205926	3.11429770860627e-11 \\
206306	2.92920687705589e-11 \\
206686	2.75481304434777e-11 \\
207066	2.59049448558812e-11 \\
207446	2.43567388480415e-11 \\
207826	2.28979613048352e-11 \\
208206	2.15233386668956e-11 \\
208586	2.02283190198216e-11 \\
208966	1.90080728934561e-11 \\
209346	1.78583259291543e-11 \\
209726	1.67749147905738e-11 \\
210106	1.57541757417334e-11 \\
210486	1.47923340243494e-11 \\
210866	1.38861699916504e-11 \\
211246	1.30322419522599e-11 \\
211626	1.22277188374653e-11 \\
212006	1.14695475339488e-11 \\
212386	1.0755230039905e-11 \\
212766	1.00822683535284e-11 \\
213146	9.44799793956008e-12 \\
213526	8.85047590770682e-12 \\
213906	8.28748181191941e-12 \\
214286	7.75696173960227e-12 \\
214666	7.25702831161357e-12 \\
215046	6.78601619341634e-12 \\
215426	6.34214902817121e-12 \\
215806	5.92398352594614e-12 \\
216186	5.52996537450667e-12 \\
216566	5.15859577276956e-12 \\
216946	4.80876449771017e-12 \\
217326	4.47919479285019e-12 \\
217706	4.16855439056008e-12 \\
218086	3.87584409011765e-12 \\
218466	3.5999536684983e-12 \\
218846	3.34010596958478e-12 \\
219226	3.09524628150371e-12 \\
219606	2.8645419369866e-12 \\
219986	2.64710475761376e-12 \\
220366	2.44226860957042e-12 \\
220746	2.24931184789057e-12 \\
221126	2.06740180530574e-12 \\
221506	1.89581683684992e-12 \\
221886	1.73455694252311e-12 \\
222266	1.58228985469577e-12 \\
222646	1.43890455106543e-12 \\
223026	1.30379040896855e-12 \\
223406	1.17644782804405e-12 \\
223786	1.05654374138453e-12 \\
224166	9.43467526326458e-13 \\
224546	8.36997138264906e-13 \\
224926	7.3657746568756e-13 \\
225306	6.41986463989497e-13 \\
225686	5.52835555112097e-13 \\
226066	4.68902694450435e-13 \\
226446	3.89799303945892e-13 \\
226826	3.15192316691082e-13 \\
227206	2.44970710383541e-13 \\
227586	1.78690395813419e-13 \\
227966	1.16295861829485e-13 \\
228346	5.75095526755831e-14 \\
228726	2.05391259555654e-15 \\
229106	-5.00155472593633e-14 \\
229486	-9.9253938401489e-14 \\
229866	-1.45661260830821e-13 \\
230246	-1.8934853684982e-13 \\
230626	-2.30482299912182e-13 \\
231006	-2.69340105774063e-13 \\
231386	-3.05866443284231e-13 \\
231766	-3.40227845896379e-13 \\
232146	-3.72646358215434e-13 \\
232526	-4.03288513695088e-13 \\
232906	-4.32098801184111e-13 \\
233286	-4.59299265287427e-13 \\
233666	-4.84834394853806e-13 \\
234046	-5.08926234488172e-13 \\
234426	-5.31630295341756e-13 \\
234806	-5.53057599717022e-13 \\
235186	-5.73208147613968e-13 \\
};
\addplot [line width=0.01pt, blue, forget plot]
table [row sep=\\]{%
1190	2.82854111304589 \\
2380	2.34697792644848 \\
3570	1.96805272895357 \\
4760	1.6455241875242 \\
5950	1.37612440203763 \\
7140	1.14992323603675 \\
8316	0.961383959216057 \\
9486	0.804014343066838 \\
10656	0.668713025069309 \\
11826	0.556306852287667 \\
12996	0.46658336982209 \\
14166	0.391486468841531 \\
15336	0.328990034644079 \\
16506	0.275557838005126 \\
17676	0.22975224015734 \\
18846	0.191617169405417 \\
20016	0.16110075114142 \\
21186	0.138131376484607 \\
22356	0.117350664751377 \\
23526	0.0984909110945683 \\
24696	0.082316616104555 \\
25866	0.0703035086024751 \\
27036	0.0595106146652164 \\
28206	0.0510420846788696 \\
29376	0.0435203854859322 \\
30546	0.037093011561419 \\
31716	0.0321002756459737 \\
32886	0.0276752402269411 \\
34056	0.0236800353393342 \\
35226	0.0203551837325457 \\
36396	0.0173250083805164 \\
37566	0.0148554314341962 \\
38736	0.0128080537175642 \\
39906	0.0111494967461115 \\
41076	0.00959740166272272 \\
42246	0.00851075398794593 \\
43416	0.0078142134992516 \\
44586	0.00718168941190395 \\
45756	0.00659192336490361 \\
46926	0.00604049637972026 \\
48096	0.00560402406357446 \\
49266	0.00520513012052892 \\
50436	0.0048353411518342 \\
51606	0.00449535230886977 \\
52776	0.00417650238767731 \\
53946	0.00387732932204521 \\
55116	0.00359649372341431 \\
56286	0.00334688379838849 \\
57456	0.00311928986629834 \\
58626	0.00290753649876935 \\
59796	0.00270879276231023 \\
60966	0.00252220970624162 \\
62136	0.00234700056329545 \\
63306	0.00218243454319517 \\
64476	0.00202783219199248 \\
65646	0.00188718499944479 \\
66816	0.00175956681929484 \\
67986	0.00163993214755281 \\
69156	0.00152771928575668 \\
70326	0.00142243998165309 \\
71496	0.00132364241632493 \\
72666	0.00123090685579752 \\
73836	0.00114384301379611 \\
75006	0.00106208772233224 \\
76176	0.00098530282119319 \\
77346	0.000913173239191101 \\
78516	0.000845405245983477 \\
79686	0.00078172485625333 \\
80856	0.000721876370443297 \\
82026	0.000665621038240638 \\
83196	0.000617775701564238 \\
84366	0.000573855797707468 \\
85536	0.000532631620935309 \\
86706	0.000493924730557704 \\
87876	0.000457575465561078 \\
89046	0.000423435266981131 \\
90216	0.000391365487562023 \\
91386	0.000361236649679741 \\
92556	0.000332927785493986 \\
93726	0.000306325832039489 \\
94896	0.00028449118243451 \\
96066	0.000264599786677144 \\
97236	0.000246105401154162 \\
98406	0.000228904826247045 \\
99576	0.000212903664955355 \\
100746	0.000198015241705263 \\
101916	0.000184159774805626 \\
103086	0.000171263683883871 \\
104256	0.000159258999464895 \\
105426	0.000148082853371923 \\
106596	0.000137677034052486 \\
107766	0.000127987594758572 \\
108936	0.000118964505336661 \\
110106	0.000110561340480431 \\
111276	0.000102734998867593 \\
112446	9.54454487712142e-05 \\
113616	8.86554966272324e-05 \\
114786	8.23305757058912e-05 \\
115956	7.64385525497402e-05 \\
117126	7.09495492380308e-05 \\
118296	6.58357798422049e-05 \\
119466	6.10713996836987e-05 \\
120636	5.66323661921886e-05 \\
121806	5.24963103236131e-05 \\
122976	4.86424176207589e-05 \\
124146	4.50513181017853e-05 \\
125316	4.17049842558748e-05 \\
126486	3.85866364917553e-05 \\
127656	3.56806554524525e-05 \\
128826	3.29725006646409e-05 \\
129996	3.04486350372057e-05 \\
131166	2.8096454765647e-05 \\
132336	2.5904224237705e-05 \\
133506	2.38610155698371e-05 \\
134676	2.19566524307102e-05 \\
135846	2.01816578394021e-05 \\
137016	1.85272056463792e-05 \\
138186	1.69850754331846e-05 \\
139356	1.55620943403378e-05 \\
140526	1.44519613976546e-05 \\
141696	1.34232926248812e-05 \\
142866	1.24695660776997e-05 \\
144036	1.15849259098577e-05 \\
145206	1.07640467135739e-05 \\
146376	1.00020681425561e-05 \\
147546	9.29454311632272e-06 \\
148716	8.6373942179141e-06 \\
149886	8.02687645434785e-06 \\
151056	7.45954512354485e-06 \\
152226	6.93222782915948e-06 \\
153396	6.44199990540439e-06 \\
154566	5.98616267533103e-06 \\
155736	5.56222409087148e-06 \\
156906	5.17791843485105e-06 \\
158076	4.82369494658741e-06 \\
159246	4.49443468342192e-06 \\
160416	4.18830313836382e-06 \\
161586	3.90361242647019e-06 \\
162756	3.63880521975224e-06 \\
163926	3.39244341746037e-06 \\
165096	3.16319816395794e-06 \\
166266	2.94984085819072e-06 \\
167436	2.7512350205261e-06 \\
168606	2.56632891099118e-06 \\
169776	2.39414881225697e-06 \\
170946	2.2337929032612e-06 \\
172116	2.08442566151934e-06 \\
173286	1.94527274105516e-06 \\
174456	1.81561627649041e-06 \\
175626	1.69479057465782e-06 \\
176796	1.58217815743322e-06 \\
177966	1.47720612292401e-06 \\
179136	1.3793427977582e-06 \\
180306	1.28809465355095e-06 \\
181476	1.20300346728719e-06 \\
182646	1.12364370241647e-06 \\
183816	1.04962009322973e-06 \\
184986	9.8056541614211e-07 \\
186156	9.16138432061064e-07 \\
187326	8.56021986239686e-07 \\
188496	7.9992125201489e-07 \\
189666	7.47562108327493e-07 \\
190836	6.98689639089256e-07 \\
192006	6.53066746181263e-07 \\
193176	6.10472865425482e-07 \\
194346	5.7070277881266e-07 \\
195516	5.3356551493744e-07 \\
196686	4.98883331589983e-07 \\
197856	4.66490772454975e-07 \\
199026	4.36233793532637e-07 \\
200196	4.07968954785343e-07 \\
201366	3.81562669848901e-07 \\
202536	3.5689051020027e-07 \\
203706	3.33836561450251e-07 \\
204876	3.12292824822258e-07 \\
206046	2.92158663206532e-07 \\
207216	2.73340285961154e-07 \\
208386	2.55750271960231e-07 \\
209556	2.39307126115307e-07 \\
210726	2.23934867649156e-07 \\
211896	2.09562648789685e-07 \\
213066	1.96124398832431e-07 \\
214236	1.83558495348013e-07 \\
215406	1.71807457594042e-07 \\
216576	1.60817661631896e-07 \\
217746	1.5053907626017e-07 \\
218916	1.40925016711702e-07 \\
220086	1.31931916225181e-07 \\
221256	1.23519113104376e-07 \\
222426	1.15648653542522e-07 \\
223596	1.08285106992234e-07 \\
224766	1.01395395468717e-07 \\
225936	9.4948633844183e-08 \\
227106	8.89159821881869e-08 \\
228276	8.32705071562856e-08 \\
229446	7.79870540923433e-08 \\
230616	7.30421269579118e-08 \\
231786	6.84137771433946e-08 \\
232956	6.40814998842387e-08 \\
234126	6.00261372274424e-08 \\
235296	5.62297886586016e-08 \\
236466	5.2675726336382e-08 \\
237636	4.93483176544629e-08 \\
238806	4.62329519113069e-08 \\
239976	4.33159728641108e-08 \\
241146	4.05846149464928e-08 \\
242316	3.80269444266723e-08 \\
243486	3.56318046179638e-08 \\
244656	3.3388763920339e-08 \\
245826	3.12880686359485e-08 \\
246996	2.93205978385558e-08 \\
248166	2.74778220732408e-08 \\
249336	2.57517637769489e-08 \\
250506	2.41349622509546e-08 \\
251676	2.26204383002582e-08 \\
252846	2.12016640910306e-08 \\
254016	1.98725322309024e-08 \\
255186	1.86273297342332e-08 \\
256356	1.74607109881819e-08 \\
257526	1.63676746600672e-08 \\
258696	1.53435408267732e-08 \\
259866	1.43839300470461e-08 \\
261036	1.34847443766795e-08 \\
262206	1.26421482726791e-08 \\
263376	1.18525519954282e-08 \\
264546	1.11125958990321e-08 \\
265716	1.04191353877958e-08 \\
266886	9.76922631679145e-09 \\
268056	9.16011316798304e-09 \\
269226	8.58921611612828e-09 \\
270396	8.05411970450365e-09 \\
271566	7.55256307494179e-09 \\
272736	7.08242842151208e-09 \\
273906	6.64173377407096e-09 \\
275076	6.22862250665435e-09 \\
276246	5.84135645409489e-09 \\
277416	5.47830752983813e-09 \\
278266	5.1379511201155e-09 \\
278636	4.81885953362848e-09 \\
279006	4.51969539572161e-09 \\
279376	4.2392060417562e-09 \\
279746	3.97621846559559e-09 \\
280116	3.7296334354231e-09 \\
280486	3.49842166347258e-09 \\
280856	3.28161875451372e-09 \\
281226	3.07832098700445e-09 \\
281596	2.88768225997771e-09 \\
281966	2.70890931908241e-09 \\
282336	2.54125942511507e-09 \\
282706	2.3840361906835e-09 \\
283076	2.23658752629419e-09 \\
283446	2.09830219866092e-09 \\
283816	1.96860744372529e-09 \\
284186	1.84696652416605e-09 \\
284556	1.73287656446419e-09 \\
284926	1.62586605290116e-09 \\
285296	1.52549345378006e-09 \\
285666	1.43134476493501e-09 \\
286036	1.34303168586314e-09 \\
286406	1.2601908405685e-09 \\
286776	1.18248116853792e-09 \\
287146	1.10958325860722e-09 \\
287516	1.04119740607089e-09 \\
287886	9.77043113081777e-10 \\
288256	9.16857090249579e-10 \\
288626	8.60392590507075e-10 \\
288996	8.07418354398237e-10 \\
289366	7.57717444344053e-10 \\
289736	7.11086745042167e-10 \\
290106	6.67335464665797e-10 \\
290476	6.26285079352584e-10 \\
290846	5.87767834403508e-10 \\
291216	5.51626910816339e-10 \\
291586	5.17714815462256e-10 \\
291956	4.85893436596996e-10 \\
292326	4.56033322215887e-10 \\
292696	4.28012902897734e-10 \\
293066	4.01718436293663e-10 \\
293436	3.77043007926403e-10 \\
293806	3.53886531190284e-10 \\
294176	3.3215497019512e-10 \\
294546	3.11760395277361e-10 \\
294916	2.9262003931052e-10 \\
295286	2.74656353216329e-10 \\
295656	2.57796950453582e-10 \\
296026	2.41973274750507e-10 \\
296396	2.27121710327793e-10 \\
296766	2.13182027586356e-10 \\
297136	2.00098160263451e-10 \\
297506	1.87817261743106e-10 \\
297876	1.76289871589574e-10 \\
298246	1.65469582480426e-10 \\
298616	1.55312762650794e-10 \\
298986	1.45778555893372e-10 \\
299356	1.36828492980356e-10 \\
299726	1.28426991263808e-10 \\
300096	1.20540022408022e-10 \\
300466	1.13135945056797e-10 \\
300836	1.06185171766526e-10 \\
301206	9.96596694058383e-11 \\
301576	9.35334587559566e-11 \\
301946	8.7781948376886e-11 \\
302316	8.23821566520166e-11 \\
302686	7.73124897435196e-11 \\
303056	7.25525750588929e-11 \\
303426	6.8083538806718e-11 \\
303796	6.3887395373996e-11 \\
304166	5.994760243766e-11 \\
304536	5.62482282973065e-11 \\
304906	5.27746735201617e-11 \\
305276	4.9513060318418e-11 \\
305646	4.6450399082687e-11 \\
306016	4.35745328708492e-11 \\
306386	4.08739153634485e-11 \\
306756	3.8337999441751e-11 \\
307126	3.595657105393e-11 \\
307496	3.37202488154276e-11 \\
307866	3.16201509420466e-11 \\
308236	2.96478397388e-11 \\
308606	2.77957656891203e-11 \\
308976	2.60562127429864e-11 \\
309346	2.44225195622505e-11 \\
309716	2.28882468533698e-11 \\
310086	2.144734390086e-11 \\
310456	2.00938710115395e-11 \\
310826	1.88227766706461e-11 \\
311196	1.76289538522667e-11 \\
311566	1.65076285973953e-11 \\
311936	1.54543045027822e-11 \\
312306	1.44650402766899e-11 \\
312676	1.35356725827762e-11 \\
313046	1.26629262631184e-11 \\
313416	1.18430265594327e-11 \\
313786	1.10729203584015e-11 \\
314156	1.03494435244045e-11 \\
314526	9.66993152218265e-12 \\
314896	9.03138674956949e-12 \\
315266	8.43175529396945e-12 \\
315636	7.86837262012341e-12 \\
316006	7.33918481543583e-12 \\
316376	6.84191592270622e-12 \\
316746	6.37473407394396e-12 \\
317116	5.93602944576332e-12 \\
317486	5.52369261441754e-12 \\
317856	5.13633580112582e-12 \\
318226	4.77251571595616e-12 \\
318596	4.43062253552284e-12 \\
318966	4.10932399219632e-12 \\
319336	3.80756537410321e-12 \\
319706	3.52390339131148e-12 \\
320076	3.25744986540144e-12 \\
320446	3.00709457334847e-12 \\
320816	2.7718383144304e-12 \\
321186	2.55084842137876e-12 \\
321556	2.34318120462262e-12 \\
321926	2.14794848574229e-12 \\
322296	1.96465066437668e-12 \\
322666	1.79228853980362e-12 \\
323036	1.6304735339645e-12 \\
323406	1.4782619572884e-12 \\
323776	1.33515420941421e-12 \\
324146	1.2008727345858e-12 \\
324516	1.07452935438346e-12 \\
324886	9.55846513051029e-13 \\
325256	8.44435632529894e-13 \\
325626	7.39519556702817e-13 \\
325996	6.40987263267334e-13 \\
326366	5.48450174164827e-13 \\
326736	4.61353177882984e-13 \\
327106	3.79585252119341e-13 \\
327476	3.0275781881528e-13 \\
327846	2.30482299912182e-13 \\
328216	1.62647673107585e-13 \\
328586	9.88653603428702e-14 \\
328956	3.88578058618805e-14 \\
329326	-1.75970349403087e-14 \\
329696	-7.05546732149287e-14 \\
330066	-1.20292664718136e-13 \\
330436	-1.67199587508549e-13 \\
330806	-2.11330952737399e-13 \\
331176	-2.52464715799761e-13 \\
331546	-2.91378032812872e-13 \\
331916	-3.2790437032304e-13 \\
332286	-3.62376795237651e-13 \\
332656	-3.94684285254243e-13 \\
333026	-4.2499337382651e-13 \\
333396	-4.53526105559376e-13 \\
333766	-4.80393502755305e-13 \\
334136	-5.05651076565528e-13 \\
334506	-5.29298826990043e-13 \\
334876	-5.51614309785009e-13 \\
335246	-5.72597524950424e-13 \\
335616	-5.92359494788752e-13 \\
335986	-6.10844708148761e-13 \\
336356	-6.28275209635376e-13 \\
336726	-6.44595488097366e-13 \\
337096	-6.60083099290887e-13 \\
337466	-6.74515998611014e-13 \\
337836	-6.88171741813903e-13 \\
338206	-7.00939306597093e-13 \\
};
\addplot [line width=0.01pt, blue, forget plot]
table [row sep=\\]{%
1190	2.32384269576584 \\
2380	1.86756335122291 \\
3570	1.5207744139916 \\
4760	1.2314500489168 \\
5950	0.996139767792307 \\
7130	0.799395405052283 \\
8310	0.637128159831611 \\
9490	0.522125635710809 \\
10670	0.426921705988063 \\
11850	0.347990680359415 \\
13030	0.283670006625776 \\
14210	0.234665036179963 \\
15390	0.193489736581687 \\
16570	0.161160744132567 \\
17750	0.134480691020676 \\
18930	0.114768642976205 \\
20110	0.0990882742724595 \\
21290	0.0865548003652329 \\
22470	0.0755183146107195 \\
23650	0.0659060055970046 \\
24830	0.0578647465073498 \\
26010	0.0511521398635229 \\
27190	0.0455472699156217 \\
28370	0.0413751174603269 \\
29550	0.0375093520859953 \\
30730	0.0339215011282285 \\
31910	0.0307939569272229 \\
33090	0.0280782951532974 \\
34270	0.0256853629558197 \\
35450	0.0234914175129972 \\
36630	0.021596919292539 \\
37810	0.0198993008606773 \\
38990	0.0183212674035952 \\
40170	0.0168557971358 \\
41350	0.0155149628328962 \\
42530	0.0142684142809582 \\
43710	0.0131078413977347 \\
44890	0.0120273940655945 \\
46070	0.0110218222720226 \\
47250	0.0101054608131609 \\
48430	0.00925814945729381 \\
49610	0.00846772741871848 \\
50790	0.00773058014716771 \\
51970	0.00704418338747986 \\
53150	0.0064043838271603 \\
54330	0.00580712119672017 \\
55510	0.00525320566079679 \\
56690	0.00477348541927913 \\
57870	0.00433208063352519 \\
59050	0.00394590229394659 \\
60230	0.00358981094290795 \\
61410	0.00325885732580911 \\
62590	0.00295106670668271 \\
63770	0.00266441768907294 \\
64950	0.00239713547153941 \\
66130	0.00214757153381362 \\
67310	0.00191534926624604 \\
68490	0.00170330492671866 \\
69670	0.00150841159542314 \\
70850	0.00132840474397439 \\
72030	0.00116034711019936 \\
73210	0.00100823760249569 \\
74390	0.000894385948281951 \\
75570	0.000824159205952746 \\
76750	0.000767636011412109 \\
77930	0.00071513498208442 \\
79110	0.000666299967041817 \\
80290	0.000620875273017296 \\
81470	0.000578625131262223 \\
82650	0.000539318659988408 \\
83830	0.000502725003895588 \\
85010	0.000468640073479021 \\
86190	0.000436913023083518 \\
87370	0.000407357979384793 \\
88550	0.000379822429310173 \\
89730	0.000354168606899685 \\
90910	0.000330270154122458 \\
92090	0.000307998729980108 \\
93270	0.000287233510234908 \\
94450	0.000267871166344469 \\
95630	0.000249817153268161 \\
96810	0.00023298436742869 \\
97990	0.000217284652039629 \\
99170	0.000202641829820327 \\
100350	0.000188983781668428 \\
101530	0.000176241287165146 \\
102710	0.000164352360053133 \\
103890	0.000153256789979239 \\
105070	0.000142899621773274 \\
106250	0.000133234263135329 \\
107430	0.000124212469791773 \\
108610	0.00011579068999773 \\
109790	0.000107926715463391 \\
110970	0.000100584061735964 \\
112150	9.37266128692715e-05 \\
113330	8.73232520114398e-05 \\
114510	8.13439869751797e-05 \\
115690	7.57584631390085e-05 \\
116870	7.0540909491601e-05 \\
118050	6.56680053594538e-05 \\
119230	6.11145631973509e-05 \\
120410	5.6861847319345e-05 \\
121037	5.28876365820108e-05 \\
121427	4.91735201217924e-05 \\
121817	4.57030355335286e-05 \\
122207	4.24599958567251e-05 \\
122597	3.94293104347687e-05 \\
122987	3.67248103031459e-05 \\
123377	3.43252037792174e-05 \\
123767	3.2085210331656e-05 \\
124157	2.99936251954014e-05 \\
124547	2.80401540804776e-05 \\
124937	2.6215300383936e-05 \\
125327	2.45102853705936e-05 \\
125717	2.29169807731133e-05 \\
126107	2.14278505055177e-05 \\
126497	2.00358997135552e-05 \\
126887	1.87346298315405e-05 \\
127277	1.75179986222629e-05 \\
127667	1.6380384398329e-05 \\
128057	1.53165537901123e-05 \\
128447	1.43216325610429e-05 \\
128837	1.33910790607872e-05 \\
129227	1.2520659990134e-05 \\
129617	1.1706428203806e-05 \\
130007	1.09447023269316e-05 \\
130397	1.0232047993608e-05 \\
130787	9.56526054712858e-06 \\
131177	8.94134906143096e-06 \\
131567	8.35752156425063e-06 \\
131957	7.81117135356668e-06 \\
132347	7.2998643179667e-06 \\
132737	6.82132717316764e-06 \\
133127	6.37343654363853e-06 \\
133517	5.9542088213238e-06 \\
133907	5.56179073996077e-06 \\
134297	5.19445061220036e-06 \\
134687	4.85057017790558e-06 \\
135077	4.54032731667109e-06 \\
135467	4.25122502556441e-06 \\
135857	3.98078468205609e-06 \\
136247	3.72777109275413e-06 \\
136637	3.49103672203643e-06 \\
137027	3.26951286161448e-06 \\
137417	3.06220362006338e-06 \\
137807	2.86818056816074e-06 \\
138197	2.6865778491314e-06 \\
138587	2.51658769701013e-06 \\
138977	2.35745631588236e-06 \\
139367	2.20848008830599e-06 \\
139757	2.06900207638805e-06 \\
140147	1.93840878964791e-06 \\
140537	1.81612719785118e-06 \\
140927	1.7016219623911e-06 \\
141317	1.59439287289453e-06 \\
141707	1.49397246734795e-06 \\
142097	1.39992382425236e-06 \\
142487	1.31183851176386e-06 \\
142877	1.22933468005293e-06 \\
143267	1.15205528872231e-06 \\
143657	1.07966645723767e-06 \\
144047	1.01185592898956e-06 \\
144437	9.48331640993061e-07 \\
144827	8.88820392397349e-07 \\
145217	8.33066602978771e-07 \\
145607	7.80831155788864e-07 \\
145997	7.31890318350636e-07 \\
146387	6.86034737018559e-07 \\
146777	6.43068497341304e-07 \\
147167	6.02808249039466e-07 \\
147557	5.65082388381821e-07 \\
147947	5.29730294962505e-07 \\
148337	4.96601619492942e-07 \\
148727	4.65555618500701e-07 \\
149117	4.36460532993177e-07 \\
149507	4.09193008199527e-07 \\
149897	3.83637551282234e-07 \\
150287	3.59686024409278e-07 \\
150677	3.37237172298721e-07 \\
151067	3.16196178185013e-07 \\
151457	2.96474251815226e-07 \\
151847	2.77988241592642e-07 \\
152237	2.60660273698754e-07 \\
152627	2.4441741391934e-07 \\
153017	2.29191351508451e-07 \\
153407	2.14918103536021e-07 \\
153797	2.01537738997448e-07 \\
154187	1.88994118854868e-07 \\
154577	1.77234655673875e-07 \\
154967	1.66210086305352e-07 \\
155357	1.55874260499012e-07 \\
155747	1.46183942784095e-07 \\
156137	1.37098626507015e-07 \\
156527	1.28580361580255e-07 \\
156917	1.20593590835494e-07 \\
157307	1.13104998922253e-07 \\
157697	1.0608336969975e-07 \\
158087	9.94994529546211e-08 \\
158477	9.33258403335024e-08 \\
158867	8.75368482144978e-08 \\
159257	8.21084089608348e-08 \\
159647	7.70179680587013e-08 \\
160037	7.22443889711322e-08 \\
160427	6.77678630434109e-08 \\
160817	6.35698259587869e-08 \\
161207	5.96328792457079e-08 \\
161597	5.59407160594105e-08 \\
161987	5.24780531252489e-08 \\
162377	4.92305653465586e-08 \\
162767	4.61848260191466e-08 \\
163157	4.3328250043384e-08 \\
163547	4.06490408555449e-08 \\
163937	3.81361405787928e-08 \\
164327	3.57791840044364e-08 \\
164717	3.35684546826087e-08 \\
165107	3.14948440105489e-08 \\
165497	2.95498134850192e-08 \\
165887	2.7725358064945e-08 \\
166277	2.6013973752903e-08 \\
166667	2.44086252321196e-08 \\
167057	2.29027167786278e-08 \\
167447	2.14900646167138e-08 \\
167837	2.01648710507207e-08 \\
168227	1.89217003732089e-08 \\
168617	1.77554562719173e-08 \\
169007	1.6661360013881e-08 \\
169397	1.56349310720394e-08 \\
169787	1.46719683069563e-08 \\
170177	1.37685323697845e-08 \\
170567	1.29209298260768e-08 \\
170957	1.21256965024408e-08 \\
171347	1.13795846079512e-08 \\
171737	1.06795485232958e-08 \\
172127	1.00227321442325e-08 \\
172517	9.40645689118114e-09 \\
172907	8.82821088454833e-09 \\
173297	8.28563906374313e-09 \\
173687	7.77653175187965e-09 \\
174077	7.29881766137197e-09 \\
174467	6.85055340232665e-09 \\
174857	6.42991687671568e-09 \\
175247	6.03519906272609e-09 \\
175637	5.66479674279918e-09 \\
176027	5.31720678598191e-09 \\
176417	4.99101876494379e-09 \\
176807	4.68491001548443e-09 \\
177197	4.39764030746304e-09 \\
177587	4.12804557203827e-09 \\
177977	3.87503445997694e-09 \\
178367	3.63758334565034e-09 \\
178757	3.41473133103065e-09 \\
179147	3.20557785871145e-09 \\
179537	3.00927766039294e-09 \\
179927	2.82503820336899e-09 \\
180317	2.65211547167965e-09 \\
180707	2.48981202322085e-09 \\
181097	2.33747338151957e-09 \\
181487	2.19448531568744e-09 \\
181877	2.06027195304159e-09 \\
182267	1.93429289252478e-09 \\
182657	1.81604103977051e-09 \\
183047	1.70504071972388e-09 \\
183437	1.60084567824015e-09 \\
183827	1.50303725021672e-09 \\
184217	1.41122241670288e-09 \\
184607	1.32503263916561e-09 \\
184997	1.24412208313274e-09 \\
185387	1.16816617490301e-09 \\
185777	1.09686071336768e-09 \\
186167	1.02991953854215e-09 \\
186557	9.67074753610575e-10 \\
186947	9.08074782035584e-10 \\
187337	8.52683534890986e-10 \\
187727	8.00679578194519e-10 \\
188117	7.51855078195973e-10 \\
188507	7.06014802176469e-10 \\
188897	6.62975785381548e-10 \\
189287	6.22566165286997e-10 \\
189677	5.84624848531945e-10 \\
190067	5.49000456206983e-10 \\
190457	5.15551046298413e-10 \\
190847	4.84143281020977e-10 \\
191237	4.54652260284405e-10 \\
191627	4.26960300448087e-10 \\
192017	4.0095771147719e-10 \\
192407	3.76540743030063e-10 \\
192797	3.53612361614353e-10 \\
193187	3.32081695475495e-10 \\
193577	3.11863090907138e-10 \\
193967	2.92876334295755e-10 \\
194357	2.75046097009124e-10 \\
194747	2.58301768862879e-10 \\
195137	2.42576958520146e-10 \\
195527	2.27809382469246e-10 \\
195917	2.13940420934478e-10 \\
196307	2.00915284409575e-10 \\
196697	1.88682514057348e-10 \\
197087	1.77193926198527e-10 \\
197477	1.66403779644497e-10 \\
197867	1.56269441831114e-10 \\
198257	1.46751055751793e-10 \\
198647	1.37811206890603e-10 \\
199037	1.29414257088456e-10 \\
199427	1.21527454766124e-10 \\
199817	1.14119491634312e-10 \\
200207	1.07161279849777e-10 \\
200597	1.00625396903808e-10 \\
200987	9.44861411333875e-11 \\
201377	8.87193651877283e-11 \\
201767	8.33023650059772e-11 \\
202157	7.82138798172127e-11 \\
202547	7.34338700958403e-11 \\
202937	6.89436285838951e-11 \\
203327	6.47254472241343e-11 \\
203717	6.07628947157934e-11 \\
204107	5.70403724253765e-11 \\
204497	5.3543336431261e-11 \\
204887	5.02580199679414e-11 \\
205277	4.71715999594835e-11 \\
205667	4.427203048607e-11 \\
206057	4.15477652282448e-11 \\
206447	3.89883680895764e-11 \\
206837	3.65838470628432e-11 \\
207227	3.4324765252336e-11 \\
207617	3.22021298515551e-11 \\
208007	3.02078917435722e-11 \\
208397	2.83341128337611e-11 \\
208787	2.6573632183613e-11 \\
209177	2.49194553880727e-11 \\
209567	2.33651986647487e-11 \\
209957	2.19047002758543e-11 \\
210347	2.05325756397201e-11 \\
210737	1.92429960854668e-11 \\
211127	1.80314652098446e-11 \\
211517	1.68929870092427e-11 \\
211907	1.58230650804114e-11 \\
212297	1.48177026204621e-11 \\
212687	1.38729583376573e-11 \\
213077	1.29852240071671e-11 \\
213467	1.21508914041613e-11 \\
213857	1.13669629264734e-11 \\
214247	1.06302189273322e-11 \\
214637	9.93788384917593e-12 \\
215027	9.2871266232919e-12 \\
215417	8.67561578132836e-12 \\
215807	8.10090883263115e-12 \\
216197	7.56078533115101e-12 \\
216587	7.05319136429239e-12 \\
216977	6.57607301945973e-12 \\
217367	6.12770945096486e-12 \\
217757	5.70626879081715e-12 \\
218147	5.31019672678212e-12 \\
218537	4.93788343547408e-12 \\
218927	4.58799664926346e-12 \\
219317	4.25909307821826e-12 \\
219707	3.95000698816261e-12 \\
220097	3.65946162261821e-12 \\
220487	3.38634675856042e-12 \\
220877	3.12971870641832e-12 \\
221267	2.88830070971358e-12 \\
221657	2.66148214578266e-12 \\
222047	2.44820830275216e-12 \\
222437	2.24792406910979e-12 \\
222827	2.05946371067967e-12 \\
223217	1.88232762710072e-12 \\
223607	1.71584968455818e-12 \\
223997	1.55936374923726e-12 \\
224387	1.4122036873232e-12 \\
224777	1.27392540960614e-12 \\
225167	1.14391829342253e-12 \\
225557	1.02173824956253e-12 \\
225947	9.06885677665059e-13 \\
226337	7.988054662178e-13 \\
226727	6.97220059464598e-13 \\
227117	6.01685368195604e-13 \\
227507	5.11868325503428e-13 \\
227897	4.27546886783148e-13 \\
228287	3.4822145167368e-13 \\
228677	2.73669975570101e-13 \\
229067	2.03503880413791e-13 \\
229457	1.37390099297363e-13 \\
229847	7.55506768257419e-14 \\
230237	1.71529457304587e-14 \\
230627	-3.76365605347928e-14 \\
231017	-8.92064200286313e-14 \\
231407	-1.37778677355982e-13 \\
231797	-1.8313128791192e-13 \\
232187	-2.26096918964913e-13 \\
232577	-2.66342503607575e-13 \\
232967	-3.04312131049755e-13 \\
233357	-3.39894778988992e-13 \\
233747	-3.73312492030209e-13 \\
234137	-4.04842825929563e-13 \\
234527	-4.34485780687055e-13 \\
234917	-4.62407889756378e-13 \\
235307	-4.88498130835069e-13 \\
235697	-5.13145081981747e-13 \\
236087	-5.36348743196413e-13 \\
236477	-5.58109114479066e-13 \\
236867	-5.785927292834e-13 \\
};
\addplot [line width=0.01pt, blue, forget plot]
table [row sep=\\]{%
1190	2.28329669976641 \\
2380	1.85953713133177 \\
3570	1.52180327382378 \\
4760	1.25500936840845 \\
5950	1.03833533408539 \\
7135	0.853615045810532 \\
8315	0.702093694509291 \\
9495	0.576274725870828 \\
10675	0.469862240840327 \\
11855	0.38553934840267 \\
13035	0.320865749253305 \\
14215	0.269654228249542 \\
15395	0.224266993987461 \\
16575	0.187399796749321 \\
17755	0.155344267585059 \\
18935	0.127170169928826 \\
20115	0.104088598133188 \\
21295	0.0872350180094469 \\
22475	0.0725838054932682 \\
23655	0.0613720739084126 \\
24835	0.0539859906120342 \\
26015	0.0477123779697823 \\
27195	0.0419885529776869 \\
28375	0.0369743148203547 \\
29555	0.0323818499721287 \\
30735	0.0281570641822612 \\
31915	0.0244616344423824 \\
33095	0.0213036113526724 \\
34275	0.0186908463617827 \\
35455	0.0163310073151406 \\
36635	0.0142287108800434 \\
37815	0.0123224579030644 \\
38995	0.0106704175460643 \\
40175	0.00924513971589502 \\
41355	0.0080076723167074 \\
42535	0.00687584011138537 \\
43715	0.00620002380612189 \\
44895	0.00564568928999581 \\
46075	0.00514397017027862 \\
47255	0.00468446914734005 \\
48435	0.00426272308735504 \\
49615	0.00387496890264505 \\
50795	0.00352539517315265 \\
51975	0.00321133699499093 \\
53155	0.00291855461609852 \\
54335	0.00264547156231237 \\
55515	0.00239064552722462 \\
56695	0.00215275334206944 \\
57875	0.0019311462187358 \\
59055	0.00172418314725531 \\
60235	0.00153079144104595 \\
61415	0.00135001706882198 \\
62595	0.00118097988629967 \\
63775	0.00102286712994171 \\
64955	0.000874927585606866 \\
66135	0.000736466345424458 \\
67315	0.000615858573348282 \\
68495	0.000507271456873903 \\
69675	0.000427411136623423 \\
70855	0.00039150142064398 \\
72035	0.000364065345135445 \\
73215	0.000338868229397216 \\
74395	0.000315658901989258 \\
75575	0.000294239745083691 \\
76755	0.000274440077614235 \\
77935	0.00025611056351682 \\
79115	0.000239119693051348 \\
80295	0.000223351019936757 \\
81475	0.000208700931276451 \\
82655	0.000195076828306051 \\
83835	0.000182602225444473 \\
85015	0.000171109722937213 \\
86195	0.000160391193604392 \\
87375	0.000150386912500411 \\
88555	0.000141043007709785 \\
89735	0.000132310660389934 \\
90915	0.00012416561606049 \\
92095	0.000116562583148572 \\
93275	0.000109444959184579 \\
94455	0.000102779000274056 \\
95635	9.65337370957275e-05 \\
96815	9.06806616604583e-05 \\
97995	8.51934770588758e-05 \\
99175	8.0047882597134e-05 \\
100355	7.52213869842921e-05 \\
101535	7.06931446946557e-05 \\
102715	6.64438116986243e-05 \\
103895	6.24554175311376e-05 \\
105075	5.87112512531185e-05 \\
106255	5.51957593247776e-05 \\
107435	5.18944537649113e-05 \\
108615	4.87938292540457e-05 \\
109795	4.58812880617665e-05 \\
110975	4.31450718583748e-05 \\
112155	4.05741996133968e-05 \\
113335	3.8158410904654e-05 \\
114515	3.58932327663708e-05 \\
115695	3.37695306004693e-05 \\
116875	3.1772668449892e-05 \\
118055	2.98949281408722e-05 \\
119235	2.81290830809744e-05 \\
120415	2.64683617339223e-05 \\
121433	2.49064161457202e-05 \\
121803	2.34372932614857e-05 \\
122173	2.20554085881841e-05 \\
122543	2.07555219292055e-05 \\
122913	1.95327149669078e-05 \\
123283	1.83823705025565e-05 \\
123653	1.73001531977302e-05 \\
124023	1.62819916753043e-05 \\
124393	1.53240618628869e-05 \\
124763	1.44227714746203e-05 \\
125133	1.35747455369262e-05 \\
125503	1.27768128783123e-05 \\
125873	1.20259935084133e-05 \\
126243	1.13194868193167e-05 \\
126613	1.06546605517233e-05 \\
126983	1.00290404682624e-05 \\
127353	9.44030068672497e-06 \\
127723	8.88625462652781e-06 \\
128093	8.36484652533276e-06 \\
128463	7.87414349046056e-06 \\
128833	7.41232804607472e-06 \\
129203	6.97769114493818e-06 \\
129573	6.56862561487781e-06 \\
129943	6.18362000970318e-06 \\
130313	5.82125283998813e-06 \\
130683	5.48018715945675e-06 \\
131053	5.15916548193829e-06 \\
131423	4.85700501157149e-06 \\
131793	4.57259316305469e-06 \\
132163	4.3048833551218e-06 \\
132533	4.05289106047979e-06 \\
132903	3.81569009505478e-06 \\
133273	3.59240913311298e-06 \\
133643	3.38222843188074e-06 \\
134013	3.18437675589367e-06 \\
134383	2.99812848547631e-06 \\
134753	2.822800900526e-06 \\
135123	2.65775162749948e-06 \\
135493	2.50237623922178e-06 \\
135863	2.35610600068936e-06 \\
136233	2.21840574787802e-06 \\
136603	2.08877189517009e-06 \\
136973	1.96673056207519e-06 \\
137343	1.85183581080661e-06 \\
137713	1.74366799121639e-06 \\
138083	1.64183218204217e-06 \\
138453	1.54595672802182e-06 \\
138823	1.45569186177363e-06 \\
139193	1.3707084091652e-06 \\
139563	1.29069657134329e-06 \\
139933	1.2153647797053e-06 \\
140303	1.14443861842783e-06 \\
140673	1.07765981111063e-06 \\
141043	1.01478526759458e-06 \\
141413	9.55586188511326e-07 \\
141783	8.99847220514616e-07 \\
142153	8.47365664524702e-07 \\
142523	7.97950728770314e-07 \\
142893	7.51422826683612e-07 \\
143263	7.07612917039135e-07 \\
143633	6.66361882173394e-07 \\
144003	6.27519944063071e-07 \\
144373	5.90946113598889e-07 \\
144743	5.56507674165374e-07 \\
145113	5.24079693309254e-07 \\
145483	4.93544566604331e-07 \\
145853	4.64791584831037e-07 \\
146223	4.37716529855248e-07 \\
146593	4.122212918789e-07 \\
146963	3.88213511226532e-07 \\
147333	3.65606240004812e-07 \\
147703	3.44317623857116e-07 \\
148073	3.24270603480059e-07 \\
148443	3.05392632349299e-07 \\
148813	2.87615412153386e-07 \\
149183	2.70874643104602e-07 \\
149553	2.55109789404351e-07 \\
149923	2.40263858197753e-07 \\
150293	2.26283192128474e-07 \\
150663	2.13117272940266e-07 \\
151033	2.00718538290179e-07 \\
151403	1.89042207277001e-07 \\
151773	1.7804611851524e-07 \\
152143	1.6769057598065e-07 \\
152513	1.57938204403685e-07 \\
152883	1.48753813544733e-07 \\
153253	1.40104270240915e-07 \\
153623	1.31958377613817e-07 \\
153993	1.24286761882253e-07 \\
154363	1.1706176544779e-07 \\
154733	1.10257346530584e-07 \\
155103	1.03848984578381e-07 \\
155473	9.78135912266254e-08 \\
155843	9.21294261435612e-08 \\
156213	8.67760187595046e-08 \\
156583	8.1734093437813e-08 \\
156953	7.69855001414577e-08 \\
157323	7.25131476531082e-08 \\
157693	6.83009426794001e-08 \\
158063	6.4333730953603e-08 \\
158433	6.05972425016255e-08 \\
158803	5.70780395725556e-08 \\
159173	5.37634688435595e-08 \\
159543	5.06416144019362e-08 \\
159913	4.77012560007317e-08 \\
160283	4.49318270923094e-08 \\
160653	4.23233775803666e-08 \\
161023	3.98665374046203e-08 \\
161393	3.75524825679818e-08 \\
161763	3.5372903828268e-08 \\
162133	3.33199759450231e-08 \\
162503	3.13863296463879e-08 \\
162873	2.95650253168134e-08 \\
163243	2.7849527572954e-08 \\
163613	2.62336815604058e-08 \\
163983	2.471169141538e-08 \\
164353	2.32780985598424e-08 \\
164723	2.19277623281222e-08 \\
165093	2.0655841426187e-08 \\
165463	1.94577767231863e-08 \\
165833	1.83292742095276e-08 \\
166203	1.72662901198883e-08 \\
166573	1.62650159452049e-08 \\
166943	1.53218647214182e-08 \\
167313	1.44334585949757e-08 \\
167683	1.35966161107781e-08 \\
168053	1.28083409434154e-08 \\
168423	1.20658111280036e-08 \\
168793	1.13663687351107e-08 \\
169163	1.07075108224386e-08 \\
169533	1.00868797758835e-08 \\
169903	9.50225498286272e-09 \\
170273	8.95154539382048e-09 \\
170643	8.43278130657765e-09 \\
171013	7.94410731641548e-09 \\
171383	7.48377643189357e-09 \\
171753	7.05014280288907e-09 \\
172123	6.64165655805959e-09 \\
172493	6.25685758759431e-09 \\
172863	5.89437104681068e-09 \\
173233	5.55290158299471e-09 \\
173603	5.23122961615385e-09 \\
173973	4.92820578790187e-09 \\
174343	4.64274801936782e-09 \\
174713	4.37383718132622e-09 \\
175083	4.12051287534965e-09 \\
175453	3.88187093580683e-09 \\
175823	3.65705921101522e-09 \\
176193	3.44527584239529e-09 \\
176563	3.24576476806726e-09 \\
176933	3.05781405751659e-09 \\
177303	2.88075358012563e-09 \\
177673	2.71395139694874e-09 \\
178043	2.55681265048935e-09 \\
178413	2.40877695567576e-09 \\
178783	2.26931667901553e-09 \\
179153	2.13793444059363e-09 \\
179523	2.0141617818048e-09 \\
179893	1.89755772206368e-09 \\
180263	1.78770653835869e-09 \\
180633	1.68421687707365e-09 \\
181003	1.58672014416439e-09 \\
181373	1.49486878431304e-09 \\
181743	1.40833567030541e-09 \\
182113	1.32681265974099e-09 \\
182483	1.25000931827657e-09 \\
182853	1.17765230900346e-09 \\
183223	1.10948378262421e-09 \\
183593	1.04526121091908e-09 \\
183963	9.84756221011907e-10 \\
184333	9.27752930035552e-10 \\
184703	8.74048888821477e-10 \\
185073	8.23452805942537e-10 \\
185443	7.75784769757593e-10 \\
185813	7.3087513818848e-10 \\
186183	6.88564150141957e-10 \\
186553	6.48701259375883e-10 \\
186923	6.11144856943469e-10 \\
187293	5.75761271992548e-10 \\
187663	5.4242471625443e-10 \\
188033	5.1101672893239e-10 \\
188403	4.81425510567846e-10 \\
188773	4.53546145084971e-10 \\
189143	4.27279267523062e-10 \\
189513	4.0253161914805e-10 \\
189883	3.79215159274082e-10 \\
190253	3.57247231796975e-10 \\
190623	3.36549899060401e-10 \\
190993	3.17049331233221e-10 \\
191363	2.98676472443304e-10 \\
191733	2.81365875043349e-10 \\
192103	2.65056254722396e-10 \\
192473	2.49689657838559e-10 \\
192843	2.35211350396725e-10 \\
193213	2.21570151115458e-10 \\
193583	2.08717543248582e-10 \\
193953	1.96607952140937e-10 \\
194323	1.85198245628015e-10 \\
194693	1.74448178125175e-10 \\
195063	1.64319391426915e-10 \\
195433	1.54776025329539e-10 \\
195803	1.45784218030798e-10 \\
196173	1.37312217152186e-10 \\
196543	1.2932965809398e-10 \\
196913	1.21808396702505e-10 \\
197283	1.14721787625172e-10 \\
197653	1.0804462879932e-10 \\
198023	1.01753383496828e-10 \\
198393	9.58256252125977e-11 \\
198763	9.02403152203135e-11 \\
199133	8.49776360389853e-11 \\
199503	8.00191024552532e-11 \\
199873	7.53470619230256e-11 \\
200243	7.09447500746307e-11 \\
200613	6.67968458323287e-11 \\
200983	6.28884722075895e-11 \\
201353	5.92059179460591e-11 \\
201723	5.57360269048957e-11 \\
202093	5.24665866308283e-11 \\
202463	4.93859397820984e-11 \\
202833	4.64832061730647e-11 \\
203203	4.37481717519006e-11 \\
203573	4.11710665559895e-11 \\
203943	3.87426202230756e-11 \\
204313	3.64545615916256e-11 \\
204683	3.42985639889548e-11 \\
205053	3.22670778984957e-11 \\
205423	3.03528868705882e-11 \\
205793	2.85492185447822e-11 \\
206163	2.68497446498372e-11 \\
206533	2.52483034479667e-11 \\
206903	2.37392883128962e-11 \\
207273	2.23174256852587e-11 \\
207643	2.09776640502923e-11 \\
208013	1.97152849601423e-11 \\
208383	1.85257365004077e-11 \\
208753	1.74047998235949e-11 \\
209123	1.6348644660269e-11 \\
209493	1.53533297186925e-11 \\
209863	1.44155243297917e-11 \\
210233	1.35318978244925e-11 \\
210603	1.26992305560236e-11 \\
210973	1.1914635944521e-11 \\
211343	1.11752274101207e-11 \\
211713	1.04785624621684e-11 \\
212083	9.8220320765563e-12 \\
212453	9.20358234068885e-12 \\
212823	8.62060423045818e-12 \\
213193	8.07126587787366e-12 \\
213563	7.55379092609587e-12 \\
213933	7.06601444022681e-12 \\
214303	6.60643761918323e-12 \\
214673	6.17339512842818e-12 \\
215043	5.76538816687844e-12 \\
215413	5.38086242229951e-12 \\
215783	5.01848562706186e-12 \\
216153	4.67698102468717e-12 \\
216523	4.3552383921508e-12 \\
216893	4.0520919952769e-12 \\
217263	3.76637609988961e-12 \\
217633	3.49714701641801e-12 \\
218003	3.24346105529116e-12 \\
218373	3.00443003808937e-12 \\
218743	2.77911027524169e-12 \\
219113	2.56694665523582e-12 \\
219483	2.36688446619837e-12 \\
219853	2.17831308546579e-12 \\
220223	2.00073291267699e-12 \\
220593	1.83336679171475e-12 \\
220963	1.67560409991552e-12 \\
221333	1.52694523691821e-12 \\
221703	1.38694611351298e-12 \\
222073	1.25494059588505e-12 \\
222443	1.1305401059758e-12 \\
222813	1.01330055457538e-12 \\
223183	9.02833363625177e-13 \\
223553	7.98749955066569e-13 \\
223923	7.00661750840936e-13 \\
224293	6.08291195192123e-13 \\
224663	5.2119419891028e-13 \\
225033	4.39148717390481e-13 \\
225403	3.6187719487657e-13 \\
225773	2.88880031007466e-13 \\
226143	2.20268248085631e-13 \\
226513	1.55486734598753e-13 \\
226883	9.44799793956008e-14 \\
227253	3.7025937871249e-14 \\
227623	-1.71529457304587e-14 \\
227993	-6.81121825607534e-14 \\
228363	-1.16351372980716e-13 \\
228733	-1.61648472385423e-13 \\
229103	-2.0433654768226e-13 \\
229473	-2.44582132324922e-13 \\
229843	-2.82551759767102e-13 \\
230213	-3.18245430008801e-13 \\
230583	-3.51885187654943e-13 \\
230953	-3.83637566159223e-13 \\
231323	-4.13669098975333e-13 \\
};
\addplot [line width=0.01pt, blue, forget plot]
table [row sep=\\]{%
1190	2.58725878419534 \\
2380	2.15227634443394 \\
3570	1.78658605761926 \\
4760	1.49636247105782 \\
5950	1.26261261004458 \\
7140	1.06499777961715 \\
8326	0.892976574345495 \\
9506	0.753034216760218 \\
10686	0.632184524260899 \\
11866	0.531281271170288 \\
13046	0.446583363472156 \\
14226	0.374583675957227 \\
15406	0.314607386169868 \\
16586	0.263761874353383 \\
17766	0.219156493707025 \\
18946	0.182952975313241 \\
20126	0.150476934837809 \\
21306	0.12362863814551 \\
22486	0.101616160761967 \\
23666	0.0823683824995704 \\
24846	0.0664406987174117 \\
26026	0.05537994674247 \\
27206	0.0463512318417369 \\
28386	0.038766742820167 \\
29566	0.0327147928813122 \\
30746	0.0280666180971017 \\
31926	0.0242735236569611 \\
33106	0.0209235979920107 \\
34286	0.0178982928276766 \\
35466	0.0159645882174454 \\
36646	0.0141817003650932 \\
37826	0.0127623793738052 \\
39006	0.0116296952364778 \\
40186	0.0106007727923862 \\
41366	0.00965933780734668 \\
42546	0.00878917679251617 \\
43726	0.00798442545334288 \\
44906	0.00723909533429701 \\
46086	0.00654879535239206 \\
47266	0.00596176400014059 \\
48446	0.0054454703077072 \\
49626	0.00496897090332543 \\
50806	0.00455006517546908 \\
51986	0.0041641713193184 \\
53166	0.00380725430155676 \\
54346	0.00350251871397528 \\
55526	0.00324217618623118 \\
56706	0.00300352440748547 \\
57886	0.00278345985765954 \\
59066	0.00258137546871073 \\
60246	0.00239405196152548 \\
61426	0.00222027382522522 \\
62606	0.00205894448426364 \\
63786	0.00190907151702635 \\
64966	0.00176975443780097 \\
66146	0.00164130374264021 \\
67326	0.00152582468338691 \\
68506	0.00141858764247071 \\
69686	0.0013189413516706 \\
70866	0.00122629516442718 \\
72046	0.00114011143452081 \\
73226	0.00105989967906567 \\
74406	0.000985211599790314 \\
75586	0.000915636772460648 \\
76766	0.000850798886193938 \\
77946	0.000790503263263631 \\
79126	0.000735396054660609 \\
80306	0.000684086759398628 \\
81486	0.000636284310583657 \\
82666	0.000591726692279304 \\
83846	0.000550174769824108 \\
85026	0.000511409626747727 \\
86206	0.000475230440426688 \\
87386	0.000442555400642364 \\
88566	0.000412342784580466 \\
89746	0.000384235283647449 \\
90926	0.000358076234783955 \\
92106	0.000333722067901665 \\
93286	0.000311040953869635 \\
94466	0.000289911673926224 \\
95646	0.000270222625417815 \\
96826	0.000251870941223642 \\
98006	0.000234761706811748 \\
99186	0.000218807261678 \\
100366	0.000203926574065094 \\
101546	0.000190044679597268 \\
102726	0.000177092175891902 \\
103906	0.00016500476639042 \\
105086	0.00015372284762466 \\
106266	0.000143191134946541 \\
107446	0.000133358322434718 \\
108626	0.000124176773259321 \\
109806	0.000115602237273971 \\
110986	0.000107593593011379 \\
112166	0.000100112611612191 \\
113346	9.31237405111807e-05 \\
114526	8.65939049667919e-05 \\
115706	8.04923257354262e-05 \\
116886	7.47903513884651e-05 \\
118066	6.94613039316505e-05 \\
119246	6.45726170816485e-05 \\
120426	6.01711779856795e-05 \\
121606	5.60770924579734e-05 \\
122707	5.22679529594283e-05 \\
123097	4.87232302063823e-05 \\
123487	4.54239557390324e-05 \\
123877	4.23525849559958e-05 \\
124267	3.94928819315021e-05 \\
124657	3.68298152360058e-05 \\
125047	3.43494630369867e-05 \\
125437	3.20389264614751e-05 \\
125827	2.98862503658182e-05 \\
126217	2.78803507763326e-05 \\
126607	2.60109483556925e-05 \\
126997	2.4270755399125e-05 \\
127387	2.26487196412495e-05 \\
127777	2.11365396597385e-05 \\
128167	1.972657793331e-05 \\
128557	1.84117451906274e-05 \\
128947	1.71854592269094e-05 \\
129337	1.60416071099956e-05 \\
129727	1.497451039284e-05 \\
130117	1.39788930592077e-05 \\
130507	1.30498519656519e-05 \\
130897	1.21828295687187e-05 \\
131287	1.13735887431465e-05 \\
131677	1.06181895208612e-05 \\
132067	9.91296759011995e-06 \\
132457	9.2545144171341e-06 \\
132847	8.63965885683493e-06 \\
133237	8.06545013648519e-06 \\
133627	7.5291421027246e-06 \\
134017	7.0281786355153e-06 \\
134407	6.56018013650561e-06 \\
134797	6.12293101009964e-06 \\
135187	5.71436806068304e-06 \\
135577	5.33256973628271e-06 \\
135967	4.97574615437868e-06 \\
136357	4.64222985158136e-06 \\
136747	4.33046720083041e-06 \\
137137	4.03901044915278e-06 \\
137527	3.76651032563124e-06 \\
137917	3.51170918189148e-06 \\
138307	3.27343462136476e-06 \\
138697	3.05059358468585e-06 \\
139087	2.84216685492167e-06 \\
139477	2.64720395415363e-06 \\
139867	2.46481840054935e-06 \\
140257	2.29418330238707e-06 \\
140647	2.13452726133267e-06 \\
141037	1.9951824468678e-06 \\
141427	1.87116220312644e-06 \\
141817	1.75514025174683e-06 \\
142207	1.64655936557567e-06 \\
142597	1.54491240400567e-06 \\
142987	1.4497310946715e-06 \\
143377	1.36058210670198e-06 \\
143767	1.27706394831284e-06 \\
144157	1.19880423510299e-06 \\
144547	1.12545725317048e-06 \\
144937	1.05670177591444e-06 \\
145327	9.9223909910684e-07 \\
145717	9.31791266256532e-07 \\
146107	8.75099461894457e-07 \\
146497	8.21922551630383e-07 \\
146887	7.72035752771849e-07 \\
147277	7.25229420794893e-07 \\
147667	6.81307939232045e-07 \\
148057	6.40088702930086e-07 \\
148447	6.01401184574524e-07 \\
148837	5.65086075987598e-07 \\
149227	5.30994498593174e-07 \\
149617	4.98987275276974e-07 \\
150007	4.68934257980003e-07 \\
150397	4.40713707139384e-07 \\
150787	4.1421171764755e-07 \\
151177	3.89321687277455e-07 \\
151567	3.65943824964798e-07 \\
151957	3.43984693784716e-07 \\
152347	3.2335678756823e-07 \\
152737	3.03978138660455e-07 \\
153127	2.85771951546998e-07 \\
153517	2.68666264013895e-07 \\
153907	2.52593631233644e-07 \\
154297	2.37490831722642e-07 \\
154687	2.23298593171606e-07 \\
155077	2.09961338759612e-07 \\
155467	1.97426948234103e-07 \\
155857	1.85646536920991e-07 \\
156247	1.74574248945625e-07 \\
156637	1.64167064276022e-07 \\
157027	1.54384619088788e-07 \\
157417	1.45189037015214e-07 \\
157807	1.36544772266767e-07 \\
158197	1.28418463307689e-07 \\
158587	1.20778795187348e-07 \\
158977	1.13596371253966e-07 \\
159367	1.0684359386115e-07 \\
159757	1.00494551125152e-07 \\
160147	9.45249130079873e-08 \\
160537	8.89118323965654e-08 \\
160927	8.36338529541791e-08 \\
161317	7.86708237443534e-08 \\
161707	7.40038181845648e-08 \\
162097	6.96150581624977e-08 \\
162487	6.54878439254603e-08 \\
162877	6.16064870229138e-08 \\
163267	5.79562491331842e-08 \\
163657	5.45232826110187e-08 \\
164047	5.12945774189255e-08 \\
164437	4.82579081140244e-08 \\
164827	4.54017874407242e-08 \\
165217	4.27154205895341e-08 \\
165607	4.01886629530779e-08 \\
165997	3.78119804911314e-08 \\
166387	3.55764130377523e-08 \\
166777	3.34735392182317e-08 \\
167167	3.14954431424042e-08 \\
167557	2.96346847616924e-08 \\
167947	2.78842703371751e-08 \\
168337	2.62376255721897e-08 \\
168727	2.46885702992472e-08 \\
169117	2.32312946657487e-08 \\
169507	2.18603368185022e-08 \\
169897	2.05705614764184e-08 \\
170287	1.93571410012083e-08 \\
170677	1.82155360795022e-08 \\
171067	1.71414787364377e-08 \\
171457	1.61309562374257e-08 \\
171847	1.51801951009389e-08 \\
172237	1.42856474427688e-08 \\
172627	1.34439767096595e-08 \\
173017	1.26520454113432e-08 \\
173407	1.19069033521768e-08 \\
173797	1.1205775862777e-08 \\
174187	1.05460536414803e-08 \\
174577	9.92528281784644e-09 \\
174967	9.34115551576298e-09 \\
175357	8.79150124921679e-09 \\
175747	8.27427859562135e-09 \\
176137	7.78756764630018e-09 \\
176527	7.3295622904368e-09 \\
176917	6.89856410884815e-09 \\
177307	6.49297549060179e-09 \\
177697	6.11129341576699e-09 \\
178087	5.7521045704334e-09 \\
178477	5.41407918497328e-09 \\
178867	5.09596642661592e-09 \\
179257	4.79658984753328e-09 \\
179647	4.51484255536982e-09 \\
180037	4.24968327195074e-09 \\
180427	4.00013266954602e-09 \\
180817	3.76526920753406e-09 \\
181207	3.54422602377724e-09 \\
181597	3.33618765946397e-09 \\
181987	3.14038678395079e-09 \\
182377	2.95610158573822e-09 \\
182767	2.78265288589097e-09 \\
183157	2.61940186208065e-09 \\
183547	2.46574727302828e-09 \\
183937	2.32112357112513e-09 \\
184327	2.18499862647548e-09 \\
184717	2.05687161747292e-09 \\
185107	1.93627153199927e-09 \\
185497	1.82275489146733e-09 \\
185887	1.7159047516202e-09 \\
186277	1.61532870412984e-09 \\
186667	1.52065721126249e-09 \\
187057	1.4315430507672e-09 \\
187447	1.34765909542978e-09 \\
187837	1.26869759142778e-09 \\
188227	1.19436904810755e-09 \\
188617	1.12440107224998e-09 \\
189007	1.05853703580294e-09 \\
189397	9.96535909347784e-10 \\
189787	9.38170485742518e-10 \\
190177	8.83227158077204e-10 \\
190567	8.31504698428631e-10 \\
190957	7.82814091326856e-10 \\
191347	7.36977034954123e-10 \\
191737	6.93825941144866e-10 \\
192127	6.53203102718436e-10 \\
192517	6.14960193878744e-10 \\
192907	5.78957548569292e-10 \\
193297	5.45063383317057e-10 \\
193687	5.13153963765944e-10 \\
194077	4.83112994054125e-10 \\
194467	4.54830839657916e-10 \\
194857	4.28204194324877e-10 \\
195247	4.03136135584958e-10 \\
195637	3.79534903505174e-10 \\
196027	3.57315010912629e-10 \\
196417	3.36395022948466e-10 \\
196807	3.16698889335498e-10 \\
197197	2.98154945177487e-10 \\
197587	2.8069552238108e-10 \\
197977	2.64257227211573e-10 \\
198367	2.48780052114483e-10 \\
198757	2.34207819804766e-10 \\
199147	2.20487461621843e-10 \\
199537	2.07569073040759e-10 \\
199927	1.95405747138722e-10 \\
200317	1.83953352550503e-10 \\
200707	1.73170144890378e-10 \\
201097	1.63016766752122e-10 \\
201487	1.53456636287075e-10 \\
201877	1.44455059025717e-10 \\
202267	1.35979227877669e-10 \\
202657	1.27998278642849e-10 \\
203047	1.20483512056069e-10 \\
203437	1.1340750560862e-10 \\
203827	1.06744502126332e-10 \\
204217	1.00470576303024e-10 \\
204607	9.45628575443891e-11 \\
204997	8.89998075237486e-11 \\
205387	8.37616087601134e-11 \\
205777	7.88288878617038e-11 \\
206167	7.41840477935796e-11 \\
206557	6.98101576546151e-11 \\
206947	6.56915077890119e-11 \\
207337	6.18129991636351e-11 \\
207727	5.81608095018282e-11 \\
208117	5.4721449593842e-11 \\
208507	5.14826514752542e-11 \\
208897	4.84329243377601e-11 \\
209287	4.55607218619036e-11 \\
209677	4.28561630627655e-11 \\
210067	4.03091449108217e-11 \\
210457	3.79106190884215e-11 \\
210847	3.56518148336704e-11 \\
211237	3.35246830296398e-11 \\
211627	3.15215076263087e-11 \\
212017	2.96350166628656e-11 \\
212407	2.7858326756558e-11 \\
212797	2.61853316807503e-11 \\
213187	2.46096476530511e-11 \\
213577	2.31257235583371e-11 \\
213967	2.17283413483926e-11 \\
214357	2.04123384861532e-11 \\
214747	1.91728855014617e-11 \\
215137	1.80054304799171e-11 \\
215527	1.69061986632357e-11 \\
215917	1.58708046704703e-11 \\
216307	1.48958068102445e-11 \\
216697	1.39774858354258e-11 \\
217087	1.31125665880916e-11 \\
217477	1.22979959549241e-11 \\
217867	1.15308318449081e-11 \\
218257	1.08082987004821e-11 \\
218647	1.01278430086893e-11 \\
219037	9.48696676772443e-12 \\
219427	8.8832829980845e-12 \\
219817	8.31484880947642e-12 \\
220207	7.77933273354847e-12 \\
220597	7.27506943576373e-12 \\
220987	6.80000500352662e-12 \\
221377	6.35275165805638e-12 \\
221767	5.9313665090599e-12 \\
222157	5.53451728890764e-12 \\
222547	5.16070519651635e-12 \\
222937	4.80865347540771e-12 \\
223327	4.47708536910341e-12 \\
223717	4.16477963227635e-12 \\
224107	3.87057053075068e-12 \\
224497	3.59340335265301e-12 \\
224887	3.3325009418661e-12 \\
225277	3.08664205306286e-12 \\
225667	2.85516055242852e-12 \\
226057	2.6370572392409e-12 \\
226447	2.43166597968525e-12 \\
226837	2.23815410649308e-12 \\
227227	2.0559665081521e-12 \\
227617	1.88421500624258e-12 \\
228007	1.72256653385716e-12 \\
228397	1.57029944602982e-12 \\
228787	1.42674760894579e-12 \\
229177	1.29168897800014e-12 \\
229567	1.1644574193781e-12 \\
229957	1.04455333271858e-12 \\
230347	9.31532628811738e-13 \\
230737	8.2528428535511e-13 \\
231127	7.24920123928996e-13 \\
231517	6.30551166835858e-13 \\
231907	5.41677813714614e-13 \\
232297	4.57966997657877e-13 \\
232687	3.79030140607028e-13 \\
233077	3.04645197957143e-13 \\
233467	2.34756658556989e-13 \\
233857	1.68698388591793e-13 \\
234247	1.06692432666478e-13 \\
234637	4.81281681175005e-14 \\
235027	-6.88338275267597e-15 \\
235417	-5.89528426075958e-14 \\
235807	-1.07802655691103e-13 \\
236197	-1.53876911213047e-13 \\
236587	-1.9728663147589e-13 \\
236977	-2.38087327630865e-13 \\
237367	-2.76667577736589e-13 \\
237757	-3.12971870641832e-13 \\
238147	-3.47111228649055e-13 \\
};
\addplot [line width=0.01pt, blue, forget plot]
table [row sep=\\]{%
1190	2.00844481238494 \\
2380	1.65334005537688 \\
3570	1.36944297311608 \\
4760	1.13604559174763 \\
5942	0.937464522100174 \\
7092	0.777454158901303 \\
8242	0.644059445649168 \\
9392	0.531434506731886 \\
10542	0.445825414812687 \\
11692	0.374656397562241 \\
12842	0.310474601641202 \\
13992	0.255610050661959 \\
15142	0.209429499199479 \\
16292	0.17249197416662 \\
17442	0.142184849390874 \\
18592	0.116847107563467 \\
19742	0.0961176410045594 \\
20892	0.0801276192293758 \\
22042	0.0700476073207264 \\
23192	0.0628857866835122 \\
24342	0.0564272967583235 \\
25492	0.0507341948296875 \\
26642	0.0457602160917045 \\
27792	0.0412175029378246 \\
28942	0.0371437103746088 \\
30092	0.0335552890012238 \\
31242	0.0302764607682067 \\
32392	0.0274334011516178 \\
33542	0.0248424920460112 \\
34692	0.0224760958911637 \\
35842	0.0203075524011841 \\
36992	0.0182830771780219 \\
38142	0.0163874987682797 \\
39292	0.0147139150403676 \\
40442	0.0133411392116988 \\
41592	0.0120558223532894 \\
42742	0.0108535157450665 \\
43892	0.00973356818613857 \\
45042	0.00872983362648971 \\
46192	0.00788317648974651 \\
47342	0.00709133249398092 \\
48492	0.006350398065115 \\
49642	0.005687605823746 \\
50792	0.00516233393808491 \\
51942	0.00469424594269885 \\
53092	0.00428020813998897 \\
54242	0.00389797924454627 \\
55392	0.00354501970224191 \\
56542	0.00321808756564579 \\
57692	0.00292061412825984 \\
58842	0.00265149934864484 \\
59992	0.00244965440771244 \\
61142	0.00227217709856081 \\
62292	0.0021080199303265 \\
63442	0.00195578399699747 \\
64592	0.00181460083673507 \\
65742	0.00168345551638915 \\
66892	0.0015619562861004 \\
68042	0.00144945262071533 \\
69192	0.00134633921953792 \\
70342	0.00125043981454565 \\
71492	0.00116159209124428 \\
72642	0.00107907396811791 \\
73792	0.00100219873486768 \\
74942	0.000930781354578525 \\
76092	0.000868274205908348 \\
77242	0.00081189513411617 \\
78392	0.000759101397922179 \\
79542	0.000709765079837388 \\
80692	0.000663741736184276 \\
81842	0.000620704863827259 \\
82992	0.000580612512893708 \\
84142	0.000543112211925878 \\
85292	0.000508015923552652 \\
86442	0.000475285885222387 \\
87592	0.000444578904989057 \\
88742	0.000415930949969223 \\
89892	0.000389107586036608 \\
91042	0.00036414652942246 \\
92192	0.000340719313797477 \\
93342	0.0003188306400011 \\
94492	0.000298358783972996 \\
95642	0.000279146244180128 \\
96792	0.000261182301226737 \\
97942	0.000244352691424199 \\
99092	0.000228583221808132 \\
100242	0.000213848666253125 \\
101392	0.000200091158733651 \\
102542	0.000187211703878309 \\
103692	0.000175148009631809 \\
104842	0.000163830630342698 \\
105992	0.000153254753247345 \\
107142	0.000143304020593316 \\
108292	0.000134026358851125 \\
109442	0.000125324619102862 \\
110592	0.000117166038232586 \\
111742	0.000109560964545563 \\
112892	0.000102433216375375 \\
114042	9.57543023119012e-05 \\
115192	8.94876379151643e-05 \\
116342	8.3636201540116e-05 \\
117492	7.81368696540219e-05 \\
118642	7.30023783271916e-05 \\
119792	6.81595938274993e-05 \\
120942	6.36382857226447e-05 \\
122092	5.93977608224772e-05 \\
123242	5.54241905361685e-05 \\
124392	5.17142236606261e-05 \\
125542	4.82245781723845e-05 \\
126692	4.49590760818186e-05 \\
127842	4.18988135477694e-05 \\
128992	3.90240342607595e-05 \\
130142	3.6341427693598e-05 \\
131292	3.38212885200861e-05 \\
132442	3.14588652957082e-05 \\
133592	2.923498877061e-05 \\
134742	2.71470223086046e-05 \\
135892	2.51911069962807e-05 \\
137042	2.3358865923695e-05 \\
138192	2.16451883115476e-05 \\
139342	2.00347373957577e-05 \\
140492	1.85254399071577e-05 \\
141642	1.71177209354334e-05 \\
142792	1.58955404532257e-05 \\
143942	1.48563875553842e-05 \\
145092	1.38903038948701e-05 \\
146242	1.29846524875021e-05 \\
147392	1.21413875600052e-05 \\
148542	1.1351348634725e-05 \\
149692	1.06114186558259e-05 \\
150842	9.92264168087065e-06 \\
151992	9.28085708878923e-06 \\
153142	8.68110930257737e-06 \\
154292	8.12006044759173e-06 \\
155442	7.5982092810345e-06 \\
156592	7.1089422885029e-06 \\
157742	6.650155725485e-06 \\
158892	6.23047113440522e-06 \\
160042	5.84383600993332e-06 \\
161192	5.48258276467095e-06 \\
162342	5.14302754844875e-06 \\
163492	4.82552058750363e-06 \\
164642	4.528667244319e-06 \\
165792	4.25019016536687e-06 \\
166942	3.98919816790855e-06 \\
168092	3.74492118954128e-06 \\
169242	3.51570512185928e-06 \\
170392	3.30083521365987e-06 \\
171542	3.09960107469642e-06 \\
172692	2.9107126077732e-06 \\
173842	2.7339057319109e-06 \\
174992	2.56779590407374e-06 \\
176142	2.41202546302777e-06 \\
177292	2.26595652824679e-06 \\
178442	2.1286872555204e-06 \\
179592	1.99992308191899e-06 \\
180742	1.87930775180645e-06 \\
181892	1.76579309008051e-06 \\
183042	1.65911763366777e-06 \\
184192	1.55914486715814e-06 \\
185342	1.4652694373396e-06 \\
186492	1.37718077086735e-06 \\
187642	1.29447092184209e-06 \\
188792	1.21682610071394e-06 \\
189942	1.14383611121704e-06 \\
191092	1.07525217807325e-06 \\
192242	1.01085790676914e-06 \\
193392	9.50370420027546e-07 \\
194542	8.93536361290526e-07 \\
195692	8.40144284253164e-07 \\
196842	7.90043434950594e-07 \\
197992	7.42936118236504e-07 \\
199142	6.98633934315129e-07 \\
200292	6.57004252391236e-07 \\
201442	6.17846227624064e-07 \\
202592	5.81040981939562e-07 \\
203742	5.46437475656969e-07 \\
204892	5.1396505290624e-07 \\
206042	4.83430332542945e-07 \\
207192	4.54722429987786e-07 \\
208342	4.27726335350265e-07 \\
209492	4.02380345043696e-07 \\
210642	3.78516035959287e-07 \\
211792	3.56056180539355e-07 \\
212942	3.34958827064025e-07 \\
214092	3.15121495064741e-07 \\
215242	2.96477250694149e-07 \\
216392	2.78934217001758e-07 \\
217542	2.62439779741275e-07 \\
218692	2.46944170334551e-07 \\
219842	2.32351004880726e-07 \\
220992	2.18631312109974e-07 \\
222142	2.05714416667835e-07 \\
223292	1.93570631468187e-07 \\
224442	1.82155761641045e-07 \\
225592	1.71411737637239e-07 \\
226742	1.61306953017082e-07 \\
227892	1.51798503045253e-07 \\
229042	1.42853822882039e-07 \\
230192	1.3444431778975e-07 \\
231342	1.26529338617676e-07 \\
232492	1.19075820037562e-07 \\
233642	1.12073321234529e-07 \\
234792	1.05476801626203e-07 \\
235942	9.92809116584503e-08 \\
237092	9.34523026185907e-08 \\
238242	8.79604427073133e-08 \\
239392	8.27883665510676e-08 \\
240542	7.79251823068705e-08 \\
241692	7.33449818657306e-08 \\
242842	6.90411255943602e-08 \\
243992	6.49878606484577e-08 \\
245142	6.11734221611648e-08 \\
246292	5.75817846182325e-08 \\
247442	5.42019126803517e-08 \\
248592	5.10274766329921e-08 \\
249742	4.80357485455052e-08 \\
250892	4.52193996691719e-08 \\
252042	4.25682172289221e-08 \\
253192	4.00735582495315e-08 \\
254342	3.77247103555689e-08 \\
255492	3.5513709306656e-08 \\
256642	3.34329076090434e-08 \\
257792	3.1478014639319e-08 \\
258942	2.96341988725857e-08 \\
260092	2.79004019332341e-08 \\
261242	2.62701896347473e-08 \\
262392	2.47329280944619e-08 \\
263542	2.32860759563636e-08 \\
264692	2.19234634335486e-08 \\
265842	2.0641792886078e-08 \\
266992	1.94367261174833e-08 \\
268142	1.83007144438641e-08 \\
269292	1.72310350987637e-08 \\
270442	1.62238539824777e-08 \\
271592	1.52761244764932e-08 \\
272742	1.43844055000564e-08 \\
273892	1.35445246685073e-08 \\
275042	1.27536882166268e-08 \\
276192	1.20086881549852e-08 \\
277342	1.13081264352388e-08 \\
278492	1.06478461048276e-08 \\
279642	1.0026415753206e-08 \\
280792	9.44228290000737e-09 \\
281942	8.89154394556613e-09 \\
283092	8.37288532951064e-09 \\
284242	7.88463982992482e-09 \\
285392	7.42516298446461e-09 \\
286542	6.99215907395967e-09 \\
287692	6.58499688110936e-09 \\
288842	6.20132756257163e-09 \\
289992	5.83987047608758e-09 \\
291142	5.4990608733263e-09 \\
292292	5.17844528369338e-09 \\
293442	4.87666335091674e-09 \\
294592	4.59275190101138e-09 \\
295742	4.32519453585911e-09 \\
296892	4.07338818231295e-09 \\
298042	3.83597031916594e-09 \\
299192	3.61292024075865e-09 \\
300342	3.40252470643065e-09 \\
301492	3.20458570790549e-09 \\
302642	3.01798525059738e-09 \\
303792	2.8426174747409e-09 \\
304942	2.67687272348383e-09 \\
306092	2.52119009003238e-09 \\
307242	2.374381191661e-09 \\
308392	2.23604795790422e-09 \\
309542	2.10616440954325e-09 \\
310692	1.98355099012559e-09 \\
311842	1.86818016612023e-09 \\
312992	1.75959063986042e-09 \\
314142	1.65726810053002e-09 \\
315292	1.56095181225169e-09 \\
316442	1.47006712358788e-09 \\
317592	1.3846432889153e-09 \\
318742	1.30417854293796e-09 \\
319892	1.22840698724147e-09 \\
321042	1.15713705284293e-09 \\
322192	1.08984515856392e-09 \\
323342	1.02662572887269e-09 \\
324492	9.66884294850701e-10 \\
325642	9.10581388069431e-10 \\
326792	8.57548254629137e-10 \\
327942	8.07660216484152e-10 \\
329092	7.60810470179507e-10 \\
330242	7.16579962034558e-10 \\
331392	6.74914901743762e-10 \\
332542	6.35682451122221e-10 \\
333692	5.98685934161125e-10 \\
334842	5.63895208305354e-10 \\
335992	5.31112986923432e-10 \\
337142	5.00211039256016e-10 \\
338292	4.71157668435751e-10 \\
339442	4.43732273147646e-10 \\
340592	4.17926471207863e-10 \\
341742	3.93622245908887e-10 \\
342892	3.70784680736591e-10 \\
344042	3.49198225890746e-10 \\
345192	3.28905180868588e-10 \\
346342	3.0977526099818e-10 \\
347492	2.91752011438717e-10 \\
348642	2.7476593222886e-10 \\
349792	2.58760846083561e-10 \\
350942	2.43718045744856e-10 \\
352092	2.2955704004346e-10 \\
353242	2.16191009538846e-10 \\
354392	2.03606353998964e-10 \\
355542	1.91762217216507e-10 \\
356692	1.80605808086653e-10 \\
357842	1.70067460114609e-10 \\
358992	1.60168489582446e-10 \\
360142	1.50825685274469e-10 \\
361292	1.42032108296775e-10 \\
362442	1.33751953956818e-10 \\
363592	1.25949473073206e-10 \\
364742	1.18600407272851e-10 \\
365892	1.11693709836658e-10 \\
367042	1.05173036946127e-10 \\
368192	9.9024455302299e-11 \\
369342	9.32470212156034e-11 \\
370492	8.77957151423914e-11 \\
371642	8.26685941923699e-11 \\
372792	7.78293540726338e-11 \\
373942	7.32747751364116e-11 \\
375092	6.89919787966176e-11 \\
376242	6.49445497380441e-11 \\
377392	6.11433681463325e-11 \\
378542	5.75563485760711e-11 \\
379692	5.41800493358835e-11 \\
380842	5.09801090231576e-11 \\
381992	4.7969128669223e-11 \\
383142	4.51439996496106e-11 \\
384292	4.24773549667634e-11 \\
385442	3.99643096393731e-11 \\
386592	3.75963149501501e-11 \\
387742	3.53658768936782e-11 \\
388892	3.32692762228248e-11 \\
390042	3.12979642203004e-11 \\
391192	2.94372859421799e-11 \\
392342	2.76826894740623e-11 \\
393492	2.60270138774388e-11 \\
394642	2.44689268846798e-11 \\
395792	2.300232226915e-11 \\
396942	2.16205386927015e-11 \\
398092	2.0315304993801e-11 \\
399242	1.90888971296488e-11 \\
400392	1.79329329164091e-11 \\
401542	1.68441927073104e-11 \\
402692	1.58134616512484e-11 \\
403842	1.48450141068679e-11 \\
404992	1.39330214032896e-11 \\
406142	1.30755961613715e-11 \\
407292	1.22682419778641e-11 \\
408442	1.15061293826102e-11 \\
409592	1.07882036637363e-11 \\
410742	1.01119668194372e-11 \\
411892	9.47542044826832e-12 \\
413042	8.87534490345843e-12 \\
414192	8.30951973895822e-12 \\
415342	7.7760020644746e-12 \\
416492	7.27540250267111e-12 \\
417642	6.80305811684434e-12 \\
418792	6.35691499439872e-12 \\
419942	5.93725069109041e-12 \\
421092	5.54301049504602e-12 \\
422242	5.17158538215767e-12 \\
423392	4.82097695098105e-12 \\
424542	4.48974191158413e-12 \\
425692	4.17837986432801e-12 \\
426842	3.88450382970973e-12 \\
427992	3.60783625197314e-12 \\
429142	3.34721139694238e-12 \\
430292	3.10240722001254e-12 \\
431442	2.87092571937819e-12 \\
432592	2.65359956230782e-12 \\
433742	2.44881892541571e-12 \\
434892	2.25491847416492e-12 \\
436042	2.07212025316039e-12 \\
437192	1.90003568434349e-12 \\
438342	1.73805414505068e-12 \\
439492	1.58545399031595e-12 \\
440642	1.44162459747577e-12 \\
441792	1.30667698883258e-12 \\
442942	1.17905685215192e-12 \\
444092	1.05876418743378e-12 \\
445242	9.4529939431709e-13 \\
446392	8.38551450499381e-13 \\
447542	7.38020755619573e-13 \\
448692	6.43152198165353e-13 \\
449842	5.53890266985491e-13 \\
450992	4.69679850567672e-13 \\
452142	3.90520948911899e-13 \\
453292	3.15858450505857e-13 \\
454442	2.45470310744622e-13 \\
455592	1.79134485023269e-13 \\
456742	1.1651790643441e-13 \\
457892	5.77315972805081e-14 \\
459042	2.27595720048157e-15 \\
460192	-4.98490138056695e-14 \\
};
\addplot [line width=0.01pt, blue, forget plot]
table [row sep=\\]{%
1190	2.37662764520576 \\
2380	1.89203118311225 \\
3570	1.52144960604196 \\
4760	1.23951110340744 \\
5950	1.00901133757787 \\
7130	0.83221505309802 \\
8310	0.689828503199825 \\
9490	0.573346047747917 \\
10670	0.472693897665519 \\
11850	0.390234545662644 \\
13030	0.32299469402552 \\
14210	0.267929669689773 \\
15390	0.223355592765433 \\
16570	0.18961810935365 \\
17750	0.163492096310956 \\
18930	0.140335891095933 \\
20110	0.120587241055863 \\
21290	0.10409347342165 \\
22470	0.0905085839385605 \\
23650	0.0793670418101067 \\
24830	0.0693545206151702 \\
26010	0.0608971353347165 \\
27190	0.0539174773292814 \\
28370	0.0477308836797416 \\
29550	0.0429975734495726 \\
30730	0.0390659554735286 \\
31910	0.0353918494855318 \\
33090	0.0320107155254634 \\
34270	0.0289066587911249 \\
35450	0.0259956692794809 \\
36630	0.0234779876008027 \\
37810	0.0211185423373428 \\
38990	0.0189340899571342 \\
40170	0.0168929945989018 \\
41350	0.0149740496918983 \\
42530	0.0131727151786774 \\
43710	0.0115794103582337 \\
44890	0.0103920355657024 \\
46070	0.00932847483509625 \\
47250	0.00835683979826229 \\
48430	0.00744983163452545 \\
49610	0.00663167432709277 \\
50790	0.00587828333616014 \\
51970	0.00522374703623502 \\
53150	0.00466311462014191 \\
54330	0.00414592472750069 \\
55510	0.00384508078067519 \\
56690	0.00356602624246155 \\
57870	0.00332013577514695 \\
59050	0.00309088751906822 \\
60230	0.00287704373795766 \\
61410	0.00267882536541936 \\
62590	0.00249395262004282 \\
63770	0.0023213830832316 \\
64950	0.00216022932543569 \\
66130	0.00200967553152925 \\
67310	0.00186987106993775 \\
68490	0.00174242335128194 \\
69670	0.00162380839486581 \\
70850	0.00151403890611729 \\
72030	0.00141148696383847 \\
73210	0.00131614644094225 \\
74390	0.00123086460139465 \\
75570	0.00115121475180613 \\
76750	0.00107677791571442 \\
77930	0.00100717454302518 \\
79110	0.000942058764330067 \\
80290	0.000881114315377185 \\
81470	0.000824051175070561 \\
82650	0.000770602733026426 \\
83830	0.000720523377703175 \\
85010	0.000673586423422357 \\
86190	0.000629582313375787 \\
87370	0.000588317049728115 \\
88550	0.000549610812485168 \\
89730	0.000513296736804891 \\
90910	0.000479219824539301 \\
92090	0.000447235970490512 \\
93270	0.000417211087494429 \\
94450	0.000390054166557963 \\
95630	0.000364811961686751 \\
96810	0.000341224037580135 \\
97990	0.00031917591516134 \\
99170	0.00029856182744481 \\
100350	0.000279283912480721 \\
101530	0.000261251538281049 \\
102710	0.00024438069850713 \\
103890	0.000228593467828331 \\
105070	0.000213817509796654 \\
106250	0.000199985631197885 \\
107430	0.0001870353776533 \\
108610	0.000174908665939111 \\
109790	0.000163551449067545 \\
110970	0.000152913410679911 \\
112150	0.0001429476857287 \\
113330	0.000133610604789114 \\
114510	0.000124861459661285 \\
115690	0.000116662288196512 \\
116870	0.000108977676515298 \\
118050	0.000101774576993563 \\
119230	9.50221405685014e-05 \\
120410	8.86915620764706e-05 \\
121044	8.27559374648734e-05 \\
121444	7.72744612767551e-05 \\
121844	7.21587396582124e-05 \\
122244	6.73714717819851e-05 \\
122644	6.28909443482484e-05 \\
123044	5.86969500016576e-05 \\
123444	5.47706589015817e-05 \\
123844	5.10945198331503e-05 \\
124244	4.76521698734023e-05 \\
124644	4.44283508141408e-05 \\
125044	4.14088317413075e-05 \\
125444	3.85803372842664e-05 \\
125844	3.59304810993999e-05 \\
126244	3.34477041815107e-05 \\
126644	3.11212176382636e-05 \\
127044	2.89409495851656e-05 \\
127444	2.69392707281102e-05 \\
127844	2.51768369439742e-05 \\
128244	2.35320505738157e-05 \\
128644	2.19966691709095e-05 \\
129044	2.05631049184363e-05 \\
129444	1.92243426585526e-05 \\
129844	1.79738867941981e-05 \\
130244	1.68057160106483e-05 \\
130644	1.57142434430435e-05 \\
131044	1.46942812927486e-05 \\
131444	1.37410091622936e-05 \\
131844	1.28499455455677e-05 \\
132244	1.20169220192978e-05 \\
132644	1.12380597770967e-05 \\
133044	1.05097482118177e-05 \\
133444	9.82862530779505e-06 \\
133844	9.19155964074214e-06 \\
134244	8.59563381933048e-06 \\
134644	8.03812922517411e-06 \\
135044	7.51651192787417e-06 \\
135444	7.02841966832013e-06 \\
135844	6.57164981809899e-06 \\
136244	6.14414822946996e-06 \\
136644	5.74399890534982e-06 \\
137044	5.36941442014216e-06 \\
137444	5.01872703506656e-06 \\
137844	4.6903804499232e-06 \\
138244	4.38292214710589e-06 \\
138644	4.0949962781256e-06 \\
139044	3.82533705650667e-06 \\
139444	3.57276261497841e-06 \\
139844	3.33616929681924e-06 \\
140244	3.11452634582654e-06 \\
140644	2.90687096871078e-06 \\
141044	2.71230374082609e-06 \\
141444	2.52998433203366e-06 \\
141844	2.3591275284951e-06 \\
142244	2.19899952980107e-06 \\
142644	2.0489145015623e-06 \\
143044	1.90823136553275e-06 \\
143444	1.776350809779e-06 \\
143844	1.65271250468502e-06 \\
144244	1.54556086684465e-06 \\
144644	1.45069293738809e-06 \\
145044	1.36187383215081e-06 \\
145444	1.27868405569442e-06 \\
145844	1.20074272497162e-06 \\
146244	1.12769828758408e-06 \\
146644	1.0592254656161e-06 \\
147044	9.9502287981279e-07 \\
147444	9.34810969521749e-07 \\
147844	8.78330142617401e-07 \\
148244	8.25339122767499e-07 \\
148644	7.7561346656374e-07 \\
149044	7.28944228978179e-07 \\
149444	6.85136755496085e-07 \\
149844	6.4400958910138e-07 \\
150244	6.05393475239246e-07 \\
150644	5.69130455929656e-07 \\
151044	5.35073041318945e-07 \\
151444	5.03083452785269e-07 \\
151844	4.73032927994499e-07 \\
152244	4.44801082022384e-07 \\
152644	4.18275321378836e-07 \\
153044	3.93350302607676e-07 \\
153444	3.69927433407913e-07 \\
153844	3.47914411724659e-07 \\
154244	3.27224800422687e-07 \\
154644	3.07777632102546e-07 \\
155044	2.8949704461434e-07 \\
155444	2.72311941718062e-07 \\
155844	2.56155679501102e-07 \\
156244	2.40965773723456e-07 \\
156644	2.26683629422908e-07 \\
157044	2.13254287839693e-07 \\
157444	2.00626192103925e-07 \\
157844	1.88750968410645e-07 \\
158244	1.77583222737976e-07 \\
158644	1.67080350554905e-07 \\
159044	1.57202359962749e-07 \\
159444	1.47911706493975e-07 \\
159844	1.39173138735682e-07 \\
160244	1.30953554888791e-07 \\
160644	1.23221867320833e-07 \\
161044	1.15948877443817e-07 \\
161444	1.0910715825263e-07 \\
161844	1.02670943802341e-07 \\
162244	9.66160270121641e-08 \\
162644	9.09196626874831e-08 \\
163044	8.55604783134289e-08 \\
163444	8.05183889007743e-08 \\
163844	7.57745183821434e-08 \\
164244	7.13111253936027e-08 \\
164644	6.71115341632778e-08 \\
165044	6.31600693412615e-08 \\
165444	5.94419951593927e-08 \\
165844	5.59434582547702e-08 \\
166244	5.26514340459805e-08 \\
166644	4.9553676773062e-08 \\
167044	4.66386716468925e-08 \\
167444	4.38955918835582e-08 \\
167844	4.13142559607671e-08 \\
168244	3.88850891486214e-08 \\
168644	3.65990874828803e-08 \\
169044	3.44477824598677e-08 \\
169444	3.24232097836941e-08 \\
169844	3.05178782245008e-08 \\
170244	2.87247420294179e-08 \\
170644	2.70371738331221e-08 \\
171044	2.54489395667967e-08 \\
171444	2.39541750324257e-08 \\
171844	2.25473639758889e-08 \\
172244	2.12233168261911e-08 \\
172644	1.99771518216707e-08 \\
173044	1.88042760806972e-08 \\
173444	1.77003686152588e-08 \\
173844	1.66613641772173e-08 \\
174244	1.56834382702975e-08 \\
174644	1.47629922175874e-08 \\
175044	1.38966401164176e-08 \\
175444	1.3081196570397e-08 \\
175844	1.23136637553145e-08 \\
176244	1.15912212605984e-08 \\
176644	1.09112151536195e-08 \\
177044	1.02711480987061e-08 \\
177444	9.66867014229322e-09 \\
177844	9.10157010869383e-09 \\
178244	8.56776694035943e-09 \\
178644	8.06530259245264e-09 \\
179044	7.59233403924142e-09 \\
179444	7.14712744542823e-09 \\
179844	6.728050838678e-09 \\
180244	6.33356816992503e-09 \\
180644	5.9622340953247e-09 \\
181044	5.61268875820531e-09 \\
181444	5.28365162733024e-09 \\
181844	4.97391788867318e-09 \\
182244	4.68235389350369e-09 \\
182644	4.40789227340588e-09 \\
183044	4.14952872063168e-09 \\
183444	3.90631799129793e-09 \\
183844	3.67737046369498e-09 \\
184244	3.46184853006193e-09 \\
184644	3.25896415409588e-09 \\
185044	3.06797570681638e-09 \\
185444	2.88818474691865e-09 \\
185844	2.71893441095017e-09 \\
186244	2.5596057495747e-09 \\
186644	2.40961695041619e-09 \\
187044	2.26841989636739e-09 \\
187444	2.13549877781105e-09 \\
187844	2.01036809421851e-09 \\
188244	1.89257082228167e-09 \\
188644	1.7816769726231e-09 \\
189044	1.67728148037227e-09 \\
189444	1.57900348352058e-09 \\
189844	1.48648404696416e-09 \\
190244	1.39938538534778e-09 \\
190644	1.3173899748864e-09 \\
191044	1.24019833291911e-09 \\
191444	1.16752890688687e-09 \\
191844	1.09911674206487e-09 \\
192244	1.03471203827254e-09 \\
192644	9.74079983340204e-10 \\
193044	9.16999309819033e-10 \\
193444	8.63262183958824e-10 \\
193844	8.12672429351124e-10 \\
194244	7.65045582440393e-10 \\
194644	7.20208115367882e-10 \\
195044	6.77996547793214e-10 \\
195444	6.38256836271722e-10 \\
195844	6.00844318743299e-10 \\
196244	5.65622715331671e-10 \\
196644	5.32463351188284e-10 \\
197044	5.0124593364842e-10 \\
197444	4.71856054229391e-10 \\
197844	4.44187242543137e-10 \\
198244	4.18138135227508e-10 \\
198644	3.93614252303109e-10 \\
199044	3.70526220816458e-10 \\
199444	3.48789774839986e-10 \\
199844	3.28325810983188e-10 \\
200244	3.09059833281111e-10 \\
200644	2.909216756386e-10 \\
201044	2.73845224274538e-10 \\
201444	2.57768362210697e-10 \\
201844	2.42632525182529e-10 \\
202244	2.28382701639163e-10 \\
202644	2.14966988654197e-10 \\
203044	2.02336480903398e-10 \\
203444	1.90445159642394e-10 \\
203844	1.79249781684376e-10 \\
204244	1.68709657355492e-10 \\
204644	1.58786317427939e-10 \\
205044	1.49443735164567e-10 \\
205444	1.40647993251974e-10 \\
205844	1.32366895222447e-10 \\
206244	1.24570409543168e-10 \\
206644	1.17230281038161e-10 \\
207044	1.10319697821382e-10 \\
207444	1.03813291296717e-10 \\
207844	9.76876912694991e-11 \\
208244	9.19205267457812e-11 \\
208644	8.64908145103982e-11 \\
209044	8.13788481046629e-11 \\
209444	7.65658092483079e-11 \\
209844	7.20344894844516e-11 \\
210244	6.77683464900269e-11 \\
210644	6.37517261203868e-11 \\
211044	5.99700289427574e-11 \\
211444	5.6409654725087e-11 \\
211844	5.30576693691387e-11 \\
212244	4.9901693888188e-11 \\
212644	4.69303484962325e-11 \\
213044	4.41328640299332e-11 \\
213444	4.149885990401e-11 \\
213844	3.9019232289661e-11 \\
214244	3.6684377757723e-11 \\
214644	3.44861916801165e-11 \\
215044	3.24166804510639e-11 \\
215444	3.04681835316956e-11 \\
215844	2.86335954946537e-11 \\
216244	2.69063105129419e-11 \\
216644	2.52801668487734e-11 \\
217044	2.37490582755129e-11 \\
217444	2.23074891891883e-11 \\
217844	2.09502970527353e-11 \\
218244	1.96723748402405e-11 \\
218644	1.84693371707567e-11 \\
219044	1.73364655964292e-11 \\
219444	1.62700408701255e-11 \\
219844	1.52660106778058e-11 \\
220244	1.4320544750035e-11 \\
220644	1.34304234400417e-11 \\
221044	1.25924271010547e-11 \\
221444	1.18033360863024e-11 \\
221844	1.10604303493744e-11 \\
222244	1.03610453550118e-11 \\
222644	9.70240554565294e-12 \\
223044	9.08251251985348e-12 \\
223444	8.49859072005188e-12 \\
223844	7.9489748117112e-12 \\
224244	7.43149985993341e-12 \\
224644	6.94422297442543e-12 \\
225044	6.48553433180155e-12 \\
225444	6.0535465529199e-12 \\
225844	5.6468718589997e-12 \\
226244	5.26395593780649e-12 \\
226644	4.9034665217107e-12 \\
227044	4.56412685423402e-12 \\
227444	4.24454915659567e-12 \\
227844	3.94351218346856e-12 \\
228244	3.66023877873545e-12 \\
228644	3.39361871937172e-12 \\
229044	3.14237524889904e-12 \\
229444	2.90589774465388e-12 \\
229844	2.68324251706531e-12 \\
230244	2.47374343231854e-12 \\
230644	2.27634577854019e-12 \\
231044	2.09049444421794e-12 \\
231444	1.91546778438578e-12 \\
231844	1.7508772209851e-12 \\
232244	1.59566804214251e-12 \\
232644	1.4496737144043e-12 \\
233044	1.31217259280447e-12 \\
233444	1.18272058813318e-12 \\
233844	1.06087361118057e-12 \\
234244	9.46076550434327e-13 \\
234644	8.37885316684606e-13 \\
235044	7.36244398780173e-13 \\
235444	6.40432151755022e-13 \\
235844	5.50171019852996e-13 \\
236244	4.65238958469172e-13 \\
236644	3.85358411847392e-13 \\
237044	3.10085290777806e-13 \\
237444	2.39086528353027e-13 \\
237844	1.72473146875518e-13 \\
238244	1.09579012530503e-13 \\
238644	5.04041253179821e-14 \\
239044	-5.32907051820075e-15 \\
239444	-5.76760861292769e-14 \\
239844	-1.07192033027559e-13 \\
};
\addplot [line width=0.01pt, blue, forget plot]
table [row sep=\\]{%
1190	2.19340956510453 \\
2380	1.7499868663628 \\
3570	1.41303552544972 \\
4760	1.16404031147423 \\
5950	0.972781657324805 \\
7133	0.818522333373063 \\
8313	0.689708922499973 \\
9493	0.58029223477946 \\
10673	0.488582099699691 \\
11853	0.406995116469708 \\
13033	0.337192478021931 \\
14213	0.282677581414032 \\
15393	0.233916623229256 \\
16573	0.191947112696827 \\
17753	0.156953027910099 \\
18933	0.129715215883778 \\
20113	0.10819324040783 \\
21293	0.0923094201049257 \\
22473	0.0804880444220034 \\
23653	0.0708302672307636 \\
24833	0.0623072840925783 \\
26013	0.054405055572276 \\
27193	0.0477373926336903 \\
28373	0.0424699404167467 \\
29553	0.0377112946732913 \\
30733	0.0333474873609045 \\
31913	0.0292985552501634 \\
33093	0.0255457695937593 \\
34273	0.0222212522224285 \\
35453	0.0201448843781517 \\
36633	0.0182231654681415 \\
37813	0.0164225355391488 \\
38993	0.0149815724858277 \\
40173	0.0136416441545498 \\
41353	0.0124700016916451 \\
42533	0.011412562128405 \\
43713	0.0104265136848194 \\
44893	0.00950651451650336 \\
46073	0.00865342301602079 \\
47253	0.00787083494579549 \\
48433	0.00714077924100009 \\
49613	0.00645950184335636 \\
50793	0.00582354537915303 \\
51973	0.00522972076234424 \\
53153	0.00468817384481929 \\
54333	0.00419847275271329 \\
55513	0.00374116077471226 \\
56693	0.00331398993689913 \\
57873	0.00292492304094299 \\
59053	0.00257471233318679 \\
60233	0.00225079287454366 \\
61413	0.00201454893591235 \\
62593	0.00184399157905907 \\
63773	0.00168646530051947 \\
64953	0.00154179300511925 \\
66133	0.00140929283952423 \\
67313	0.00128659685664184 \\
68493	0.001172880213336 \\
69673	0.00106740719851545 \\
70853	0.000969516441805596 \\
72033	0.000879822880986481 \\
73213	0.000798369810746991 \\
74393	0.000723050130923775 \\
75573	0.000653367894755952 \\
76753	0.000588873429610559 \\
77933	0.000529253610063951 \\
79113	0.000481436182110662 \\
80293	0.00043942686194115 \\
81473	0.000400894357012904 \\
82653	0.000365520858955137 \\
83833	0.000333032533853306 \\
85013	0.000303181922210072 \\
86193	0.00027574449966955 \\
87373	0.000250516223457786 \\
88553	0.000227311425273047 \\
89733	0.000205960970795893 \\
90913	0.000186310641118348 \\
92093	0.000168219701385919 \\
93273	0.000153788157993495 \\
94453	0.000142475952669308 \\
95633	0.000132024109598117 \\
96813	0.000122340098313856 \\
97993	0.000113362994002275 \\
99173	0.00010503840019388 \\
100353	9.73165245932717e-05 \\
101533	9.01516667924374e-05 \\
102713	8.35018354277683e-05 \\
103893	7.73284124965801e-05 \\
105073	7.1595854220563e-05 \\
106253	6.62714231242645e-05 \\
107433	6.13249471652089e-05 \\
108613	5.67286024851232e-05 \\
109793	5.24567169271073e-05 \\
110973	4.84855919106741e-05 \\
112153	4.47933406214629e-05 \\
113333	4.15012593625863e-05 \\
114513	3.85040859326446e-05 \\
115693	3.57288214940521e-05 \\
116873	3.3158343174633e-05 \\
118053	3.07769661931911e-05 \\
119233	2.85702664625798e-05 \\
120413	2.65249705413417e-05 \\
121273	2.4628858610154e-05 \\
121653	2.28706763235609e-05 \\
122033	2.12400545106517e-05 \\
122413	1.97274359642896e-05 \\
122793	1.832400865176e-05 \\
123173	1.70216447430427e-05 \\
123553	1.58128449182504e-05 \\
123933	1.46906874655683e-05 \\
124313	1.36487817314901e-05 \\
124693	1.26812255290543e-05 \\
125073	1.17825661439808e-05 \\
125453	1.09477646187961e-05 \\
125833	1.01721630214047e-05 \\
126213	9.45145443387307e-06 \\
126593	8.78165542284037e-06 \\
126973	8.15908077456129e-06 \\
127353	7.5803202971847e-06 \\
127733	7.0422175122431e-06 \\
128113	6.54185007370467e-06 \\
128493	6.0765117647632e-06 \\
128873	5.64369594230341e-06 \\
129253	5.24108030253156e-06 \\
129633	4.86651286030204e-06 \\
130013	4.51799903827821e-06 \\
130393	4.19368977550016e-06 \\
130773	3.89187056970552e-06 \\
131153	3.61095137724199e-06 \\
131533	3.34945730023906e-06 \\
131913	3.10601999720106e-06 \\
132293	2.87936975734704e-06 \\
132673	2.66832818590634e-06 \\
133053	2.47180145168668e-06 \\
133433	2.2887740504518e-06 \\
133813	2.11830304425176e-06 \\
134193	1.9595127388472e-06 \\
134573	1.81158976336748e-06 \\
134953	1.6737785227261e-06 \\
135333	1.54537699093016e-06 \\
135713	1.42573282102543e-06 \\
136093	1.31423974569778e-06 \\
136473	1.21033424610451e-06 \\
136853	1.11349246900705e-06 \\
137233	1.02322737183247e-06 \\
137613	9.39086079176921e-07 \\
137993	8.60647434652861e-07 \\
138373	7.89543461388043e-07 \\
138753	7.35811823193711e-07 \\
139133	6.85932164012915e-07 \\
139513	6.39581800465638e-07 \\
139893	5.96489820536839e-07 \\
140273	5.56410290231124e-07 \\
140653	5.19118303066168e-07 \\
141033	4.84407745471316e-07 \\
141413	4.52089425340141e-07 \\
141793	4.2198943706051e-07 \\
142173	3.93947722721055e-07 \\
142553	3.67816798296783e-07 \\
142933	3.43460625162884e-07 \\
143313	3.20753600069335e-07 \\
143693	2.99579655027582e-07 \\
144073	2.79831446514667e-07 \\
144453	2.61409627111409e-07 \\
144833	2.44222187695264e-07 \\
145213	2.28183861639142e-07 \\
145593	2.13215584909943e-07 \\
145973	1.99244005627541e-07 \\
146353	1.86201037088995e-07 \\
146733	1.74023450594252e-07 \\
147113	1.62652503132854e-07 \\
147493	1.52033596934054e-07 \\
147873	1.42115968326806e-07 \\
148253	1.32852400358541e-07 \\
148633	1.24198961948263e-07 \\
149013	1.16114765358333e-07 \\
149393	1.08561744593949e-07 \\
149773	1.01504451288648e-07 \\
150153	9.49098652447411e-08 \\
150533	8.8747221738128e-08 \\
150913	8.298784986982e-08 \\
151293	7.76050253503691e-08 \\
151673	7.25738323326119e-08 \\
152053	6.78710372348235e-08 \\
152433	6.34749710015647e-08 \\
152813	5.93654205793825e-08 \\
153193	5.55235275534471e-08 \\
153573	5.19316954439297e-08 \\
153953	4.8573502164917e-08 \\
154333	4.54336203659089e-08 \\
154713	4.24977421586981e-08 \\
155093	3.9752509672919e-08 \\
155473	3.71854507741354e-08 \\
155853	3.47849191673077e-08 \\
156233	3.25400384415531e-08 \\
156613	3.04406501672183e-08 \\
156993	2.84772659342458e-08 \\
157373	2.66410219995628e-08 \\
157753	2.49236378757622e-08 \\
158133	2.33173773067641e-08 \\
158513	2.18150114084104e-08 \\
158893	2.04097858058638e-08 \\
159273	1.90953884926515e-08 \\
159653	1.78659209093546e-08 \\
160033	1.67158697994552e-08 \\
160413	1.56400827844294e-08 \\
160793	1.4633743550263e-08 \\
161173	1.36923503091246e-08 \\
161553	1.28116945941059e-08 \\
161933	1.19878426074749e-08 \\
162313	1.12171164579067e-08 \\
162693	1.04960778402052e-08 \\
163073	9.82151193706926e-09 \\
163453	9.1904130417042e-09 \\
163833	8.59997084656783e-09 \\
164213	8.04755745376085e-09 \\
164593	7.53071582870746e-09 \\
164973	7.04714797628014e-09 \\
165353	6.59470528185935e-09 \\
165733	6.17137829728165e-09 \\
166113	5.77528797007787e-09 \\
166493	5.40467670617772e-09 \\
166873	5.05790115346016e-09 \\
167253	4.73342381956954e-09 \\
167633	4.42980707671126e-09 \\
168013	4.1457064448025e-09 \\
168393	3.87986387462291e-09 \\
168773	3.63110314038906e-09 \\
169153	3.39832384455008e-09 \\
169533	3.18049653280639e-09 \\
169913	2.97665869730679e-09 \\
170293	2.7859095030891e-09 \\
170673	2.60740662394454e-09 \\
171053	2.44036191254793e-09 \\
171433	2.28403801427746e-09 \\
171813	2.13774559165714e-09 \\
172193	2.0008392165316e-09 \\
172573	1.87271537166467e-09 \\
172953	1.75280939762601e-09 \\
173333	1.64059299478936e-09 \\
173713	1.53557205839761e-09 \\
174093	1.43728395851639e-09 \\
174473	1.34529637429992e-09 \\
174853	1.25920440741112e-09 \\
175233	1.17862947179859e-09 \\
175613	1.10321723978402e-09 \\
175993	1.0326362542834e-09 \\
176373	9.66576263472518e-10 \\
176753	9.04747221586177e-10 \\
177133	8.46877346027952e-10 \\
177513	7.92712839814413e-10 \\
177893	7.42015893173686e-10 \\
178273	6.94564017411636e-10 \\
178653	6.50149045711146e-10 \\
179033	6.08576355975998e-10 \\
179413	5.69663538563248e-10 \\
179793	5.33240118727463e-10 \\
180173	4.99146557420005e-10 \\
180553	4.67233585155213e-10 \\
180933	4.37361258320834e-10 \\
181313	4.09399014689171e-10 \\
181693	3.83224396660609e-10 \\
182073	3.58722829219005e-10 \\
182453	3.35787397887088e-10 \\
182833	3.14317516458829e-10 \\
183213	2.94219426599795e-10 \\
183593	2.75405365179893e-10 \\
183973	2.57793120184147e-10 \\
184353	2.41305864179253e-10 \\
184733	2.25871377157461e-10 \\
185113	2.11422601648081e-10 \\
185493	1.9789614391641e-10 \\
185873	1.85233106630989e-10 \\
186253	1.73378478240949e-10 \\
186633	1.62280133775283e-10 \\
187013	1.5189016711048e-10 \\
187393	1.42163003591378e-10 \\
187773	1.33056510254193e-10 \\
188153	1.24530719070037e-10 \\
188533	1.16548659612192e-10 \\
188913	1.09075637411138e-10 \\
189293	1.02079067421101e-10 \\
189673	9.55285295312081e-11 \\
190053	8.93956020320275e-11 \\
190433	8.36533620152125e-11 \\
190813	7.82770515073139e-11 \\
191193	7.32434113359659e-11 \\
191573	6.85303480629784e-11 \\
191953	6.41176556293033e-11 \\
192333	5.99857385985558e-11 \\
192713	5.61171109580982e-11 \\
193093	5.24947307845025e-11 \\
193473	4.91029994442727e-11 \\
193853	4.59271509711812e-11 \\
194233	4.29535296220251e-11 \\
194613	4.01689792539628e-11 \\
194993	3.75616759917818e-11 \\
195373	3.51201845383287e-11 \\
195753	3.28340687971718e-11 \\
196133	3.06933367610895e-11 \\
196513	2.86887735789776e-11 \\
196893	2.68114974666389e-11 \\
197273	2.50537368629011e-11 \\
197653	2.34076091842894e-11 \\
198033	2.18662310480511e-11 \\
198413	2.0422608049131e-11 \\
198793	1.90708004943474e-11 \\
199173	1.7804979712821e-11 \\
199553	1.6619428055975e-11 \\
199933	1.55092050313499e-11 \\
200313	1.44695366799397e-11 \\
200693	1.34957600650409e-11 \\
201073	1.25839894060675e-11 \\
201453	1.17299503443746e-11 \\
201833	1.09301456774347e-11 \\
202213	1.01810782027201e-11 \\
202593	9.47947276230821e-12 \\
202973	8.82244277633504e-12 \\
203353	8.20710166493654e-12 \\
203733	7.63078489285363e-12 \\
204113	7.09104996943211e-12 \\
204493	6.58545440401781e-12 \\
204873	6.11194428401518e-12 \\
205253	5.66841018567743e-12 \\
205633	5.2529647298627e-12 \\
206013	4.86399809318527e-12 \\
206393	4.4994563630496e-12 \\
206773	4.15817380527983e-12 \\
207153	3.83842957418778e-12 \\
207533	3.53900242444638e-12 \\
207913	3.2584490661236e-12 \\
208293	2.99565927619483e-12 \\
208673	2.74957834278666e-12 \\
209053	2.51904053172325e-12 \\
209433	2.30299113113119e-12 \\
209813	2.10070849604449e-12 \\
210193	1.91119342574098e-12 \\
210573	1.73372427525464e-12 \\
210953	1.56730184386333e-12 \\
211333	1.41164857581089e-12 \\
211713	1.26570975922391e-12 \\
212093	1.12893028259009e-12 \\
212473	1.00086605669958e-12 \\
212853	8.80961970040062e-13 \\
213233	7.68496377645533e-13 \\
213613	6.63191723759837e-13 \\
213993	5.64603919173123e-13 \\
214373	4.72177852373079e-13 \\
214753	3.85580456452317e-13 \\
215133	3.04478664503449e-13 \\
215513	2.28372876165395e-13 \\
215893	1.57263091438153e-13 \\
216273	9.0649709960644e-14 \\
216653	2.79776202205539e-14 \\
217033	-3.05311331771918e-14 \\
217413	-8.53761505936745e-14 \\
217793	-1.36834987785051e-13 \\
218173	-1.8490764475132e-13 \\
218553	-2.30038210702332e-13 \\
218933	-2.72448730243013e-13 \\
219313	-3.12083692222132e-13 \\
219693	-3.49165141244612e-13 \\
220073	-3.8397063306661e-13 \\
220453	-4.16555678839359e-13 \\
220833	-4.47031300865319e-13 \\
221213	-4.7573056605188e-13 \\
221593	-5.02486940945346e-13 \\
221973	-5.27577981301874e-13 \\
222353	-5.51114709423928e-13 \\
222733	-5.73319169916431e-13 \\
223113	-5.93913807023227e-13 \\
223493	-6.13342709954168e-13 \\
223873	-6.31439345255558e-13 \\
224253	-6.48536779834785e-13 \\
224633	-6.64468480238156e-13 \\
225013	-6.79345468768133e-13 \\
225393	-6.93389790029642e-13 \\
225773	-7.06434910568987e-13 \\
226153	-7.18813897293558e-13 \\
226533	-7.30304705598428e-13 \\
226913	-7.41184891239755e-13 \\
227293	-7.51287920763843e-13 \\
227673	-7.60780327624389e-13 \\
228053	-7.69717622972621e-13 \\
228433	-7.77988784506078e-13 \\
228813	-7.85815856829686e-13 \\
229193	-7.93198839943443e-13 \\
229573	-7.99971200393657e-13 \\
229953	-8.06466005087714e-13 \\
230333	-8.1246120942069e-13 \\
230713	-8.18067835695047e-13 \\
231093	-8.23396906213247e-13 \\
231473	-8.28337398672829e-13 \\
231853	-8.33000335376255e-13 \\
232233	-8.37385716323524e-13 \\
232613	-8.41438030363406e-13 \\
232993	-8.45268299798363e-13 \\
233373	-8.48876524628395e-13 \\
233753	-8.5220719370227e-13 \\
234133	-8.55482351624914e-13 \\
};
\addplot [line width=0.01pt, blue, forget plot]
table [row sep=\\]{%
1190	3.04793683049409 \\
2380	2.34553105529016 \\
3570	1.84612261179922 \\
4760	1.46181011708265 \\
5950	1.15714520609296 \\
7140	0.926719716393281 \\
8309	0.747490270677982 \\
9469	0.593572493969494 \\
10629	0.470251536398663 \\
11789	0.370344094255359 \\
12949	0.299701600788282 \\
14109	0.24952915062169 \\
15269	0.209739926147582 \\
16429	0.176997783172371 \\
17589	0.151812235559405 \\
18749	0.130424035916017 \\
19909	0.112275531084425 \\
21069	0.0973767987526111 \\
22229	0.0860082955616313 \\
23389	0.0763193356479927 \\
24549	0.0678703568536568 \\
25709	0.0601237097877496 \\
26869	0.0530112365996125 \\
28029	0.0465750702390623 \\
29189	0.0411714606507018 \\
30349	0.0364060577784152 \\
31509	0.0324198926334433 \\
32669	0.029029585946901 \\
33829	0.0259119987184213 \\
34989	0.0231144125069225 \\
36149	0.0207468429315954 \\
37309	0.0186313520270778 \\
38469	0.0170014947558583 \\
39629	0.0154831922006976 \\
40789	0.0140684747038595 \\
41949	0.0127509113632039 \\
43109	0.0115233390626761 \\
44269	0.0103795119224709 \\
45429	0.00931277129343716 \\
46589	0.00831766113239563 \\
47749	0.00741194068672862 \\
48909	0.00684877581151094 \\
50069	0.0063232470585029 \\
51229	0.00583133931595914 \\
52389	0.00537064815011895 \\
53549	0.00493903313232996 \\
54709	0.00453453442148916 \\
55869	0.00415534746834945 \\
57029	0.00379980502438831 \\
58189	0.00346636283384516 \\
59349	0.00315567843702014 \\
60509	0.00287871325650402 \\
61669	0.0026207832757788 \\
62829	0.00237999553939799 \\
63989	0.00215493626246355 \\
65149	0.00194438330907265 \\
66309	0.00174725663822328 \\
67469	0.00156365122309671 \\
68629	0.00139337911293741 \\
69789	0.00123437937118487 \\
70949	0.0010858378660798 \\
72109	0.000947013609948422 \\
73269	0.000817228455388452 \\
74429	0.000696840259996556 \\
75589	0.000628219350858605 \\
76749	0.000584289537577221 \\
77909	0.000543743234998673 \\
79069	0.000506240454716533 \\
80229	0.000471499276179899 \\
81389	0.000439276177899017 \\
82549	0.000409358061886411 \\
83709	0.000381556613426315 \\
84869	0.000355704007806823 \\
86029	0.00033164959269949 \\
87189	0.000309257295842968 \\
88349	0.000288403575777918 \\
89509	0.000268975781718639 \\
90669	0.000250870823815086 \\
91829	0.000233994080754252 \\
92989	0.00021825849045648 \\
94149	0.000203583783422279 \\
95309	0.000189895828425324 \\
96469	0.000177126067720412 \\
97629	0.000165211024453382 \\
98789	0.000154091869050688 \\
99949	0.000143714034405273 \\
101109	0.000134026871941473 \\
102269	0.00012498334234512 \\
103429	0.000116539736022714 \\
104589	0.000108655419327752 \\
105749	0.000101292603329972 \\
106909	9.44161324779591e-05 \\
108069	8.79932909429448e-05 \\
109229	8.1993624782839e-05 \\
110389	7.63887783371553e-05 \\
111549	7.11523434843153e-05 \\
112709	6.62597205710047e-05 \\
113869	6.16879899688616e-05 \\
115029	5.74157933345143e-05 \\
116189	5.34232237500154e-05 \\
117349	4.96917240093708e-05 \\
118509	4.62039923870283e-05 \\
119669	4.29438952906369e-05 \\
120829	3.98963862555668e-05 \\
121989	3.70474307868074e-05 \\
123149	3.43839365991028e-05 \\
124309	3.18936888428767e-05 \\
125469	2.95652899398591e-05 \\
126629	2.73881036821355e-05 \\
127789	2.5352203277762e-05 \\
128949	2.34483230516225e-05 \\
130109	2.16678135310233e-05 \\
131269	2.00025996698883e-05 \\
132429	1.84451419836251e-05 \\
133589	1.69884003823895e-05 \\
134749	1.56258005093446e-05 \\
135909	1.43512024040615e-05 \\
137069	1.31588713248587e-05 \\
138229	1.20434505753719e-05 \\
139389	1.09999361946889e-05 \\
140549	1.00236533789322e-05 \\
141709	9.2306295697564e-06 \\
142869	8.58619452953624e-06 \\
144029	7.98868374995987e-06 \\
145189	7.43424978932872e-06 \\
146349	6.91944515535114e-06 \\
147509	6.44115560638481e-06 \\
148669	6.01242288750248e-06 \\
149829	5.6141047390379e-06 \\
150989	5.2437684298634e-06 \\
152149	4.89925854252338e-06 \\
153309	4.57861483554867e-06 \\
154469	4.28004793834313e-06 \\
155629	4.00192062377513e-06 \\
156789	3.74273193237684e-06 \\
157949	3.50110350100241e-06 \\
159109	3.27576768877069e-06 \\
160269	3.06555718504464e-06 \\
161429	2.86939585342205e-06 \\
162589	2.68629061717096e-06 \\
163749	2.51532422823608e-06 \\
164909	2.35564879419448e-06 \\
166069	2.20647996146406e-06 \\
167229	2.06709166755692e-06 \\
168389	1.93681139482038e-06 \\
169549	1.815015864437e-06 \\
170709	1.70112712227777e-06 \\
171869	1.59460897275476e-06 \\
173029	1.49496372531255e-06 \\
174189	1.40172921997417e-06 \\
175349	1.31447610651758e-06 \\
176509	1.23280535008208e-06 \\
177669	1.15634594471947e-06 \\
178829	1.08475281279663e-06 \\
179989	1.01770487531683e-06 \\
181149	9.54903277061714e-07 \\
182309	8.96069753508755e-07 \\
183469	8.40945125701875e-07 \\
184629	7.89287914082504e-07 \\
185789	7.40873059568159e-07 \\
186949	6.95490743884974e-07 \\
188109	6.52945299939311e-07 \\
189269	6.13054205123031e-07 \\
190429	5.75647152223357e-07 \\
191589	5.40565188777986e-07 \\
192749	5.07659921100689e-07 \\
193909	4.76792778258961e-07 \\
195069	4.47834330341568e-07 \\
196229	4.20663656131026e-07 \\
197389	3.95167757794024e-07 \\
198549	3.71241018426449e-07 \\
199709	3.48784697512539e-07 \\
200869	3.27706464686717e-07 \\
202029	3.0791996497026e-07 \\
203189	2.89344416815052e-07 \\
204349	2.71904237014731e-07 \\
205509	2.55528693648976e-07 \\
206669	2.40151581953896e-07 \\
207829	2.25710924173228e-07 \\
208989	2.1214868961561e-07 \\
210149	1.99410534196254e-07 \\
211309	1.87445558186283e-07 \\
212469	1.76206080115815e-07 \\
213629	1.65647427163851e-07 \\
214789	1.55727738704314e-07 \\
215949	1.46407783840896e-07 \\
217109	1.37650791320887e-07 \\
218269	1.29422290606751e-07 \\
219429	1.21689963439309e-07 \\
220589	1.14423506503147e-07 \\
221749	1.07594501363995e-07 \\
222909	1.01176294897698e-07 \\
224069	9.51438864915666e-08 \\
225229	8.94738228507386e-08 \\
226389	8.41441000209997e-08 \\
227549	7.91340720729394e-08 \\
228709	7.44243647265996e-08 \\
229869	6.99967955819503e-08 \\
231029	6.58342993453687e-08 \\
232189	6.19208572749663e-08 \\
233349	5.82414323990754e-08 \\
234509	5.47819073992706e-08 \\
235669	5.15290279334835e-08 \\
236829	4.84703481795634e-08 \\
237989	4.55941814303529e-08 \\
239149	4.28895522985862e-08 \\
240309	4.03461530296134e-08 \\
241469	3.79543019790596e-08 \\
242629	3.57049055321745e-08 \\
243789	3.35894208558507e-08 \\
244949	3.15998230360215e-08 \\
246109	2.97285725481267e-08 \\
247269	2.79685857251799e-08 \\
248429	2.63132068356597e-08 \\
249589	2.47561819932685e-08 \\
250749	2.32916338993583e-08 \\
251909	2.19140399715378e-08 \\
253069	2.06182099726782e-08 \\
254229	1.93992654717867e-08 \\
255389	1.82526220804391e-08 \\
256549	1.71739700793871e-08 \\
257709	1.61592589309478e-08 \\
258869	1.52046807366801e-08 \\
260029	1.43066564151084e-08 \\
261189	1.34618207137116e-08 \\
262349	1.26670101630033e-08 \\
263509	1.19192508085675e-08 \\
264669	1.12157459986051e-08 \\
265829	1.05538669470384e-08 \\
266989	9.93114140923623e-09 \\
268149	9.34524480022958e-09 \\
269309	8.79399159048333e-09 \\
270469	8.27532620206739e-09 \\
271629	7.7873161807851e-09 \\
272789	7.32814353643363e-09 \\
273949	6.89609908066657e-09 \\
275109	6.48957504401082e-09 \\
276269	6.10705980230719e-09 \\
277429	5.74713138190575e-09 \\
278589	5.40845224161757e-09 \\
279749	5.08976460977806e-09 \\
280909	4.78988532170987e-09 \\
282069	4.50770160087544e-09 \\
283229	4.24216661798482e-09 \\
284389	3.9922958272598e-09 \\
285549	3.75716296963091e-09 \\
286709	3.5358969086019e-09 \\
287869	3.32767791100252e-09 \\
289029	3.13173509347564e-09 \\
290189	2.94734320283041e-09 \\
291349	2.7738200070182e-09 \\
292509	2.61052351957503e-09 \\
293669	2.45684983468664e-09 \\
294829	2.31223090674249e-09 \\
295989	2.17613227437852e-09 \\
297149	2.04805100656458e-09 \\
298309	1.92751409278102e-09 \\
299469	1.81407638910613e-09 \\
300629	1.70731900839272e-09 \\
301789	1.60684809902278e-09 \\
302949	1.51229279099496e-09 \\
304109	1.42330430774606e-09 \\
305269	1.33955430081656e-09 \\
306429	1.26073418371675e-09 \\
307589	1.18655324454764e-09 \\
308749	1.11673814640056e-09 \\
309909	1.05103148406727e-09 \\
311069	9.89191228928377e-10 \\
312229	9.30989507708091e-10 \\
313389	8.7621221389611e-10 \\
314549	8.24657397924256e-10 \\
315709	7.76135322677618e-10 \\
316869	7.30467519804989e-10 \\
318029	6.87485623984685e-10 \\
319189	6.47031928036057e-10 \\
320349	6.08957217984596e-10 \\
321509	5.73121439195745e-10 \\
322669	5.39392974729935e-10 \\
323829	5.07647757164165e-10 \\
324989	4.77768991036243e-10 \\
326149	4.49646819777882e-10 \\
327309	4.23177881625492e-10 \\
328469	3.98265032064415e-10 \\
329629	3.74816677695122e-10 \\
330789	3.52746554188599e-10 \\
331949	3.31973670775199e-10 \\
333109	3.12421644110827e-10 \\
334269	2.94018753788095e-10 \\
335429	2.76697276202498e-10 \\
336589	2.6039376210818e-10 \\
337749	2.45048314972962e-10 \\
338909	2.30604424444891e-10 \\
340069	2.170094659526e-10 \\
341229	2.04212979859619e-10 \\
342389	1.9216855884352e-10 \\
343549	1.80831627449862e-10 \\
344709	1.70160718848678e-10 \\
345869	1.60116753189499e-10 \\
347029	1.50662704534454e-10 \\
348189	1.4176410045863e-10 \\
349349	1.33388133871648e-10 \\
350509	1.25504107106877e-10 \\
351669	1.18083154365678e-10 \\
352829	1.11098019672795e-10 \\
353989	1.04523223409814e-10 \\
355149	9.83343961813432e-11 \\
356309	9.25091669934375e-11 \\
357469	8.70259420082675e-11 \\
358629	8.18646817002389e-11 \\
359789	7.70066233002353e-11 \\
360949	7.2433781195258e-11 \\
362109	6.8129391017635e-11 \\
363269	6.40776876004168e-11 \\
364429	6.02640715108294e-11 \\
365589	5.66741653607039e-11 \\
366749	5.3295146074106e-11 \\
367909	5.01144681308574e-11 \\
369069	4.71205297003507e-11 \\
370229	4.4302284063491e-11 \\
371389	4.16495171684517e-11 \\
372549	3.91525145637672e-11 \\
373709	3.68020058871821e-11 \\
374869	3.45894979325578e-11 \\
376029	3.25068860718147e-11 \\
377189	3.05465097660829e-11 \\
378349	2.87013190991559e-11 \\
379509	2.69641531325249e-11 \\
380669	2.53291831953106e-11 \\
381829	2.37900255051215e-11 \\
382989	2.2341239969137e-11 \\
384149	2.09776085391411e-11 \\
385309	1.96938021446158e-11 \\
386469	1.8485490915765e-11 \\
387629	1.73479564047341e-11 \\
388789	1.6277368342088e-11 \\
389949	1.52693968580309e-11 \\
391109	1.43207667946399e-11 \\
392269	1.34276478824802e-11 \\
393429	1.2586987008234e-11 \\
394589	1.17957310585837e-11 \\
395749	1.10508824313627e-11 \\
396909	1.03495545467069e-11 \\
398069	9.68958246971852e-12 \\
399229	9.06835717628951e-12 \\
400389	8.48332515346328e-12 \\
401549	7.9328210667029e-12 \\
402709	7.41456895880788e-12 \\
403869	6.92668145063635e-12 \\
405029	6.46754871880262e-12 \\
406189	6.03522787301358e-12 \\
407349	5.62827562333723e-12 \\
408509	5.24530419099278e-12 \\
409669	4.88459273029207e-12 \\
410829	4.54525306281539e-12 \\
411989	4.22573087632827e-12 \\
413149	3.92491594780608e-12 \\
414309	3.64180907652667e-12 \\
415469	3.3753000394654e-12 \\
416629	3.12444514705135e-12 \\
417789	2.88818968741111e-12 \\
418949	2.66597854903239e-12 \\
420109	2.45659048658808e-12 \\
421269	2.25958141086835e-12 \\
422429	2.07428518805841e-12 \\
423589	1.89959159513364e-12 \\
424749	1.73527858748912e-12 \\
425909	1.58062452015884e-12 \\
427069	1.43501877047925e-12 \\
428229	1.29785071578681e-12 \\
429389	1.16878728917413e-12 \\
430549	1.04738440143137e-12 \\
431709	9.32920407592519e-13 \\
432869	8.25395307657573e-13 \\
434029	7.24031945509296e-13 \\
435189	6.28663787693995e-13 \\
436349	5.3884674500182e-13 \\
437509	4.54247750525383e-13 \\
438669	3.7470027081099e-13 \\
439829	2.99815727800024e-13 \\
440989	2.29372076887557e-13 \\
442149	1.62980740014973e-13 \\
443309	1.00475183728577e-13 \\
444469	4.16888745746746e-14 \\
445629	-1.36557432028894e-14 \\
446789	-6.59472476627343e-14 \\
447949	-1.14963594199935e-13 \\
449109	-1.6109336087311e-13 \\
450269	-2.04558592287185e-13 \\
451429	-2.45525821895853e-13 \\
452589	-2.83995049699115e-13 \\
453749	-3.20354853755589e-13 \\
454909	-3.54549722914044e-13 \\
456069	-3.86635168325711e-13 \\
457229	-4.16944256897978e-13 \\
458389	-4.45421477479613e-13 \\
459075	-4.7223336352431e-13 \\
459445	-4.97435426183301e-13 \\
459815	-5.21305221212742e-13 \\
460185	-5.43620704007708e-13 \\
460555	-5.64603919173123e-13 \\
};
\addplot [line width=2.0pt, blue!50!black]
table [row sep=\\]{%
1190	2.62066802672088 \\
2380	2.12547472927112 \\
3570	1.75275681138171 \\
4760	1.43733717100311 \\
5950	1.16968972813378 \\
7135.5	0.960649204682426 \\
8313	0.787908833008655 \\
9492	0.650739046058268 \\
10671	0.542108376790482 \\
11851	0.453251047532564 \\
13031	0.381907681828066 \\
14211	0.317468024126866 \\
15391	0.261320207473955 \\
16571	0.215584966548177 \\
17751	0.179725686703642 \\
18931	0.151403214466119 \\
20111	0.12802428999267 \\
21291	0.108649643969263 \\
22471	0.092456304817031 \\
23651	0.0785027841976917 \\
24831	0.0671555277855342 \\
26011	0.0568367453331872 \\
27191	0.0490378418463728 \\
28371	0.0430134240478587 \\
29551	0.0377282604312294 \\
30731	0.0337595648822727 \\
31911	0.0300869239429631 \\
33091	0.0264800082502876 \\
34271	0.0232113215355946 \\
35451	0.0202500340553487 \\
36631	0.0177774367101144 \\
37811	0.0158350513782053 \\
38991	0.0143123721713706 \\
40171	0.0129027430891606 \\
41351	0.0116721268812253 \\
42531	0.010419217910629 \\
43711	0.00914118143455128 \\
44891	0.008022040761898 \\
46071	0.00716167413623553 \\
47251	0.00638710221352418 \\
48431	0.00567928699253789 \\
49611	0.00512920362663966 \\
50791	0.00469270316365161 \\
51971	0.00432976181409411 \\
53151	0.00399187834461706 \\
54331	0.00370180013811022 \\
55511	0.00344182787458769 \\
56691	0.00318942199754174 \\
57871	0.00289146877127888 \\
59051	0.00262344766663031 \\
60231	0.00244037997343999 \\
61411	0.00227007125654644 \\
62591	0.002111536318056 \\
63771	0.00196387855408309 \\
64951	0.00182628005723562 \\
66131	0.00169855796743817 \\
67311	0.00158145356767847 \\
68491	0.00147231499847328 \\
69671	0.001356160232304 \\
70851	0.00123033726780603 \\
72031	0.0011129746503003 \\
73211	0.000977625606222055 \\
74391	0.000855807201835201 \\
75571	0.000744158531128081 \\
76751	0.000651136113680451 \\
77931	0.000583367315837469 \\
79111	0.000542684833296614 \\
80291	0.000504874974835701 \\
81471	0.000469698600250146 \\
82651	0.000436947109010433 \\
83831	0.000406432490269615 \\
85011	0.000377990391698779 \\
86191	0.00035145973231826 \\
87371	0.000326711121857393 \\
88551	0.000303613466284991 \\
89731	0.00028204925011166 \\
90911	0.000261917951238033 \\
92091	0.000243112345943153 \\
93271	0.000225553857620076 \\
94451	0.000209736315860576 \\
95631	0.00019510362809233 \\
96811	0.000181476916787882 \\
97991	0.000168798423389016 \\
99171	0.000156995927458048 \\
100351	0.000146005565674623 \\
101531	0.000135774149258527 \\
102711	0.000126243912039137 \\
103891	0.000117362450443192 \\
105071	0.000109093793448756 \\
106251	0.000101390054901451 \\
107431	9.42101991304178e-05 \\
108611	8.75122426863828e-05 \\
109791	8.127122271534e-05 \\
110971	7.54525348863333e-05 \\
112151	7.00266308925612e-05 \\
113331	6.49849385899182e-05 \\
114511	6.03054132937708e-05 \\
115691	5.5944782775208e-05 \\
116871	5.18826449140142e-05 \\
118051	4.81007570082848e-05 \\
119231	4.45753060316356e-05 \\
120411	4.14239434338715e-05 \\
121118	3.86845027687266e-05 \\
121653	3.6118874003066e-05 \\
122033	3.37158630209777e-05 \\
122615	3.14782339210229e-05 \\
123220	2.93820582497606e-05 \\
123618.5	2.74180231809029e-05 \\
123998.5	2.55776661998164e-05 \\
124378.5	2.38530821429772e-05 \\
124758.5	2.22368830615771e-05 \\
125138.5	2.07221629152654e-05 \\
125518.5	1.93024649621232e-05 \\
125898.5	1.7971751513135e-05 \\
126278.5	1.66717266137373e-05 \\
126658.5	1.54681729172546e-05 \\
127038.5	1.42213753425158e-05 \\
127418.5	1.2997126264791e-05 \\
127798.5	1.18801961107029e-05 \\
128183	1.09959281847649e-05 \\
128568	1.01675854178862e-05 \\
128953	9.45178286876347e-06 \\
129338	8.82987657396805e-06 \\
129723	8.2498093059713e-06 \\
130108	7.70851070308609e-06 \\
130493	7.20605815873743e-06 \\
130878	6.74311224152513e-06 \\
131263	6.31857408150971e-06 \\
131648	5.92200193810788e-06 \\
132033	5.55137736435452e-06 \\
132418	5.20487511324053e-06 \\
132803	4.88081734734136e-06 \\
133188	4.57765770578433e-06 \\
133573	4.29004270741107e-06 \\
133958	3.98238732318701e-06 \\
134343	3.69687531315277e-06 \\
134728	3.43185806001012e-06 \\
135113	3.18581498937043e-06 \\
135500.5	2.95734295413519e-06 \\
135895.5	2.74514661419145e-06 \\
136290.5	2.54802969185119e-06 \\
136685.5	2.36511561158004e-06 \\
137080.5	2.21244285464328e-06 \\
137475.5	2.07793178136351e-06 \\
137870.5	1.95177097844779e-06 \\
138265.5	1.83342583764956e-06 \\
138660.5	1.72239755674708e-06 \\
139055.5	1.61822057737071e-06 \\
139450.5	1.52046022666719e-06 \\
139845.5	1.4287105408739e-06 \\
140240.5	1.34259225786915e-06 \\
140635.5	1.26175096071268e-06 \\
141030.5	1.18585536218463e-06 \\
141425.5	1.11459571561223e-06 \\
141820	1.04768234321373e-06 \\
142205	9.84844272300389e-07 \\
142590	9.2582797101004e-07 \\
142975	8.7039617308049e-07 \\
143360	8.18326789275847e-07 \\
143745	7.69411894141481e-07 \\
144130	7.23456785312049e-07 \\
144515	6.80279108489223e-07 \\
144900	6.39708043648213e-07 \\
145285	6.01583548143214e-07 \\
145670	5.65755651160682e-07 \\
146055	5.32083797633032e-07 \\
146440	5.00436235784107e-07 \\
146825	4.7068944847295e-07 \\
147210	4.42727620508787e-07 \\
147595	4.15700775635575e-07 \\
147980	3.89604986605896e-07 \\
148365	3.6517368368072e-07 \\
148750	3.42298487443493e-07 \\
149135	3.20878263238011e-07 \\
149520	3.0081862084641e-07 \\
149905	2.82031450138387e-07 \\
150290	2.64434489805065e-07 \\
150675	2.47950926957063e-07 \\
151060	2.32509025144267e-07 \\
151445	2.18041778632383e-07 \\
151830	2.04486591492969e-07 \\
152215	1.91784978509357e-07 \\
152600	1.79882287731914e-07 \\
152984.5	1.68727442406702e-07 \\
153364.5	1.58272700168105e-07 \\
153744.5	1.48473430050533e-07 \\
154124.5	1.3928790404405e-07 \\
154504.5	1.30677103415966e-07 \\
154884.5	1.22604538244087e-07 \\
155264.5	1.1503608005059e-07 \\
155644.5	1.07939804150359e-07 \\
156024.5	1.01285845210963e-07 \\
156404.5	9.50462601401192e-08 \\
156784.5	8.91949025194627e-08 \\
157164.5	8.37073035886426e-08 \\
157544.5	7.85605631103969e-08 \\
157924.5	7.37332458422557e-08 \\
158304.5	6.92052863349168e-08 \\
158684.5	6.49578996148037e-08 \\
159064.5	6.09734972512044e-08 \\
159444.5	5.72356105288385e-08 \\
159824.5	5.37288168955818e-08 \\
160204.5	5.04386727939732e-08 \\
160584.5	4.73516500454352e-08 \\
160964.5	4.44550771194763e-08 \\
161344.5	4.17370836225395e-08 \\
161724.5	3.91865487281429e-08 \\
162106.5	3.67930533262673e-08 \\
162491.5	3.45468347262567e-08 \\
162876.5	3.24387447458996e-08 \\
163261.5	3.04602104095331e-08 \\
163646.5	2.86031973106837e-08 \\
164031.5	2.68601747510644e-08 \\
164416.5	2.52240845433072e-08 \\
164801.5	2.36883101467633e-08 \\
165186.5	2.224664902295e-08 \\
165571.5	2.08932861012201e-08 \\
165956.5	1.96227698534557e-08 \\
166341.5	1.84299882577399e-08 \\
166726.5	1.73101483702531e-08 \\
167111.5	1.62587554530802e-08 \\
167496.5	1.5271594488997e-08 \\
167881.5	1.43447125844354e-08 \\
168266.5	1.34744023161382e-08 \\
168651.5	1.26571863545699e-08 \\
169036.5	1.18898030310177e-08 \\
169421.5	1.11691930704261e-08 \\
169806.5	1.04924861576983e-08 \\
170191.5	9.85699005751073e-09 \\
170576.5	9.26017879043783e-09 \\
170961.5	8.69968225236661e-09 \\
171347	8.17327733271256e-09 \\
171737	7.67887731178973e-09 \\
172127	7.21452497742803e-09 \\
172517	6.77838302154399e-09 \\
172907	6.36872776738073e-09 \\
173297	5.98394128692448e-09 \\
173687	5.62250546121135e-09 \\
174077	5.28299543001154e-09 \\
174467	4.96407381866959e-09 \\
174857	4.66610500238218e-09 \\
175247	4.39320607670268e-09 \\
175637	4.13628814532885e-09 \\
176027	3.89441334736063e-09 \\
176417	3.66669911100459e-09 \\
176807	3.45231510046062e-09 \\
177197	3.25047927463018e-09 \\
177587	3.06045594422599e-09 \\
177977	2.88155260763645e-09 \\
178367	2.7131174529238e-09 \\
178757	2.55453685982232e-09 \\
179147	2.40523340133691e-09 \\
179537	2.26466334574127e-09 \\
179927	2.13231515777679e-09 \\
180317	2.00770716718424e-09 \\
180707	1.89038606990266e-09 \\
181097	1.77992504069024e-09 \\
181487	1.67592256739013e-09 \\
181877	1.57800023048438e-09 \\
182267	1.48528261911807e-09 \\
182657	1.39531169951468e-09 \\
183047	1.31080696297303e-09 \\
183437	1.23143478747423e-09 \\
183827	1.15688264523683e-09 \\
184217	1.08685643818163e-09 \\
184607	1.02108060895389e-09 \\
184997	9.59296142521993e-10 \\
185387	9.0126001106583e-10 \\
185777	8.46743952731543e-10 \\
186167	7.95533583453079e-10 \\
186557	7.47427897351827e-10 \\
186947	7.02237934468997e-10 \\
187337	6.59786281165253e-10 \\
187727	6.19906515009205e-10 \\
188117	5.82442261087834e-10 \\
188507	5.4724652587268e-10 \\
188897	5.14181641708689e-10 \\
189287	4.83118101080038e-10 \\
189677	4.5393427905438e-10 \\
190067	4.26516044704783e-10 \\
190457	4.00756317020523e-10 \\
190847	3.76554176728661e-10 \\
191237	3.53815310383254e-10 \\
191627	3.32451011164636e-10 \\
192017	3.12377623767901e-10 \\
192407	2.93517155025569e-10 \\
192797	2.75795886128805e-10 \\
193187	2.59144761205476e-10 \\
193577	2.43499054253249e-10 \\
193967	2.2879781402807e-10 \\
194357	2.14983641999567e-10 \\
194747	2.02003136440254e-10 \\
195137	1.8980567118021e-10 \\
195527	1.7834372867398e-10 \\
195917	1.67573011022881e-10 \\
196307	1.57451662818886e-10 \\
196697	1.47940160122317e-10 \\
197087	1.39001976595665e-10 \\
197477	1.30602251235956e-10 \\
197867	1.22708343486266e-10 \\
198257	1.15289833235721e-10 \\
198647	1.08318021219134e-10 \\
199037	1.01765873505855e-10 \\
199427	9.56080214997712e-11 \\
199817	8.98207064281564e-11 \\
200207	8.43815017859129e-11 \\
200597	7.92694243578751e-11 \\
200987	7.44647121742048e-11 \\
201377	6.99489910438444e-11 \\
201767	6.57044973984e-11 \\
202157	6.17151330040144e-11 \\
202547	5.79654102494942e-11 \\
202937	5.44409517466704e-11 \\
203327	5.11282682857939e-11 \\
203717	4.80143147463252e-11 \\
204107	4.50874337865059e-11 \\
204497	4.23361345980311e-11 \\
204887	3.97499810844693e-11 \\
205277	3.73188702162963e-11 \\
205667	3.50336981647104e-11 \\
206057	3.28855276343631e-11 \\
206447	3.08661429748724e-11 \\
206837	2.89677726250659e-11 \\
207227	2.71833111575859e-11 \\
207617	2.55056531450748e-11 \\
208007	2.39285258274435e-11 \\
208397	2.24458784892079e-11 \\
208787	2.10521045040934e-11 \\
209177	1.97415972458259e-11 \\
209567	1.85096937777018e-11 \\
209957	1.73513980961104e-11 \\
210347	1.62624913535581e-11 \\
210737	1.52386991914e-11 \\
211127	1.42761913402012e-11 \\
211517	1.3371193041678e-11 \\
211907	1.25203736267565e-11 \\
212297	1.17202914040604e-11 \\
212687	1.09681708160281e-11 \\
213077	1.02609032381906e-11 \\
213467	9.59587964644015e-12 \\
213857	8.97071306127373e-12 \\
214247	8.38268343628101e-12 \\
214637	7.82990339232015e-12 \\
215027	7.31009697219065e-12 \\
215417	6.82137679675066e-12 \\
215807	6.36174446455584e-12 \\
216197	5.92953464106927e-12 \\
216587	5.523081991754e-12 \\
216977	5.14582820798637e-12 \\
217367	4.79699613364915e-12 \\
217757	4.46814807375517e-12 \\
218147	4.15839584988476e-12 \\
218537	3.86640719440834e-12 \\
218927	3.59129392890623e-12 \\
219317	3.33211236380748e-12 \\
219707	3.08780778723872e-12 \\
220097	2.85765855423392e-12 \\
220487	2.64072097522217e-12 \\
220877	2.43638442753991e-12 \\
221267	2.24381624391867e-12 \\
221657	2.06229477939246e-12 \\
222047	1.89132043360019e-12 \\
222437	1.73017156157584e-12 \\
222827	1.57823754065589e-12 \\
223217	1.43524081508417e-12 \\
223607	1.30034871759221e-12 \\
223997	1.1733392035751e-12 \\
224387	1.05354613921804e-12 \\
224777	9.40802991067358e-13 \\
225167	8.34499136459499e-13 \\
225557	7.34245997335847e-13 \\
225947	6.39877040242709e-13 \\
226337	5.50892664819003e-13 \\
226727	4.67126337611035e-13 \\
227117	3.88022947106492e-13 \\
227507	3.13693515607838e-13 \\
227897	2.43527420451528e-13 \\
228287	1.77358128183869e-13 \\
228677	1.15130127653629e-13 \\
229067	5.63438184997267e-14 \\
229457	1.0547118733939e-15 \\
229847	-5.10147479815259e-14 \\
230237	-1.0019762797242e-13 \\
230627	-1.46438416948058e-13 \\
231017	-1.90070181815827e-13 \\
231407	-2.31203944878189e-13 \\
231797	-2.69950728437607e-13 \\
232187	-3.06477065947774e-13 \\
232577	-3.40782957408692e-13 \\
232967	-3.73201469727746e-13 \\
233357	-4.03732602904938e-13 \\
233747	-4.32820446150117e-13 \\
234137	-4.60353977160821e-13 \\
234527	-4.86388707088281e-13 \\
234917	-5.10869124781266e-13 \\
235307	-5.33739719088544e-13 \\
235697	-5.55444579219966e-13 \\
236087	-5.75817171721837e-13 \\
236477	-5.94968518896621e-13 \\
236867	-6.1300964304678e-13 \\
};
\end{axis}

\end{tikzpicture}
}
\end{center}
\caption{20 runs of \adaalgo~and their median (in bold) on 1D-TV-regularized logistic regression \eqref{eq:logtv}
}
\label{fig:tvcomp}
\end{figure}
